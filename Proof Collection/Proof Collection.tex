\documentclass[a4paper]{article}
\usepackage{amsmath}
\usepackage{amsthm}
\usepackage[english]{babel}
\usepackage[utf8]{inputenc}
\usepackage{amssymb}
\usepackage{amsmath}
\usepackage{tikz-cd}
\usepackage[shortlabels]{enumitem}
\usepackage{enumitem}
\usepackage{tikz}
\usepackage{mathrsfs}
\usepackage{mathtools}
\usepackage{mathabx}
\usepackage[margin=1in]{geometry}
%\usepackage{geometry} %These two package 
%\usepackage{marginnote} % intended to make a notes on the margin


\DeclareTextFontCommand{\textbfit}{%
  \fontseries\bfdefault % change series without selecting the font yet
  \itshape
}
\newtheorem{theorem}{Theorem}[section]
\newtheorem{corollary}{Corollary}[theorem]
\newtheorem{lemma}[theorem]{Lemma}
\theoremstyle{remark}
\newtheorem*{remark}{Remark}
\newtheorem{definition}{Definition}[section]
\newtheorem{prop}{Proposition} 


\newcommand{\er}{\mathbb{R}} %real number
\newcommand{\rn}{\mathbb{R}^n} %n-Euclidean space
\newcommand{\rk}{\mathbb{R}^k} %k-Euclidean Space
\newcommand{\rem}{\mathbb{R}^m}%m-Euclidean space
\newcommand{\h}{\mathbb{H}}	   %Half real 
\newcommand{\hn}{\mathbb{H}^n} %n-Half space
\newcommand{\hk}{\mathbb{H}^k} %k-Half space
\newcommand{\hm}{\mathbb{H}^m} %m-Half space
\newcommand{\rational}{\mathbb{Q}} % rational Q
\newcommand{\Z}{\mathbb{Z}} % Integer Z
\newcommand{\Zn}{\mathbb{Z}^n} % Z^n
\newcommand{\ssatu}{\mathbb{S}^1} % Circle S1
\newcommand{\s}{\mathbb{S}} % Letter mathbb S
\newcommand{\openball}{\mathbb{B}} %open ball
\newcommand{\closedball}{\bar{\mathbb{B}}} %closed ball

\newcommand{\map}{\rightarrow} % --->
\newcommand{\hmap}{\hookrightarrow} % hookrightarrow
\newcommand{\doo}{\partial}    % operator do
\newcommand{\subhim}{\subseteq} % subseteq  
\newcommand{\CM}{C^{\infty}(M)} % Space of smooth function
\newcommand{\VF}{\mathfrak{X}} % Space of VF
\newcommand{\Coo}{C^{\infty}}  % Just C^infty
\renewcommand{\thefootnote}{\fnsymbol{footnote}} 
\renewcommand\qedsymbol{q.e.d.} % q.e.d. replace square
\newcommand{\glie}{\mathfrak{g}} % Lie algebra g
\newcommand{\hlie}{\mathfrak{h}} % Lie algebra h
\newcommand{\GL}{\text{GL}(n,\er)} % GL(n,R)
\newcommand{\GLkr}{\text{GL}(k,\er)} % GL(k,R)
\newcommand{\GLnC}{\text{GL}(n,\mathbb{C})} % GL(n,C)
\newcommand{\GLsaja}{\text{GL}} % GL
\newcommand{\GLV}{\text{GL}(V)} % GL(V)
\newcommand{\LieGL}{\text{Lie}(\text{GL}(n,\er))} % Lie(GL(n,R))
\newcommand{\TInGL}{T_{I_n}\text{GL}(n,\er)} %T_In GL(n,R)
\newcommand{\Mtrix}{\text{M}(n,\er)} % Space of Matrix
\newcommand{\lieMatrix}{\mathfrak{gl}(n,\er)} % Lie algebra gl(n,R) of M(n,R) 
\newcommand{\isomorphic}{\cong} % Isomorphic symbol
\newcommand{\Ltegak}{\text{L}} % L tegak Left Invari
\newcommand{\Id}{\text{Id}} % Identitiy Map
\newcommand{\Lie}{\text{Lie}} % Lie...
\newcommand{\LieG}{\text{Lie}(G)} % Lie(G)
\newcommand{\Ualpha}{U_{\alpha}} % U_{\alpha}
\newcommand{\Ubeta}{U_{\beta}} % U_{\beta}
\newcommand{\alfa}{\alpha}  %alpha letter shortcut
\newcommand{\VarphiAlpha}{\varphi_{\alpha}} %varphialpha map
\newcommand{\VarphiBeta}{\varphi_{\beta}} % varphibeta map
\newcommand{\gmma}{\Gamma} %gamma letter shortcut
\newcommand{\Esection}{\Gamma(E)} % section
\newcommand{\Eprimesection}{\Gamma(E')}
\newcommand{\Fkeriting}{\mathscr{F}} %F keriting
\newcommand{\ddxip}{\frac{\partial}{\partial x^i}\bigg|_p} %do/do xi|_P
\newcommand{\ddxi}{\frac{\partial}{\partial x^i}} % do/dox^i
\newcommand\norm[1]{\left\lVert#1\right\rVert} % making norm
\newcommand\metric[1]{\langle#1\rangle}
\newcommand{\tpm}{T_pM} %T_pM tangent space
\newcommand{\tpstarm}{T^*_pM} % T*_pM Cotagent space
\newcommand\wtilde[1]{\widetilde{#1}} %wide tilde
\newcommand{\seperdua}{\frac{1}{2}} % 1/2
\newcommand{\Inter}{\text{Int }} % Interior
\newcommand{\Conv}{%
	\mathop{\scalebox{1.5}{\raisebox{-0.2ex}{$\Asterisk$}}
	}
}
\newcommand{\In}{\text{In }} % denote the interior of simplex
\newcommand{\Bd}{\text{Bd }} % denote the boundary of simplex 
%\newcommand{\minusset}{\smallsetminus} 
%=======================================================
\title{Differential Geometry : A Workbook}

\author{Kelvin Lois}

\date{December 23, 2017}
\begin{document}
\maketitle  \tableofcontents 

\begin{abstract}
A collection of proofs and details of arguments from topology, smooth manifold theory, and Riemannian geometry. I mostly follows Lee's books \cite{LeeTM},\cite{LeeSM}, and \cite{LeeRM} (if the reference is not stated, then it must from Lee's). But somewhere along the way i added some of the result from other books that seems relevant. Such as \cite{LeeJeff},  \cite{LWTu01},\cite{MilnorDT},\cite{Kosinski} for manifold theory and \cite{WPoor}, \cite{LWTu02}, \cite{LeeJeff}, \cite{doCarmo} for differential geometry and Riemannian geometry. I also add some algebraic topology from \cite{LeeTM},\cite{Rotman},\\ \cite{bredon}, and \cite{Hatcher}. Also there are some Morse Theory from \cite{MilnorM}, \cite{Audin} and \cite{YukioM}.
\end{abstract}


\section{Topology and Topological Manifold}

\subsection*{Chapter 1 Topological Spaces}

\begin{prop}[\textbf{Interior, Closure and Boundary Operation} \cite{bredon} dan \textbf{WikiProof}] Here some useful relation involving interior, closure and boundary. Let $A,B$ be any subsets of a topological space $X$.
\begin{enumerate}[nolistsep]
	\item[(a)] Interior operation 
	\begin{enumerate}[nolistsep]
	\item [(I)] $\Inter (A \cap B) = \Inter (A) \cap \Inter (B), \qquad \Inter (\bigcap_{\alpha} A_{\alpha}) \subhim \bigcap_{\alpha} \Inter (A_{\alpha})$,
	\item [(II)] $\Inter (\bigcup_{\alpha} A_{\alpha}) \supseteq \bigcup_{\alpha} \Inter(A_{\alpha})$.
    \end{enumerate}
\item[(b)] Closure Operation
    \begin{enumerate}[nolistsep]
	\item[(I)] $\overline{\bigcap_{\alpha} A_{\alpha}} \subhim \bigcap_{\alpha}\overline{A}_{\alpha}$
	\item[(II)] $\overline{A \cup B} = \overline{A} \cup \overline{B}, \qquad \overline{\bigcup_{\alpha} A_{\alpha}} \supseteq \bigcup_{\alpha} \overline{A}_{\alpha}$.   
    \end{enumerate}		
\item [(c)] Boundary Operation
\begin{enumerate}[nolistsep]
	\item[(I)] $\doo (A \cap B) \subhim \doo A \cup \doo B$,
	\item[(II)] $\doo (A \cup B) \subhim \doo A \cup \doo B$.
\end{enumerate}
\end{enumerate}
\end{prop}
\begin{remark}
	Used a lot in Proposition 16.8 \cite{LeeSM}.
\end{remark}


\begin{prop}[Proposition 2.30 \cite{LeeTM} (also as Problem 2-6)]
	Suppose $X$ and $Y$ are topological spaces, and $f : X \to Y$ is any map. For any $A \subhim X$ and $B \subhim Y$,
	\begin{enumerate}[nolistsep]
		\item[(a)] $f$ is continous $\Leftrightarrow$ $f(\overline{A}) \subhim \overline{f(A)}$.
		\item[(b)] $f$ is closed $\Leftrightarrow$ $f(\overline{A}) \supseteq \overline{f(A)}$.
		\item[(c)] $f$ is continous $\Leftrightarrow$ $f^{-1}(\text{Int }B) \subhim \text{Int }f^{-1}(B)$.
		\item[(d)] $f$ is open $\Leftrightarrow$ $f^{-1}(\text{Int }B) \supseteq \text{Int }f^{-1}(B)$.
	\end{enumerate} 
\end{prop}
\begin{proof}
	To prove (a), let $f$ be a continous map and $A\subhim X$. Choose any point $y\in f(\overline{A})$ and any neighbourhood $U\subhim Y$ of $y$. By definition, there exists $a \in \overline{A}$ such that $f(a)=y$. Since $f$ is continous, $f^{-1}(U)$ is a neighbourhood of $a \in \overline{A}$, which is contain point of $A$, say $a'$. This means that $f(a') \in f(A)$ and $f(a') \in U$, which concludes that $f(\overline{A}) \subhim \overline{f(A)}$. Conversely, let $U$ be any closed subset in $Y$. The preimage $f^{-1}(U)$ is closed in $X$ $\Leftrightarrow$ $f^{-1}(U) = \overline{f^{-1}(U)}$. Since $f^{-1}(U) \subhim \overline{f^{-1}(U)}$ always holds, we only need to show that $f^{-1}(U) \supseteq \overline{f^{-1}(U)}$. By hypothesis and the fact that $f(f^{-1}(U)) \subhim U$ we have 
	$$
	f(\overline{f^{-1}(U)}) \subhim \overline{f(f^{-1}(U))} \subhim \overline{U} = U,
	$$
	which completes the proof for (a).
	
	For (b), suppose $f$ is a closed map and $A \subhim X$. It is naturally holds that $f(A) \subhim f(\overline{A})$. But since $f(\overline{A})$ is closed, it must contain the closure $\overline{f(A)}$. The converse immidiate.
	
	For (c), let $f$ be a continous map and $B \subhim Y$. Naturally $f^{-1}(\text{Int }B) \subhim f^{-1}(B)$. But since $f^{-1}(\text{Int }B)$ is open and $\text{Int }f^{-1}(B)$ is the largest open subset contained in $f^{-1}(B)$, then $f^{-1}(\text{Int }B) \subhim \text{Int }f^{-1}(B)$. The converse is also immidiate.
	
	For (d), let $f$ is an open map and $B \subhim Y$. By the fact that $f(f^{-1}(B)) \subhim B$ we have
	$$
	f(\text{Int }f^{-1}(B)) \subhim f(f^{-1}(B)) \subhim B.
	$$
	By hypotheis and definition of $\text{Int }B$ we have $f(\text{Int }f^{-1}(B)) \subhim \text{Int }B$ which is same as syaing that $\text{Int }f^{-1}(B) \subhim f^{-1}(\text{Int }B)$. Conversely, let $A$ be an open subset in $X$. The hypothesis $\text{Int }f^{-1}(B) \subhim f^{-1}(\text{Int }B)$ implies $f(\text{Int }f^{-1}(B)) \subhim f(f^{-1}(\text{Int }B))$. Choose $B = f(A)$. By the fact that $A = \text{Int }A$ and $A \subhim f^{-1}(f(A))$ we have
	$$
	\text{Int }f(A) \supseteq f(\text{Int }f^{-1}(f(A))) \supseteq f(\text{Int }A) = f(A).
	$$
	This completes the proof.  
\end{proof}
\begin{remark}
	Used in Proposition 16.3 \cite{LeeSM} and Lemma 4.59 \cite{LeeTM}. From this proposition, there are nice properties of closure, interior and boundary operation involving homeomorphism as proposition below shows.
\end{remark}

\begin{prop}[Exercise 10.H \cite{Viro}] Let $f : X \to Y$ be a homeomorphism. Then for every $A \subhim X$,
	\begin{enumerate}[nolistsep]
		\item[(a)] $ f(\overline{A}) = \overline{f(A)}$
		\item[(b)] $ f(\Inter A) = \Inter (f(A))$
		\item[(c)] $ f(\doo A) = \doo (f(A)) $
	\end{enumerate}
\end{prop}
\begin{proof}
	(a) and (b) follow from Proposition 2.30 above, and (c) follow from (a) and (b), $ f(\doo A) = f(\overline{A} \smallsetminus \Inter A) = f(\overline{A}) \smallsetminus f(\Inter A) = \overline{f(A)} \smallsetminus \Inter (f(A)) = \doo (f(A))$.
\end{proof}
\begin{remark}
	Used in Proposition 16.8 \cite{LeeSM}.
\end{remark}

\begin{prop}[Exercise 2.31 \textbf{Properties of Local Homeomorphism}] :
	\begin{enumerate}[nolistsep]
		\item [(a)] Every homeomorphism is a local homeomorphism.
		\item [(b)] Every local homeomorphism is continous and open.
		\item [(c)] Every bijective local homeomorphism is a homeomorphism.
	\end{enumerate}
\end{prop}
\begin{proof}
	For(a) it is obvious (any point in the open subset $X$ and $f : X\to f(X)=Y$ is homeomorphism). 
	
	For (b), suppose $f : X \to Y$ is a local homeomorphism and $V \subhim Y$ is any open subset.  Let $x \in f^{-1}(V) \subhim X$ arbitrary and $U$ is a neighbourhood such that $f(U)$ is open and $f|_U : U \to f(U)$ is a homeomorphism. Since $V \cap f(U) \subhim f(U)$ is open in $f(U)$, then
	\begin{align*}
	(f|_U)^{-1}(V \cap f(U))&= (f \circ \iota)^{-1}(V \cap f(U))\\ &= \iota^{-1}(f^{-1}(V \cap f(U))) \\ &= U \cap f^{-1}(V \cap f(U))
	\end{align*}
	is open in $U$, hence open in $X$. Moreover, $U \cap f^{-1}(V \cap f(U)) \subhim f^{-1}(V)$. We have showed that for any $x \in f^{-1}(V)$, there exists neighbourhood $W:= U \cap f^{-1}(V \cap f(U))$ such that $W \subhim f^{-1}(V)$. So $f^{-1}(V)$ is open and $f$ is continous. To show $f$ is an open map, let $A \subhim X$ is any open subset and $y \in f(A) \subhim Y$ is arbitrary. Let $x \in A$ such that $f(x) = y$. By hypothesis, there exists a neighbourhood $U \subhim X$ of $x$ such that $f(U)$ is open and $f|_U : U \to f(U)$ is a homeomorphism. Taking the intersection $U \cap A$ we have an open neighbourhood $f(U \cap A) \subhim f(A)$ of $y$. Therefore $f(A)$ is open. 
	
	For (c), since $f$ is bijective, it has inverse $f^{-1}$. From (b), $f$ and $f^{-1}$ is continous. So $f$ is a homeomorphism.
\end{proof}

\begin{prop}[Exercise 2.35 \cite{LeeTM}]
Suppose $X$ is a topological space, and for every $p \in X$ there exists continous function $f : X \to \er$ such that $f^{-1}(0)=\{p\}$. Show that $X$ is Hausdorff. 
\end{prop}
\begin{proof}
Let $p,q \in X$ be arbitrary distinct elements. Let $f : X \to \er$ be a continous function such that $f^{-1}(0)=\{p\}$. Since $f(q) \neq 0 $ and $\er$ is Hausdorff, then we can find disjoint neighbourhoods $U,V \subhim \er$ such that $0 \in V$ and $f(q) \in U$. By continuity, $f^{-1}(U)$ and $f^{-1}(V)$ are neighbourhoods of $q$ and $p$ respectively. They are also disjoint, since $f^{-1}(U) \cap f^{-1}(V)= f^{-1}(U \cap V) = f^{-1}(\emptyset) = \emptyset$. Because $p$ and $q$ are arbitrary, then $X$ is Hausdorff.
\end{proof}

\begin{prop}[Problem 2-14 \cite{LeeTM} : Prove Lemma 2.48 \textbf{Sequence Lemma}]
Suppose $X$ is a first countable space, $A$ is any subset of $X$ and $x$ is any point of $X$.
\begin{enumerate}[nolistsep]
\item[(a)] $x \in\bar{A}$ if and only if $x$ is a limit of a sequence of points in $A$.
\item[(b)] $x \in \text{Int }A$ if and only if every sequence in $X$ converging to $x$ is eventually in $A$.
\item[(c)] $A$ is closed in $X$ if and only if $A$ contains every limit of every convergent sequence of points in $A$.
\item[(d)] $A$ is open in $X$ if and only if every sequence in $X$ converging to a point of $A$ is eventually in $A$. 
\end{enumerate}
\end{prop}
\begin{proof}
For (a), let  $\{x_i\}_{i=1}^\infty$ be a sequence of points in $A$ such that $x_i \to x$. Then for any neighbourhood $U$ of $x$, there exists $N$ such that $x_n \in U$ for all $n \geq N$. This shows that $x \in \bar{A}$. Conversely, suppose that $x\in \bar{A}$. Let $\mathscr{B}_x = \{B_i\}_{i=1}^{\infty}$ be a countable neighbourhood basis for $X$ at $x$. By setting $U_i = \bigcap_{j=1}^{i} B_j$, we have a countable nested neighbourhood basis $\mathscr{U}_x = \{U_i\}_{i=1}^{\infty}$ for $X$ at $x$.  Since $x \in \bar{A}$, we can choose $x_i \in A \cap U_i $ for all $U_i \in \mathscr{U}_x$. We have to show that the sequence $x_i \to x$. Since $\mathscr{U}_x $ is a neighbourhood basis at $x$, then for any neighbourhood $U$ of $x$ there exists $U_N \in \mathscr{U}_x $ such that $x \in U_N \subhim U$. Since $\mathscr{U}_x $ is also nested, then $x \in U_n \subhim U$ for all $n \geq N$. Therefore $x_n \in U$ for all $n \geq N$. So $x_i \to x$.   
\end{proof}

\begin{prop}[Example 2.49 \cite{LeeTM}]
Every Euclidean space is second countable.
\end{prop}
\begin{proof}
The collection of subsets of $\rn$
$$
\mathcal{B}  = \{ B_r(x) : r \text{ is rational and }x \text{ has rational coordinates}  \}.
$$
form a countable basis for $\rn$ with Euclidean topology. First we have to show that $\mathcal{B}$ is a basis for $\rn$ with Euclidean topology and then shows that $\mathcal{B}$ itself is a countable set. 

It is obvious that each element of $\mathcal{B}$ is open, since they are open balls. Suppose that $U$ is an arbitrary open subset of $\rn$. We have to show that $U$ is a union of some elements of $\mathcal{B}$. Its enough to show that we can find an element $B \in \mathcal{B}$ such that $x \in B \subhim U$ for any point $x \in U$. Let $x=(x_1,\dots,x_n)$ be any point in $U$. Since $U$ is open, there exists an open ball $B_r(x) \subhim U$ centered at $x$. By choosing smaller radius, we can assume that the radius $r$ is rational. Since $\mathbb{Q}$ dense in $\mathbb{R}$, for each $i$ we can choose a rational number $y_i$ with $x_i < y_i < x_i+ \frac{r}{2\sqrt{n}}$. The point $y = (y_1,\dots,y_n)$ has rational coordinates and contain in $B_r(x)$, since
$$
\norm{y-x}^2 = \sum_{i=1}^{n} (y_i-x_i)^2 < n\cdot \frac{r^2}{4n} = \frac{r^2}{4} \implies \norm{y-x} <\frac{r}{2}.
$$
This also means that $x \in B_{r/2}(y)$. So we have an element $B_{r/2}(y) \in \mathcal{B}$ that contain $x$. Moreover, $B_{r/2}(y) \subhim B_{r}(x) \subhim U$. To see this, let $z \in B_{r/2}(y)$ be arbitrary. By triangle inequality,
$$
\norm{z-x} \leq \norm{z-y} + \norm{y-x} < \frac{r}{2} + \frac{r}{2} = r \implies z \in B_r(x).
$$ 
Since we can do this for all $x \in U$, then $\mathcal{B}$ is a basis for $\rn$ with Euclidean topology. 

To see the countablity of $\mathcal{B}$, we write $\mathcal{B}$ as 
$$
\mathcal{B} = \bigcup_{x \in \mathbb{Q}^n} \mathcal{B}_x, \quad  \text{with} \quad \mathcal{B}_x = \{ B_{r}(x) : r \in \mathbb{Q}^{+} \}.
$$
Since $\mathbb{Q}^{+}$ and $\mathbb{Q}^n$ are both countable, then so is $\mathcal{B}$. 
\end{proof}

\begin{prop}[Theorem 2.50 : \textbf{Properties of Second Countable Spaces}]
Suppose $X$ is a second countable space.
\begin{enumerate}[nolistsep]
\item[(a)] $X$ is first countable.
\item[(b)] $X$ is separable ($X$ contains a countable dense subset).
\item[(c)] $X$ is Lindel\"{o}f (Every open cover of $X$ has a countable subcover).
\end{enumerate}
\end{prop}
\begin{proof}
Let $\mathcal{B}$ be a countable basis for $X$. To prove (a), for any $p\in X$, just note that the collection of elements of $\mathcal{B}$ that caontain $p$ is a neighbourhood basis at $p$. Since it is a subset of the countable set $\mathcal{B}$, then it is countable.

For (b), we obtain a countable dense subset of $X$ as follows. For each $B_i \in \mathcal{B}$, choose any point $p_i \in B_i$. The collection $V = \{p_i\}_{i=1}^{\infty}$ is a countable subset of $X$. It is also dense in $X$, since for any open subset $U \subhim X$ is a union of some elements of $\mathcal{B}$, so $p_i \in B_i \subhim U$ for some $i$.  Therefore $U \cap V \neq \emptyset$. 

For (c), let $\mathcal{U}$ is an open cover for $X$. Let $\mathcal{B'} \subhim \mathcal{B}$ defined by $B \in \mathcal{B}$ is in $\mathcal{B}'$ if and only if $B$ contained in some element of $\mathcal{U}$. Because $\mathcal{B}'$ is a subset of $\mathcal{B}$ then $\mathcal{B}'$ is countable.  For each $B \in \mathcal{B}'$, choose an element $U_B \in \mathcal{U}$ such that $B \subhim U_B$. Its not hard to verify that the collection $\mathcal{U}' = \{ U_B: B \in \mathcal{B}' \}$ is a countable subcover of $\mathcal{U}$. 
\end{proof}

\subsubsection*{Problems Chapter 2}
\begin{prop}[Problem 2-11 \cite{LeeTM}]
	Let $f : X \to Y$ be a continous map between topological spaces, and let $\mathcal{B}$ be a basis for the topology of $X$. Let $f(\mathcal{B})$ denote the collection $\{ f(B) \mid B \in \mathcal{B}\}$ of the subsets of $Y$. Show that $f(\mathcal{B})$ is a basis for the topology of $Y$ if and only if $f$ is surjective and open.
\end{prop}
\begin{proof}
	First suppose that $f(\mathcal{B})$ is a basis for the topology of $Y$ and let $U$ be any open subset of $X$. By definition of basis, we have $f(X) = f(\bigcup \mathcal{B}) = \bigcup \{f(B) : B \in \mathcal{B} \} = Y$. So $f$ is surjective. Let $U = \bigcup \mathcal{B'}$ for some subcollection $\mathcal{B}' = \{B_i \in \mathcal{B} : i \in I \} \subhim \mathcal{B}$. Therefore $f(U) = f(\bigcup \mathcal{B'}) = \bigcup_{i\in I} f(B_i)$. Since $f(B_i)$ is open for each $i \in I$, therefore $f(U)$ is open in $Y$. So $f$ is an open map.
	Conversely, let $f$ be a surjective open map. Then the set $f(\mathcal{B})$ is a collection of open subsets of $Y$. Since $f(X) = Y$, then $\bigcup f(\mathcal{B}) = f(\bigcup \mathcal{B})= f(X) = Y$. For any open subset $V \subhim Y$, $f^{-1}(V) \subhim X$ is open and $V = f(f^{-1}(V))$. Since $\mathcal{B}$ is a basis for $X$, then there is a subcollection $\mathcal{B}' \subhim \mathcal{B}$ such that $\bigcup \mathcal{B'} = f^{-1}(V)$. Therefore $V = f(\bigcup \mathcal{B'}) = \bigcup \{f(B') : B' \in \mathcal{B'}\}$. So we conclude that $f(\mathcal{B'})$ is a basis for the topology of $Y$.
\end{proof}

\begin{prop}[Problem 2-15 \cite{LeeTM}]
Let $X$ and $Y$ be topological spaces.
\begin{enumerate}[nolistsep]
\item[(a)] Suppose $f : X \to Y$ continous and $p_n \to p$ in $X$. Show that $f(p_n) \to f(p)$ in $Y$.
\item[(b)] Prove that if $X$ is first countable, the converse is true.
\end{enumerate}
\end{prop}
\begin{proof}
Let $U$ be any neighbourhood of $f(p) \in Y$. By continuity, $f^{-1}(U)$ is open subset in $X$ contain $p$. Since $p_n \to p$, there are $N \in \mathbb{N}$ such that $p_n \in f^{-1}(U)$ for all $n \geq N$. Which is equivalent as $f(p_n) \in U$ for all $n \geq N$. This proves (a). 

For (b), let $U$ be any open subset in $Y$ and $p$ be any point in $A = f^{-1}(U)$.  Because $X$ is first countable, we have nested neighbourhood basis $\mathcal{B} = \{B_n : n\in \mathbb{N}\}$ at $p$ (by Lemma 2.47). Let $(p_n)$ be any sequence in $X$ converge to $p$. By hypothesis, this imply that $f(p_n)\to f(p)$. So there exists $N \in \mathbb{N}$ such that $f(p_n) \in U$ for all $n\geq N$, which means $p_n \in f^{-1}(U)=A$ for all $n\geq N$.  To show $A$ is open, we need to show that $p \in \text{Int} A$. If $p \notin \text{Int}A$ then $p \in \text{Ext}A$ or $p \in \doo A$. If $p \in \text{Ext}A$, then we have a neighbourhood $V$ of $p$ such that $V \cap A = \emptyset$, which is not true because $p_n \to p$ and $p_n \in A$ for $n\geq N$ for some $N$. If $p \in \doo A$ then for every $B_n \in \mathcal{B}$, $B_n \smallsetminus A \neq \emptyset$. So we can choose a point $x_n \in B_n \smallsetminus A$ for every $n \in \mathbb{N}$. This sequence $(x_n)$ is converge to $p$, since for any neighbourhood $V$ of $p$ there are $m\in \mathbb{N}$ such that $B_m \subseteq V$, and $x_n \in V$ for all $n\geq m$. By construction $(x_n)$ not in $A$. Contradiction. Therefore $p \in \text{Int}A$.
\end{proof}
\begin{remark}
In metric space, these two definitions is equivalent since metric space is first countable. The proof in part (b) actually can be shortened by referring to Lemma 2.48 (b).
\end{remark}

\begin{prop}[Problem 2-19 \cite{LeeTM}]
	Let $X$ be a topological space and $\mathcal{U}$ is an open cover of $X$. 
	\begin{enumerate}[nolistsep]
		\item [(a)] Suppose we are given a basis for each $U \in \mathcal{U}$ (considered as a subspace of $X$). Show that the union of all those basis is a basis for $X$.
		\item [(b)] Show that if $\mathcal{U}$  is countable and each $U \in \mathcal{U}$ is second countable, then $X$ is second countable.
	\end{enumerate}
\end{prop}
\begin{proof}
	Let $\mathcal{B}_i$ be the countable basis for each $U_i \in \mathcal{U}$. Since $U_i$ is open in $X$, then each elements in $\mathcal{B}_i$ are also open in $X$. Then the union $\mathcal{B} = \bigcup \mathcal{B}_i$ is a countable collection of open subsets in $X$. To show $\mathcal{B}$ is a basis for the topology of $X$, we have to show that every open subset of $X$ is the union of some collection of elements in $\mathcal{B}$. Let $U$ be any open subset of $X$ and $x$ be any point in $U$. Since $\mathcal{U}$ cover $X$, then there is $U_i$ such that $x \in U_i$. Since $x \in U_i \cap U \subhim U_i$, then there are $B \in \mathcal{B}_i$ such that $x \in B \subhim U_i \cap U$. Therefore for any $x \in U$, we have $B_x \in \mathcal{B}$ contained in $U$. Hence we have $\bigcup_{x\in U}B_x = U$. So $\mathcal{B}$ is a countable basis for the topology of $X$. So $X$ is second  countable.
\end{proof}


\begin{prop}[Problem 2-23 \cite{LeeTM}]
	Show that every manifold has a basis of coordinate balls.
\end{prop}
\begin{proof}
	Let $p\in M$ be arbitrary, and $(U_p,\varphi_p)$ be the coordinate chart of $p$. W.l.o.g., we may assume that the map $\varphi_p : U_p \to \hat{U}_p$ takes $U_p$ to an open ball centered at $\hat{p}$ with radius $r_p$. That is we set $\hat{U}_p = \varphi_p(U_p) = B(\hat{p},r_p)$, where
	$$
	B(\hat{p},r_p) := \{ x \in \rn : ||x-\hat{p}|| < r_p \}.
	$$ 
	Let $B_{p,r} := \varphi_p^{-1} (B(\hat{p},r))$, where $B(\hat{p},r)$ is an open ball centered at $\hat{p}$ with radius $r<r_p$. Since $\varphi_p : U_p \to \hat{U}_p$ is a homeomorphism, $B_{p,r}$ is a coordinate ball. By collect all such coordinate balls, we obtain a neighbourhood basis at $p$
	$$
	\mathcal{B}_p := \{ B_{p,r} : \forall r\in \rational  \text{ s.t. } r<r_p\}.
	$$
	The collection of all such neighbourhood basis 
	$$\mathcal{B} = \bigcup_{p \in M} \mathcal{B}_p $$ is a basis of coordinate balls for $M$. To see this, let $U$ be an open subset of $M$. For any point  $p \in U$, there exists an element $B_{p,r} \in \mathcal{B}_p$ such that $B_{p,r}\subhim U$. Therefore $U = \bigcup_{p \in U}B_{p,r}$. Each element in $\mathcal{B}$ is a coordinate ball by construction. Hence $\mathcal{B}$ is the desired basis for $M$.
	\begin{remark}
		Look the improvement of this basis on Proposition 4.60 \cite{LeeTM}.
	\end{remark}
	
\end{proof}

\begin{prop}[Problem 2.25 \cite{LeeTM}]
If $M$ is an $n$-dimensional manifold with boundary, then Int$\,M$ is an open subset of $M$, which itself an $n$-dimensional manifold without boundary.
\end{prop}
\begin{proof}
Let $p \in \text{Int} M \subset M$. By definition of Int$\,M$, $p$ is in the domain of some interior chart $(U,\varphi)$. For any other point $q \in U$ implies $q \in \text{Int} M$. Therefore $p \in U \subset \text{Int} M$. Because $p$ arbitrary, Int$\,M$ is open subset of $M$.

To show that Int$\, M$ is a $n$-manifold we have to show that Int$\, M$ is Hausdorff and second countable. Because $M$ itself is a Hausdorff space then for any two point $p,q \in \text{Int} M \subset M$ there are disjoint open subset of $M$, $U$ and $V$ contain $p$ and $q$ respectively. By taking their intersection with Int$\, M$, we have disjoint open subset of Int$\, M$, $U \cap \text{Int}\,M$ and $V \cap \text{Int}\, M$ contain $p$ and $q$ respectively. Hence Int$\,M$ is Hausdorff.  

And finally by throwing out all the basis set that is not contained in Int$\,M$ from the original countable basis for $M$, the remaining basis set is the countable basis for Int$\,M$.
\end{proof}
\begin{remark}
By this result and Theorem 2.59 (Invarince of Boundary) we have the important Corollary 2.60. This result will appear again in page 27 \cite{LeeSM}.
\end{remark}

\subsection*{Chapter 3 New Spaces from Old}

\begin{prop}[Exercise 3.13 \cite{LeeTM}]
	Let $X$ be any topological space and let $S$ be a subspace of $X$. Show that the inclusion map $S \hookrightarrow X$ is a topological embedding.
\end{prop}
\begin{proof}
	The inclusion map $S \hookrightarrow X$ is a continous map by characteristic property of subspace topology. The restriction to its image is just identity map, which is obvoiusly a homeomorphism.
\end{proof}

\begin{prop}[Exercise 3.17 \cite{LeeTM}]
	Give an example of a topological embedding that is neither an open nor a closed map.
\end{prop}
\begin{proof}
	Consider the inclusion map of the subspace $Y\smallsetminus (0,0) \subset \er^2$, where $ Y = \{(0,y): \forall y \in \er \}$. 
\end{proof}

\subsubsection*{Disjoint Union Spaces}
 \begin{prop}[Theorem 3.41 (also as Problem 3.10) \textbf{Characteristic Property of Disjoint Union Spaces}]
 	Suppose that $(X_{\alpha})_{\alpha \in A}$ is an indexed family of topological spaces, and $Y$ is any topological space. A map $f : \bigsqcup_{\alpha \in A} X_{\alpha} \to Y$ is continous if and only if its restrictions to each $X_{\alpha}$ is continous. The disjoint union topology is the unique topology on $\bigsqcup_{\alpha \in A} X_{\alpha}$ with this property. 
 \end{prop}
\begin{proof}
	By definition of disjoint union topology on $\bigsqcup_{\alpha \in A} X_{\alpha}$, 
	\begin{align*}
	&f: \bigsqcup_{\alpha \in A} X_{\alpha} \to Y \text{ continous } \\&\Leftrightarrow  \forall V \subhim Y \text{ open, }  \iota_{\alpha}^{-1}(f^{-1}(V)) =(f \circ \iota_{\alpha})^{-1}(V)=(f|_{X_{\alpha}})^{-1}(V) \text{ open } \forall \alpha \in A \\ &\Leftrightarrow f|_{X_{\alpha}} :X_{\alpha} \to Y  \text{ continous } \forall \alpha \in A.
	\end{align*} 
	
	To show that the disjoint union topology is the only topology that satisfy the characteristic property, denote $X:=\bigsqcup_{\alpha \in A} X_{\alpha}$ and let $(X, \mathcal{D})$ be the disjoint union topology and $(X,\mathcal{T})$ be the other topology on $\bigsqcup_{\alpha \in A} X_{\alpha}$ that satisfy the property. Let $\Id : (X,\mathcal{T}) \to (X,\mathcal{D})$ be the identity map on $X$. By applying the characteristic property on the following two diagrams
	\[
	\begin{tikzcd}
	(X,\mathcal{D}) \arrow[r,"Id_{\mathcal{D}}"] &  (X,\mathcal{D})\\
	X_{\alpha} \arrow[u,"\iota_{\alpha}"] \arrow[ur,"\iota_{\alpha}",swap]& 
	\end{tikzcd}
	\qquad
	\begin{tikzcd}
	(X,\mathcal{T}) \arrow[r,"Id_{\mathcal{T}}"] &  (X,\mathcal{T})\\
	X_{\alpha} \arrow[u,"\iota'_{\alpha}"] \arrow[ur,"\iota'_{\alpha}",swap]& 
	\end{tikzcd}
	\]
    we conclude that both canonical injection maps $\iota_{\alpha} : X_{\alpha} \to (X,\mathcal{D})$ and $\iota_{\alpha}' : X_{\alpha} \to (X,\mathcal{T})$ are continous. Then by same argument
    \[
    \begin{tikzcd}
    (X,\mathcal{T}) \arrow[r,"Id"] &  (X,\mathcal{D})\\
    X_{\alpha} \arrow[u,"\iota'_{\alpha}"] \arrow[ur,"\iota_{\alpha}",swap]& 
    \end{tikzcd}
    \qquad
    \begin{tikzcd}
    (X,\mathcal{D}) \arrow[r,"Id^{-1}"] &  (X,\mathcal{T})\\
    X_{\alpha} \arrow[u,"\iota_{\alpha}"] \arrow[ur,"\iota'_{\alpha}",swap]& 
    \end{tikzcd}
    \]
    we conclude that $\Id : (X,\mathcal{T}) \to (X,\mathcal{D})$ is a homeomorphism.
\end{proof}

\begin{prop}[Exercise 3.43 \textbf{Other Properties of Disjoint Union Spaces} \cite{LeeTM}]
	Let $(X_{\alpha})_{\alpha \in A}$ be an indexed family of topological spaces.
	\begin{enumerate}[nolistsep]
		\item[(a)] A subset of $\bigsqcup_{\alpha \in A} X_{\alpha}$ is closed if and only if its intersection with each $X_{\alpha}$ is closed.
		\item [(b)] Each canonical injection $\iota_{\alpha} : X_{\alpha} \to \bigsqcup_{\alpha \in A} X_{\alpha}$ is a topological embedding and an open and closed map.
		\item [(c)] If each $X_{\alpha}$ is Hausdorff, then so is $\bigsqcup_{\alpha \in A} X_{\alpha}$.
		\item [(d)] If each $X_{\alpha}$ is first countable, then so is $\bigsqcup_{\alpha \in A} X_{\alpha}$.
		\item [(e)] If each $X_{\alpha}$ is second countable and the indexed set $A$ is countable, then $\bigsqcup_{\alpha \in A} X_{\alpha}$ is second countable.
	\end{enumerate}
\end{prop}
\begin{proof}
	Let $X: = \bigsqcup_{\alpha \in A} X_{\alpha}$. For (a), let $A \subhim X$ be a subset. Then
	\begin{align*}
	A \text{ closed } &\Leftrightarrow X \smallsetminus A \text{ open } \Leftrightarrow \iota_{\alpha}^{-1}(X \smallsetminus A) \text{ open in }X_{\alpha}, \forall \alpha \in A  \\ &\Leftrightarrow X_{\alpha}\smallsetminus \iota_{\alpha}^{-1}(A) \text{ is open in }X_{\alpha}, \forall \alpha \in A \\
	&\Leftrightarrow \iota_{\alpha}^{-1}(A) \text{ is closed in }X_{\alpha}, \forall \alpha \in A.
	\end{align*}
	For (b), the canonical injection $\iota_{\alpha} : X_{\alpha} \to \bigsqcup_{\alpha \in A} X_{\alpha}$ are continous (by characteristic property of disjoint union spaces). Considered $X_{\alpha}^* = \iota_{\alpha} (X_{\alpha}) \subhim \bigsqcup_{\alpha \in A} X_{\alpha}$ as a subspace, the map $\bar{\iota}_{\alpha} : X_{\alpha} \to X_{\alpha}^*$, defined as $\bar{\iota}_{\alpha} (x) = \iota_{\alpha}(x)$, is a continous bijective map. Apply the characteristic property of subspace topology to the following diagram
	\[
	\begin{tikzcd}
	X^*_{\alpha} \arrow[r,"(\bar{\iota}_{\alpha})^{-1}"] &  X_{\alpha} \arrow[r,"\iota_{\alpha}"] & \bigsqcup_{\alpha \in A} X_{\alpha}
	\end{tikzcd}
	\]
	we conclude that $(\bar{\iota}_{\alpha})^{-1}$ is continous since the composition $\iota_{\alpha} \circ (\bar{\iota}_{\alpha})^{-1}$ is equal to inclusion map of $X^*_{\alpha} \subhim \bigsqcup_{\alpha \in A} X_{\alpha}$ which is continous. Hence $\iota_{\alpha}$ are topological embedding. To show that $\iota_{\alpha} : X_{\alpha} \to \bigsqcup_{\alpha \in A} X_{\alpha}$ is an open and closed map, let $U\subhim X_{\alpha}$ be an open subset. Since $\iota_{\alpha}(U) \subset X_{\alpha}^*$ is open in $X_{\alpha}^*$ (since $\iota_{\alpha}$ is a topological embedding), then it is open in $\bigsqcup_{\alpha \in A} X_{\alpha}$ by definition of disjoint union topology.   
	
	For (c),  let $y_1$ and $y_2$ be two distinct points in $X_{\alpha}^*$ for some $\alpha \in A$. Since $\iota_{\alpha}$ is injective, there are distinct $x_1,x_2 \in X_{\alpha}$ and by hypothesis there are disjoint open subset $U_1,U_2 \subhim X_{\alpha}$ such that $x_1 \in U_1$ and $x_2 \in U_2$. By (b), the images $V_1=\iota_{\alpha}(U_1)$ and $V_2=\iota_{\alpha}(U_2)$ are the open subsets in $X_{\alpha}^*$ where $y_1 \in V_1$ and $y_2 \in V_2$, and $V_1 \cap V_2 = \iota_{\alpha}(U_1 \cap U_2) = \emptyset$. For two points such that $y_1 \in X_{\alpha}^*$ and $y_2 \in X_{\beta}^*$, $\alpha \neq \beta$, then the desired disjoint open subsets are $X_{\alpha}^*$ and $X_{\beta}^*$.  
	
	For (d), any point $p_{\alpha} \in X_{\alpha}^* \subhim \bigsqcup_{\alpha \in A} X_{\alpha}$ has a countable neighbourhood basis 
	$$
	\mathcal{B}_{p_{\alpha}} = \{ \iota_{\alpha}(B) : \forall B\in \mathcal{B}_p \}
	$$
	where $\mathcal{B}_p$ is a countable neighbourhood basis at $p = \iota_{\alpha}^{-1}(p_{\alpha}) \in X_{\alpha}$. Th elements are open subsets (by (b)) and certainly countable since $\mathcal{B}_p$ is countable. To verify that it is a neighbourhood basis, let $U\subhim \bigsqcup_{\alpha \in A} X_{\alpha}$ is a neighbourhood of $p_{\alpha}$ (possibly not entirely contained in $X_{\alpha}^*$). By (b), $\iota_{\alpha}^{-1}(U)$ is a neighbourhood of $p \in X_{\alpha}$. By hypothesis, $\exists B \in \mathcal{B}_p$ such that $B \subhim \iota_{\alpha}^{-1}(U)$. Therefore $\iota_{\alpha}(B) \subhim \iota_{\alpha} \iota_{\alpha}^{-1}(U) \subhim U$.  
	
	For (e), suppose that $A$ is a countable indexed set and $\mathcal{B}_{\alpha} = \{U_{\alpha}^i\}_{i=1}^{\infty}$ is the countable basis for $X_{\alpha}$. It is clear that 
	$$
	\mathcal{B} = \{\iota_{\alpha}(U_{\alpha}^i) : \forall U_{\alpha}^i \in \mathcal{B}_{\alpha}, \forall \alpha \in A \}
	$$
	is a countable basis for $\bigsqcup_{\alpha \in A} X_{\alpha}$.
\end{proof}

\subsubsection*{Quotient Spaces}
\begin{prop}[Proposition 3.56 \cite{LeeTM}]
	Suppose $P$ is a second countable space and $M$ is a quotient space of $P$. If $M$ is locally euclidean, then it is second countable. Thus, if $M$ is locally euclidean and Hausdorff, then it is a manifold.
\end{prop}
\begin{proof}
	Let $q : P \to M$ be its associated quotient map. Since $M$ is locally euclidean, $M$ is covered by coordinate balls $\mathcal{U}$. It is easy to see that the collection $\mathcal{A} = \{q^{-1}(U) : U \in \mathcal{U} \}$ is an open cover for $P$. By second countability, we have a countable subcover $\mathcal{A}' \subhim \mathcal{A}$ of $P$, and we can write it as $\mathcal{A}' = \{ A_i = q^{-1}(U_i) : U_i \in \mathcal{U} \}$. Since $q$ surjective, for any $U \in \mathcal{U}$, $q(q^{-1}(U)) = U$. So for any $A_i \in \mathcal{A}'$, $q(A_i) = q(q^{-1}(U_i)) = U_i$. We obtain countable cover for $M$ as 
	$$
	\mathcal{U}' = \{ q(A_i) :  \forall A_i \in \mathcal{A}'  \} \subhim \mathcal{U}.
	$$
	Each coordinate ball $U_i \in \mathcal{U}'$ is homeomorphic to an euclidean ball (which is second countable) in $\mathbb{R}^n$. It is not hard to verify that those $U_i$ are second countable. By Problem 2-19, the union of all basis for each $U_i$ is a countable basis for $M$.   
\end{proof}

\begin{prop}[Exercise 3.59 \cite{LeeTM}]
	Let $q : X \to Y$ is a map. For any subset $U \subhim X$, show that the following equivalent.
	\begin{enumerate}[nolistsep]
		\item [(a)] $U$ is saturated 
		\item [(b)] $U = q^{-1}(q(U))$.
		\item [(c)] $U$ is union of fibers.
		\item [(d)] If $x \in U$, then every point $x'\in X$ such that $q(x) = q(x')$ is also in $U$.
	\end{enumerate}
\end{prop}
\begin{proof}
	We will show that $(a) \Leftrightarrow (b) \Rightarrow (c) \Rightarrow (d) \Rightarrow (b)$. For $(a) \Rightarrow (b)$, suppose that $U = q^{-1}(V)$ for some subset $V \subhim Y$. Then $$q(U) = q(q^{-1}(V)) \subhim V \implies q^{-1}(q(U)) \subhim q^{-1}(V) = U.$$
	Since $U \subhim q^{-1}(q(U))$ always holds, then $U = q^{-1}(q(U))$. $() \Rightarrow (a)$ immidiate.
	
	For $(b) \Rightarrow (c)$,
	$$
	U = q^{-1}(q(U)) = q^{-1}\big( \bigcup_{y \in q(U)} \{y\}  \big) = \bigcup_{y \in q(U)} q^{-1}(\{y\}).
	$$
	For $(c) \Rightarrow (d)$, suppose $(c)$ holds and $x \in U$. For any $x' \in X$ such that $q(x') = q(x)=y$, then $x' \in q^{-1}(\{y\}) \subhim U$.
	
	For $(d) \Rightarrow (b)$, we only need to show that $q^{-1}(q(U)) \subhim U$. Suppose that $(d)$ holds and let $x' \in q^{-1}(q(U))$ be arbitrary. This means $y = q(x') \in q(U)$. Since $y \in q(U)$, there exists $x \in U$ such that $q(x) = y = q(x')$. Since $(d)$ holds, then $x'\in U$. Therefore $q^{-1}(q(U)) \subhim U$.  
\end{proof}

\begin{prop}[Exercise 3.61 \cite{LeeTM} \textbf{Characterization of Quotient Map in Terms of Saturated Open (Closed) Subsets}]\label{Proposition 12}
	A continous surjective map $q: X \to Y$ is a quotient map $\Leftrightarrow$ it takes saturated open subsets to open subsets, or saturated closed subsets to closed subsets.  
\end{prop}
\begin{proof}
	Suppose that $q$ is a quotient map and $U$ is a saturated open subset. The subset $q(U) \subhim Y$ is open if and only if $q^{-1}(q(U))$ is open. Since $U$ saturated open subset, then $U = q^{-1}(q(U))$ is open. 
	
	Conversely, suppose that $q$ takes any saturated open subset to open subsets. The map $q$ is a quotient map if we can show that : any subset $V \subhim Y$ is open $\Leftrightarrow $ $q^{-1}(V)$ is open in $X$. Since $q$ is continous by hypothesis, then we only need to show that if $q^{-1}(V)$ is open, then $V\subhim Y$ is open. Since $U = q^{-1}(V)$ is saturated open subset and $q$ is surjective, then $q(U) = q(q^{-1}(V)) = V$ is open by hypothesis.  
  \end{proof}

\begin{prop}[Exercise 3.63 : \textbf{Properties of Quotient Maps}]:
	\begin{enumerate}[nolistsep]
		\item [(a)] Any composition of quotient maps is a quotient map.
		\item [(b)] An injective quotient map is a homeomorphism.
		\item [(c)] If $q : X \to Y$ is a quotient map, a subset $K \subhim Y$ is closed if and only if $q^{-1}(K)$ is closed in $X$.
		\item [(d)] If $q:X\to Y$ is a quotient map and $U\subhim X$ is a saturated open or closed subset, then the restriction $q|_U : U \to q(U)$ is a quotient map.
		\item [(e)] If $\{q_{\alpha} : X_{\alpha} \to Y_{\alpha} \}_{\alpha \in A}$ is an indexed family of quotient maps, then the map $q:\bigsqcup_{\alpha \in A} X_{\alpha} \to \bigsqcup_{\alpha \in A} Y_{\alpha}$ whose restriction to each $X_{\alpha}$ is equal to $q_{\alpha}$ is a quotient map.
	\end{enumerate}
\end{prop}
\begin{proof}
	For (a), let $q : X \to Y$ and $r : Y \to Z$ are quotient maps and $U\subhim X$ is a saturated open subset with respect to the composition map $r \circ q : X\to Z$. Since by hypothesis, $Z$ has quqotient topology induced by $r$ and $Y$ has quotient topology induced by $q$, then
	\begin{align*}
	(r \circ q)(U) \text{ is open in $Z$ }&\Leftrightarrow r^{-1}(r \circ q(U)) \text{ is open in $Y$ }\\
	&\Leftrightarrow q^{-1}\big( r^{-1}(r \circ q(U)) \big) \text{ is open in $X$} \\
	&\Leftrightarrow (r \circ q)^{-1}\big(r \circ q(U)\big) \text{ is open in }X.
	\end{align*} 
	But since $U$ is saturated w.r.t $r \circ q$, then $(r \circ q)^{-1}\big(r \circ q(U)\big) = U$ which is open. Therefore by Proposition 3.60, $r \circ q$ is a quotient map.
	
	For (b), an injective quotient map $q : X \to Y$ is certainly a continous bijective map by definition. For any $U \subhim X$ open, the image $q(U)$ is open $\Leftrightarrow q^{-1}(q(U))$ is open in $X$. By surjectivity alone, $q^{-1}(q(U)) = U$ which is open. Therefore $q$ is a homeomorphism.
	
	For (c), let $q: X \to Y$ be a quotient map and $K \subhim Y$ be a subset. Then
	\begin{align*}
	K \text{ is closed } &\Leftrightarrow Y\smallsetminus K \text{ is open } \Leftrightarrow q^{-1}(Y \smallsetminus K) \text{ is open in }X \\ &\Leftrightarrow X\smallsetminus q^{-1}(K) \text{ is open in }X \\&\Leftrightarrow q^{-1}(K) \text{ is closed in }X.
	\end{align*}
	
	For (d), by Corollary 3.10, the restriction $q|_U : U \to q(U)$ is a continous surjective map. Let $A\subhim U$ be a saturated open subset with respect to $q|_U$. By Proposition 3.60, we only need to show that $(q|_U)(A)$ is open in $q(U)$. By definition, there exists $B \subhim q(U)$ such that $A = (q|_U)^{-1}(B)$. Note that
	$$
	A = (q \circ \iota_U)^{-1}(B) = \iota_U^{-1}(q^{-1}(B)) = q^{-1}(B) \cap U = q^{-1}(B), 
	$$  
	where the last follows from $B \subhim q(U) \Rightarrow q^{-1}(B) \subhim q^{-1}(q(U))=U$. So $A$ is a saturated subset with respect to $q: X \to Y$. Since $A$ is open in $U$ and $U$ is open in $X$, then $A$ is a saturated open subset with respect to $q$. Since $q$ is a quotient map, Proposition 3.60 implies that $q(A)$ is open in $X$, hence open in $q(U)$. But because
	$$
	(q|_U)(A) = (q \circ \iota_U)(A) = q(A),
	$$ 
	then $q|_U : U \to q(U)$ is a quotient map.
	
	For (e), let $\{ q_{\alpha} : X_{\alpha} \to Y_{\alpha} \}$ be quotient maps and $q : \bigsqcup_{\alpha \in A} X_{\alpha} \to \bigsqcup_{\alpha \in A} Y_{\alpha} $ be a map whose restriction to each $X_{\alpha}$ is equal to $q_{\alpha}$. Technically, this means that $q|_{X_{\alpha}} = q \circ \iota_{\alpha} = \tilde{\iota}_{\alpha} \circ q_{\alpha}$, where $\iota_{\alpha} : X_{\alpha} \to \bigsqcup_{\alpha \in A} X_{\alpha}$ and $\tilde{\iota}_{\alpha} : Y_{\alpha} \to \bigsqcup_{\alpha \in A} Y_{\alpha}$ are the canonical injection. For convenient, denote $X= \bigsqcup_{\alpha \in A} X_{\alpha}$ and $Y=\bigsqcup_{\alpha \in A} Y_{\alpha}$ . The map $q$ is surjective and continous (by characteristic property). So it is a quotient map if we can show that for any saturated open subset $U \subhim X$ the image $q(U) \subhim Y$ is open. By the fact that $U = q^{-1}(q(U))$, we have
	\begin{align*}
	U \text{ open in }X &\Leftrightarrow \iota_{\alpha}^{-1}(U) \text{ open in }X_{\alpha}, \forall \alpha \\
	&\Leftrightarrow \iota_{\alpha}^{-1} q^{-1}(q(U)) \text{ open in }X_{\alpha}, \forall \alpha \\
	&\Leftrightarrow  q_{\alpha}^{-1} (\tilde{\iota}_{\alpha})^{-1}(q(U)) \text{ open in }X_{\alpha}, \forall \alpha \\
	&\Leftrightarrow (\tilde{\iota}_{\alpha})^{-1}(q(U)) \text{ open in }Y_{\alpha}, \forall \alpha \\
	&\Leftrightarrow q(U) \text{ open in }Y.
	\end{align*}
	Therefore by Proposition 3.60, $q : X \to Y$ is a quotient map.
\end{proof}




\begin{prop}[\textbf{Example 3.64 }: $\mathbb{S}^n$ is quotient space of $\mathbb{R}^{n-1}$]
	To show this, we have to construct a quotient map $q : \mathbb{R}^{n-1} \to \mathbb{S}^n$. This can be done if we contruct a continous surjective map which takes saturated open subsets in $\mathbb{R}^{n-1}$ to open subsets in $\mathbb{S}^n$ (Proposition 3.60). The obvious continous surjective map is
	$$
	q : x \mapsto \frac{x}{|x|}.
	$$
	Now we use Exercise 3.59 to observe whats the saturated open subset looks like. For any point in $\mathbb{S}^n$, the fiber is an (radial) open ray in $\mathbb{R}^{n-1}$ pass through that point. By Exercise 3.59, the saturated subset is union of these fibers. Take any saturated open subset. The image of this subset under $q$ is exactly the intersection of this subset with $\mathbb{S}^n$. With subspace topology on $\mathbb{S}^n \subhim \er^{n-1}$, the image is open in $\mathbb{S}^n$. Therefore $q$ takes saturated open subsets to open subsets. Hence $\mathbb{S}^n$ is a quotient space of $\er^{n-1}$.
\end{prop}

From Proposition \ref{Proposition 12}, it follows a sufficient condition for a surjective continous map to be a quotient map. The proof is immidate.

\begin{prop}[Proposition 3.67 \cite{LeeTM}]
	If $q : X \to Y$ is a surjective continous map that is also an open or closed map, then it is a quotient map.
\end{prop}

The relation between the last two propositions then can be observe as follows:
If $q : X \to Y$ is a continous surjective map,
\[
\begin{tikzcd}
&  q \text{ open map} \arrow[d] \\
q \text{ quotient map} \arrow[r,leftrightarrow] & U\subhim X \text{ saturated \& open} \Rightarrow q(U) \text{ open}
\end{tikzcd}
\]

\textbf{Example :} The projection map $\pi_i : X_1 \times \cdots \times X_k \to X_i.$

\begin{prop}[Proposition 3.69 \cite{LeeTM} \textbf{Initiation of Closed Map Lemma}]
	Suppose $X$ and $Y$ are topological spaces, and $f : X \to Y$ is a continous map that is either open or closed.
	\begin{enumerate}[nolistsep]
		\item [(a)] If $f$ is \textit{injective}, it is a \textit{topological embedding}
		\item [(b)] If $f$ is \textit{surjective}, it is a \textit{quotient map}.
		\item [(c)] If $f$ is \textit{bijective}, it is a \textit{homeomorphism}. 
	\end{enumerate}
\end{prop}
\begin{remark}
	This result become \textbf{Closed Map Lemma}(Lemma 4.50) if $X$ is compact and $Y$ is hausdorff. This implies that $f$ is a closed map, and therefore the result follows. 
\end{remark}

\begin{prop}[Exercise 3.72 \cite{LeeTM} \textbf{Uniqueness of Quotient Topology}]
	Given a topological space $X$, a set $Y$, and a surjective map $q : X \to Y$, the quotient topology is the only topology on $Y$ for which the characteristic property holds.
\end{prop}
\begin{proof}
	Denote $(Y,\mathcal{Q})$ as the set $Y$ endowed with quotient topology induced by $q : X \to Y$. Suppose that there exists another topology $(Y,\mathcal{T})$ such that the characteristic property holds. We will show that the identity map $\Id : (Y,\mathcal{Q}) \to (Y,\mathcal{T})$ is a homeomorphism. 
	
	By hypothesis, the following diagram commute
	\[
	\begin{tikzcd}
	(X,\mathcal{O}) \arrow[d,"q",swap] \arrow[dr,"q"] &  \\
	(Y,\mathcal{T}) \arrow[r,"\Id^{-1}",swap] & (Y,\mathcal{Q})
	\end{tikzcd}
	\]
	and $\Id^{-1}$ is continous. However, since the identity map $i : (Y,\mathcal{T}) \to (Y,\mathcal{T})$ is continous and the following diagram commute
	\[
	\begin{tikzcd}
	(X,\mathcal{O}) \arrow[d,"q",swap] \arrow[dr,"q"] &  \\
	(Y,\mathcal{T}) \arrow[r,"i",swap] & (Y,\mathcal{T})
	\end{tikzcd}
	\] 
	then the map $q : (X,\mathcal{O}) \to (Y,\mathcal{T})$ is continous by characteristic property. Finally, by observe the following diagram
	\[
	\begin{tikzcd}
	(X,\mathcal{O}) \arrow[d,"q",swap] \arrow[dr,"q"] &  \\
	(Y,\mathcal{Q}) \arrow[r,"\Id",swap] & (Y,\mathcal{T})
	\end{tikzcd}
	\]
	we conclude that $\Id : (Y,\mathcal{Q}) \to (Y,\mathcal{T})$ is continous, and therefore $(Y,\mathcal{Q})$ and $(Y,\mathcal{T})$ are homeomorphic.  
\end{proof}

\subsubsection*{Adjunction Spaces}

\begin{prop}[Proposition 3.77 \textbf{Properties of Adjunction Spaces}]
	Let $X \cup_f Y$ be an adjuction space, and let $q: X \sqcup Y \to X \cup_f Y$ be the associated quotient map.
	\begin{enumerate}[nolistsep]
		\item[(a)] The restriction $ q|_X : X \to X \cup_f Y$ is a \textbf{topological embedding}, whose image set $q(X)$ is a \textbf{closed subspace} of $X \cup_f Y$.
		\item [(b)] The restriction $q|_{Y \smallsetminus A} : Y \smallsetminus \to X \cup_f Y$ is a \textbf{topological embedding}, whose image set $q(Y \smallsetminus A)$ is an \textbf{open subspace} of $X \cup_f Y$.
		\item [(c)] $X \cup_f Y$ is the \textbf{disjoint union} of $q(X)$ and $q(Y \smallsetminus A)$. 
	\end{enumerate}
\end{prop}
\begin{proof}
	For (a), we will first show that $q|_X$ is a closed map. Let $B \subhim X$ is any closed subset. Note that
	\begin{align*}
	&(q|_X)(B) = q(B)\subhim (X\sqcup Y)/{\sim} \text{ closed }\\ &\Leftrightarrow q^{-1}(q(B)) \subhim X \sqcup Y \text{ closed} \\ &\Leftrightarrow q^{-1}(q(B)) \cap X \text{ and } q^{-1}(q(B)) \cap Y \text{ both closed}
	\end{align*}
	Let $x \in X \cap q^{-1}(q(B))$. This means that $q(x) \in q(B) \subhim X \cup_f Y = (X \sqcup Y)/{\sim}$. So there exists $b \in B\subhim X$ such that $q(x) = q(b) \in (X \sqcup Y)/{\sim}$. If $b \in X \smallsetminus f(A)$, then $q(b)=[b] = \{b\} = q(x) = [x]$. So $x {\sim} b$ but $b \notin f(A)$, this implies $x=b$. If $b \in f(A)$, then $\exists a \in A$ such that $f(a) = b$. So  $q(b) = [b]=[f(a)] =q(x) = [x]$ for some $a \in A \subhim Y$. We have $x {\sim} f(a)$ and  $x \in X$, hence $b = f(a) = x$. We conclude $X \cap q^{-1}(q(B)) \subhim B$. Since $B \subhim q^{-1}(q(B))$ always holds, then $q^{-1}(q(B)) = B$, which is closed subset. We showed that $q|_X$ is a closed map. In particlular, $q(X)$ is closed subspace of $X \cup_f Y$.
	
	To show that $q|_X$ is a topological embedding, it is enough (by Proposition 3.69) to show that $q|_X$ is injective. Given any two points $x_1,x_2 \in X$, the condition $[x_1] = q(x_1) = q(x_2) = [x_2]$ immidiately imply that $x_1=x_2$. This completes the proof for (a).
	
	 For (b), just note that $Y \smallsetminus A$ is a saturated open subset of $X \sqcup Y$. So the restriction $q|_{Y \smallsetminus A} : Y \smallsetminus A \to q(Y \smallsetminus A)$ is a quotient map by Proposition 3.62(d). Since it is bijective, it is a homeomorphism. By definition of quotient topology, $q(Y \smallsetminus A)$ is open in $(X \sqcup Y)/{\sim}$ $\Leftrightarrow$ $q^{-1}(q(Y \smallsetminus A))$ is open in $X \sqcup Y$. Since $Y \smallsetminus A$ is saturated open subset, then $q^{-1}(q(Y \smallsetminus A)) = Y \smallsetminus A$ is open.
	
	For (c), it is a consequence of the equivalence relation on $X \sqcup Y$.
\end{proof}

\begin{prop}[\textbf{Example 3.78(b) Adjunction Spaces}]
	Let $A=\ssatu$, $Y=\closedball^2$, and $X=\closedball^2$. So $A \subhim Y$ and define a map $f: A \hookrightarrow \closedball^2$ be the inclusion map. Then the adjuction space $\closedball ^2 \cup_f \closedball^2 = (\closedball^2 \sqcup \closedball^2)/{\sim}$ is homeomorphic to $\s^2$. To show this, we need to exhibit a quotient map $\tilde{q} : \closedball^2 \sqcup \closedball^2 \to \s^2$ that make the same identification as the quotient map $q: \closedball^2 \sqcup \closedball^2 \to \closedball ^2 \cup_f \closedball^2$ (and then use Theorem 3.75).
	\[
	\begin{tikzcd}
	\closedball^2 \sqcup \closedball^2 \arrow[d,"q",swap] \arrow[dr,"\tilde{q}"] & \\
	\closedball ^2 \cup_f \closedball^2 \arrow[r,dashrightarrow] & \s^2
	\end{tikzcd}
	\]
    From the characteristic property of disjoint union space (Theorem 3.41), the map $\tilde{q} : \closedball^2 \sqcup \closedball^2 \to \s^2$ is continous $\Leftrightarrow$ its restriction to each $\closedball^2$ is continous. Define the restriction map $\tilde{q}_1 = \tilde{q}|_X : \closedball^2 \to \s^2$ and $\tilde{q}_2 = \tilde{q}|_Y : \closedball^2 \to \s^2$ as
    $$
    \tilde{q}_1(x_1,x_2) = (x_1,x_2,+\sqrt{1-x_1^2-x_2^2}), \quad \tilde{q}_2(x_1,x_2) = (x_1,x_2,-\sqrt{1-x_1^2-x_2^2})
    $$
    Its easy to see that the restriction of $\tilde{q}_1$ and $\tilde{q}_2$ to $\Inter \closedball^2$ is continous. Since $\closedball^2$ is first countable, we can check the continuity at the point $(x_1,x_2) \in \doo \closedball^2$ by limit sequence. That is, if $(x_1,x_2) \to (x_1',x_2') \in \doo \closedball^2$, then $\tilde{q}_i(x_1,x_2) \to \tilde{q}_i(x_1'x_2')= (x_1',x_2',0)$. Therefore both map are continous, and $\tilde{q} : \closedball^2 \sqcup \closedball^2 \to \s^2$ is a continous map. Moreover $\tilde{q}$ is surjective and make the same identification as $q$. The saturated open subsets $U = \tilde{q}^{-1}(\tilde{q}(U))$ of $\closedball^2 \sqcup \closedball^2$ are in the form of $U = V_1 \sqcup V_2$ where $V_1$ and $V_2$ are open subsets s.t. $V_1,V_2\subhim \Inter \closedball^2$. It can be checked directly that $\tilde{q}(U)$ is open in $\s^2$. Therefore $\tilde{q}$ is a quotient map, and $\closedball^2 \cup_f \closedball^2 \approx \s^2$ follows from uniqueness property of quotient space.
\end{prop}

\subsubsection*{Problems Chapter 3}
\begin{prop}[Problem 3-4 \cite{LeeTM}]
	Show that every closed ball in $\rn$ is an $n$-dimensional manifold with boundary, as is the complement of every open ball. Assuming the theorem on the invariance of the boundary, show that the manifold boundary of each is equal to its topological boundary as a subset of $\rn$, namely a sphere. [Hint : for the unit ball in $\rn$, consider the map $\pi \circ \sigma^{-1} : \rn \to \rn$, where $\sigma$ is the stereographic projection and $\pi$ is a projection from $\er^{n+1}$ to $\rn$ that omits some coordinate other than the last.]
\end{prop}
\begin{proof}
	It is enough to solve this for the closed unit ball $\bar{\mathbb{B}}^n$ in $\rn$, since any other closed ball $\bar{B}_r(p)$ in $\rn$ is homeomorphic to $\bar{\mathbb{B}}^n$ by composition of translation $T : \bar{B}_r(p) \to \bar{B}_r(0)$, defined as $x \mapsto x - p$ together with dilation $D : \bar{B}_r(0) \to \bar{\mathbb{B}}^n$, defined as $x \mapsto \frac{x}{r}$. As a subspace of $\rn$, $\closedball^n$ is a second countable Hausdorff space. For any point $p\in \openball^n$, the identity map on $\openball^n$ serve as the homeomorphism. So we only need to construct homeomorphisms between neighbourhood of points on $\doo \closedball^n=\s^{n-1}$ with open subsets in $\hn$.
	
	Consider the stereographic projection (Ex. 3.21) $\sigma : \s^n\smallsetminus \{S\} \to \rn$, which is a homeomorphism defined as $\sigma(x) = -\tilde{\sigma}(-x)$, where $\tilde{\sigma} : \s^n \smallsetminus \{N\} \to \er^n$ defined as
	$$
	\tilde{\sigma}(x_1,\dots,x_{n+1}) = \frac{(x_1,\dots,x_n)}{1-x_{n+1}}, \quad \tilde{\sigma}^{-1}(u_1,\dots,u_n)= \frac{(2u_1,\dots,2u_n,|u|^2-1)}{|u|^2+1}.
	$$
	For $i=1,\dots,n$, let $U_i^{+} =\{ (x_1,\dots,x_n) \in \rn : x_i > 0 \}$  and $U_i^{-} =\{ (x_1,\dots,x_n) \in \rn : x_i < 0 \}$ be some open subsets of $\rn$. Observe that $\sigma^{-1}(U_i^{\pm}) = (\s^n\smallsetminus\{S\}) \cap \widetilde{U}_i^{\pm}$, for each $i=1,\dots,n$, where $\widetilde{U}_i^+ = \{(x_1,\dots,x_{n+1}) \in \er^{n+1} : x_i>0 \}$ and $\widetilde{U}_i^- = \{(x_1,\dots,x_{n+1}) \in \er^{n+1} : x_i<0 \}$. That is $\sigma^{-1}$ map $U_i^+$ to the $x_i>0$ open hemisphere of $\s^n$. In particular, $\sigma^{-1}$ map $\doo \closedball^n = \s^{n-1} = \{(x_1,\dots,x_{n+1}) \in \er^{n+1} : x_{n+1} = 0 \text{ and }\sum_{i=1}^{n} x_i^2 = 1 \}$ to itself and region inside it, $\openball^n \cap U_i^+$ to the upper part ($x_{n+1}>0$) of the hemisphere $\sigma^{-1}(U_i^+)$ and the region outside it, $U_i^+ \smallsetminus \openball^n$, to the lower part ($x_{n+1}<0$) of hemisphere $\sigma^{-1}(U_i^+)$.     
	
	
	
	Since these hemispheres $\sigma^{-1}(U^{\pm}_i)$ homeomorphic to open unit ball $\openball^{n-1}$ via projection map $\pi_i : (x_1,\dots,x_i,\dots,x_{n+1}) \mapsto (x_1,\dots,\hat{x_i},\dots,x_{n+1})$, then by restricting the composition map $\pi_i \circ \sigma^{-1} : \rn \to \rn$ to $U_i^{\pm} \cap \closedball^n$, we obtain the desired homeomorphisms
	$$
	\varphi:=\pi_i \circ (\sigma^{-1})|_{U_i^{\pm}\cap \closedball^n} : U_i^{\pm}\cap \closedball^n \to \hn, 
	$$
	with domains cover $\doo \closedball^n = \s^{n-1}$. By construction, any $p\in \s^{n-1}$ must contained in one such neighbourhoods,  $\varphi(p)=0 \in \doo \hn$ and $\varphi(U_i^{\pm}\cap \closedball^n)$ is an open half unit ball in $\hn$. Therefore, $\closedball^n$ is an $n$-manifold with boundary with manifold boundary is equal to its topological boundary $\s^{n-1}$.
	
	With the same technique, the complement of a unit open ball $\rn \smallsetminus \openball^n$ is second countable Hausdorff space. Its interior $\rn \smallsetminus \closedball^n$ is locally euclidean (with identitiy map serves as the homeomorphism), and for each of its boundary points $p \in \s^{n-1}$, we can take the map 
	$$
	\psi:= \pi_i \circ (\tilde{\sigma}^{-1})|_{U_i^{\pm} \cap (\rn \smallsetminus \openball^n)} : U_i^{\pm} \cap (\rn \smallsetminus \openball^n) \to \hn,
	$$
	only this time, we use $\tilde{\sigma}$ instead of $\sigma$. As before, the complement of any other open ball in $\rn$ is homeomorphic to $\rn \smallsetminus \openball^n$ via translation and dilation.
\end{proof}

\subsection*{Chapter 4 Connectedness and Compactness}

\subsubsection*{Components and Path Components}

\begin{prop}[Exercise 4.22 \textbf{Properties of Path Components}]
	Let $X$ be any space.
	\begin{enumerate}[nolistsep]
		\item[(a)] The path components of $X$ form a partition of $X$.
		\item [(b)] Each path component is contained in a single component, and each component is a disjoint union of path components.
		\item [(c)] Any nonempty path-connected subset of $X$ is contained in a single path component.
	\end{enumerate}
\end{prop}
\begin{proof}
	Let $(P_{\alpha})_{\alpha \in A}$ be the path components of $X$. 
	
	For (a), we have to show that $P_{\alpha} \cap P_{\beta} = \emptyset$ for any $\alpha \neq \beta$ and $\bigcup_{\alpha} P_{\alpha} = X$. Suppose that $P_{\alpha} \cap P_{\beta}\neq \emptyset$ for $\alpha \neq \beta$ and $p \in P_{\alpha} \cap P_{\beta}$. Since both are path connected subset that has one point in common, then by Proposition 4.13(d), $P_{\alpha} \cup P_{\beta}$ is path connected. By maximality, $P_{\alpha} \cup P_{\beta} = P_{\alpha} = P_{\beta}$. So $\alpha = \beta$, contradiction. To show $\bigcup_{\alpha} P_{\alpha} = X$, let $p \in X$ arbitrary. There exists at least a path connected  subset contain $p$, that is $\{p\}$. Let $U$ be the union of all path connected subset contain $p$. Again, by Proposition 4.13(d), $U$ is path connected and maximal by construction. Therefore $U\in \{P_{\alpha}\}$. This shows that $X \subhim \bigcup_{\alpha} P_{\alpha}$, hence $X = \bigcup_{\alpha} P_{\alpha}$.   
	
	For first part of (b), we have to show that for any $P_{\alpha}$, there exists a unique connected component $C$ of $X$ such that $P_{\alpha} \subhim C$. Note that all path components $(P_{\alpha})$ are connected by Theorem 4.15. Since the connected components form a partition of $X$, then any $P_{\alpha}$ has a point in common with some component $C$. By Proposition 4.13(d) $P_{\alpha} \cup C$ is connected, hence by maximality $P_{\alpha} \cup C=C$. So $P_{\alpha} \subhim C$. The uniqueeness follow from the fact that all components are disjoint. For the second part, suppose that $\{P_{\alpha}:\alpha \in A_C\subseteq A \} \subseteq \{P_{\alpha}: \alpha \in A \}$ be the path components of $X$ that contain in a component $C$. So $\bigcup_{\alpha \in A_C} P_{\alpha} \subhim C$. We need to show that $\bigcup_{\alpha \in A_C} P_{\alpha} \supseteq C$. Take any element $p \in C$. Since path components cover $X$, there exists a path component $P_{\alpha_0}$ such that $p \in P_{\alpha_0}$. By the first part of (b), $P_{\alpha_0} \subhim C$. So $\alpha_0 \in A_C$. Hence $C \subhim \bigcup_{\alpha \in A_C} P_{\alpha}$.
	
	For (c), let $A$ be a path connected subset of $X$. By (a), there exist a path component $P_{\alpha}$ that has a point in common with $A$. By Proposition 4.13(d), $A \cup P_{\alpha}$ is path connected subset contain $P_{\alpha}$. By maximality, $A \cup P_{\alpha} = P_{\alpha}$. Therefore $A \subhim P_{\alpha}$.
\end{proof}

\begin{prop}[Exercise 4.24 \cite{LeeTM}]
	Every manifold (with or without boundary) is locally connected and locally path-connected.
\end{prop}
\begin{proof}
	By Problem 2-23, we know that every manifold admit a basis of coordinate balls. Since each coordinate ball is path-connected (hence connected), the conclusion follows.
\end{proof}

\subsubsection*{Local Compactness}
\begin{prop}[Proposition 4.63 \cite{LeeTM}]
Let $X$ be a Hausdorff space. The following are equivalent.
\begin{enumerate}[nolistsep]
	\item[(a)] $X$ is locally compact.
	\item[(b)]  Each point of $X$ has a precompact neighbourhood.
	\item[(c)]  $X$ has a basis of precompact open subsets.
\end{enumerate}	
\end{prop}

\begin{proof}
	It is  obvious that $(c) \Rightarrow (b)\Rightarrow (a)$. So we only need to show that $(c) \Rightarrow (a)$. First we will show that every point $p \in X$ has a neighbourhood basis consists of precompact open subsets $\mathcal{B}_p$. If we can show this, then the union of such collections $\mathcal{B} =\bigcup_{p \in X} \mathcal{B}_p$ is the desired basis for $X$. Let $p \in X$ be arbitrary and $V$ be any open subset contain $p$. By $(a)$, there exists a compact subset $K$ containing a neighbourhood $U$ of $p$. The subset $K$ is compact in Hausdorff space $X$, so $K$ closed in $X$. Since $V' = V \cap U \subset K$, then $\overline{V'} \subset K$ is closed in $K$ and therefore $\overline{V'}$ is compact. By this, the collection of open subsets $\mathcal{B}_p = \{ V' = V \cap U : \text{for all neighbourhood } V \text{ of }p \}$ is easily seen to be the desired neighbourhood basis of $p$. This completes the proof. 
\end{proof}

\begin{prop}[Proposition 4.64 \cite{LeeTM}]
	Every manifold with or without boundary is locally compact.
\end{prop}
\begin{proof}
	Proposition 4.60 showed that every manifold has a basis of regular coordinate balls. Every regular coordinate ball is precompact since its closure is homeomorphic to a closed ball $\overline{B}_r(x) \subhim \rn$. Therefore by Proposition 4.63(b), every manifold is locally compact.
\end{proof}

\begin{prop}[Lemma 4.65 \cite{LeeTM}]
	Let $X$ be a locally compact Hausdorff space. If $x \in  X$ and $U$ is any neighborhood of $x$, there exists a precompact neighborhood $V$ of $x$ such that $ \overline{V} \subhim U$.
\end{prop}
\begin{proof}
	To be add.
\end{proof}

\begin{prop}[Proposition 4.66 \cite{LeeTM}]
	Any open or closed subset of a locally compact Hausdorff space is a locally compact Hausdorff space.
\end{prop}
\begin{proof}
	Suppose $U$ is an open subset of $X$. Then $U$ is locally compact by Proposition 4.63(b) and Lemma 4.65. For the case $U$ is closed in $X$, we can proved it directly from definition (as in \cite{Munkres}) or by Proposition 4.63(b) as Lee did. I find Munkres proof's more natural. 
\end{proof}

The following proposition made (by myself) as a substitute for Proposition A.60 in Proposition 6.3 \cite{LeeSM}.

\begin{prop}[Proposition (Idea from \cite{Munkres})]
 Any noncompact, open or closed subspace $A$ of a locally compact, second countable, Hausdorff space $X$ is a countable union of compact subsets of $A$.
\end{prop}
\begin{proof}
	The proof is similar to the proof of Proposition 4.66. Suppose that $A$ is an open subspace of $X$ and let $x$ be any point in $A$. By Lemma 4.65, there exits a precompact neighbourhood $V_x$ in $X$ such that $\overline{V}_x \subhim A$. Since $A$ open, then $V_x$ is a neighbourhood of $x$ in $A$. Therefore we have $A = \bigcup_{x \in A} V_x$ and also $A= \bigcup_{x \in A} \overline{V}_x$. Since $A$ is second countable, then it is Lindelof. This means that $A = \bigcup_{i=1}^{\infty} V_{x_i}$. In particular $A =\bigcup_{i=1}^{\infty} \overline{V}_{x_i} $, that is $A$ is a countable union of compact subsets of $A$.
	
	Suppose that $A$ is closed in $X$ and let $x$ be any point in $A$. By local compactness of $X$, there exits a compact subset $K_x\subhim X$ contain a neighbourhood $U_x$ (in $X$) of $x$. Since $K_x$ is compact in $X$ then $K_x$ is closed in $X$. The subsets $U_x \cap A$ is a neighbourhood of $x$ in $A$ and $K_x\cap A$ is compact, since it is a closed subset of $K_x$. So we can write $A= \bigcup_{x \in A} (U_x \cap A)$. By second countability of $A$, we have $A = \bigcup_{i =1}^{\infty} (U_{x_i}\cap A)$. Because $U_x \cap A \subhim K_x \cap A$, therefore $A = \bigcup_{i =1}^{\infty} (K_{x_i} \cap A)$. This completes the proof.   
\end{proof}

\subsubsection*{Paracompactness}
\begin{prop}[Exercise 4.73 \cite{LeeTM}]
	Suppose $\mathcal{A}$ is an open cover for $X$ such that every element of $\mathcal{A}$ intersect only finitely many others. Show that $\mathcal{A}$ is locally finite. Give a counterexample to show that this need not be true when the elements are not open.
\end{prop}
\begin{proof}
	For any point $x \in X$, choose an element $A \in \mathcal{A}$ such that $x \in A$. Since $A$ is open and only intersect finitely many others, then this is the desired neighbourhood. Counterexample : Consider $\mathbb{R}$ with standard topology and choose the cover to be $\mathcal{A} = \{ \{x\} : \forall x \in \mathbb{R} \}$. The intersections between elements of $\mathcal{A}$ is empty, but any neighbourhood of any point intersect infinitely many of them. 
\end{proof}

\subsubsection*{Problems Chapter 4}

\begin{prop}[Problem 4-1 \cite{LeeTM}]
	Show that for $n>1$, $\rn$ is not homeomorphic to any open subset of $\er$.
\end{prop}
\begin{proof}
	We knew that $\rn$ is path-connected, hence connected. So $\rn$ cannot be homeomorphic to any disconnected open subset of $\er$. So consider connected open subset $U \subhim \er$ (open interval). Suppose that we have a homeomorphism $\varphi : \rn \to  U$ and let $\varphi(0) = x \in U$. The subsets $\rn \smallsetminus \{0\} \subhim \rn$ and $U \smallsetminus \{x\} \subhim U$ are open. So the map $\varphi|_{\rn \smallsetminus \{0\}} : \rn \smallsetminus \{0\} \to U \smallsetminus \{x\}$ is still a homeomorphism. But, $\rn \smallsetminus \{0\}$ is path-connected (hence connected) whereas $U \smallsetminus \{x\}$ is disconnected, which is means that they cannot be homeomorphic. Contradiction.
\end{proof}

\begin{prop}[Problem 4-2 \cite{LeeTM}]
	 \textbf{Invariance of Dimension, 1-Dimensional Case}: Prove that a non-empty topological space cannot be both a 1-manifold and an $n$-manifold for some $n>0$.
\end{prop}
\begin{proof}
	If a topological space $X$ is a 1-manifold and $n$-manifold, then any point $p\in X$ has neighbourhoods $U \subhim X$ homeomorphic to $\rn$ and $V\subhim X$ homeomorphic to open subset in $\er$. Then $U \cap V$ is homeomorphic to open subset in $\rn$ and open subset in $\er$. This implies that we have open subset in $\rn$ homeomorphic to open subset in $\er$, but this is impossible by Problem 4-1. 
 \end{proof}

\begin{prop}[Problem 4-3 \cite{LeeTM}]
	 \textbf{Invariance of The Boundary, 1-Dimensional Case}:
	Suppose $M$ is a 1-dimensional manifold with boundary. Show that a point cannot be both a boundary point and an interior point.
\end{prop}
\begin{proof}
	Let $p \in M$ that is both a boundary point and an interior point. By definition, we have homeomorphisms $\varphi : U \to [0,a)$, for some $a>0$ and $\varphi(p)=0$, and $\psi : V \to (b,c)$, for some $b<c$. The intersection $U \cap V \subhim$ is an open subset contain $p$. So $\varphi(U \cap V)= [0,a')$ for some $0<a'<a$, and $\psi(U \cap V) = (b',c')$, for $b<b'<\psi(p)<c'<c$. We have a homeomorphism 
	$$
	f:=\psi|_{U \cap V} \circ (\varphi^{-1})|_{[0,a')} : [0,a') \to (b',c')
	$$
	such that $b'<f(0)<c'$. Then the restriction $f|_{(0,a')} : (0,a') \to (b',c')\smallsetminus \{f(0)\}$ is still a homeomorphism between connected and disconnected space which is impossible.   
\end{proof}

\begin{prop}[Problem 4-8 \cite{LeeTM}]
	Show that a locally connected topological space is homeomorphic to the disjoint union of its components.
\end{prop}
\begin{proof}
	Let $X$ is a locally connected topological space and $\{C_{\alpha}: \alpha \in A \}$ be its components. Denote $i_{\alpha} : C_{\alpha} \to X$ as the inclusion map of the components. By regard $C_{\alpha}$ as subspace of $X$, we have their disjoint union space $\bigsqcup_{\alpha \in A} C_{\alpha}$. Denote their canonical injection as $\iota_{\alpha} : C_{\alpha} \to \bigsqcup_{\alpha \in A} C_{\alpha}$. Define a map
	$$
	f: \bigsqcup_{\alpha \in A} C_{\alpha} \to X
	$$
	defined as $f(p) = i_{\alpha} \circ \iota_{\alpha}^{-1} (p)$ whenever $p \in C^*_{\alpha}$. This map is well defined and bijective. It is also continous by the characteristic property
	\[
	\begin{tikzcd}
	\bigsqcup_{\alpha \in A} C_{\alpha} \arrow[r,"f"] &  X\\
	C_{\alpha} \arrow[u,"\iota_{\alpha}"] \arrow[ur,"f \circ \iota_{\alpha} = i_{\alpha}",swap]& 
	\end{tikzcd}
	\]
	For any open subset $U \subhim \bigsqcup_{\alpha \in A} C_{\alpha}$, the image is $f(U) = \bigcup_{\alpha \in A} i_{\alpha}(U\cap C_{\alpha}) = U \cap \big(\bigcup_{\alpha \in A} C_{\alpha}\big) = U \cap X=U$. By definition $U$ is open in $\bigsqcup_{\alpha \in A} C_{\alpha}$ if and only if $U\cap C_{\alpha}$ is open in $C_{\alpha}$ for each $\alpha \in A$. By hypothesis, the components $C_{\alpha}$ are all open in $X$. So $U \cap C_{\alpha}$ is open in $X$. So $U= \bigcup_{\alpha \in A} (U \cap C_{\alpha})$ is open in $X$. Therefore $f$ is a homeomorphism and we conclude $X \approx \bigsqcup_{\alpha \in A} C_{\alpha}$.
\end{proof}

\begin{prop}[Problem 4-9 \cite{LeeTM}]
	Show that every $n$-manifold is homeomorphic to a disjoint union of countably many connected $n$-manifolds, and every $n$-manifold with boundary is homeomorphic to a disjoint union of countably many connected $n$-manifolds with (possibly empty) boundaries.
\end{prop}
\begin{proof}
	Suppose that $M$ is an $n$-manifold with or without boundary. By Proposition 4.23 and Proposition 4.25, every components of $M$ is open. So the components itself is a $n$-manifold (with or without boundary) by Proposition 2.53. If we can show that the components of $M$ are countable, then by Problem 4-8, we can conclude that $M$ is homeomorphic to disjoint union of countably many connected $n$-manifolds (its components) with (possibly empty) boundaries.
	To show that the components $\{C_{\alpha}\}_{\alpha \in A}$ of $M$ are countable, let $\mathcal{B}=\{B_i\}_{i=1}^{\infty}$ be the countable basis for $M$. Define an equivalence relation on $\mathcal{B}$ as  : $\forall B_i,B_j \in \mathcal{B}$, $B_i {\sim} B_j \Leftrightarrow $ $B_i$ and $B_j$ are contained in the same component. Denote the set of the equivalence classes as $K = \{[B_i]: \forall B_i \in \mathcal{B}\}$ and define a map
	$$
	f : \{C_{\alpha}\}_{\alpha \in A} \to K
	$$
	as $f(C_{\alpha}) = [B_i]$ for any $B_i \in \mathcal{B}$ such that $B_i \subhim C_{\alpha}$. This map is well defined since any components (open subset in $M$) must contain some $B_i$, by definition of basis. And for $B_i,B_j \subhim C_{\alpha}$, then $f(C_{\alpha}) = [B_i]=[B_j]$. This map is injective : for any $\alpha, \beta$ such that $f(C_{\alpha}) = f(C_{\beta}) = [B_i]$, then $B_i \subhim C_{\alpha} \cap C_{\beta} \Rightarrow C_{\alpha} = C_{\beta}$. Also $f$ is surjective by definition. Therefore we have a bijective map between the set of components and countable set $K$. So $\{C_{\alpha}\}_{\alpha \in A}$ is countable. 
	
\end{proof}

\begin{prop}[Problem 4-11 \cite{LeeTM}]
	Let $X$ be a topological space, and let $CX$ be the cone on $X$. 
	\begin{enumerate}[nolistsep]
		\item[(a)] Show that $CX$ is path-connected.
		\item [(b)] Show that $CX$ is locally connected if and only if $X$ is, and locally path-connected if and only if $X$ is. 
	\end{enumerate}
 \end{prop}
\begin{proof}
	To show (a), it is enough to show that there is a point $p \in CX$ such that any other point in $CX$ can be joined by a path from $p$ to that point. Let $p$ be the vertex of $CX = X \times I / (X \times \{1\})$. I.e., $p$ is the image of the subspace $X \times \{1\}$ under quotient map $q : X \times I \to CX$. Let $p' \in CX$ be arbitrary. If $p'=p$, then we just define the path to be the constant path at $p$. So let $p' \neq p$. Define a path $\alpha : I \to X \times I$ from $(x,s) = q^{-1}(\{p'\})$ to $(x,1)$ as
	$$
	\alpha(t) = (x, s + t(1-s)).
	$$
	Then composition $q \circ \alpha : I \to CX $ is a path joining $p'$ and $p$. 
	
	For (b), first suppose that $CX$ is locally connected and $\mathcal{B}$ is the connected basis for it. By definition, there exists a subcollection $\mathcal{B'} \subhim \mathcal{B}$ such that $q(X \times [0,1) = \bigcup \mathcal{B}'$. Let $\mathcal{U} = \{ q^{-1}(B) \mid \forall B \in \mathcal{B}' \}$ and $\pi : X \times I \to X$ be the projection map $(x,t) \mapsto x$. Since $X \times [0,1)$ is homeomorphic to its image under $q$ and $\pi$ is an open map, then $\mathcal{V} = \{ \pi(U)\mid \forall U \in \mathcal{U} \}$ is a collection of connected open subset of $X$ whose union is $X$. By Problem 2-11, $\mathcal{U}$ is a connected basis for $X \times [0,1)$ and hence $\mathcal{V}$ is a connected basis for $X$. Therefore $X$ is locally connected.
	
	Now suppose that $X$ is locally connected and let $\mathcal{B} = \{B_{\alpha} \mid \alpha \in A\}$ be its connected basis. The subset $X \times [0,1) \subhim X \times I$ is locally connected saturated open subset. So $X \times [0,1)$ is homeomorphic to its image under quotient map $q : X \times I \to CX$. So we have connected basis for $q(X \times [0,1)) \subhim CX$. Let $p$ be the vertex of $CX$. We need to find a connected neighbourhood basis for $p$. Define a collection of open subset of $X \times I$ containing $X \times \{1\}$ as 
	$$
	\mathcal{U} = \{ \bigcup_{\alpha \in A} B_{\alpha} \times (t_{\alpha},1] : \text{where }  0 \leq t_{\alpha} < 1, \text{ for each }\alpha \in A \}.
	$$
	So by definition of $\mathcal{U}$, for any open subset  $V \subhim X \times I$ contain $X \times \{1\}$ there exists $U \in \mathcal{U}$ such that $U \subhim V$. With this, the collection $\mathcal{V} = \{ q(U) \mid \forall U \in \mathcal{U} \}$ is a connected neighbourhood basis for the vertex $p$. The union of connected basis of $q(X \times [0,1))$ with $\mathcal{V}$ is a connected basis for $CX$. Therefore $CX$ is  locally connected. The path-connected case is similar as the connected case.
 \end{proof}

\begin{prop}
[\text{Problem 4.12 }\cite{LeeTM}] Suppose $S\subseteq X$ is an open and closed subset of a topological space $X$. Then $S$ is a union of components of $X$.
\end{prop}
\begin{proof}
By hypothesis $S$ is open and closed. So $X = S \cup S^c$. For any $x \in S$, let $C_x$ be the components that contain $x$. This set is not empty because every singeton $\{x\}$ is connected. Its clear that $S \subset \bigcup_{x \in S} C_x$. To show $S \supset \bigcup_{x \in S} C_x$, let $x$ be any element of $S$. Because $C_x \subset X=S \cup S^c$ is connected, then $C_x \subset S$. Otherwise (i.e $C_x$ contain both elements of $S$ and $S^c$) the subset $C_x \cap S$ and $C_x\cap S^c$ will disconnect $C_x$ (this is a special case of Proposition 4.9(a) \cite{LeeTM}). Because $x \in S$ is arbitrary, $S = \bigcup_{x \in S} C_x$.  
\end{proof}
\begin{remark}
This result used in Lemma 7.2 \cite{LeeSM}.
\end{remark}

\begin{prop}[Problem 4-17 \cite{LeeTM}]
	Suppose $M$ is a manifold of dimension $n \geq 1$, and $B \subhim M$ is a regular coordinate ball. Show that $M \smallsetminus B$ is an $n$-manifold with boundary, whose
	boundary is homeomorphic to $\s^{n-1}$. (You may use the theorem on invariance of the boundary.)
\end{prop}
\begin{proof}
	We will show that $M \smallsetminus B = (M\smallsetminus \bar{B}) \cup \doo B$ is a $n$-manifold with boundary, with manifold boundary $\doo B$. As a subspace of $M$, it is second countable and Hausdorff. As a manifold, any point $p\in M$ has a local chart $(U,\varphi)$ with $p\in U \subhim M$. Since $M\smallsetminus \bar{B}$ is open in $M$, then any point $p \in M \smallsetminus \bar{B}$ has a neighbourhood $U \cap (M\smallsetminus \bar{B})$ homeomorphic to an open subset $\varphi(U \cap (M\smallsetminus \bar{B}))$. Now consider points in $\doo (M\smallsetminus B) = \doo \bar{B}$. By hypothesis, $B$ is a regular coordinate ball. So there exists a coordinate ball $(B',\varphi)$ such that $B\subhim B'$, that takes $\varphi(B) = B_r(0)$, $\varphi(\bar{B}) = \bar{B}_r(0)$ and $\varphi(B') = B_{r'}(0)$ for some $r'>r>0$. So $B' \smallsetminus B$ is an open subset in $M\smallsetminus B$ homeomorphic to $B_{r'}(0) \smallsetminus B_r(0)$, which is an $n$-manifold with boundary, with $\doo \bar{B}_r(0)$  as the manifold boundary, by result of Problem 3-4. Let $p \in \doo \bar{B} \subhim B' \smallsetminus B$ be arbitrary and let $(V,\psi)$ be any boundary chart for  $\varphi(p) \in \doo \bar{B}_r(0) \subhim B_{r'}(0) \smallsetminus B_r(0)$. Then $\varphi^{-1}(V)$ is an open subset in $M\smallsetminus B$ contain $p$ homeomorphic to $\psi(V) \subhim \hn$, by composition map $\psi \circ \varphi|_{\varphi^{-1}(V)} : \varphi^{-1}(V) \to \psi(V)$, taking $p$ to $\psi(\varphi(p)) \in \doo \hn$. 
\end{proof}

\begin{prop}[Problem 4-18 \cite{LeeTM}]
	Let $M_1$ and $M_2$ be $n$-manifolds. For $i =1,2$, let $B_i \subhim M_i$ be regular coordinate balls, and let $M_i' = M_i \smallsetminus B_i$. Choose a homeomorphism $f : \doo M_2' \to \doo M_1'$ (such a homeomorphism exists by Problem 4-17). Let $M_1 \# M_2$ (called
	a connected sum of $M_1$ and $M_2$) be the adjunction space $M_1' \cup_f M_2'$.
	\begin{enumerate}[nolistsep]
		\item [(a)] Show that $M_1 \# M_2$ is an $n$-manifold (without boundary).
		\item [(b)] Show that if $M_1$ and $M_2$ are connected and $n > 1$, then $M_1 \# M_2$ is connected.
		\item [(c)] Show that if $M_1$ and $M_2$ are compact, then $M_1 \# M_2$ is compact.
	\end{enumerate}
\end{prop}
\begin{proof}
	Part (a) follow from Theorem 3.79. For (b), suppose that $M_1$ and $M_2$ are connected and $n>1$. For each $M_i'=M_i\smallsetminus B_i$, $i=1,2$, denote their embedding as $e_i : M_i' \to M_1 \# M_2 $. If we can show that $M_i \smallsetminus B_i$ are connected, then since $e_1(M_1') \cap e_2(M_2') =e_1(\doo M_1') = e_2(\doo M_2') \neq \emptyset$ and $e_1(M_1') \cup e_2(M_2')  = M_1\# M_2$, then $M_1 \# M_2$ connected. To show that $M_i' = M_i \smallsetminus B_i$ are connected, we will show that they are path-connected. Let $p,q \in M_i'$ be arbitrary and let $\alpha : [0,1] \to M$ be a path connecting them (guaranteed by hypothesis). If this path does not intersect $B_i$, then we are done. That is we obtain a path in $M_i'$  connecting $p$ and $q$. If it does intersect $B_i$, do the following: let $t_1$ and $t_2$ be the minimum and the maximum of the closed subset $\alpha^{-1}(\bar{B}_i) \subhim [0,1]$ respectively. That is $\alpha(t_1)$ is the point where the path  hit $\bar{B}_i$ for the first time and $\alpha(t_2)$ is the point where the path is on $\bar{B}_i$ for the last time. Both of this points must lie on the boundary $\doo \bar{B}_i$. Since $B_i$ is a regular coordinate ball, $\doo \bar{B}_i$ is homeomorphic to $\mathbb{S}^{n-1}$ (this is where the dimensional restriction comes in), and hence $\doo \bar{B}_i$ is connected. Choose a path (after rescaling the parameter) $\beta : [t_1,t_2] \to \doo \bar{B}_i$ connecting $\alpha(t_1)$ and $\alpha(t_2)$, define a new path $\alpha' : [0,1] \to M_i\smallsetminus B_i$ as
	\begin{equation*}
	\alpha'(t) =  \left\{
	\begin{array}{rl}
	\alpha(t) & \text{for } 0 \leq t \leq t_1,\\
	\beta(t) & \text{for } t_1 \leq t \leq t_2,\\
	\alpha(t) & \text{for } t_2 \leq t \leq 1.
	\end{array} \right.
	\end{equation*}
	Therefore we obtained a path in $M_i'$ joining $p$ and $q$. This proves that $M_i'$ is connected.
	
	For (c), suppose $M_i$ is compact for each $i=1,2$. Since $M_I'=M_i\smallsetminus B_i$ is a closed subspace of a compact space, then $M_i'$ is compact as well as their image $e_i(M_i')$. Therefore $M_1 \# M_2 = e_1(M_1') \cup e_2(M_2')$ is compact.
\end{proof}

\begin{prop}[Problem 4-19 \cite{LeeTM}]
	Let $M_1 \# M_2$ be a connected sum of $n$-manifolds $M_1$ and $M_2$. Show that there are open subsets $U_1,U_2  \subhim M_1 \# M_2$ and points $p_i  \in M_i$ such that $U_i \approx M_i \smallsetminus \{p_i\}$, $U_1 \cap U_2 \approx \rn \smallsetminus \{0\}$, and $U_1\cup U_2 =  M_1 \# M_2$.
\end{prop}
\begin{proof}
	Let $M_i' = M_i \smallsetminus B_i$, with $B_i \subhim M_i$ be the regular coordinate balls. Denote the homeomorphisms as $\varphi_i : B_i' \to \varphi_i(B_i') = B_{r_i}(0)$ with $B_i' \supseteq B_i$ such that $\varphi_i(B_i) = \openball^n$ and $\varphi_i(\bar{B}_i) = \closedball^n$ for $r_2 >1>0$. The connected sum $M_1 \# M_2$ is an adjunction space $M_1' \cup_f M_2'$ under some homeomorphism $f : \doo B_2 \to \doo B_1$. By Theorem 3.79 we have embeddings $e_i : M_i' \to M_1 \#M_2$, which is just the restriction of the quotient map $q : M_1' \sqcup M_2' \to M_1' \cup_f M_2'$ to each $M_i'$.  So
	$$
	e_i = q|_{M_i'} : M_i' \to M_1\# M_2, \quad i=1,2.
	$$
	We claim that 
	$$
	U_1 = e_1(M_1') \cup e_2(B_2' \smallsetminus B_2), \quad U_2 = e_2(M_2') \cup e_1(B_1' \smallsetminus B_1)
	$$
	are the desired open subsets. 
	
	$\textbf{Showing that }\boldmath{U_1} \textbf{ and }\boldmath{U_2} \textbf{ are open subsets : }$ To see that they are open subsets, denote $\iota_i : M_1' \to M_1' \sqcup M_2'$ as the canonical injection. Note that 
	\begin{align*}
	U_1 &= e_1(M_1') \cup e_2(B_2' \smallsetminus B_2) \\
	&= q(\iota_1(M_1') \cup q(\iota_2(B_2'\smallsetminus B_2)) \\
	&= q\big(\iota_1(M_1') \cup \iota_2(B_2'\smallsetminus B_2) \big) \\
	&= q(M_1' \sqcup (B_2'\smallsetminus B_2)).
	\end{align*}
	But since $M_1' \sqcup (B_2'\smallsetminus B_2) \subhim M_1' \sqcup M_2'$ is saturated, then $M_1' \sqcup (B_2'\smallsetminus B_2) = q^{-1}q(M_1' \sqcup (B_2'\smallsetminus B_2)) = q^{-1}(U_1)$. By definition of disjoint union topology, $M_1' \sqcup (B_2'\smallsetminus B_2)=q^{-1}(U_1)$ is open in $M_1' \sqcup M_2'$. This implies that $U_1$ is open. By same arguments, $U_2$ is also open.  
	
	$\textbf{Showing } \mathbf{U_i \approx M_i \smallsetminus \{p_i\}}$ :  Next, we want to show that $U_i \approx M_i \smallsetminus \{p_i\}$ where $p_i$ is the center of $B_i$. The idea is to construct a quotient map $\bar{q} : M_1' \sqcup (B_2' \smallsetminus B_2) \to M_1 \smallsetminus \{p_1\}$ such that it makes same identification as the quotient map $ \widetilde{q}:  M_1' \sqcup (B_2' \smallsetminus B_2) \to  M_1' \cup_f (B_2' \smallsetminus B_2)$. 
	\[
	\begin{tikzcd}
	& M_1' \sqcup (B_2' \smallsetminus B_2) \arrow[d,"\widetilde{q}",swap] \arrow[dr,"\bar{q}"] \arrow[dl,"q|_{M_1' \sqcup (B_2' \smallsetminus B_2)}",swap] & \\
	q(M_1' \sqcup (B_2' \smallsetminus B_2))&M_1' \cup_f (B_2' \smallsetminus B_2) \arrow[r,dashrightarrow] \arrow[l,dashrightarrow] & M_1 \smallsetminus \{p_1\}
	\end{tikzcd}
	\]
	If we managed to show this, then by characteristic property of quotient space, $M_1' \cup_f (B_2' \smallsetminus B_2) \approx  M_1 \smallsetminus \{p_1\}$. By noting that the restriction of a quotient map $q : M_1' \sqcup M_2' \to M_1' \cup_f M_2'$ to a saturated open subset $M_1' \sqcup (B_2' \smallsetminus B_2)$ is also a quotient map onto its image $q(M_1' \sqcup (B_2' \smallsetminus B_2)) = U_1$, then $U_1 \approx M_1' \cup_f (B_2' \smallsetminus B_2) \approx M_1 \smallsetminus \{p_1\}$.
	
	We want a continous surjective map $\bar{q} : M_1' \sqcup (B_2' \smallsetminus B_2) \to M_1 \smallsetminus \{p_1\}$ that makes the same identification as $\widetilde{q}: M_1' \sqcup (B_2'\smallsetminus B_2) \to M_1 \cup_f (B_2'\smallsetminus B_2)$. I.e., For any $x,x' \in M_1' \sqcup (B_2'\smallsetminus B_2)$ such that $\widetilde{q}(x) = \widetilde{q}(x')$, then $\bar{q}(x) = \bar{q}(x')$. More clearly
	\begin{enumerate}[nolistsep]
		\item [$\bullet$] If $x\in \doo B_2$, then $\bar{q}^{-1}(\bar{q}(x)) = \{x, f(x) \}$ (Since $f$ is bijective, then if $x \in \doo B_1$, $\bar{q}^{-1}(\bar{q}(x)) = \{x, f^{-1}(x)\}$). 
		\item [$\bullet$] If $x \notin \doo B_1 \sqcup \doo B_2$, then $\bar{q}^{-1}(\bar{q}(x)) = \{x\}$.
	\end{enumerate}
    Before going further, note that
    \begin{enumerate}
    	\item [$\diamond$] $B_2' \smallsetminus B_2 \approx B_{r_2}(0) \smallsetminus \bar{\mathbb{B}}^n$ through the restriction of the given homeomorphism $\varphi_2 : B_2' \to \varphi_2(B_2') = B_{r_2}(0)$.
    	\item [$\diamond$] $B_{r_2}(0) \smallsetminus \openball^n \approx \closedball^n \smallsetminus \bar{B}_{r}(0)$ for some $0<r<r_2$, say $r=1/r_2$ by $g : x \mapsto \hat{x}/|x| = x/|x|^2$.
    	\item [$\diamond$] $\closedball^n \smallsetminus \bar{B}_{r}(0) \approx \closedball^n \smallsetminus \{0\}$ by $h : x \mapsto (|x|-r)/(1-r) \hat{x}$.
    	\item [$\diamond$] $\bar{B}_1 \smallsetminus \{p_1\} \approx \closedball^n \smallsetminus \{0\}$ by the restriction of $\varphi_1 : B_1' \to \varphi_1(B_1')$ to $\bar{B}_1 \smallsetminus \{p_1\}$.
    \end{enumerate}
    So we have homeomorphisms $\psi = h \circ g \circ \varphi_2: B_2' \smallsetminus B_2 \to \closedball^n \smallsetminus \{0\}$ and $\varphi := \varphi_1|_{\bar{B}_1 \smallsetminus \{p_1\}} : \bar{B}_1 \smallsetminus \{p_1\} \to \closedball^n \smallsetminus \{0\}$. Define $\bar{q}: M_1' \sqcup (B_2' \smallsetminus B_2) \to M_1 \smallsetminus \{p_1\} $ so that 
    \begin{align*}
    \bar{q}|_{M_1'}&= \iota_{M_1'} : M_1' \to M_1 \smallsetminus \{p_1\},\\
    \bar{q}|_{B_2' \smallsetminus B_2} &= \iota_{\bar{B}_1 \smallsetminus \{p_1\}} \circ F : B_2' \smallsetminus B_2 \to M_1\smallsetminus \{p_1\},
    \end{align*}
    for some homeomorphism 
    $$
    F : B_2' \smallsetminus B_2 \to \bar{B}_1 \smallsetminus \{p_1\},\quad \text{s.t. }F|_{\doo B_2} = f.
    $$
    In the other words, $F$ is the extension of $f: \doo B_2 \to \doo B_1$ to $B_1' \small B_2$ and $\bar{B}_1 \smallsetminus \{p_1\}$. To do this, we extend the following homeomorphism on $\s^{n-1}$
    \[
    \begin{tikzcd}
    \doo B_2 \arrow[r,"f"] & \doo B_1 \arrow[d,"\varphi|_{\doo B_1}"] \\
    \s^{n-1} \arrow[u,"\psi^{-1}|_{\s^{n-1}}"] \arrow[r,"\hat{f}"] & \s^{n-1}
    \end{tikzcd}
    \]
    where $\hat{f} = \varphi|_{\doo B_1} \circ f \circ (\psi^{-1})|_{\s^{n-1}}$, to homeomorphism on $\closedball^n\smallsetminus \{0\}$. It is clear that the extension is $\widetilde{f} : \closedball^n\smallsetminus \{0\} \to \closedball^n\smallsetminus \{0\}$ defined as
    $$
    \widetilde{f}(x) = |x| \hat{f}\bigg(\frac{x}{|x|}\bigg).
    $$
    So we define $F : B_2' \smallsetminus B_2 \to \bar{B}_1 \smallsetminus \{p_1\}$ as $F = \varphi^{-1} \circ \widetilde{f} \circ \psi$.
    \[
    \begin{tikzcd}
    \closedball^n\smallsetminus \{0\}  \arrow[r,"\widetilde{f}"] & \closedball^n\smallsetminus \{0\} \arrow[d,"\varphi^{-1}"] \\
    B_1'\smallsetminus B_2 \arrow[u,"\psi"] \arrow[r,"F"] & \bar{B}_2 \smallsetminus \{p_1\}
    \end{tikzcd}
    \]
    The restriction to $\doo B_2$ is certainly agree with $f$ as follows
    \begin{align*}
    F|_{\doo B_2} &= ( \varphi^{-1} \circ \widetilde{f} \circ \psi)|_{\doo B_2} \\
    &= \varphi^{-1} \circ \hat{f} \circ \psi|_{\doo B_2} \\
    &= \varphi^{-1} \circ (\varphi \circ f \circ \psi^{-1}) \circ \psi \\
    &= f.
    \end{align*}
    So we obtained a continous surjective map $\bar{q}: M_1' \sqcup (B_2' \smallsetminus B_2) \to M_1 \smallsetminus \{p_1\} $ defined as
    \begin{align*}
    \bar{q}|_{M_1'}&= \iota_{M_1'} : M_1' \to M_1 \smallsetminus \{p_1\},\\
    \bar{q}|_{B_2' \smallsetminus B_2} &= \iota_{\bar{B}_1 \smallsetminus \{p_1\}} \circ F : B_2' \smallsetminus B_2 \to M_1\smallsetminus \{p_1\}.
    \end{align*}
    It can check directly that it is a closed map. Hence $\bar{q}$ is a quotient map. So we can conclude that  $U_1 \approx M_1 \smallsetminus \{p_1\}$. By same arguments $U_2 \approx M_2 \smallsetminus \{p_2\}$.
    
    $\textbf{Showing that }\mathbf{U_1 \cap U_2 \approx \rn \smallsetminus \{0\}}$ : To start with, note that $U_1 \cap U_2$ is open and 
    \begin{align*}
     U_1 \cap U_2 &= \big( e_1(M_1') \cup e_2(B_2'\smallsetminus B_2) \big)\cap \big( e_2(M_2') \cup e_1(B_1' \smallsetminus B_1) \big) \\
     &=  \big(q \circ \iota_1(M_1') \cup q\circ \iota_2(B_2'\smallsetminus B_2) \big) \cap \big( q\circ \iota_2(M_2') \cup q\circ \iota_1(B_1'\smallsetminus B_1) \big)  \\
     &= q\circ \iota_1(B_1' \smallsetminus B_1) \cup q\circ \iota_2(B_2' \smallsetminus B_2) \\
     &= q\big((B_1'\smallsetminus B_1) \sqcup (B_2'\smallsetminus B_2) \big).
    \end{align*}
    So $U_1 \cap U_2$ is a saturated open subset of $M_1 \# M_2$. Therefore we have a quotient map
    $$
    q|_{(B_1'\smallsetminus B_1) \sqcup (B_2'\smallsetminus B_2)} : (B_1'\smallsetminus B_1) \sqcup (B_2'\smallsetminus B_2) \to  q\big((B_1'\smallsetminus B_1) \sqcup (B_2'\smallsetminus B_2) \big).
    $$
    For each $i=1,2$, we know that we have homeomorphisms $\varphi_i : B_i' \smallsetminus B_i \approx B_{r_i}(0) \smallsetminus \openball^n$. Denote $A_i:=B_i'\smallsetminus B_i$ and $C_i:= B_{r_i}(0) \smallsetminus \openball^n$,  we have a homeomorphism
    $$
    \varphi : A_1 \sqcup A_2 \to C_1 \sqcup C_2
    $$ 
    defined as
    \begin{equation*}
    \varphi(x) =  \left\{
    \begin{array}{rl}
    \iota_{B_1} \circ \varphi_1 \circ \iota_{A_1}^{-1} (x) & \text{for } x \in \iota_{A_1}(A_1),\\
    \iota_{B_2} \circ \varphi_2 \circ \iota_{A_2}^{-1} (x) & \text{for } x \in \iota_{A_2}(A_2).
    \end{array} \right.
    \end{equation*}
    So $(B_1'\smallsetminus B_1) \sqcup (B_2'\smallsetminus B_2) \approx (B_{r_1}(0) \smallsetminus \openball^n) \sqcup (B_{r_2}(0) \smallsetminus \openball^n)$. Also, since $B_{r_2} \smallsetminus \openball^n \approx \closedball^n \smallsetminus \bar{B}_{r}(0) $, with $r = 1/r_1$, then $(B_{r_1}(0) \smallsetminus \openball^n) \sqcup (B_{r_2}(0) \smallsetminus \openball^n) \approx (B_{r_1}(0) \smallsetminus \openball^n) \sqcup (\closedball^n \smallsetminus \bar{B}_{r}(0))$. In total we have homeomorphism 
    $$
    \mu : (B_1'\smallsetminus B_1) \sqcup (B_2'\smallsetminus B_2) \to (B_{r_1}(0) \smallsetminus \openball^n) \sqcup (\closedball^n \smallsetminus \bar{B}_{r}(0)).
    $$
    The composition map 
    $$
    q_0:=q \circ \mu^{-1} :  (B_{r_1}(0) \smallsetminus \openball^n) \sqcup (\closedball^n \smallsetminus \bar{B}_{r}(0)) \to U_1 \cap U_2
    $$
    is a quotient map, with the topology same as the quotient topology induced by $q$.
    \[
    \begin{tikzcd}
    (B_1'\smallsetminus B_1) \sqcup (B_2'\smallsetminus B_2) \arrow[r,"\mu"] \arrow[dr,"q",swap] &  (B_{r_1}(0) \smallsetminus \openball^n) \sqcup (\closedball^n \smallsetminus \bar{B}_{r}(0)) \arrow[d,,"q_0"] \arrow[dr,"Q"]& \\
    & U_1 \cap U_2 \arrow[r,dashrightarrow] & B_{r_1}(0) \smallsetminus \bar{B}_r(0). 
    \end{tikzcd}
    \]
    As before, our strategy is to construct a quotient map $$
    Q :  (B_{r_1}(0) \smallsetminus \openball^n) \sqcup (\closedball^n \smallsetminus \bar{B}_{r}(0)) \to  B_{r_1}(0) \smallsetminus \bar{B}_r(0)
    $$
    that makes the same identification as $q_0$. Denote the embeddings as
    $$
    \iota_I : B_{r_1}(0) \smallsetminus \openball^n \to B_{r_1}(0) \smallsetminus \bar{B}_r(0),\quad \iota_{II} : \closedball^n \smallsetminus \bar{B}_{r}(0) \to B_{r_1}(0) \smallsetminus \bar{B}_r(0).
    $$
    As before we define $Q$ such that 
    $$
    Q_1 = Q|_{ B_{r_1}(0) \smallsetminus \openball^n} = \iota_I, \quad Q_2=Q|_{\closedball^n \smallsetminus \bar{B}_{r}(0)} = \iota_{II} \circ \widetilde{f}
    $$
    for some homeomorphism $\widetilde{f} : \closedball^n \smallsetminus \bar{B}_{r}(0) \to \closedball^n \smallsetminus \bar{B}_{r}(0)$ such that $\widetilde{f}|_{\doo (\closedball^n \smallsetminus \bar{B}_{r})}  = \hat{f} := \mu_1 \circ f \circ \mu_2^{-1} : \s^{n-1} \to \s^{n-1}$. As before the extension is clear, that is 
    $$
    \widetilde{f} (x) = |x| \hat{f} \Big( \frac{x}{|x|} \Big). 
    $$
    It can be check that $Q$ is an open map. Hence $Q$ is a quotient map. Therefore we have $U_1 \cap U_1 \approx B_{r_1}(0) \smallsetminus \bar{B}_r(0) \approx \rn \smallsetminus \{0\}$. 
    
    $\textbf{Showing that }\mathbf{U_1 \cup U_2 = M_1 \# M_2 }$ : This part is easy
    \begin{align*}
    U_1 \cup U_2 &= \big( e_1(M_1') \cup e_2(B_2'\smallsetminus B_2) \big)\cup \big( e_2(M_2') \cup e_1(B_1' \smallsetminus B_1) \big) \\
    &= q(M_1' \sqcup M_2') \\
    &= M_1 \# M_2. 
    \end{align*}
    This finishes the proof. 
\end{proof}
\begin{remark}
	\textbf{Working with disjoint union spaces : }For any $A_1 \subhim X_1$, $A_2 \subhim X_2$, we have
	$$
	A_1 \sqcup A_2 = \iota_1(A_1) \cup \iota_2(A_2) \subhim X_1 \sqcup X_2,
	$$
	with $\iota_1 : A_i \to X_1\sqcup X_2$ be their canonical injection.
\end{remark}

\begin{prop}[Problem 4-30 \cite{LeeTM}]
	Prove the following generalization of the gluing lemma: suppose $X$ is a topological space and $\{X_{\alpha}\}_{\alpha \in A}$ is a locally finite closed cover of $X$. If for each $\alpha \in A$ we are given a continuous map $f_{\alpha}: X_{\alpha} \to Y$ such that $f_{\alpha}|_{X_{\alpha}\cap X_{\beta}} = f_{\beta}|_{X_{\alpha}\cap X_{\beta}}$ for all $\alpha,\beta \in A$, then there exists a unique continuous map $f: X\to Y$ whose restriction to each $X_{\alpha}$ is $f_{\alpha}$.
\end{prop}
\begin{proof}
	The map $f: X\to Y$ defined as $f(p) = f_{\alpha}(p)$, for any $\alpha \in A$ where $p \in X_{\alpha}$, is well defined since $\{f_{\alpha} : \alpha \in A\}$ are all agrees in their common domain. To show that $f$ is continous, it is enough if we can show that it is continous in a neighbourhood of each point (Proposition 2.19). By hypothesis, for any point $p \in X$ we have a neighbourhood $U_p$ contain $p$ such that $U_p$ intersect only finitely many sets in $\{X_{\alpha}\}$, say $X_{\alpha_1},\dots,X_{\alpha_k}$. So $U_p = U_p \cap \big( \bigcup_{i=1}^k X_{\alpha_i} \big)$. Let $U\subhim Y$ be any open subset. Then 
	\begin{align*}
	(f|_{U_p})^{-1}(U) &= f^{-1}(U) \cap U_p\\
	&= f^{-1}(U) \cap \Big( \bigcup_{i=1}^k X_{\alpha_i} \Big) \cap U_p\\
	&=\Big( \bigcup_{i=1}^k (f^{-1}(U) \cap X_{\alpha_i} )  \Big) \cap U_p \\
	&= \Big( \bigcup_{i=1}^k f_{\alpha_i}^{-1}(U)  \Big) \cap U_p
	\end{align*}
	is open in $U_p$, since $f_{\alpha}$ are continous by hypothesis. So $f$ is continous locally and we are done.
\end{proof}

\begin{prop}[Problem 4-33 \cite{LeeTM}]
	Suppose $X$ is a topological space with the property that for every open cover of $X$, there exists a partition of unity subordinate to it. Prove that $X$ is paracompact.
\end{prop}
\begin{proof}
	Let $\mathcal{U}$ be an open cover for $X$. To show that $X$ is paracompact we need to find a locally finite open refinement for $\mathcal{U}$. That is another open cover $\mathcal{V}$ which is locally finite and for any $V \in \mathcal{V}$, there exists some $U \in \mathcal{U}$ such that $V \subhim U$. Let $\{\psi_{\alpha}: X \to \er\}$ be a partition of unity subordinate to the open cover $\mathcal{U} = \{ U_{\alpha} : \alpha \in A \}$. These set of funtions satisfy the following properties :
	\begin{enumerate}[nolistsep]
		\item [(1)] $0 \leq \psi_{\alpha}(p) \leq 1$, $\forall p\in X$ and $\forall \alpha \in A$.
		\item [(2)] supp $\psi_{\alpha} \subhim U_{\alpha}$, $\forall \alpha \in A$.
		\item [(3)] $\{\text{supp }\psi_{\alpha} \}_{\alpha \in A}$ is locally finite.
		\item [(4)] $\sum_{\alpha \in A}\psi_{\alpha}(p) = 1$, $\forall p \in X$.
	\end{enumerate}
    By definition supp $\psi_{\alpha}=\overline{\{p\in X : \psi_{\alpha}(p) \neq 0\}} = \overline{\psi_{\alpha}^{-1}(0,\infty)}$. So $V_{\alpha}:=\psi_{\alpha}^{-1}(0,\infty)$ is an open subset contain in $U_{\alpha}$. We will show that $\mathcal{V} = \{V_{\alpha}: \alpha \in A\}$ is the desired open cover. 
    
    To show $\mathcal{V}$ is a cover for $X$, let $p\in X$ be arbitrary. Since $\mathcal{U}$ is a cover, $\exists \alpha \in A$ such that $p \in U_{\alpha}$. If $\psi_{\alpha}(p) \neq 0$, then $p \in V_{\alpha}$. If $\psi_{\alpha}(p) = 0$ then the property (4) tells us that there exists at least one $\alpha_0 \in A$ where $\psi_{\alpha_0}(p) \neq 0$. Therefore any $p\in X$ must contained in at least an element in $\mathcal{V}$. So $\mathcal{V}$ is an open cover. Since $V_{\alpha} \subhim U_{\alpha}, \forall \alpha \in A$, then $\mathcal{V}$ is an open refinement of $\mathcal{U}$.
    
    To show that $\mathcal{V}$ is locally finite, let $p \in X$ be arbitrary. If every neigbourhood of $p$ intersect infinitely many $V_{\alpha}$, then since $V_{\alpha} \subhim \text{supp }\psi_{\alpha}$, every neighbourhood of $p$ intersect infinitely many supp $\psi_{\alpha}$. By (3), this cannot happen. So $\mathcal{V}$ must be locally finite. This completes the proof.
\end{proof}

\subsection*{Chapter 5 Cell Complexes}
\begin{prop}[Exercise 5.3 : \textbf{Proposition 5.2} \cite{LeeTM}]
	Suppose that $X$ is a topological space whose topology is coherent with family $\mathcal{B}$ of subspaces.
	\begin{enumerate}[nolistsep]
		\item [(a)] If $Y$ is another topological space, then the map $f : X \to Y$ is continous $\Leftrightarrow$ $f|_B : B \to Y$ is continous for every $B \in \mathcal{B}$.
		\item [(b)] The map $\bigsqcup_{B \in \mathcal{B}} B \to X$ induced by inclusion of each set $B \hookrightarrow X$ is a quotient map.
	\end{enumerate} 
\end{prop}
\begin{proof}
	Part (a) is obvious since $X$ is coherent with $\mathcal{B}$. For (b), the map $q: \bigsqcup_{B \in \mathcal{B}} B \to X$ defined by $q|_B  = q \circ i_B = \iota_B : B \hookrightarrow X$ is continous, by characteristic property of disjoint union space. Also it is certainly onto since the union of all subspaces in $\mathcal{B}$ is $X$.  
	\[
	\begin{tikzcd}
	\bigsqcup_{B \in \mathcal{B}} B \arrow[r,"q"] & X \\
	B \arrow[u,"i_B"] \arrow[ur,"\iota_B",hookrightarrow,swap]
	\end{tikzcd}
	\]
	Suppose that $U \subhim \bigsqcup_{B \in \mathcal{B}} B$ is a saturated open subset with respect to $q$. Since $X$ is coherent with $\mathcal{B}$
	\begin{align*}
	q(U) \subhim X \text{ open }&\Leftrightarrow q(U)\cap B \text{ open in }B, \forall B \in \mathcal{B} \\
	&\Leftrightarrow \iota_{B}^{-1}(q(U)) \text{ open in }B, \forall B \in \mathcal{B} \\
	&\Leftrightarrow (q \circ i_{B})^{-1}(q(U)) \text{ open in }B, \forall B \in \mathcal{B}\\
	&\Leftrightarrow i_B^{-1} (q^{-1}q(U))  \text{ open in }B, \forall B \in \mathcal{B}\\
	&\Leftrightarrow i_B^{-1} (U)  \text{ open in }B, \forall B \in \mathcal{B} \\
	&\Leftrightarrow U \cap B \text{ open in }B, \forall B \in \mathcal{B}\\
	&\Leftrightarrow U \text{ open in }X.
	\end{align*}
	Therefore $q$ is a quotient map.
\end{proof}


\begin{prop}[Proposition 5.4 \cite{LeeTM}]
	Let $X$ be a Hausdorff space, and let $\mathcal{E}$ be a cell decomposition of $X$. If $\mathcal{E} $ is locally finite, then it is a CW decomposition.
\end{prop}
\begin{proof}
	Here i just write some technical details that skipped by Lee in the book. (C) condition is easy. To prove (W) condition, note that since $\mathcal{E}$ is a partition for $X$, then we can write 
	$$
	A = A \cap X = A \cap \big(\bigcup_{e \in \mathcal{E}} \bar{e}\big) = \bigcup_{e \in \mathcal{E}} (A \cap \bar{e}).  
	$$
	So 
	$$
	W \smallsetminus A = W \smallsetminus \big( \bigcup_{e \in \mathcal{E}} (A \cap \bar{e}) \big) = \bigcap_{e \in \mathcal{E}} W\smallsetminus (A \cap \bar{e})  
	$$
    But for any $e \in \mathcal{E}\smallsetminus \{e_1,\dots,e_m \}$, we have 
    $$
    W \smallsetminus (A \cap \bar{e}) = (W \smallsetminus A) \cup (W \smallsetminus \bar{e}) = (W \smallsetminus A) \cup W = W.
    $$
    Therefore
    \begin{align*}
    W \smallsetminus A &= \bigcap_{e \in \mathcal{E}} W\smallsetminus (A \cap \bar{e}) = W \cap \big( \bigcap_{i=1}^{m} W\smallsetminus (A \cap \bar{e}_i) \big)\\ &= W \cap  \big( W \smallsetminus \bigcup_{i=1}^m (A \cap \bar{e}_i) \big) \\ &=  W \smallsetminus \bigcup_{i=1}^m (A \cap \bar{e}_i).
    \end{align*}
\end{proof}

\begin{prop}[Theorem 5.6 \cite{LeeTM}]
	Suppose $X$ is a CW complex and $Y$ is a subcomplex of $X$. Then $Y$ is closed in $X$, and with the subspace topology and cell decomposition that it inherit from $X$, it is a CW complex.
\end{prop}
\begin{proof}
	Missing step : Note that since $S \subhim Y$, then $S \cap \bar{e} = (S \cap Y) \cap \bar{e} = S \cap (Y \cap \bar{e})$. We know that $\bar{e}\smallsetminus e \subhim \bigcup_{i=1}^m e_i$, where some of them may contain in $Y$ some may not. So we wrote
	
	$$\bar{e} = e \cup (\bar{e}\smallsetminus e) = e \cup \big( \bigcup_{e_i \subhim Y} e_i \cup \bigcup_{e_j \nsubseteq Y} e_j  \big)$$
	For those $e_i$ that contain in $Y$, then by definition of $Y$, $\bar{e}_i \subhim Y$. Therefore
	$$
	Y \cap \bar{e} \subhim Y \cap \Big( e \cup \bigcup_{e_i \subhim Y} e_i \cup \bigcup_{e_j \nsubseteq Y} e_j  \Big)  =  \bigcup_{e_i \subhim Y} e_i \subhim  \bigcup_{e_i \subhim Y} \bar{e}_i
	$$
	But $\bar{e} \cap \bigcup_{e_i \subhim Y} \bar{e}_i = \bigcup_{e_i \subhim Y} \bar{e}_i$. So
	$$
	Y \cap \bar{e} \subhim  \big( \bigcup_{e_i \subhim Y} \bar{e}_i \big) \cap \bar{e}.
	$$ 
	Also, since $\bigcup_{e_i \subhim Y} \bar{e}_i \subhim Y$, then $\Big( \bigcup_{e_i \subhim Y} \bar{e}_i \Big) \cap \bar{e} \subhim Y \cap \bar{e}$. Therefore
	$$
	Y \cap \bar{e} =  \big( \bigcup_{e_i \subhim Y} \bar{e}_i \big) \cap \bar{e}.
	$$
\end{proof}

\begin{prop}[Example 5.8(a) \textbf{CW Complexes} \cite{LeeTM}]
\textbf{A $0$-dimensional CW complex is just a discrete space}.
\end{prop}
\begin{proof}[Why ?] 
	Suppose $(X,\mathcal{E})$ be a 0-dimensional CW complex. Since the largest dimension of cells in the CW complex $X$ is $0$, all of the cells of $X$ has dimension $0$. But a $0$-cell is just a point. So cell decomposition $\mathcal{E}$ of $X$ is just a union of singletons. We write (by definition)
	$$
	X \text{ is a Hausdorff Space }, \quad \mathcal{E} = \big\{ \{p\} : p \in X \big\}, \quad  \text{satisfy } (C) \text{ and }(W).
	$$
	The Hausdorff and (C) condition implies the same thing, that is $\{p\} $ is closed for each $p \in X$. So $\overline{\{p\}} = \{p\}$. (W) condition implies that $X$ has discrete topology. To see this, let $A \subhim X$ be an arbitrary subset of $X$. The topology $X$ is coherent with $\{ \overline{\{p\}} : \{p\} \in \mathcal{E} \}$. Since for $p \in A$, $A\cap\{p\} = \{p\}$ and for $p \notin A$, $A \cap \{p\} = \emptyset$ both closed in $\{p\}$, then $A$ is closed in $X$. Hence any arbitrary subset of $X$ is closed. So $X$ has discrete topology.   
\end{proof}

\begin{prop}[Example 5.8(b) \textbf{CW Complexes} \cite{LeeTM}]
	$$\textit{Bouquet of Circles }\quad  \s^1 \vee \cdots \vee \s^1 = (\s^1 \sqcup \cdots \sqcup \s^1)/{\sim}$$
\end{prop}
\begin{proof}[Why ?]
	Suppose that we have a $k$-tuple of circles $(\s^1)_{i=1}^k =( \s^1, \dots, \s^1)$ each of them considered as subspace of $\er^2$. Their disjoint union spaces is the set
	$$
	\s^1 \sqcup \cdots \sqcup \s^1  = \{ (p,i) : \forall p \in \s^1, i = 1,\dots,k \}    
	$$
	endowed with disjoint union topology. For each $i$, we have canonical injection $\iota_i : \s^1 \hookrightarrow \s^1 \sqcup \cdots \sqcup \s^1 $ defined as $\iota_i(p) = (p,i)=p_i$. For each $i$, let $p_i$ be a specific point in the $i$-th circle $\s^1$.  By defining an equivalence relation $\sim$ on $\s^1 \sqcup \cdots \sqcup \s^1$ as : $\forall x,y \in \s^1 \sqcup \cdots \sqcup \s^1$
	$$
	 x \sim y \Leftrightarrow x=y \text{ or } x,y \in \{p_1,\dots,p_k\}.
	$$
	Hence the partition associated to this relation is the collection of singletons $[x]$ for $x \notin \{p_1,\dots,p_k\}$, together with single set $\{p_1,\dots,p_k\}$. The collection of all of the equivalence classes endowed with quotient topology induced by the canonical projection
	$$
	q :  \s^1 \sqcup \cdots \sqcup \s^1 \to \s^1 \vee \cdots \vee \s^1
	$$ 
	is called \textit{bouquet of circles} $\s^1 \vee \cdots \vee \s^1=(\s^1 \sqcup \cdots \sqcup \s^1)/{\sim}$.
	
	$\bullet \quad \s^1 \vee \cdots \vee \s^1 $ as a CW complex : It is easy to check case by case that $\s^1 \vee \cdots \vee \s^1$ ie Hausdorff (since $\s^1 \sqcup \cdots \sqcup \s^1$ is). The cell decomposition is consist of the base point $e_0:=[p_i]$, as a 0-cell, and $k$-tuples of 1-cells $e_j:= (q \circ \iota_j)(\s^1 \smallsetminus \{p\})$, for each $j =1,\dots,k$. To see that each of them are (open) 1-cells, consider the composition maps
	\[
	\begin{tikzcd}
	I \arrow[r,"\omega"] &  \s^1  \arrow[r,"\iota_j",hookrightarrow] & \s^1 \sqcup \cdots \sqcup \s^1 \arrow[r,"q"] & \s^1 \vee \cdots \vee \s^1.
	\end{tikzcd}
	\]	
%	$$
%	[0,1] \xrightarrow{\omega_j} \s^1 \xhookrightarrow{\iota_j} \s^1 \sqcup \cdots \sqcup \s^1 \xrightarrow{q} \s^1 \vee \cdots \vee \s^1.
%	$$
	where $\omega : I \to \s^1$ is a quotient map defined by $\omega(s) = p e^{2\pi i s}$. Since $(0,1) \subhim I$ is a saturated open subset, then the restriction $\omega|_{(0,1)} : (0,1) \to \s^1 \smallsetminus \{p\}$ is an injective quotient map and hence a homeomorphism\footnote[2]{by Proposition 3.62}. The canonical injection map $\iota_j$ is a topological embedding. Therefore $\s^1 \smallsetminus \{p\} \approx \iota_j (\s^1 \smallsetminus \{p\})$. By same argument as we did with $\omega_j$, the restriction $q|_{\iota_j (\s^1 \smallsetminus \{p\})} : \iota_j (\s^1 \smallsetminus \{p\}) \to q(\iota_j (\s^1 \smallsetminus \{p\}))$ is a homoemomorphism. Therefore
	$$
	(0,1) \approx \s^1 \smallsetminus \{p\} \approx \iota_j (\s^1 \smallsetminus \{p\}) \approx (q \circ \iota_j)(\s^1 \smallsetminus \{p\})
	$$   
	so each subset $(q \circ \iota_j) (\s^1 \smallsetminus \{p\})$ is an open 1-cell. With this observation, the characteristic map for each 1-cell is the composition map
	$$
	\Phi_j = q \circ \iota_j \circ \omega : I \to \s^1 \vee \cdots \vee \s^1.
	$$
	We have seen that $\Phi_j (0,1) = (q \circ \iota_j)(\s^1 \smallsetminus \{p\})$. Also, $\Phi_j(\{0,1\}) = (q \circ \iota_j)(p_j)=[p_j]$. So the space $\s^1 \vee \cdots \vee \s^1$ together with this cell decomposition is a \textit{cell complex}. 
	
	(C) condition is satisfied : For any 1-cell $e_j = (q\circ \iota_j)(\s^1 \smallsetminus \{p\})$, the base point $[p_j]$ is its boundary point. But since $e_j \cup [p_j] = (q \circ \iota_j)(\s^1)$ and this subset is a closed subset, then $(q \circ \iota_j)(\s^1) \subset \bar{e}_j = e_j \cup \doo e_j$. So 
	$$
	\bar{e}_j = (q \circ \iota_j)(\s^1) = e_j \cup [p_j] = e_j \cup e_0.
    $$
    (W) condition is satisfied : Let $A \subhim \s^1 \vee \cdots \vee \s^1$ be any subset such that $A \cap \bar{e}_i$ is closed in $\bar{e}_i$ (hence in $\s^1 \vee \cdot \vee \s^1$) for any $i =0,\dots,k$. Nothing new from considering A $\cap \bar{e}_0$, so lets focus on $A \cap \bar{e}_j$ for $j =1,\dots,k$. By definition
    \begin{align*}
    A\cap \bar{e}_j \text{ is closed } &\Leftrightarrow q^{-1}(A\cap \bar{e}_j) \text{ is closed in } \s^1 \sqcup \cdots \sqcup \s^1 \\ &\Leftrightarrow q^{-1}(A \cap \bar{e}_j) \cap \iota_i(\s^1) \text{ is closed in }\iota_i(\s^1), \forall i=1,\dots,k.
    \end{align*}
    By note that
    \begin{align*}
    q^{-1}(A \cap \bar{e}_j) &= q^{-1}(A) \cap q^{-1}(\bar{e}_j) \\ &= q^{-1}(A) \cap q^{-1} (q(\iota_j(\s^1))) \\&= q^{-1}(A) \cap \big( \iota_j(\s^1) \cup \bigcup_{l=1}^{k} \{p_l\} \big)\\
    &= \big( q^{-1}(A) \cap \iota_j(\s^1) \big) \cup \big( q^{-1}(A) \cap \bigcup_{l=1}^{k} \{p_l\} \big),
    \end{align*}
    then we have
    \begin{equation*}
    q^{-1}(A \cap \bar{e}_j) \cap \iota_i(\s^1) = \left\{
    \begin{array}{rl}
    q^{-1}(A)\cap \iota_j(\s^1) & \text{if } i=j,\\
    \{p_i\} \text{ or }\emptyset \phantom{XX} & \text{if } i \neq j.
    \end{array} \right.
    \end{equation*}
    So to sum up, the subsets $A \cap \bar{e}_j \subhim \s^1 \vee \cdots \vee \s^1$
    \begin{align*}
     A\cap \bar{e}_j \text{ is closed }\forall j &\Leftrightarrow q^{-1}(A\cap \bar{e}_j) \text{ is closed in } \s^1 \sqcup \cdots \sqcup \s^1, \forall j \\ &\Leftrightarrow q^{-1}(A \cap \bar{e}_j) \cap \iota_i(\s^1) \text{ is closed in }\iota_i(\s^1),  \forall i,j
     \\ &\Leftrightarrow q^{-1}(A)\cap \iota_j(\s^1) \text{ is closed in }\iota_j(\s^1) \forall j
     \\ &\Leftrightarrow q^{-1}(A) \text{ is closed in }\s^1 \sqcup \cdots \sqcup \s^1 \\
     &\Leftrightarrow A \text{ is closed in }\s^1 \vee \cdots \vee \s^1.
    \end{align*}
	I.e., $\s^1 \vee \cdots \vee \s^1$ is coherent with closed subspaces $\{\bar{e}_0,\bar{e}_1,\dots,\bar{e}_k \}$.
\end{proof}

\begin{prop}[Exercise 5.19 \cite{LeeTM}]
	Suppose $X$ is an $n$-dimensional CW complex with $n\geq 1$, and $e$ is any $n$-cell of $X$. Show that $X \smallsetminus e$ is a subcomplex, and $X$ is homeomorphic to an adjunction space obtain from $X\smallsetminus e$ by attaching a single $n$-cell.
\end{prop}
\begin{proof}
	Suppose that $e'$, with $1\leq k \leq n$, is any $k$-cell of an $n$-dimensional CW complex $(X,\mathcal{E})$. By definition, there exists a continous map  $\Phi : D \to X$, where $D$ is a closed $k$-cell, such that $\Phi|_{\Inter D} : \Inter D \to e'$ is a homeomorphism and $\Phi(\partial D)$ contain in the union of all cells of $X$ of dimensions strictly less that $k$. We also know that (by continuity and closed map lemma) $\Phi(D)= \bar{e}'$. So $\bar{e}' = \Phi(D) = e' \sqcup \Phi(\doo D)$ and together with $(C)$,
	$$
	\bar{e}' \smallsetminus e' = \Phi(\doo D) \subhim \bigcup_{i=1}^{m} \{e_i \in \mathcal{E} : \text{dim }e_i < k \}.
	$$
	Since $e$ has dimension $n$, the closure of any $k$-cell different from $e$ must contained in $X \smallsetminus e$. So $X \smallsetminus e$  is a subcomplex.
	
	Let $\Phi : D \to X $ be the characteristic map for $e$. The map $\varphi := \Phi|_{\doo D} : \doo D \to X$ takes values in $X\smallsetminus e$. So we can form the adjunction space $(X\smallsetminus e) \cup_{\varphi} D$. Define a map $\tilde{\Phi} : (X \smallsetminus e) \sqcup D \to X$ so that $\tilde{\Phi}|_{X\smallsetminus e} = \iota : X\smallsetminus e \hookrightarrow X$ and $\tilde{\Phi}|_D  = \Phi$. This map is continous (by characteristic property) and onto. If we can show that it is a quotient map, then  by uniqueness of quotient topology $(X\smallsetminus e) \cup_{\varphi} D \approx X$.
	\[
	\begin{tikzcd}
	 (X \smallsetminus e) \sqcup D \arrow[d,"q",swap] \arrow[dr,"\tilde{\Phi}"] & \\
	(X\smallsetminus e) \cup_{\varphi} D \arrow[r,dashrightarrow] & X.
	\end{tikzcd}
	\]
	Suppose that $A \subhim (X \smallsetminus e) \sqcup D $ is a saturated closed subset. By (W) condition, $\tilde{\Phi}(A)$ closed in $X$ if and only if $\tilde{\Phi}(A) \cap \bar{e}$ is closed in $\bar{e}$ for any cell $e \in \mathcal{E}$.
	
	By hypothesis, $A \cap (X \smallsetminus e)$ is closed in $X \smallsetminus e$ and $A \cap D$ is closed in $D$. Since $\Phi : D \to X$ is a closed map, then $\Phi(A \cap D)$ is closed in $X$ hence in $\bar{e}$. We know that $\Phi(A \cap D) =\tilde{\Phi}(A \cap D) \subhim \tilde{\Phi}(A) \cap \tilde{\Phi}(D) = \tilde{\Phi}(A) \cap \bar{e}$ always holds. To show that $\Phi(A \cap D) \supseteq \tilde{\Phi}(A) \cap \bar{e}$, let $x \in \tilde{\Phi}(A) \cap \bar{e}$ be arbitrary. There exists $y \in D$ such that $\tilde{\Phi}(y) = x \in \tilde{\Phi}(A)$. But, since $A=\tilde{\Phi}^{-1}(\tilde{\Phi}(A))$, then $y \in A$. So $x \in \tilde{\Phi}(A \cap D) = \Phi(A \cap D)$. So we conclude that $\Phi(A \cap D) = \tilde{\Phi}(A) \cap \bar{e}$ is closed in $\bar{e}$.
	
	Now we have to show that $\tilde{\Phi}(A) \cap \bar{e}'$ is closed in $\bar{e}'$ for any cell $e' \subhim X \smallsetminus e$. Suppose $e'$ is any such cell. Since the inclusion map is a topological embedding, then the fact that $A \cap (X \smallsetminus e)$ is closed in $X \smallsetminus e$ imply that $\tilde{\Phi}(A \cap (X \smallsetminus e))$ is closed in $\iota(X \smallsetminus e) = X \smallsetminus e$. By same arguments as the previous paragraph, $\tilde{\Phi}(A \cap (X \smallsetminus e)) = \tilde{\Phi}(A) \cap \tilde{\Phi} (X \smallsetminus e) = \tilde{\Phi}(A) \cap  (X \smallsetminus e) $ is closed in $X \smallsetminus e$. Since by Theorem 5.6, $X \smallsetminus e$ is closed in $X$, then $\tilde{\Phi}(A) \cap  (X \smallsetminus e)$ is closed in $X$. Therefore $\tilde{\Phi}(A) \cap  (X \smallsetminus e) \cap \bar{e}' = \tilde{\Phi}(A) \cap  \bar{e}'$ is closed in $\bar{e}'$. So we conclude that $\tilde{\Phi}(A)$ is closed in $X$, and hence $\tilde{\Phi}$ is a quotient map. This completes the proof.
\end{proof}

\begin{prop}[Theorem 5.20 \textbf{CW Construction Theorem}]
Suppose $X_0 \subhim X_1 \subhim \cdots \subhim X_{n-1} \subhim X_n \subhim \cdots$ is a sequence of topological spaces satisfying the following conditions:
\begin{enumerate}[nolistsep]
	\item [(i)] $X_0$ is a nonempty discrete space.
	\item [(ii)] For each $n \geq 1$, $X_n$ is obtained by $X_{n-1}$ by attaching a (possibly empty) collection of $n$-cells. 
\end{enumerate}
Then $X = \bigcup_n X_n$ has a unique topology coherent with the family $\{X_n\}$, and a unique cell decomposition making it into a CW complex whose $n$-skeleton is $X_n$ for each $n$. 	
\end{prop}
\begin{proof}
	\textbf{Give $X$ a topology $\mathcal{C}$ by }declaring a subset $B \subhim X$ is closed $\Leftrightarrow$ $B \cap X_n$ is closed in $X_n$ for each $n$. It is immidiate that:
	\begin{enumerate}[nolistsep]
		\item [$\bullet$] $\mathcal{C}$ is a topology for $X$.
		\item [$\bullet$] $\mathcal{C}$ is the unique topology coherent with $\{X_n\}$.
		\item [$\bullet$] $X_n$ is a subspace of $X$, for all $n$ : if $B$ is closed in $X$, then $B \cap X_n$ is closed in $X_n$ by definition; conversely, if $B$ is closed in $X_n$, then for $m\leq n$ $B \cap X_m$ is closed in $X_m$. For $m> n$, note that $X_{m-1}$ is closed subspace of $X_m$ for $m \geq 1$, then since $B \subhim X_n \subhim X_m$ we have $B$ is closed in $X_m$. Therefore $B$ is closed in $X$.
	\end{enumerate} 
Next, we define \textbf{the cell decomposition of $X=\bigcup_{n} X_n$} (this part is clear from the text).


\end{proof}

\begin{prop}[Exercise 5.31 \cite{LeeTM} \text{Prove Proposition 5.30}]
	Every simplex is a convex hull of its vertices.
\end{prop}
\begin{proof}
	Let $[v_0,\dots,v_k]$ be the simplex of an affinely independent $k+1$ points $W = \{v_1,\dots v_k\}$ in $\rn$, and $A$ be the convex hull of $W$. Observe that for any $x = \sum_{i=0}^k t_i v_i$ and $y = \sum_{i=0}^k s_i v_i$ in $[v_1,\dots,v_n]$, the line segment is
	$$
    x+(1-a)y = \sum_{i=0}^k (t_i(1-a) + as_i) v_i = \sum_{i=0}^k t_i' v_i, \quad 0\leq a \leq 1.
	$$ 
	It can be check directly that $t_i' \geq 0$ and $\sum_{i=0}^k t_i' = 1$. Therefore $x+(1-a)y \in [v_1,\dots,v_n]$ for any $a\in [0,1]$. So $[v_1,\dots,v_n]$ is convex and $A \subhim [v_1,\dots,v_n]$ by definition of $A$.
	
	To show $[v_1,\dots,v_n] \subhim A$, we must show that $[v_1,\dots,v_n] \subhim C$ for any convex subset $C$ contain $W$. Let $x \in [v_1,\dots,v_n]$ arbitrary. If $x = v_i$ for any $i=1,\dots,k$, then $x\ in C$. Suppose that any elements in the form $ \sum_{i=0}^m t_i v_i$,for some $n < k$, contain in $C$. Then for any $x = \sum_{i=0}^{n+1} t_i v_i$, we have
	\begin{align*}
	x &=  \sum_{i=0}^{n} t_i v_i + t_{n+1} v_{n+1} = \sum_{j=0}^n t_j \cdot \frac{\sum_{i=0}^{n} t_i v_i}{\sum_{j=0}^n t_j} + t_{n+1} v_{n+1}\\ &= \big(\sum_{j=0}^n t_j \big) \tilde{v} + t_{n+1} v_{n+1},
	\end{align*}
	with $\tilde{v} = \sum_{i=0}^{n} t_i v_i / \sum_{j=0}^n t_j = \sum_{i=0}^n a_i v_i$ is in $C$. Since $\sum_{i=0}^{n+1}t_i = 1$, then $x$ is just an element in the line segment between $\tilde{v} \in C$ and $v_{n+1} \in C$. Therefore $x \in C$. Hence $[v_1,\dots,v_n] \subhim A$. This completes the proof.  
\end{proof}

\begin{prop}[Exercise 5.34 \cite{LeeTM}]
	If $K$ is an Euclidean simplical complex, then the collection consisting of the interiors of the simplices of $K$ is a regular $CW$ decomposition of $|K|$. 
\end{prop}
\begin{proof}
	By definition, $K$ is a collection of simplices in $\rn$ satisfy
	\begin{enumerate}[nolistsep]
		\item [(i)] For any $\sigma \in K$, every face of $\sigma$ is also in $K$.
		\item [(ii)] For any $\sigma_1,\sigma_2 \in K$, $\sigma_1 \cap \sigma_2$ is empty or a face of each.
		\item [(iii)] $K$ is a locally finite collection.
	\end{enumerate}
    Let $\mathcal{K}$ denote the collection of interiors of the simplices of $K$. We can write $\mathcal{K} = \{e = \In \sigma = \sigma \smallsetminus \Bd \sigma : \sigma \in K \}$. We have to show that 
    \begin{enumerate}[nolistsep]
    	\item [(1)] $|K|$ is Hausdorff (obvious) and $\mathcal{K}$ is a cell decomposition of $|K|$.
    	\item [(2)] $\mathcal{K}$ satisfy $(C)$ and $(W)$ condition.
    	\item [(3)] Each cell of $\mathcal{K}$ is regular, and the closure of each is a finite subcomplex.
    \end{enumerate}
    Once we show (1) holds, then by Proposition 5.4, (2) is satisfied. So we only need to show (1) and (3). 
    
    For (1), note that each $\sigma \in K$ is a closed $k$-cell, so we have homeomorphism $\phi : \closedball^k \to \sigma$ that takes $\openball^k$ to interior of $\sigma$ in $\rk$, which is equal to $\In \sigma = e$. So each $e \in \mathcal{K}$ is an open cell. For any $e_1,e_2 \in \mathcal{K}$ we have $\sigma_1 \cap \sigma_2 $ is empty or a face of each. If it is empty, then $e_1 \cap e_2= \emptyset$. So let $\sigma_1 \cap \sigma_2 = \sigma'$, for some face $\sigma' \in K$ of $\sigma_1$ and $\sigma_2$. Since $\sigma' \subhim \Bd \sigma_i$, then $e_i \cap \sigma' = \emptyset$. Also $e_1 \cap e_2 \subhim \sigma'$, so $e_1 \cap e_2  = (e_1 \cap e_2 )\cap \sigma' = (e_1 \cap \sigma') \cap (e_2 \cap \sigma') = \emptyset$. Next, we need to show that $\bigcup \mathcal{K} = |K| = \bigcup K$. The side $\bigcup \mathcal{K} \subhim \bigcup K$ is obvious. Now suppose we have $x \in \sigma \subhim |K|$. If $x \in \In \sigma$, then $x \in \bigcup \mathcal{K}$. So let $x $ is in a face of $\sigma$. By looking at the barycentric coordinate of $x$, we obtain a face of $\sigma$ such that $x$ is in the interior of that simplex. Hence $x \in \bigcup \mathcal{K}$. Therefore we conclude that $\mathcal{K}$ is a partition of $|K|$. For each $e \in \mathcal{K}$, we have a characteristic map $\Phi : \closedball^k \to |K|$ defined as composition $\iota \circ \phi$, where $\iota : \sigma \hookrightarrow |K|$.
    
    (2) follows easily from Proposition 5.4 (since (iii) implies local finiteness of $\mathcal{K}$).
    
    For (3), since $\Phi (\doo \closedball^k) = \Bd \sigma$, then each cell in $\mathcal{K}$ is regular. The closure $\bar{e} = \sigma$ is indeed a finite subcomplex by (1) and the fact that a simplex has finite dimension.   
    
\end{proof}

\begin{prop}[Proposition 5.38 \cite{LeeTM}]
	Let $ \sigma = [v_0,\dots,v_k]$ be a $k$-simplex in $\rn$. Given any $k+1$ points $w_0,\dots,w_k$ in $\mathbb{R}^m$, there is a unique map $f : \sigma \to \er^m$ that is the restriction of an affine map and takes $v_i$ to $w_i$ for each $i$. 
\end{prop}
\begin{proof}
	By definition of $\sigma$, the vectors $\{v_1-v_0,\dots,v_k-v_0\}$ are linearly independent. Let $\{e_i\}_{i=1}^{n} = \{v_1-v_0,\dots,v_k-v_0,e_{k+1},\dots,e_n\}$ be a basis for $\rn$ obtained by completing the set $\{v_i-v_0\}_{i=1}^k$. Define a linear map  $g : \rn \to \er^m$ such that $g(v_i-v_0) =w_i-w_0$ for $i=1,\dots,k$ and $g(e_i) = a_i$ for $i=k+1 ,\dots, n$ and some vectors $a_i \in \er^m$. Since a linear map completely determined by its value on a basis of the domain, this map is well-defined.
	
	Now, we can compose the map $g$ in the previous construction with invertible affine maps $A : \rn \to \rn$ and $B : \er^m \to \er^m$ defined as $A(x) = x -v_0$ and $B(y) = y-w_0$. That is, define $f : \rn \to \er^m$ as $f(x) = B^{-1} \circ g \circ A (x)$. So we have
	$$
	f(x) = g(x-v_0) + w_0 = g(x) + w_0 - g(v_0).
	$$
	with $f(v_i) = g(v_i - v_0) + w_0 = w_i$. This map is an affine map, since $f(x) = g(x) + c ; \, c = w_0 - g(v_0)$, and therefore the restriction of $f$ to $\sigma$ is the desired map.
	
	To show uniqueness, let $F : \sigma \to \er^m$ be a map obtained by restriction of a affine map, say $F(x) = A(x) + c$, where $A : \rn \to \er^m$ is a linear map. For any $v = \sum_{i=0}^k t_i v_i \in \sigma$, we have
	\begin{align*}
		F(v) &= A(\sum t_i v_i) + c  \\ &= \sum t_i A(v_i) + c \\
		&= \sum t_i (F(v_i) - c) + c \\
		&= \sum t_i F(v_i) - (\sum t_i)c + c \\
		&= \sum_{i=0}^k t_i F(v_i).
	\end{align*}
	So this map is uniquely determined by its value on the vertices ${v_i}$.
\end{proof}

\begin{prop}[Exercise 5.40 \cite{LeeTM}]
	Let $K$ and $L$ be simplicial complexes. Suppose $f_0 :  K_0 \to  L_0$ is any map with the property that whenever $\{v_0,\dots,v_k\}$ are the vertices of a simplex of $K$, $\{f_0(v_0),\dots,f_0(v_k)\}$ are the vertices of a simplex of $L$ (possibly with repetitions). Then there is a unique simplicial map $f : |K| \to |L|$ whose vertex map is $f_0$. It is a simplicial isomorphism
	if and only if $f_0$ is a bijection satisfying the following additional condition: $\{v_0,\dots,v_k\}$ are the vertices of a simplex of $K$ if and only if $\{f_0(v_0),\dots,f_0(v_k)\}$ are
	the vertices of a simplex of $L$.  
\end{prop}
\begin{proof}
	Let $K$ be simplical complexes in $\rn$ and $L$ be simplical complexes in $\er^m$. By Proposition 5.38, for any set of vertices $\{v_0,\dots,v_k\}$ of a simplex $\sigma= [v_0,\dots,v_k]$ of $K$ and its corresponding image $\{f_0(v_0),\dots,f_0(v_k)\}$, which is also vertices of a simplex of $L$, we have a unique affine map $f_{\sigma} : \sigma \to \er^m$ such that $f_{\sigma} (v_i) = f_0(v_i)$. For any two such map $f_{\sigma}$ and $f_{\tau}$, where $\sigma \cap \tau \neq \emptyset$, we have $f_{\sigma}|_{\sigma \cap \tau} = f_{\tau}|_{\sigma \cap \tau}$ by uniqueness property of affine map in Proposition 5.38. Since $K$ is a locally finite closed cover of $|K|$, then (by Problem 4.30) we have unique continous map $f : |K| \to \er^m$ such that $f|_{\sigma} = f_{\sigma}$ is the affine map defined above. Clearly $f$ is agree with $f_0$ on $K_0$ and it is easy to see that $f_{\sigma}(\sigma) = [f_0(v_0),\dots,f_0(v_k)]$. Therefore we have the desired simplical map $f: |K| \to |L|$. 
\end{proof}

\subsubsection*{Problems Chapter 5}


\begin{prop}[Problem 5.8 \cite{LeeTM}] Prove Proposition 5.7 : If $X$ is any CW complex, the topology of $X$ is coherent with the collection of subspaces $\{X_n : n \geq 0\}$. 
\end{prop}
\begin{proof}
	Let $\mathcal{S} = \{X_n: n\geq 0 \}$ and $U \subhim X$ such that for any $n \geq 0$, $U \cap X_n$ is open in $X_n$. By definition, $U \subhim X$ is open $\Leftrightarrow$ $U \cap \bar{e}$ is open in $\bar{e}$ for each $\bar{e} \in \{\bar{e} : e \in \mathcal{E}\}$. Let $e \in \mathcal{E}$ be arbitrary and $X_n$ be a skeleton contain $e$. So $\bar{e} \subhim X_n$. Since $U \cap  \bar{e} = (U \cap X_n) \cap \bar{e}$ and $U \cap X_n$ is open in $X_n$, then $U \cap \bar{e}$ is open in $\bar{e}$. So $U$ is open in $X$ and therefore the topology of $X$ is coherent with $\mathcal{S}$. 
\end{proof}

\begin{prop}[Problem 5-11 \cite{LeeTM}]
	Prove Proposition 5.16 (a CW complex is locally compact $\Leftrightarrow$ it is locally finite).
\end{prop}
\begin{proof}
	Let $X$ be a CW complex. For $\Rightarrow$ part, suppose $X$ is locally compact. For any $p\in X$, there exists a compact subset $K$ contain a neighbourhood $U$ of $p$. By Theorem 5.14, $K$ contain in a finite subcomplex. So we have a neighbourhood $U \subhim K$ that contained in a finite subcomplex. So $X$ is locally finite.
	
	For $\Leftarrow$ part, suppose $X$ is locally finite. Any point has a neighbourhood $U$ such that $U \subhim \bigcup_{i=1}^k \bar{e}_i$. For any $e \in \mathcal{E}$,  $\bar{e}$ is compact. So $\bar{e}_i$ are closed and contained in a finite subcomplex, say $A_i$. Therefore the subset $K:=\bigcup_{i=1}^k \bar{e}_i$ is closed and contained in the finite subcomplex $\bigcup_{i=1}^k A_i$. It follows that $K$ is a compact subset contain $U$, which completes the proof.
\end{proof}

\subsection*{Chapter 6 Compact Surfaces}
 
\begin{prop}[Example 6.6 \cite{LeeTM}]
	If $M$ is any $n$-manifold, a connected sum $M \# \mathbb{S}^n$ is homeomorphic to $M$. Let $B_2 \subhim \s^n$ be the open lower hemisphere, so $(\s^n)' = \s^n \smallsetminus B_2$ is the closed upper hemisphere, which is homeomorphic to a closed ball. Then $M \# \s^n$ is obtained from by cutting out an open ball $B_1$ and pasting back to the closed ball along the boundary sphere.
\end{prop}
\begin{proof}
	We will show this by using uniqueness of quotient topology. Let $f : \doo (\s^n)' \to \doo M'$ be the homeomorphism and $q : M' \sqcup (\s^n)' \to M' \cup_f (\s^n)' = M \# \s^n$ be the quotient map. We have to construct a quotient map $Q :  M' \sqcup (\s^n)' \to M$ that makes same identification as $q$. 
	\[
	\begin{tikzcd}
	 M' \sqcup (\s^n)' \arrow[d,"q",swap] \arrow[dr,"Q"] & \\
	M \# \s^n \arrow[r,dashrightarrow] & M.
	\end{tikzcd}
	\]
	Let $\sigma_N : \s^n \smallsetminus \{N\} \to \rn$ and $\sigma_S : \s^n \smallsetminus \{S\} \to \rn$ be the stereographic projections map from the "north pole" $N =(0,\dots,0,1) \in \er^{n+1}$ and the "south pole" $S = (0,\dots,0,-1) \in \er^{n+1}$ respectively. Also by hypothesis $B_1$ is a regular coordinate ball, so let $\psi : U \to B_R(0)$ be the homeomorphism and $U \supset \overline{B_1}$ be an open subset such that $\psi(U) = B_R(0) \supset \psi(\overline{B_1}) = \overline{\mathbb{B}^n}$ for some $R>0$ and $\psi(B_1) = \mathbb{B}^n$, $\psi(\doo B_1) = \s^{n-1}$.
	
	First note that the open lower hemisphere $B_2\subhim \s^n$ is a regular coordinate ball since we can choose an open subset $V \subhim \s^n $ contain $\overline{B_2}$ defined by $V :=\s^n \cap \{x \in \er^{n+1} \mid x^{n+1} < \varepsilon \text{ for some }0 <\varepsilon <1\}$ and a homeomorphism, obtained by restriction of $\sigma_N$ to $V$ such that $\sigma_{N}(V) = B_r(0) \supset \sigma_N(\overline{B_2}) = \overline{\mathbb{B}^n}$ and $\sigma_{N}(B_2) = \mathbb{B}^n$, $\sigma_{N}(\doo B_2) = \s^{n-1}$. Let us define $\varphi_N := \sigma_N|_{\overline{B_2}}$. Also $(\s^n)' \approx \overline{\mathbb{B}^n}$ by $\varphi_S := \sigma_S|_{(\s^n)'} : (\s^n)' \to \overline{\mathbb{B}^n}$.  
	
	Define $Q :  M' \sqcup (\s^n)' \to M$ by its restrictions to $M'$ and $(\s^n)'$ as $Q|_{M'} = \iota_{M'} : M' \hookrightarrow M$ and $Q|_{(\s^n)'} := \iota_{\overline{B_1}} \circ \widetilde{f} \circ \varphi_N^{-1} \circ \varphi_S : (\s^n)' \to M$. 
	\[
	\begin{tikzcd}
	(\s^n)' \arrow[r,"\varphi_S"] &  \overline{\mathbb{B}^n}  \arrow[r,"\varphi_N^{-1}"] & \overline{B_2} \arrow[r,"\widetilde{f}"] & \overline{B_1} \arrow[r,"\iota_{\overline{B_1}}",hookrightarrow] & M.
	\end{tikzcd}
	\]	
	The map $\iota_{\overline{B_1}}  : \overline{B_1} \hookrightarrow M $ is just inclusion map and $\widetilde{f} : \overline{B_2} \to \overline{B_1}$ is an extension of $f : \doo B_2 \to \doo B_1$ defined as $\widetilde{f} := (\psi|_{\overline{B_1}})^{-1} \circ g \circ \varphi_N$ and $g : \overline{\mathbb{B}^n} \to \overline{\mathbb{B}^n}$ defined as
	\[
	g(x)=\left\{
	\begin{array}{ll}
	|x| \hat{f} \big(\frac{x}{|x|}\big),\, \, x \neq 0\\
	0  \qquad \qquad x=0
	\end{array}
	\right.
	\]
	and $\hat{f} : \s^{n-1} \to \s^{n-1}$ is the representation of $f$, i.e., $\hat{f} = \psi|_{\doo B_1} \circ f \circ (\sigma_N|_{\doo B_2})^{-1}$.
	\[
	\begin{tikzcd}
	\doo B_2  \arrow[r,"f"] & \doo B_1 \arrow[d,"\psi|_{\doo B_1}"] \\
	\s^{n-1} \arrow[u,"(\sigma_N|_{\doo B_2})^{-1}"] \arrow[r,"\hat{f}"] & \s^{n-1}
	\end{tikzcd}
	\]
    The map $Q :  M' \sqcup (\s^n)' \to M$ continous surjective map, which is also a quotient map, since it's takes saturated open subsets to open subsets. It can be verify directly that $Q$ makes same identification as $q$. Therefore by uniqueness of quotient topology $M\#\s^n \approx M$. 
\end{proof}

\subsection*{Chapter 7 Homotopy and the Fundamental Group}

\begin{prop}[Exercise 7.42 \textbf{Equivalent Definition of Contractible Space}]
	Show that the following are equivalent:
	\begin{enumerate}[nolistsep]
		\item[(a)] $X$ is contractible.
		\item [(b)] $X$ is homotopy equivalence to a one-point space.
		\item [(c)] Each point of $X$ is a deformation retract of $X$.
	\end{enumerate}
\end{prop}
\begin{proof}
	We will show $(a) \Rightarrow (b)$, $(b) \Rightarrow (c)$, and $(c) \Rightarrow (a)$. 
	
	For $(a) \Rightarrow (b)$, suppose $X$ is contractible. That is, $\Id_X$ is homotopic with a constant map $f : X \to X$ defined by $f(x) = p$, for any $x\in X$. Define a map $g : X \to \{p\}$ as $g(x) = p$, for all $x\in X$. This map is continous by characteristic property of subspace topology $\{p\} \subhim X$. Let $\iota_p : \{p\} \to X$ be the inclusion map. Since $\iota_p \circ g : X \to X$ is just the map $f$, then by hypothesis we have $\iota_p \circ g = f \simeq \Id_X$, and $g \circ \iota_p = \Id_{\{p\}}$. This prove $(a) \Rightarrow (b)$. 
	
	For $(b) \Rightarrow (c)$, suppose $(b)$ holds. Let $f : X \to \{*\}$ be the homotopy equivalence and $g : \{*\} \to X$, defined as $g(*) = p$ for some $p \in X$, be the homotopy inverse for $f$. By hypothesis, the composition $h=g \circ f : X \to X$ is a constant map homotopic to $\Id_X$. Let $H  : X \times I \to X$ be the homotopy. Note that for a fix $x \in X$, $H(x,\cdot) : I \to X$ is a path from $x$ to $p$. So $X$ is path-connected. Choose any $q \in X$, we want to show that $q$ is a deformation retract of $X$. That is we want to show that $\iota_q \circ r \simeq \Id_X$, where $r : X \to \{q\}$ is a constant map (retraction) and $\iota_q : \{q\} \to X$ is the inclusion map. We will show that $h \simeq \iota_q \circ r$. So $h \simeq \Id_X$ implies $\iota_q \circ r \simeq \Id_X$. Note that $h = g \circ f$ and $k = \iota_q \circ r$ are both constant maps to point $p$ and $q$ (resp.). Since $X$ is path-connected, we have a path $\alpha : I \to X$ from $\alpha(0) = p$ to $\alpha(1) =q$. Define a map $F : X \times I \to X$ by $F(x,t) = \alpha(t)$. So $F(x,0) = \alpha(0) = p = h(x)$, and $F(x,1) = \alpha(1) = q = k(x)$. Therefore $h \simeq k$, and the proof is complete.
	
	For $(c) \Rightarrow (a)$, suppose $(c)$ holds. Choose any point $p \in X$, we have a retraction $r : X \to \{p\}$ with $\iota_p \circ r \simeq \Id_X$. But $\iota_p \circ r : X \to X$ is a constant map $(\iota_p \circ r)(x) = p$. So identity map $\Id_X$ homotopic to constant map $\iota_p \circ r$. This prove $(c) \Rightarrow (a)$. 
\end{proof}

\subsection*{Chapter 9 Some Group Theory}
\begin{prop}[Exercise 9.15 \textbf{Properties of Free Abelian Groups}]
	Let $S$ be a nonempty set. 
	\begin{itemize}[nolistsep]
		\item[(a)] Characteristic Property : Given any abelian group $H$ and any map $\varphi : S \to H$, there exists a unique homomorphism $\Phi : \Z S \to H$ extending $\varphi$.
		\[
		\begin{tikzcd}
		\Z S \arrow[dr,"\Phi"] & \\
		S \arrow[u,"\iota"] \arrow[r,"\varphi",swap] & H.
		\end{tikzcd}
		\]
		\item[(b)] The free abelian group $\Z \{\sigma_1 ,\dots ,\sigma_n \}$ on a finite set is isomorphisc to $\Zn$ via the map $(k_1,\dots,k_n) \mapsto  k_1 \sigma_1 + \cdots + k_n \sigma_n$.
	\end{itemize}
\end{prop}
\begin{proof}
	The natural injective map $\iota : S \to \Z S$ enable us to identify $S$ as a subset of $\Z S$. Define a map $\Phi : \Z S \to H$ by 
	$$
	\Phi (\sum_{i=1}^k n_i \sigma_i) = \sum_{i=1}^{k} n_i \varphi(\sigma_i) 
	$$
	for any element $ \sum_{i=1}^k n_i \sigma_i \in \Z S$. It is easy to see that this map is a homomorphism and since $\Phi(\sigma) = \varphi(\sigma)$ for any $\sigma \in S \subhim \Z S$, then $\Phi$ extending $\varphi$. Since this map is completely determined by the values $\varphi(\sigma), \forall \sigma \in S$, then by definition $\Phi$ is unique. This proves (a). Part (b) is obvious.  
\end{proof}

\begin{prop}[Exercise 9.16] 
	Prove that for any set $S$, the identity map of $S$ induces an isomorphism between the free abelian group on $S$ and the direct sum of infinite cyclic groups generated by elements of $S$ : $\Z S \cong \bigoplus_{\sigma \in S} \Z \{\sigma\}$.  
 \end{prop}
\begin{proof}
	If we have a family of abelian groups $(G_{\alpha})_{\alpha \in A}$ their \textbf{direct sum} $\bigoplus_{\alpha \in A} G_{\alpha}$ defined as a subgroup of the direct product $\Pi_{\alpha \in A} G_{\alpha}$ consisting of those elements $(g_{\alpha})_{\alpha \in A}$ such that $g_{\alpha}$ is the identity element in $G_{\alpha}$ for all but finitely many $\alpha$. 
	
	In our case, we have $(\Z \{\sigma\})_{\sigma \in S}$ where $\Z\{\sigma\}$ is the free abelian group on singleton $\{\sigma\}$, which by definition 
	$$
	\Z \{\sigma\} = \{n\sigma : \{\sigma\} \to \Z \text{ defined by }(n\sigma)(\sigma) = n \,|\, \forall n \in \Z \}.
	$$ 
	For each $\sigma \in S$, we identify $\sigma$ with $1\sigma \in \Z \{\sigma\}$. Since identity on $\Z \{\sigma\}$ is $0\sigma$, then the direct sum $\bigoplus_{\sigma \in S} \Z\{\sigma\}$ consists of elements of $\Pi_{\sigma \in S}\Z \{\sigma\}$ in form $(n_i\sigma_i)$, interpret as elements $(n_{\sigma} \sigma)_{\sigma \in S}$ with $n_{\sigma} = 0$ for all $\sigma \in S$ except for $\{\sigma_1,\dots,\sigma_k\}$ which valued $n_{\sigma_i} = n_i$. Define a map $\Phi : \Z S \to \bigoplus_{\sigma \in S} \Z\{\sigma\}$ as
	$$
	\Phi (\sum_{i=1}^{k} n_i \sigma_i) = (n_i \sigma_i).
	$$
	By definition of $\bigoplus_{\sigma \in S} Z\{\sigma\}$ and $\Z S$, the map is surjective. Since $\Phi (\sum_{i=1}^{k} n_i \sigma_i) = (0 \sigma)_{\sigma \in S}$ implies that $n_i = 0$, then $\Phi$ also injective. Its easy to see that the map $\Phi$ is also a homomorphism. Therefore $\Z S \cong \bigoplus_{\sigma \in S} \Z \{\sigma\}$.
	\[
	\begin{tikzcd}
	\Z S \arrow[dr,"\Phi"] & \\
	S \arrow[u,"\iota"] \arrow[r,"\varphi",swap] & \bigoplus_{\sigma \in S} \Z \{\sigma\}
	\end{tikzcd}
	\]
\end{proof}

\begin{prop}[Exercise 9.17]
	\begin{itemize}[nolistsep]
		\item [(a)] Show that an abelian group is free abelian $\Leftrightarrow$ is has a basis.
		\item [(b)] Show that any two free abelian groups whose bases have the same cardinality are isomorphic.
	\end{itemize}
\end{prop}
\begin{proof}
	Let $G$ be an abelian group. First, suppose that $G$ is free abelian. That is there exists $S\subhim G$ such that the induced map $\Phi : \Z S \to G$ is an isomorphism. We claim that $S$ is a basis subset for $G$. Let $n_1\sigma_1 + \cdots+ n_k \sigma_k$ be a linear combination of elements of $S$ such that $\sum_{i=1}^{k}n_i\sigma_i = 0 \in G$. By $\Phi$, we have the corresponding formal linear combination in $\Z S$ (also denoted by the same symbol) 
	$$
	0=\sum_{i=1}^{k}n_i\sigma_i : S \to \Z.
	$$
	By definition of $\Z S$, for any $\sigma_j \in S$, we have 
	$$ 
	\sum_{i=1}^{k}n_i\sigma_i(\sigma_j) = 0(\sigma_j) \implies n_j = 0.
	$$
	Therefore $S$ is a linear independent subset. The subset $S$ is also generate $G$ since any $g \in G$, we have unique $\sum_{i=1}^{k}n_i\sigma_i \in \Z S$ such that $g=\Phi(\sum_{i=1}^{k}n_i\sigma_i) = \sum_{i=1}^{k}n_i\sigma_i$. This proves the first part.
	
	To show $\Leftarrow$, suppose that $G$ has a basis, that is a linear independent subset $S$ that generate $G$. By characteristic property, we have a unique homomorphism $\Phi : \Z S \to G$ extending the inclusion map $S \hookrightarrow G$. The injectivity and the surjectivity of $\Phi$ follows easily from the fact that $S$ is linearly independent and generates $G$ respectively. Therefore $\Z S \cong G$.
	
	For (b) let $G$ and $H$ are free abelian group whose bases have the same cardinality. Since $G \cong \Z S_1$ and $H \cong \Z S_2$, for some $S_1 \subhim G, S_2 \subhim H$, then it's enough to  show that $\Z S_1 \cong \Z S_2$. Consider the following diagrams
	\[
	\begin{tikzcd}
	\Z S_1 \arrow[dr,"\Phi_1"] & \\
	S_1 \arrow[u,"\iota_1"] \arrow[r,"\varphi_1",swap] & G.
	\end{tikzcd}
	\quad
	\begin{tikzcd}
	\Z S_2 \arrow[dr,"\Phi_2"] & \\
	S_2 \arrow[u,"\iota_2"] \arrow[r,"\varphi_2",swap] & H.
	\end{tikzcd}
    \quad
   \begin{tikzcd}
   \Z S_1 \arrow[dr,"\Phi"] & \\
   S_1 \arrow[u,"\iota_1"] \arrow[r,"\varphi",swap] & \Z S_2.
   \end{tikzcd}
	\]
   In the above diagram, we define $\varphi : S_1 \to \Z S_2$ as $\varphi(\sigma)=\iota_2 \circ \psi (\sigma)$ where $\psi : S_1 \to S_2$ is a bijective map. By characteristic property $\Phi : \Z S_1 \to \Z S_2$ is a homomorphism extending $\varphi$. It's certainly surjective, since any element $\sum_{i=1}^{k}n_1 \rho_i$, where $\rho_i \in S_2$, we have $\sum_{i=1}^{k}n_1 \sigma_i$, where $\psi(\sigma_1) = \rho_i$, such that
   $$
   \Phi (\sum_{i=1}^{k}n_1 \sigma_i) = \sum_{i=1}^{k} n_i \varphi(\sigma_i) = \sum_{i=1}^{k} n_i \iota_2(\rho_i) = \sum_{i=1}^{k} n_i \rho_i. 
   $$ 
   And for any element $\sum_{i=1}^{k} n_i \sigma_i$ such that $0=\Phi(\sum_{i=1}^{k} n_i \sigma_i)=\sum_{i=1}^{k} n_i \rho_i \in \Z S_2$, we have $n_i=0$ for all $i=1,\dots, k$. Therefore $\Phi : \Z S_1 \to \Z S_2$ is an isomorphism. This proves (b). 
\end{proof}

\begin{prop}[Problem 9-4]
	Let $G_1,G_2,H_1,H_2$ be groups, and let $f_i:  G_i \to H_i$ be group homomorphisms for $i = 1,2$.
	\begin{enumerate}
		\item [(a)] Show that there exists a unique homomorphism $f_1 * f_2 : G_1 * G_2 \to H_1 * H_2$ such that the following diagram commutes for $i = 1,2$ :
		\[
		\begin{tikzcd}
		G_1 * G_2 \arrow[r,"f_1*f_2"] & H_1 * H_2 \\
		G_i \arrow[u,"\iota_i"] \arrow[r,"f_i",swap] & H_i \arrow[u,"\iota_i'",swap]
		\end{tikzcd}
		\]
		where $\iota_i : G_i \to G_1 * G_2$ and $\iota_i' : H_i \to H_1 * H_2$ are the canonical injections.
		
		\item [(b)] Let $S_1$ and $S_2$ be disjoint sets, and let $R_i$ be a subset of free group $F(S_i)$ for $i=1,2$. Prove that $\langle S_1 \cup S_2 \mid R_1 \cup R_2 \rangle$ is a presentation of the free product group $\langle S_1 \mid R_1 \rangle * \langle S_2 \mid R_2 \rangle$. 
	\end{enumerate}
\end{prop}
\begin{proof}
	Since the maps $\iota_i ' \circ f_i : G_i \to H_1 * H_2$ are homomorphisms, then (a) follows easily from characteristic property of free product.
	
	For (b), in order to make sense of $\metric{S_1 \cup S_2 \mid R_1 \cup R_2} = F(S_1 \cup S_2)/\overline{R_1 \cup R_2}$, we have to show that $F(S_1 \cup S_2) \cong F(S_1) * F(S_2)$ by an isomorphism $\Phi : F(S_1)*F(S_2) \to F(S_1 \cup S_2)$. After showing that, we can interpret $F(S_1 \cup S_2)/\overline{R_1 \cup R_2}$ as the quotient group 
	$$
	F(S_1 \cup S_2)/\Phi(\overline{\iota_1(R_1)\cup \iota_2(R_2)}) \cong F(S_1) * F(S_2)/\overline{\iota_1(R_1)\cup \iota_2(R_2)}
	$$ 
	where $\iota_i : F(S_i) \to F(S_1)*F(S_2)$ is the canonical injection. 
	
	\textbf{Step 1: Showing $\mathbf{F(S_1 \cup S_2) \cong F(S_1) * F(S_2)}$}. Consider the following digrams.
	\[
	\begin{tikzcd}
	F(S_1)*F(S_2) \arrow[dr,"\Phi"]& \\
	F(S_i) \arrow[u,"\iota_i"] \arrow[r,"\varphi",swap] & F(S_1 \cup S_2)
	\end{tikzcd}
    \quad
	\begin{tikzcd}
	F(S_1 \cup S_2) \arrow[dr,"\Psi"]& \\
	S_1 \cup S_2 \arrow[u,"i"] \arrow[r,"\psi",swap] & F(S_1)*F(S_2)
	\end{tikzcd} 
	\]
	where $\varphi : F(S_i) \to F(S_1 \cup S_2)$ is a homomorphism defined as $\varphi (\sigma_1^{n_1}\cdots \sigma_k^{n_k}) = \sigma_1^{n_1}\cdots \sigma_k^{n_k}$, and $\psi : S_1 \cup S_2 \to F(S_1)*F(S_2)$ defined as $\psi(\sigma) = \iota_i \circ j_i (\sigma)$ when $\sigma \in S_i$, with $j_i : S_i \to F(S_i)$ as the canonical injection. Note that $\psi$ is well-defined since $S_1 \cap S_2= \varnothing$ by hypothesis. Let $i_k : S_k \to S_1 \cup S_2$ (for k =1,2) be the injections, and note that 
	$$
	\varphi \circ j_k = i \circ i_k, \quad \psi \circ i_k = \iota_k \circ j_k. 
	$$
	These relation implies that $\Phi \circ \Psi \circ i = i$ and $\Psi \circ \Phi \circ \iota_i = \iota_i$. 
	\[
	\begin{tikzcd}
	F(S_1)*F(S_2) \arrow[dr,"\Id"]& \\
	F(S_i) \arrow[u,"\iota_i"] \arrow[r,"\iota_i",swap] & F(S_1)*F(S_2) 
	\end{tikzcd}
	\quad
	\begin{tikzcd}
	F(S_1 \cup S_2) \arrow[dr,"\Id"]& \\
	S_1 \cup S_2 \arrow[u,"i"] \arrow[r,"i",swap] & F(S_1 \cup S_2)
	\end{tikzcd} 
	\]
	By uniqueness of characteristic property, we conclude that $\Psi \circ \Phi = \Id$ and $\Phi \circ \Psi = \Id$. So $F(S_1 \cup S_2) = F(S_1) * F(S_2)$. By a simple result from group theory : that is if $f: G\to H$ is an isomorphism and $N \subseteq G $ is a normal subgroup, then $G/N \cong H/f(N)$. In our case, this implies that $	F(S_1 \cup S_2)/\Phi(\overline{\iota_1(R_1)\cup \iota_2(R_2)}) \cong F(S_1) * F(S_2)/\overline{\iota_1(R_1)\cup \iota_2(R_2)}$.  
	
	\textbf{Step 2: Showing $\mathbf{F(S_1) * F(S_2)/\overline{\iota_1(R_1)\cup \iota_2(R_2)} \cong F(S_1)/\overline{R_1} * F(S_2)/\overline{R_2}}$}. By (a), we have homomorphism $p=\pi_1 * \pi_2 : F(S_1)* F(S_2) \to F(S_1)/\overline{R_1} * F(S_2)/\overline{R_2}$ such that diagram below commute.
	\[
	\begin{tikzcd}
	 F(S_1)* F(S_2) \arrow[r,"p"] & F(S_1)/\overline{R_1} * F(S_2)/\overline{R_2} \\
	F(S_i) \arrow[u,"\iota_i"] \arrow[r,"\pi_i",swap] &F(S_i)/\overline{R_i} \arrow[u,"\iota_i'",swap]
	\end{tikzcd}
	\]
	Let $\pi : F(S_1)*F(S_2) \to F(S_1)*F(S_2)/\overline{\iota_1(R_1)\cup \iota_2(R_2)}$ be the projection onto the quotient group. Since $\iota_1(R_1) \cup \iota_2(R_2) \subhim \ker p$ and $\ker p$ is normal subgroup, then $\overline{\iota_1(R_1) \cup \iota_2(R_2)} \subhim \ker p$ we have unique homomorphism $\widetilde{p} : F(S_1)*F(S_2)/\overline{\iota_1(R_1)\cup \iota_2(R_2)} \to F(S_1)/\overline{R_1} * F(S_2)/\overline{R_2}$ satisfying $\widetilde{p} \circ \pi = p$.
	\[
	\begin{tikzcd}
	F(S_1)*F(S_2) \arrow[dr,"p"] \arrow[d,"\pi",swap]& \\
	F(S_1)*F(S_2)/\overline{\iota_1(R_1)\cup \iota_2(R_2)}  \arrow[r,"\widetilde{p}",swap] &  F(S_1)/\overline{R_1} * F(S_2)/\overline{R_2}
	\end{tikzcd}
	\]
	We will show that $\widetilde{p}$ is an isomorphism by construct its inverse $q$ satisfying $q \circ \iota_i' = q_i$,
	\[
	\begin{tikzcd}
	 F(S_1)/\overline{R_1} * F(S_2)/\overline{R_2} \arrow[dr,"q"] & \\
	F(S_i)/\overline{R_i} \arrow[u,"\iota_i'"] \arrow[r,"q_i",swap]& F(S_1)*F(S_2)/\overline{\iota_1(R_1)\cup \iota_2(R_2)} 
	\end{tikzcd}
	\]
	where $q_i : F(S_i)/\overline{R_i} \to F(S_1)*F(S_2)/\overline{\iota_1(R_1)\cup \iota_2(R_2)}$ defined as induced homomorphism of the map $\pi \circ \iota_i$ with $\ker (\pi \circ \iota_i)$, which satisfy $q_i \circ \pi_i = \pi \circ \iota_i$.
	\[
	\begin{tikzcd}
	F(S_i) \arrow[dd,"\pi_i",swap] \arrow[dr,"\iota_i"] & & \\
	& F(S_1) * F(S_2) \arrow[dr,"\pi"] & \\
	F(S_i)/\overline{R_i} \arrow[rr,"q_i",swap] & & F(S_1)*F(S_2)/\overline{\iota_1(R_1)\cup \iota_2(R_2)}
	\end{tikzcd}
	\]
	So, by characteristic property of free product, we have the unique homomorphism $q$ such that $q \circ \iota_i' = q_i$.
	\[
	\begin{tikzcd}
	F(S_1)* F(S_2) \arrow[r,"p"] \arrow[d,"\pi",swap] &  F(S_1)/\overline{R_1} * F(S_2)/\overline{R_2} \arrow[dl,"q"] \\
	F(S_1)*F(S_2)/\overline{\iota_1(R_1)\cup \iota_2(R_2)} \arrow[ur,"\widetilde{p}"] & F(S_i)/\overline{R_i} \arrow[l,"q_i"] \arrow[u,"\iota_i'",swap] \\
	F(S_1)* F(S_2) \arrow[u,"\pi"] & F(S_i) \arrow[l,"\iota_i"] \arrow[u,"\pi_i'"]
	\end{tikzcd}
	\]
	We have $\widetilde{p} \circ q \circ \iota_i' = \iota_i'$ and $q \circ \widetilde{p} \circ \pi = \pi$. So by uniqueness, we conclude that $\widetilde{p} \circ q = \Id$ and $q \circ \widetilde{p} = \Id$. 
\end{proof}

\begin{prop}[Problem 9-5]
	Let $S$ be a set, let $R$ and $R'$ be subsets of the free group $F(S)$, and let $\pi : F(S) \to \metric{S \mid R}$ be the projection onto the quotient group. Prove that $\metric{S \mid R \cup R'}$ is a presentation of the quotient group $\metric{S \mid R}/\overline{\pi(R')}$.
\end{prop}
\begin{proof}
	Let $\pi' : \metric{S\mid R} \to \metric{S \mid R]/\overline{\pi(R')}}$ and $p : F(S) \to \metric{S \mid R \cup R'}$ are the canonical projections. Then the map $\pi' \circ \pi : F(S) \to \metric{S \mid R}/\overline{\pi(R')}$ is a homomorphism with $\ker (\pi' \circ \pi) \supseteq \overline{R \cup R'}$. So we have a unique homomorphism $f : \metric{S \mid R \cup R'} \to   \metric{S \mid R}/\overline{\pi(R')}$ such that $f \circ p = \pi' \circ \pi$.
	\[
	\begin{tikzcd}
	F(S) \arrow[d,"p",swap] \arrow[dr,"\pi' \circ \pi"] & \\
	\metric{S \mid R \cup R'} \arrow[r,"f",swap] &  \metric{S \mid R}/\overline{\pi(R')} 
	\end{tikzcd}
	\]
	Consider also the following diagrams:
	\[
	\begin{tikzcd}
	F(S) \arrow[d,"\pi",swap] \arrow[dr,"p"] & \\
	F(S)/ \overline{R} \arrow[r,"g",swap] &  \metric{S \mid R \cup R'}
	\end{tikzcd}
	\quad 
	\begin{tikzcd}
	F(S)/ \overline{R} \arrow[d,"\pi'",swap] \arrow[dr,"g"] & \\
	\metric{S \mid R}/\overline{\pi(R')} \arrow[r,"f'",swap] &  \metric{S \mid R \cup R'}
	\end{tikzcd}
	\]
	Since $\overline{R} \subhim \overline{R \cup R'} = \ker p$ and $\overline{\pi(R')} \subhim \ker g$, we have unique homomorphisms $g$ and $f'$ such that $g  \circ \pi = p$ and $f' \circ \pi' = g$. These relations implies that $f \circ f' \circ \pi' = \pi'$ and $f' \circ f \circ p = p$. Therefore $f \circ f' = \Id$ and $f' \circ f = \Id$ by uniqueness.
\end{proof}

\subsection*{Chapter 10 The Seifert-Van Kampen Theorem}
\begin{prop}[Exercise 10.18 \cite{LeeTM}]
Suppose $G$ is a group.
\begin{itemize}[nolistsep]
    \item [(a)] Show that $[G,G]$ is a normal subgroup of $G$.
    \item [(b)] Show that $[G,G]$ is trivial if and only if $G$ is abelian.
    \item [(c)] Show that the quotient group $G/[G,G]$ is always abelian. 
\end{itemize}
\end{prop}
\begin{proof}
(a) The subset $[G,G] \subseteq G$ is a subgroup by definition. To show that $[G,G]$ is normal, let $g \in G$ be arbitrary. We need to show that $g[G,G]g^{-1} = [G,G]$. For any $c \in [G,G]$ we have
$$
gcg^{-1} = (gcg^{-1}c^{-1})c \in [G,G]\, \text{ by closure.}
$$
Hence $g[G,G]g^{-1} \subseteq [G,G]$. Same way for other direction, that is if $c \in [G,G]$ is arbitrary, then
$$
c = g(g^{-1}cg)g^{-1} \in g[G,G]g^{-1}
$$
since $g^{-1}cg = (g^{-1}cgc^{-1})c \in [G,G]$ by closure again. Thus we have $g[G,G]g^{-1} = [G,G]$. Since this is true for any $g \in G$ we conclude that $[G,G]$ is normal subgroup of $G$.

For (b), if $[G,G] = \{e_G\}$, the for any $g,h \in G$ we have $ghg^{-1}h^{-1} = e_G$ . So $G$ is abelian. For other direction, if $G$ is abelian, then $ghg^{-1}h^{-1} = gg^{-1}hh^{-1} = e_G$ for any $g,h \in G$. Therefore $[G,G] = \langle e_G \rangle = \{e_G\}$.

For (c), for any $g,h \in G$, by denoting the elements in the quotient as $[a] = a[G,G]$ we have
$$
[g][h]=[gh] = [(hgg^{-1}h^{-1})gh] = [hg][g^{-1}h^{-1}gh]= [hg][e_G] = [hg] =[h][g].  
$$
\end{proof}

\begin{prop}[Exercise 10.20 \cite{LeeTM}]
Let $G$ be a group. For any abelian group and any homomorphism $\varphi : G \to H$, there exists a unique homomorphism $\tilde{\varphi} : \text{Ab}(G) \to H$ such that the following diagram commute
\[
\begin{tikzcd}
	G \arrow[d] \arrow[dr,"\varphi"] & \\
	\text{Ab}(G) \arrow[r,"\tilde{\varphi}",swap] &  H
\end{tikzcd}
\]
\end{prop}
\begin{proof}
First we show that $[G,G] \subseteq \text{Ker }\varphi$. It is immidiate from the definition that any $c \in [G,G]$ is in the form $c = a_1a_2\cdots a_n$, where $a_i = g_ih_ig_i^{-1}h_i^{-1}$ for some $g_i,h_i \in G$. Since $H$ is abelian we have $\varphi(a_i) = \varphi(g_i)\varphi(g^{-1}_i)\varphi(h_i)\varphi(h_i^{-1}) = \varphi(e_G) = e_H$. Therefore
$$
\varphi(c) = \varphi(a_1) \cdots \varphi(a_n)  = e_H \cdots e_H = e_H.
$$
Hence $[G,G] \subseteq \text{Ker }\varphi$ and the conclusion follows by passing $\varphi$ to the quotient $\text{Ab}(G) = G/[G,G]$.
\end{proof}

\section{Smooth Manifold Theory}

\subsection{Chapter 1 (Smooth Manifolds)}
\begin{prop}
(\text{Example 1.28} \cite{LeeSM}). Matrices of Full Rank is an open submanifold of $M(m\times n,\mathbb{R})$.  
\end{prop}
\begin{proof}
Suppose $m <n$ and $M(m\times n, \mathbb{R})$ is the space of $m \times n$ matrices. Let $M_m(m\times n, \mathbb{R})$ denote the subset of $M(m\times n, \mathbb{R}$ consisting of matrices of rank $m$ (full rank matrices) and  $A \in M_m(m\times n, \mathbb{R})$ be arbitrary such matrix. By linear algebra we know that because $\text{rank} A = m$, $A$ has some nonsingular $m\times m$ submatrix. Choose one such submatrix of $A$. By this we have choose integers $1\leq j_1 < \cdots < j_m \leq n$ such that 
$$
\begin{pmatrix}
A^1_{j_1} & \cdots & A^1_{j_m} \\
\vdots & \ddots & \vdots \\
A^m_{j_1} & \cdots & A^m_{j_m}
\end{pmatrix}
$$
is the particular nonsingular submatrix that we choose. Define a projection map 
$$
P : M(m\times n, \mathbb{R}) \rightarrow M(m\times m, \mathbb{R})
$$
defined by sending each elements in $ M(m\times n, \mathbb{R})$ to its submatrix defined as above. Because $\det(P(A))\neq 0$, then by continuity of determinant map $\det  :  M(m\times m, \mathbb{R})  \rightarrow \mathbb{R}$, we have neighbourhood of $P(A)$, say $U$, consist of nonsingular $m \times m$ matrices. And therefore by continuity of projection map $P$, $P^{-1}(U) \cap M_m(m\times n, \mathbb{R}) $ is the desired neighbourhood of $A$ in $M_m(m\times n, \mathbb{R})$. Thus $M_m(m\times n, \mathbb{R})$ is an open subset of $M(m\times n, \mathbb{R})$ and therefore an $mn-$dimensional smooth manifold.
\end{proof}

\begin{prop}[Lemma 1.35 : \textbf{Smooth Manifold Chart Lemma}]
Let $M$ be a \textbf{set}, and suppose we are given a collection $\{U_{\alpha}\}$ of subsets of $M$ together with maps $\varphi_{\alpha} : U_{\alpha} \to \rn$, such that the following properties are satisfied :
\begin{enumerate}[nolistsep]
\item[(i)] For each $\alpha$, $\varphi_{\alpha}$ is a \textbf{bijection} between $U_{\alpha}$ and an \textbf{open subset} $\varphi_{\alpha}(U_{\alpha}) \subhim \rn$. 
\item[(ii)] For each $\alpha$ and $\beta$, the sets $\varphi_{\alpha}(U_{\alpha} \cap U_{\beta})$ and $\varphi_{\beta}(U_{\alpha} \cap U_{\beta})$ are \textbf{open} in $\rn$.
\item[(iii)] Whenever $\Ualpha \cap \Ubeta \neq \emptyset$, the map $\varphi_{\beta} \circ \varphi_{\alpha}^{-1} : \varphi_{\alfa}(\Ualpha \cap \Ubeta) \to \varphi_{\beta} (U_{\alfa}\cap \Ubeta)$ is \textbf{smooth}.
\item[(iv)] \textbf{Countably} many of the sets $\Ualpha$ cover $M$.  
\item[(v)] Whenever $p,q$ are distinct points in $M$, either there exists some $\Ualpha$ containing both $p$ and $q$ or there exists disjoint sets $U_{\alfa}$, $\Ubeta$ with $p \in \Ualpha$ and $q \in \Ubeta$.
\end{enumerate}
Then $M$ has a unique smooth manifold structure such that each $(\Ualpha, \varphi_{\alpha})$ is a smooth chart.
\end{prop}
\begin{proof}
First we \textbf{define the topology} of $M$ as the topology generated by all sets of the form $\varphi_{\alpha}^{-1}(V)$, with $V$ is an open subset in $\rn$. That is we have to show that $\mathcal{B} = \{\varphi_{\alpha}^{-1}(V) : \text{ for each }\alpha \text{ and for each open subset }V \text{ in }\rn \}$ satisfies the following conditions : 
\begin{enumerate}[nolistsep]
\item[(a)] $\bigcup_{B \in \mathcal{B}} B = M$.
\item[(b)] If $B_1,B_2 \in \mathcal{B}$ and $p \in B_1 \cap B_2$, then there exists an element $B_3 \in \mathcal{B}$ such that $p \in B_3 \subhim B_1 \cap B_2$.   
\end{enumerate}
To show (a), let $p\in M$. For every
$\alpha$, $\Ualpha \in \mathcal{B}$, since by (i) we have $\Ualpha = \varphi_{\alpha}^{-1}(\varphi_{\alpha}(\Ualpha))$. Since $\{U_{\alpha}\}$ is a cover for $M$ then there exists some element in $ \{U_{\alpha}\}$ containing $p$. Therefore $p \in \bigcup_{B \in \mathcal{B}} B$, and hence $M \subhim \bigcup_{B \in \mathcal{B}} B$. An by definition any $\varphi_{\alpha}^{-1}(V) \subhim \Ualpha \subhim M$, then (a) is holds. 

We devide (b) by two cases. First we consider elements of $\mathcal{B}$ which is both subsets of the same $\Ualpha$. That is take $B_1 = \VarphiAlpha^{-1}(V_1)$ and $B_2 = \VarphiAlpha^{-1}(V_2)$ for some $V_1,V_2$ are open subsets in $\rn$. Let $p \in \VarphiAlpha^{-1}(V_1) \cap \VarphiAlpha^{-1}(V_2)$. Just observe that $\VarphiAlpha^{-1}(V_1) = \VarphiAlpha^{-1}(V_1 \cap \VarphiAlpha(\Ualpha))$ and $\VarphiAlpha^{-1}(V_2) = \VarphiAlpha^{-1}(V_2 \cap \VarphiAlpha(\Ualpha))$. So
$$
B_3 = \VarphiAlpha^{-1}(V_1) \cap \VarphiAlpha^{-1}(V_2) = \VarphiAlpha^{-1}(V_1 \cap V_2) \in \mathcal{B}.
$$
With this choice of $B_3$, we have $p \in B_3= B_1\cap B_2$. The second case we consider $B_1 = \VarphiAlpha^{-1}(V)$ and $B_2=\VarphiBeta^{-1}(W)$, for an open subsets $V,W$ in $\rn$. Without loss generality we assume that $V \subhim \VarphiAlpha(\Ualpha)$ and $W \subhim \VarphiBeta(\Ubeta)$. Let $p \in B_1 \cap B_2$. We can find $B_3 \in \mathcal{B}$ such that $p \in B_3 \subhim B_1\cap B_2$ by the following way : We suspect that we can write $B_1\cap B_2$ in the form
\begin{align*}
\VarphiAlpha^{-1}(V) \cap \VarphiBeta^{-1}(W) &= \VarphiAlpha^{-1}(V \cap \cdots),
\end{align*}
where the blank is some open subset in $\rn$ that we dont know yet. To find this, we can apply both sides by $\VarphiAlpha$. So we have by (i),
\begin{align*}
\begin{split}
\VarphiAlpha (\VarphiAlpha^{-1}(V) \cap \VarphiBeta^{-1}(W)) &= \VarphiAlpha (\VarphiAlpha^{-1}(V \cap \cdots)) \\
V \cap (\VarphiAlpha \circ \VarphiBeta^{-1})(W) &= V \cap \cdots.
\end{split}
\end{align*} 
By this, we might suspect that $\cdots = (\VarphiAlpha \circ \VarphiBeta^{-1})(W)$. I.e., the following holds
\begin{align*}
\VarphiAlpha^{-1}(V) \cap \VarphiBeta^{-1}(W) &= \VarphiAlpha^{-1}(V \cap (\VarphiAlpha \circ \VarphiBeta^{-1})(W))\\ &= \VarphiAlpha^{-1}(V \cap (\VarphiBeta \circ \VarphiAlpha^{-1})^{-1}(W)).
\end{align*}
This is true as we can check it directly. Now, by (iii) $(\VarphiBeta \circ \VarphiAlpha^{-1})^{-1}(W)$ is open in $\VarphiAlpha(\Ualpha\cap \Ubeta)$, and by (ii) it is open in $\rn$. Therefore we have $B_3 = \VarphiAlpha^{-1}(V \cap (\VarphiBeta \circ \VarphiAlpha^{-1})^{-1}(W)) \in \mathcal{B}$ such that $p \in B_3 = B_1 \cap B_2$. 

Also by definition, each map $\VarphiAlpha$ is a homeomorphism onto its image, so $M$ is \textbf{locally Euclidean of dimension $n$}.

To show that \textbf{$M$ is Hausdorff}, let $p,q \in M$. By (v) we either have $\Ualpha$ such that $p,q \in \Ualpha$, or there are disjoint $\Ualpha$,$\Ubeta$ such that $p \in \Ualpha$ and $q \in \Ubeta$. If the former holds then we can always find disjoint open subset $U,V \subhim \Ualpha$ such that $p\in U$ and $q \in V$ by taking the preimage of disjoint open subsets in $\VarphiAlpha$ each contain $\VarphiAlpha(p)$ and $\VarphiAlpha(q)$. Therefore $M$  is Hausdorff. The \textbf{second-countability of $M$} follows from (iv) and Exercise A.22.(Look Proposition \ref{Exercise A.22} in this notes). Finally, (iii) guarantees that $\{(\Ualpha,\VarphiAlpha)\}$ \textbf{is a smooth atlas}. It is clear that the topology and the smooth structure are unique once it satisfies the conclusions of the lemma.
\end{proof}

\begin{prop}[Theorem 1.37 : \textbf{Topological Invariance of the Boundary} \cite{LeeSM}]
If $M$ is a topological manifold with boundary, then each point of $M$ is either a boundary point or an interior point, but not both. Thus $\doo M$ and Int $M$ are disjoint sets whose union is $M$.
\end{prop}

\begin{prop}[Exercise 1.39 \cite{LeeSM} : Proposition 1.38]
Let $M$ be topological manifold with boundary.
\begin{enumerate}[nolistsep]
\item[(a)] Int $M$ is an open subset of $M$ and a topological $n$-manifold without boundary. 
\item[(b)] $\doo M$ is a closed subset of $M$ and a topological $(n-1)$-manifold without boundary.
\item[(c)] $M$ is a topological manifold if and only if $\doo M = \emptyset$.
\item[(d)] If $n=0$, then $\doo M = \emptyset$ and $M$ is a $0$-manifold.
\end{enumerate}
\end{prop}
\begin{proof}
For (a), by definition Int $M \subseteq M$. For any $p \in \text{Int} M$, $p$ contain in $U$ where $U$ is the domain of some interior chart $(U,\varphi)$. Because $U$ is a domain of interior chart $(U,\varphi)$, $U$ is a neighbourhood of $p$ such that $U \subseteq \text{Int }M$. Hence Int $M$ is open in $M$. Take these charts for each $p \in \text{Int }M$ as charts for Int $M$. Endowed with subspace topology, Int $M$ is a topological $n$-manifold without boundary. 

To prove (b), we use Theorem 1.37. By this, Int $M$ and $\doo M$ are disjoint and $M = \text{Int} M \cup \doo M$. By (a), $\doo M = M \smallsetminus \text{Int} M$ is closed. By definition, for any $p \in \doo M$, $p$ in the domain of some boundary chart $(U,\varphi)$ such that $\varphi(p) \in \doo \hn$. That is
$$
\varphi : U \to \varphi(U) \subseteq \hn,\quad \text{with }\varphi(U) \cap \doo \hn \neq \emptyset.
$$
Since $\varphi(U) \cap \doo \hn$ is open in $\doo \hn$, by projecting the last coordinate $x^n=0$, its also open in $\mathbb{R}^{n-1}$. Taking the preimage we have
$$
\varphi^{-1}(\varphi(U) \cap \doo \hn) = \varphi^{-1} (\varphi(U)) \cap \varphi^{-1}(\doo\hn)= U \cap (U \cap \doo M) = U \cap \doo M
$$
which is open in $\doo M$. Therefore we have $U \cap \doo M$ as neighbourhood of $p$ in $\doo M$ with a map $\varphi|_{U \cap \doo M} : U \cap \doo M \to \varphi(U) \cap \doo \hn \subseteq \hn $. Since $\doo\hn \cong \mathbb{R}^{n-1}$, we have the desired charts for $\doo M$. Therefore $\doo M$ is topological $(n-1)$-manifold without boundary.

Part (c) and (d) is obvious, since by Theorem 1.37, $M=\text{Int }M$ if and only if $\doo M = \emptyset$. 
\end{proof}

\begin{prop}[Problem 1-12 \cite{LeeSM} \,\& \cite{Kosinski} \, (1.4)] \textbf{Proposition 1.45}
Suppose $M_1,\dots,M_k$ are smooth manifolds and $N$ is a smooth manifold with boundary. Then $M_1 \times \cdots \times M_k \times N$ is a smooth manifold with boundary, and $\doo (M_1 \times \cdots \times M_k \times N) = M_1 \times \cdots \times M_k \times \doo N$. 
\end{prop}
\begin{proof}
Its enough to show for the product manifold $M \times N$, where $M$ is a smooth $m$-manifold and $N$ is a smooth $n$-manifold with boundary. The secound countability and Hausdorff property of $M \times N$ immidiately follow from properties of product topology. Note that  $N = \text{Int}N \cup \doo N$ is a disjoint union, so we consider two types of points in $M \times N$
\begin{enumerate}[nolistsep]
\item[(1)] $p = (p_1,p_2) \in M \times \text{Int}N$,
\item[(2)] $p =(p_1,p_2) \in M \times \doo N$.
\end{enumerate}
For any $p \in M \times N$, let $(U_1,\varphi_1)$, $(U_2,\varphi_2)$ denote a smooth charts for $p_1,p_2$ respectively. Their product charts is $(U_1 \times U_2, \varphi_1 \times \varphi_2)$ defined such that
$$
\varphi_1 \times \varphi_2 : U_1 \times U_2 \to \varphi_1(U_1) \times \varphi_2(U_2).
$$
For points as in (1), $(U_2,\varphi_2)$ is an interior chart. Since $\varphi_1(U_1)$ is an open subset in  $\mathbb{R}^m$ and $\varphi_2(U_2)$ is an open subset $\mathbb{H}^n$ (hence in $\mathbb{R}^n$), and by inspection $\mathbb{R}^m \times \hn = \mathbb{H}^{m+n}$, then $\varphi_1(U_1) \times \varphi_2(U_2)$ open in $\mathbb{H}^{m+n}$, hence in $\mathbb{R}^{m+n}$. Therefore any $p$ in case (1) is an interior point of $M \times N$. 

For points in (2), the chart $(U_1,\varphi_2)$ is an boundary chart. So $\varphi_1(U_1)$ is an open subset in  $\mathbb{R}^m$ and $\varphi_2(U_2)$ is an open subset $\mathbb{H}^n$ with $\varphi_2(U_2) \cap \doo \hn \neq \emptyset$ and $\varphi_2(p_2) \in \doo \hn$. Hence  $\varphi_1(U_1) \times \varphi_2(U_2)$ open subset in $\mathbb{R}^m \times \hn = \mathbb{H}^{m+n}$ with $(\varphi_1(U_1) \times \varphi_2(U_2) \cap \doo \mathbb{H}^{m+n}) \neq \emptyset$ and $(\varphi_1 \times \varphi_2)(p)=(\varphi_1(p_1), \varphi_2(p_2)) = (x^1,\dots,x^m,y^1,\dots,y^{n-1},0) \in \doo \mathbb{H}^{m+n}$. So points in (2) are boundary points, i.e. $\doo (M \times N) = M \times \doo N$. 

It leave us to check the transition maps are smooth to conclude $M \times N$ is a smooth manifold. Let $p\in U \cap V$ where $U=U_1 \times U_2$ and $V = V_1 \times V_2$ are the domain of smooth charts $(U,\varphi_1 \times \varphi_2)$ and $(V,\psi_1 \times \psi_2)$. By inspection, their transition map obey
$$
(\psi_1 \times \psi_2) \circ (\varphi_1 \times \varphi_2)^{-1} = (\psi_1 \circ \varphi_1^{-1}) \times (\psi_2 \circ \varphi_2^{-1})
$$
therefore they are smoothly compatible.
\end{proof}
\begin{remark}
Note that if both manifolds has boundary, then their product cannot generally considered as manifold with boundary. This happen because the charts as constructed above is not necessarily an atlas since the product of half-space 
$$
\mathbb{H}^m \times \hn = \{ (x^1,\dots,x^m,y^1,\dots,y^n) \in \mathbb{R}^{m+n} : x^m \geq 0 \, \text{and} \, y^n \geq 0 )  \}
$$ is not itself a half-space (think of $[0,\infty) \times [0,\infty) \subset \mathbb{R}^2$).
\end{remark}

\begin{prop}[Theorem 1.46 \textbf{Smooth Invariance of the Boundary}]
	Suppose $M$ is a smooth manifold with boundary and $p \in M$. If there is a smooth chart $(U, \varphi)$ for $M$ such that $\varphi(U) \subhim \hn$ and $\varphi(p) \in \partial\mathbb{H}^n$, then the same is true for every smooth chart whose domain contain $p$.
\end{prop}
\begin{proof}
	(Sketch) First assume the contrary that $p$ contain in the domain of a smooth interior chart, and the find a contradiction by applying inverse function theorem to the transition map between interior and boundary charts. 
\end{proof}

\begin{remark}
	This implies that $M = \Inter M \sqcup \doo M$, that is a manifold is a disjoint union of its interior and boundary points (a substitute of Theorem 1.37 Topological Invariance of the Boundary).
\end{remark}

\subsection{Chapter 2 (Smooth Maps)}

\begin{prop}[Exercise 2.19 \cite{LeeSM} : \textbf{Theorem 2.18 Diffeomorphism Invariance of the Boundary}]
Suppose that $M$ and $N$ are smooth manifold with boundary and $F : M \to N$ is a diffeomorphism. Then $F(\partial M) = \partial N$, and $F$ restrict to a diffeomorphism from Int $M$ to Int $N$.
\end{prop}
\begin{proof}
	To show $F(\doo M) = \doo N$, first  we need to show that $F(\doo M) \subhim \doo N$. Let $p \in \doo M$ be arbirtary. Choose a boundary chart $(U,\varphi)$ for $p$. Since $F$ is a diffeomorphism, $V=F(U)$ is an open subset of $N$ contain $F(p)$ and $(V,\varphi \circ F^{-1}|_V)$ is a boundary chart for $F(p)$. By Theorem 1.46 (Smooth Invariance of the Boundary), $F(p) \in \doo N$. So $F(\doo M) \subhim \doo N$. By the same argument, we have $F^{-1}(\doo N) \subhim \doo M$, so $\doo N \subhim F(\doo M)$.
	
	To show $F(\text{Int }M) = \text{Int }N$, first we need to show that $F(\text{Int }M) \subhim \text{Int }N)$. Let $p \in \text{Int }M$ be arbitrary. As before, choose a interior chart $(U,\varphi)$ whose domain contain $p$. Then $(F(U),\varphi \circ F^{-1}|_{F(U)})$ is a boundary chart for $F(p)$. So $F(p) \in \text{Int }N$. By the same argument, we have $F^{-1}(\text{Int }N) \subhim \text{Int }M$, so $\Inter N \subhim F(\Inter M)$.  
	
	And finally, by Proposition 2.15(c), the map $F|_{\Inter M} : \Inter M \to \Inter N$ is a diffeomorphism (From Exercise 1.44). 
\end{proof}
\begin{remark}
	(1) Used in Proposition 16.3. (2) We can also prove $F(\Inter M) = \Inter N$ by result $F(\doo M) = \doo N$. That is, since interior point and boundary point is disjoint (by Prop.1.46), then $F(\Inter M) = F(M \smallsetminus \doo M) = N \smallsetminus \doo N = \Inter N$.
\end{remark}

\begin{prop}[Lemma 2.26 \textbf{Extension Lemma for Smooth Function on Closed Subset}] \label{Extend f from A}
Suppose $M$ is a smooth manifold with or withour boundary, $A \subseteq M$ is a closed subset, and $f : A \to \mathbb{R}^k$ is a smooth function. For any open subset $U$ containing $A$, there exists a smooth function $\tilde{f} : M \to \mathbb{R}^k$ such that $\tilde{f}|_A = f$ and supp $\tilde{f} \subseteq U$.
\end{prop}
\begin{proof}
By definition of smoothness of $f$ on $A$, for each $p \in A$, there are a neighbourhood $W_p$ of $p$ and a smooth function $\tilde{f}_p : W_p \to \mathbb{R}^k$ that agrees with $f$ on $W_p \cap A$. Replace $W_p$ with $W_p \cap U$, we may assume that $W_p \subhim U$. The family of sets 
$$
\mathcal{U} = \{W_p : p \in A \} \cup \{ M\smallsetminus A \} 
$$ 
is an open cover for $M$. Let $\{\psi_p : p \in A \} \cup \{\psi_0\}$ be the smooth partition of unity subordinate to this open cover, with supp $\psi_p \subhim W_p$ and supp $\psi_0 \subhim M \smallsetminus A$.

For each $p \in A$, we can restrict $\psi_p : M \to \er$ to open subset $W_p$ to obtain smooth function 
$\psi_p|_{W_p} : W_p \to \er$. Because $\tilde{f}_p : W_p \to \rk$ is also a smooth function on $W_p$, then their multiplication $\psi_p|_{W_p} \tilde{f}_p : W_p \to \rk$ defined by
\begin{align*}
(\psi_p|_{W_p} \tilde{f}_p)(x) &:= (\psi_p|_{W_p})(x) \cdot (\tilde{f}_p)(x) \\ 
&= \psi_p(x) \cdot  \tilde{f}_p(x) \\
&= (\psi_p|_{W_p}(x) \tilde{f}_p^1(x), \dots, \psi_p|_{W_p}(x) \tilde{f}_p^k (x) ) \in \rk
\end{align*}
is also smooth $\rk$-valued function on $W_p$. Now for simplicity of notation, we denote the above multiplication by $\psi_p \tilde{f}_p : W_p \to \rk$. Note that supp $\psi_p \tilde{f}_p \subhim $ supp $\psi_p \cap$ supp $\tilde{f}_p$   $\subhim W_p$. By Gluing Lemma (Corollary 2.8) we can extend $\psi_p \tilde{f}_p$ to all $M$ by choosing an open cover for $M$ as $W_p \cup (M \smallsetminus \text{supp }\psi_p)$ and a zero function $\varphi : M \smallsetminus \text{supp }\psi_p \to \rk$, defined by $\varphi(x) = 0 \in \rk$. Because they are agree on the domain of intersection, we have
$$
\psi_p \tilde{f}_p : M \to \rk, \quad \text{supp }\psi_p \tilde{f}_p \subhim W_p.
$$

Finally we can define $\tilde{f} : M \to \rk$ by
$$
\tilde{f}(x) = \sum_{p \in A} (\psi_p \tilde{f}_p)(x),  \quad x\in M.
$$
$\bullet $ \textbf{$\tilde{f}$ is smooth function :} Because the collection of support $\{\text{supp } \psi_p\}$ is locally finite (by definition of partition of unity), this sum actually has only a finite number of nonzero terms in a neighbourhood of any point $p \in M$, so its restriction on that neighbourhood is a smooth function, and therefore by Proposition 2.6 $\tilde{f}$ is a smooth function on $M$. \newline
$\bullet $ \textbf{$\tilde{f}$ agrees with $f$ on $A$ :} If $x \in A$, then for each $p$ such that $\psi_p(x) \neq 0$, $(\psi_p\tilde{f}_p)(x) = \psi_p(x) \tilde{f}_p(x) = \psi_p(x) f(x)$. Also because $\psi_0(x) = 0$, therefore
$$
\tilde{f}(x) = \sum_{p\in A} \psi_p(x) f(x) = \Big(\psi_0(x) + \sum_{p \in A} \psi_p(x) \Big) f(x) = f(x),
$$
so $\tilde{f}$ indeed is an extension of $f$. \newline 
$\bullet $ \textbf{supp $\tilde{f} \subhim U$ : }It follows from Lemma 1.13(b) that 
$$
\text{supp } \tilde{f} \subhim \overline{\bigcup_{p \in A} \text{supp } \psi_p}  = \bigcup_{p \in A} \text{supp }\psi_p \subhim U. 
$$ 
The Lemma used in the equality, whereas the first subset relation coming from pointwise argument as folllows. Let $x$ be any point in $M$ such that $\tilde{f}(x) \neq 0$, that is
$$
\tilde{f}(x) = \sum_{p\in A} (\psi_p \tilde{f}_p)(x)  =  \sum_{p\in A} \psi_p(x) \tilde{f}_p(x) \neq 0.
$$
So at least there are one $p \in A$ such that $\psi_p(x) \neq 0$, which means that $x \in \text{supp }\psi_p$. This implies 
$$
\{x \in M :  \tilde{f}(x) \neq 0 \} \subseteq \bigcup_{p \in A} \text{supp } \psi_p \implies \text{supp }\tilde{f} \subseteq \overline{\bigcup_{p \in A} \text{supp } \psi_p}.
$$
The last relation $\bigcup_{p \in A} \text{supp }\psi_p \subhim U$ is clear since supp $\psi_p \subhim W_p \subhim U$.
\end{proof}

\subsection{Chapter 4 (Submersions, Immersions and  Embeddings)}

\begin{prop}
\text{(Proposition 4.1 \cite{LeeSM})}. Suppose $F : M \rightarrow N $ is a smooth map and $p \in M$. If $F$ has full rank at $p$, then there is a neighbourhood $U$ of $p$ such that $F$ has full rank there. I.e.,
\begin{itemize}
\item If $dF_p$ is surjective, then p has neighbourhood $U$ such that $F|_U$ is a submersion.
\item If $dF_p$ is injective, then p has neighbourhood $U$ such that $F|_U$ is a immersion.
\end{itemize}
\end{prop}
\begin{proof}
Choose smooth charts $(V,\varphi)$ at $p$ and $(W,\psi)$ at $F(p)$. Note that we have a continous map from $\hat{V}=\varphi(V)$ to space of matrices $M(m\times n,\mathbb{R})$,
$$
J: \hat{V} \rightarrow M(m\times n,\mathbb{R})
$$
defined by
$$
J : \hat{p} \mapsto \Bigg[\frac{\partial \hat{F}^i}{\partial x^j}(\hat{p})\Bigg] \in M(m\times n,\mathbb{R})
$$
By hypothesis $dF_p$ has full rank at $p$.This equally means that Jacobian matrix of $F$ in coordinates has full rank at $p$.  Because the set of $m\times n$ matrices of full rank is open subset of $M(m\times n, \mathbb{R})$, therefore by continuity of $J$, $\hat{p}$ has neighbourhood $\hat{U} \subset \hat{V}$ such that the Jacobian matrix is full rank there. Hence by continuity of $\varphi : V \rightarrow \hat{V}$, the Jacobian of $F$ has full rank in $\varphi^{-1}(\hat{U}) = U$ contain $p$. 
\end{proof}

\begin{prop}[Exercise 4.4 \cite{LeeSM}]
If $F : M \rightarrow N$ and $G : N \rightarrow O$ are both submersion (immersion) then $G \circ F : M \rightarrow O$ also a submersion (immersion).
\end{prop}
\begin{proof}
Let $p$ be any point in $M$. By chain rule, the differential of $G \circ F$ at $p$ is
$$
d(G \circ F)_p = dG_{F(p)} \circ dF_p : T_pM \rightarrow T_{G(F(p))}O.
$$
By definition $dF_p$ and $dG_{F(p)}$ are both surjective (injective). Because the composition of two surjective (injective) linear map is surjective (injective) then the conclusion follows. 
\end{proof}

\begin{prop}[Problem 4-1 \cite{LeeSM} : Related to Theorem 4.5]
Use the inclusion map $\mathbb{H}^n \hookrightarrow \mathbb{R}^n$ to show that Theorem 4.5 does not extend to the case which $M$ is a manifold with boundary. 
\end{prop}
\begin{proof}
Let $p \in \partial \hn$. From Lemma 3.11 the differential $d\iota_p$ is invertible. If inverse function theorem holds, we can find connected neighbourhoods $U \subset \hn$ of $p$ and $V \subset \rn$ of $\iota(p)=p$ such that $\iota|_{U} : U \map V$ is a diffeomorphism. Because this is an identity map on $U$, then $U=V$ which is not true because $U$ cannot open in $\rn$. 
\end{proof}
\begin{remark}
I tried to prove for the general case $F : M \map N$ but fail. So i guess maybe not all map (or all space $M$) this extension fail. What class of maps $F : M \to N$ or class of manifold which is the extension can be done ?
\end{remark}


\begin{prop}[Problem 4.2 \cite{LeeSM} : Related to Theorem 4.5]
Suppose that $M$ is a smooth manifold (without boundary) and $N$ ia a smooth manifold with boundary and let $F : M \rightarrow N$ be a smooth map. Show that if $p\in M$ is a point such that $dF_p$ is nonsingular, then $F(p)\in \text{Int} N$.
\end{prop}
\begin{proof}
Assume that $F(p) \in \partial N$. Choose a smooth chart $(U,\varphi)$ centered at $p$ and boundary chart $(V,\psi)$ centered at $F(p)$. Denote $\varphi(U)=\hat{U} \subset \mathbb{R}^n$ and $\psi(V)=\hat{V} \subset \mathbb{H}^n$ with $\hat{V} \cap \partial \mathbb{H}^n \neq \emptyset$, we have the following map
$$
\hat{F} = \psi \circ F \circ \varphi^{-1} : \hat{U} \rightarrow \hat{V}.
$$
By composing $\hat{F}$ with the inclusion maps $\iota : \hat{V} \hookrightarrow \mathbb{H}^n$ and $\iota' : \mathbb{H}^n \hookrightarrow \mathbb{R}^n$, we regarding the codomain $\hat{V}$ as subset of $\mathbb{R}^n$, 
$$
\iota' \circ \iota \circ \hat{F} : \hat{U} \rightarrow \mathbb{R}^n.
$$
By proposition 3.9 and lemma 3.11, then the differential of the inclusions, $d\iota_{\hat{F}(\hat{p})}$ and $d\iota'_{\hat{F}(\hat{p})}$ are isomorphisms. Therefore the differential
$$
d(\iota' \circ \iota \circ \hat{F})_{\hat{p}} = d\iota'_{\hat{F}(\hat{p})} \circ d\iota_{\hat{F}(\hat{p})} \circ d\hat{F}_{\hat{p}}
$$
is also isomorphism. By applying Inverse Function Theorem for map $\iota' \circ \iota \circ \hat{F}$, we obtain connected neighbourhoods $\hat{U}_0 \subset \hat{U} \subset \mathbb{R}^n$ of $\hat{p}$ and $\hat{V}_0 \subset \mathbb{R}^n$ of $\hat{F}(\hat{p})$ such that 
$$
(\iota' \circ \iota \circ \hat{F})|_{\hat{U}_0} : \hat{U}_0 \rightarrow \hat{V}_0
$$
is a diffeomorphism. Because by definition of smoothness $\hat{F}(\hat{U}) \subset \hat{V}$ and $\hat{U}_0 \subset \hat{U}$ then $(\iota' \circ \iota \circ \hat{F})(\hat{U}_0) \ni \hat{F}(\hat{p})$ is an open subset of $\mathbb{R}^n$ and must contained in $\hat{V}$ which is impossible because $\hat{V}$ is open subset in $\mathbb{H}^n$ and $\hat{F}(\hat{p}) \in \partial \mathbb{H}^n$.
\end{proof}

\begin{prop}[Exercise 4.7 \cite{LeeSM} : Related to Prop. 2.15]
Verify the following properties of local diffeomorphisms :
\begin{enumerate}[nolistsep]
\item[(a)] Every composition of local diffeomorphisms is a local diffeomorphism.
\item[(b)] Every finite product of local diffeomorphisms between smooth manifolds is a local diffeomorphism. 
\item[(c)] Every local diffeomorphism is a local homeomorphism and an open map.
\item[(d)] The restriction of a local diffeomorphism to an open submanifold with or without boundary is a local diffeomorphism.
\item[(e)] Every diffeomorphism is a local diffeomorphism.
\item[(f)] Every bijective local diffeomorphism is a diffeomorphism.
\item[(g)] A map between smooth manifolds with or without boundary is a local diffeomorphism if and only if in a neighbourhood of each point of its domain, it has a coordinate representation that is a local diffeomorphism. 
\end{enumerate}
\end{prop}

\begin{proof}
Part (a)-(d) is direct consequence of Proposition 2.15. For (a), suppose that $F: M \to N$ and $G : N \to P$ are local diffeomorphisms and $p \in M$. By hypothesis, there are neighbourhoods $U\subseteq M$ of $p$ such that $F(U)$ is open in $N$, and $V\subseteq N$ of $F(p)$, such that $G(V)$ is open in $P$ where the restriction maps $F|_{U} : U \to F(U)$ and $G|_{V} : V \to G(V)$ are diffeomorphisms. Because $F(U) \cap V$ is open, $W:=F^{-1}(F(U)\cap V) \subseteq U$ is an open subset in $M$ contain $p$. Now we have diffeomorphisms $F|_{W} : W \to F(W)$ and $G|_{F(W)} : F(W) \to G(F(W))$. The composition of diffeomorphisms is always a diffeomorphism. Therefore $(G \circ F)|_{W} : W \to G(F(W))$ is a diffeomorphism. Hence $G \circ F : M \to P$ is a local diffeomorphism.

For (b), suppose we have local diffeomorphisms $F_i : M_i \to N_i$ for $i=1,\dots,n$. Their product map 
$$
F := F_1 \times \cdots \times F_n : M_1 \times \cdots \times M_n \to N_1 \times \cdots \times N_n
$$
defined as $F(p_1,\dots,p_n)=(F_1(p_1),\dots,F_n(p_n))$. For any point $p=(p_1,\dots,p_n) \in  M_1 \times \cdots \times M_n $, there are neighbourhoods $U_i$ of $p_i$ such that $F_i(U_i)$ is open in $N_i$ and $F_i|_{U_i} : U_i \to F_i(U_i)$ are diffeomorphisms. The product $U:=U_1 \times \cdots \times U_n$ is a neighbourhood contain $p$ and $F(U)=F_1(U_1) \times \cdots \times F_n(U_n)$ open. Finite product of diffeomorphisms is a diffeomorphism. So $F|_U = (F_1 \times \cdots \times F_n)|_U : U_1 \times \cdots \times U_n \to F_1(U_1) \times \cdots \times F_n(U_n)$ is a diffeomorphism. 

For (c), by definition a local diffeomorphism is a diffeomorphism on some neighbourhood of each point. Because every diffeomorphism is a homeomorphism and an open map, the conlusion follows.

For (d), let $U \subset M$ be an open submanifold with or without boundary and $F : M \to N$ is a local diffeomorphism. Let $p$ be any point on $U$ and $U_0$ is a neighbourhood of $p$ such that $F(U_0)$ is open in $N$ and $G:=F|_{U_0} :U_0 \to F(U_0)$ diffeomorphism. The restriction of a diffeomorphism to an open submanifold with or witout boundary is also a diffeomorphism. So $G|_U = F|_{U_0 \cap U} : U_0 \cap U \to F(U_0 \cap U)$ is a diffeomorphism. Because $U_0\cap U$ is open subset contain $p$ and $F(U_0\cap U)$ is open. Hence $F|_U : U \to N$ is a local diffeomorphism.

Part (e) is obvious.

For (f), let $F : M \to N$ is a bijective local diffeomorphism. To show $F$ is diffeomorphism, we only need to verify that $F$ and $F^{-1}$ are smooth. By proposition 2.6 map $F : M \to N$ smooth if every point $p \in M$ has a neighbourhood $U$ such that $F|_U = F \circ \iota : U \to N$ is smooth. Because $F$ is a local diffeomorphism, we always has a neighbourhood of each point such that $F|_U : U \to F(U)$ diffeomorphism. The inclusion map $\iota : F(U) \hmap N$ is smooth. Hence the composition $F \circ \iota : U \to N$ is smooth. So $F : M \to N$ smooth. By similar manner $F^{-1} : N \to M$ is also smooth. This proves (f).

Part (g) is obvious.
\end{proof}

\begin{prop}[Exercise 4.9 \cite{LeeSM}]
Show that conclusion in Proposition 4.8 still hold if $N$ is allowed to be smooth manifold with boundary, but not if $M$ is.
\end{prop}
\begin{remark}
Used in Proposition 4.22 (d).
\end{remark}
\begin{proof}
Since Proposition 4.8 (b) follows from (a), its enough if we can show that (a) generalized or not. If $M$ manifold with boundary, then Inverse Function Theorem for Manifolds may fail for points in $\doo M$ (e.g. $\hn \hookrightarrow \rn$). So let assume $N$ is a manifold with boundary and $M$ is a manifold. Suppose $F : M \to N$ is a local diffeomorphism and let $p$ be any point in $M$. By hypothesis, there is neighbourhood $U$ of $p$ such that $F(U)$ open in $N$ and $F|_U : U \to F(U)$ diffeomorphism. The point $F(p)$ either in Int$N$ or $\doo N$ but not both. By Proposition 3.6 (d) still holds for either case. So $dF_p : T_p M \to T_{F(p)}N$ is an isomorphism. Hence $F : M \to N$ is both immersion and submersion. Conversely, if $F : M \to N$ is both smooth submersion and smooth immersion, then $dF_p$ is bijective for any $p \in M$. Then by Problem 4-2, $F(M) \subseteq \text{Int}N$. By regard $\text{Int}N$ as smooth manifold without boundary, then Inverse Function Theorem for Manifolds still apply. Hence $F : M \to \text{Int} N$ is a local diffeomorphism. Note that Proposition 4.8 (b) implies that the inlcusion map $\iota : \text{Int} N \hookrightarrow N$ is a local diffeomporphism. Hence by Proposition 4.6 (a), the map $F : M \to N$ is local diffeomorphism since it is equal to the composition of $F : M \to \text{Int}N$ with $\iota : \text{Int} N \hookrightarrow N$.
\end{proof}

\begin{prop}[Exercise 4.10 \cite{LeeSM}] Suppose $M,N$ and $P$ are smooth manifolds with or without boundary, and $F : M \to N$ is a local diffeomorphism. Prove the following :
\begin{enumerate}[nolistsep]
\item[(a)] If $G : P \to M$ is continous, then $G$ is smooth if and only if $F \circ G $ is smooth.
\item[(b)] If in addition $F$ is surjective and $G : N \to P$ is any map, then $G$ is smooth if and only if $G \circ F$ is smooth.
\end{enumerate}
\end{prop}
\begin{proof}
For (a), first assume that $G : P \to M$ is smooth. Because $F : M \to N$ is a local diffeomorphism, then at any $p \in M$ we can find a neighourhood $U $ of $p$ such that $F(U) \subseteq N$ is open and $F|_{U} : U \to F(U)$ is a diffeomorphism. Since $\iota : F(U) \hookrightarrow N$ is smooth, then $\iota \circ F|_U : U \to N$ is smooth. By this,  $F : M \to N$ is smooth by Proposition 2.6. Therefore the composition $F \circ G : P \to N$ is smooth. Conversely, suppose $F \circ G$ is smooth and let $p \in P$. By hypothesis, there are neighbourhood $U$ of $G(p) \in M$ such that $F(U)$ open and $F|_U : U \to F(U)$ diffeomorphism. Since $G : P \to M$ continous, $V = G^{-1}(U)$ is open. The maps $(F \circ G)|_V : V \to N$ and $\iota : U \hookrightarrow M$ are smooth. Hence $G|_V = (\iota \circ F|_U^{-1}) \circ (F \circ G)|_V : V \to M$ is smooth.

For (b), suppose $F : M \to N$ is a surjective local diffeomorphism. Since local diffeomorphism implies smoothness, then the right direction is obvious. Conversely, let $(G \circ F) : M \to P$ is a smooth map and $p$ be any point in $N$. Surjectivity of $F : M \to N$ implies that $F^{-1}(p) \neq \emptyset$. So take any $q \in F^{-1}(p)$ and a neighbourhood $U$ of $q$ with $V=F(U)$ open and $F|_U : U \to F(U)$ diffeomorphism. Then $G$ is smooth since $G|_V = (G \circ F) \circ F|_U^{-1} : V \to P$ is smooth. 
\end{proof}

\begin{prop}[Exercie 4.16 \cite{LeeSM}]
Show that composition of smooth embeddings is a smooth embedding.
\end{prop}
\begin{proof}
Let $F :M \to N$ and $G : N \to P$ are smooth embeddings. By exercise 4.4, $G \circ F : M \to P$ is smooth immersion. The surjectivity of $G \circ F : M \to G(F(M))$ is immidiate, since it is a map onto its image. The injectivity follow from injectivities of $F : M \to F(M)$ and $G : N \to G(N)$. Since $G \circ F : M \to P$ is continous, then its restriction to $G(F(M))$ is also continous. Since $F : M \to F(M)$ and $G : N \to G(N)$ are both open map, then so is $G \circ F : M \to G(F(M))$. 
\end{proof}

\begin{prop}[Exercise 4.24 \cite{LeeSM}]
Given an example of smooth embedding that is neither an open map nor a closed map.
\end{prop}
\begin{proof}
Consider inclusion map $\iota : [0,1) \to \mathbb{R}$. It is homeomorphism onto its image and a smooth immersion. But $[0,1)$ is open and closed in itself but not open nor closed in $\mathbb{R}$.
\end{proof}

\begin{prop}[Exercise 4.27 \cite{LeeSM}]
Give an example of a smooth map  that is topological submersion but not a smooth submersion.
\end{prop}
\begin{proof}
We want to find a topological submersion, that is a continous map $\pi : X \to Y$ which point in $X$ is in the image of a continous local section $\sigma : U \to X$. The map $\pi$ can fail to be smooth submersion if :
\begin{enumerate}[nolistsep]
\item[(1)]$\pi $ is not a smooth map, or
\item[(2)] the differential not surjective at a point in $X$.
\end{enumerate} 
So by (1) its clear that any non-smooth topological submersion is not a smooth submersion. The example for (2) is the following topological submersion : $\pi : \mathbb{R} \to \mathbb{R}$ defined as $\pi(x)=x^3$, with the (global) section (which is just its inverse) $\sigma : y \mapsto \sqrt[3]{y}$. (1) is hold because $x \to x^3$ is smooth. Whereas its clearly not surjective at $0$ so (2) is the failed criteria.
\end{proof}

\begin{prop}[Proposition 4.28 \cite{LeeSM} : \textbf{Properties of Smooth Immersion}]
Let $M$ and $N$ be smooth manifolds, and suppose $\pi : M \to N$ is a smooth submersion. Then $\pi : M \to N$ is an open map, and if it is surjective it is a quotient map.
\end{prop}
\begin{proof}
Let $U \subhim M$ be an open subset and $\pi(U)\subhim N$ be its image. Let $q$ be any point in $\pi(U)$ and $p$ be any point in $U$ such that $\pi(p) = q$. Since $\pi$ is a submersion, there is an open subset $V \subhim N$ and a smooth local section $\sigma : V \to M$ such that $p \in \sigma(V)$. So there are $x \in V$ such that $\sigma(x)=p$. But $\pi \circ \sigma = \text{Id}_V$. So $x = \pi (\sigma(x)) = \pi(p) = q$, or $\sigma(q) = p \in U$. Since $\sigma$ smooth, $\sigma^{-1}(U)$ is open in $V$, hence in $N$. For any $y \in \sigma^{-1}(U)$, $y = \pi (\sigma(y)) \in \pi(U)$. Hence $\sigma^{-1}(U) \subhim \pi(U)$ is an open subset contain $q$. Therefore $\pi(U)$ open.

Since $\pi$ continous open map, then $N$ has the quotient topology determined by $\pi$. If in addition $\pi$ is surjective, then $\pi$ is a quotient map.
\end{proof}

\begin{prop}[Problem 4.6 \cite{LeeSM}]
Lem $M$ be a compact smooth manifold. Show that there is no smooth submersion $F : M \to \rk$ for any $k>0$.
\end{prop}
\begin{proof}
Suppose that we have a smooth submersion $F : M \to \rk$. By Proposition 4.28, $F$ is an open map. Therefore $F(M)$ is a compact open subset of $\rk$. But since $\rk$ is Hausdorff, $F(M)$ must be closed. Furthermore, $\rk$ is connected so $F(M) = \rk$ which is impossible since $\rk$ is noncompact.
\end{proof}

\subsection{Chapter 5 (Submanifolds)}

\begin{prop}[Exercise 5.36 \cite{LeeSM} : Prove Proposition 5.35 \textbf{General Characterization of \boldmath$T_pS$ using Curves} ]
Suppose $M$ is a smooth manifold with or without boundary, $S \subhim M$ is an immersed or embedded submanifold. and $p \in S$. A vector $v \in T_pM$ is in $T_pS$ if and only if there is a smooth curve $\gamma : J \to M$ whose image is contained in $S$, and which is also smooth as map onto $S$, such that $0 \in J$, $\gamma(0) = p$, and $\gamma'(0)=v$. 
\end{prop}
\begin{proof}
Let $v \in T_pM$ and $\iota : S \to M$ be the inclusion map of the submanifold $S$. If $v$ is in $T_pS \subhim T_pM$ then there is a vector $v_s \in T_pS$ such that $d\iota_p (v_s) = v$. By Proposition 3.23, there is a smooth curve $\gamma_s : J \to S$ such that $\gamma_s(0)=p$ and $\gamma_s'(0) = v_s$. Therefore we have a smooth curve
$\gamma := \iota \circ \gamma_s : J \to M$ with $\gamma(0) = \iota(\gamma_s(0)) = \iota(p) = p$ and 
$$
\gamma'(0) = d\iota_p (\gamma_s'(0)) = d\iota(v_s) = v.
$$
For the converse, let $\gamma : J \to M$ be a smooth curve contain in $S$ and also smooth as a map into $S$, such that $\gamma(0)=p$ and $\gamma'(0) = v \in T_pM$. Denote the restriction of this map to $S\subhim M$ by $\gamma_s : J \to S$ defined by $\gamma_s(t) = \gamma(t)$. As a smooth curve in $S$, the velocity vector $\gamma_s'(0)$ is in $T_pS$. Since $\gamma = \iota \circ \gamma_s :J \to M$, then
$$
\gamma'(0) = d\iota_p(\gamma_s'(0)) = v.
$$ 
Hence $v$ is in the image of $T_pS$ under $d\iota_p : T_pS \to T_pM$.
\end{proof}

\begin{prop}[Proposition 5.37 : \textbf{Another Characterization of} \boldmath$T_pS$, \textbf{Embedded Case}]
Suppose $M$ is a smooth manifold, $S \subhim M$ is an embedded submanifold, and $p \in S$. As a subspace of $T_pM$, the tangent space $T_pS$ is characterized by
$$
T_pS = \{ v \in T_pM : vf=0 \text{ whenever } f \in \CM \text{ and } f|_S=0  \}
$$
\end{prop}
\begin{proof}
First, suppose that $v \in d\iota_p(T_pS) \subhim T_pM$ and $f \in \CM$ such that $f|_S=0$. Then by definition there are $w\in T_pS$ where $v = d\iota_p(w)$. This implies
$$
vf = d\iota_p(w)f = w(f \circ \iota)=w(f|_S) =0.
$$
For the converse, we need the following simple fact about tangent space of an embedded submanifold. For any point $p \in S$ we have a slice coordinates $(x^1,\dots,x^n)$ for $S$ in some neighbourhood $U$ of $p$. This means that $S\cap U$ is the subset of $U$ where $x^{k+1} = \cdots=x^n = 0$, and $(x^1,\dots,x^k)$ is a coordinates for $U \cap S$. Because the inclusion map $\iota : S\cap U \hookrightarrow M$ has coordinate representation
$$
\iota(x^1,\dots,x^k) = (x^1\dots,x^k,0,\dots,0),
$$
the coordinate basis $\{\doo/\doo x^i|_p : i=1,\dots,k\}$ of the tangent space $T_pS$, will be mapped exactly to themself
$$
d\iota_p\bigg(\frac{\doo}{\doo x^i}\bigg|_p\bigg) = \frac{\doo}{\doo x^i}\bigg|_p, \quad 1 \leq i \leq k.
$$ 
Therefore in this coordinates, the linear subspace $d\iota_p(T_pS) \subhim T_pM$ spanned by $\{\doo/\doo x^i|_p : i=1,\dots,k\}$. By this, we know that a vector $v \in T_pM$ is in $T_pS$ if and only if  $v^{k+1}=\dots = v^n = 0$ (I think this is also a characterization of the tangent space of an embedded submanifold, only this one is coordinate dependent). 

So now we have to show that for $v \in T_pM$ satisfies $vf=0$ whenever $f$ vanish on $S$, the last $(n-k)$-components of $v$ in slice coordinates are all zero. To do this, choose a smooth bump function $\varphi : U \to \er$ such that supp $ \varphi \subhim U$ and $\varphi (p) = 1$. For any $j=k+1,\dots,n$, we have a function $f = \varphi x^j : U \to \er$ with supp $f \subhim U$. By Gluing Lemma, we have its extension to $M$, $f = \varphi x^j : M \to \er$ with $\varphi x^j \equiv 0$ on $M \smallsetminus \text{supp }f$. The function $f$ is vanish on $S$. So
\begin{align*}
0&=vf =\sum_{i=1}^{n} v^i \frac{\doo}{\doo x^i}\bigg|_p f = \sum_{i=1}^{n} v^i \frac{\doo (\varphi x^j)}{\doo x^i} (p)\\ &= \sum_{i=1}^{n} v^i \bigg( \varphi(p) \delta^j_i + x^j(p) \frac{\doo \varphi }{\doo x^i} (p) \bigg) \\
&=  \sum_{i=1}^{n} v^i ( 1 \cdot \delta^j_i + 0) = v^j
\end{align*}
Since this is true for all $j=k+1,\dots,n$, then $v \in T_pS$.
\end{proof}

\begin{prop}[Exercise 5.40.] Suppose $S \subhim M$ is a level set of a smooth map $\Phi : M \to N$ with constant rank. Show that $T_pS = \text{Ker } d\Phi_p$ for each $p \in S$.
\end{prop}
\begin{proof}
	By Theorem 5.12, the level set $S= \Phi^{-1}(c) \subhim M$ is an embedded submanifold with codimension $r$ (where $r$ is the rank of $\Phi$). Constant Rank Theorem tells us that at any point $p \in M$, there are smooth charts $(U,\varphi)$ centered at $p$ and $(V,\psi)$ centered at $\Phi(p)$ and
	$$
	\psi \circ \Phi \circ \varphi^{-1} (x^1,\dots,x^r,x^{r+1},\dots,x^m) = (x^1,\dots,x^r,0,\dots,0).
	$$
	Since $S = \Phi^{-1}(c)$, this implies that for any $q \in S\cap U$, we have 
	$$
	\psi \circ \Phi \circ \varphi^{-1} (x^1,\dots,x^r,x^{r+1},\dots,x^m) = (0,\dots,0) \implies x^1=\cdots=x^r = 0.
	$$
	So in this chart, 
	$$
	\varphi(S\cap U) = \{(x^1,\dots,x^m) \in \mathbb{R}^m : x^1=\cdots=x^r=0\}.
	$$
	Define a local defining map $\Phi':=\pi^r \circ \psi \circ  \Phi|_U : U \to \mathbb{R}^r$ for $S$ on each neighborhood of $p\in S$,  where $U$ and $V$ is the domain of the smooth charts above and $\pi^r$ is just the projection to the first $r$-factor on $\mathbb{R}^n$. For any $p \in  U$, we have
	$$
	\Phi'(p) = \pi^r \circ (\psi \circ \Phi \circ \varphi^{-1})  (x^1,\dots,x^m) = \pi^r(x^1,\dots,x^r,0,\dots,0) = (x^1,\dots,x^r).
	$$    
	So $\Phi' : U \to \rk$ is a submersion. In particular, for any $p \in S\cap U$, $\Phi'(p) = (0,\dots,0)$. Therefore $S\cap U = \Phi'^{-1}(0)$, and hence $S\cap U$ is a regular zero level set of $\Phi'$. By Proposition 5.38 (with codomain $N=\mathbb{R}^r$), $T_pS = \text{Ker }d\Phi'_p$. 
	
	Next, we have to show that $\text{Ker }d\Phi'_p = \text{Ker }d\Phi_p$, for any $p \in S \cap U$. Since 
	$$
	d\Phi'_p = d(\pi^r \circ \psi)_{\Phi(p)} \circ d\Phi_p : T_pU \to T_0\mathbb{R}^r, 
	$$
	then for any $v \in \text{Ker }d\Phi_p $, then $v \in \text{Ker }d\Phi'_p$. So $\text{Ker }d\Phi_p \subseteq \text{Ker }d\Phi'_p$. Since $d\Phi'_p$ is surjective,
	\begin{align*}
	\text{dim} (\text{Ker }d\Phi'_p) &= \text{dim}(T_pU) - \text{dim} (\text{Im }d\Phi'_p) \\
	&=  \text{dim}(T_pM) -  \text{dim}(\mathbb{R}^r) \\ 
	&= \big(\text{dim}(\text{Ker }d\Phi_p\big) + \text{dim}(\text{Im }d\Phi_p)) - r .
	\end{align*}
	In local charts $U$ and $V$ above, the representation of $\Phi$ is
	$$
	\psi \circ \Phi \circ \varphi^{-1} (x^1,\dots,x^m) = (x^1,\dots,x^r,0\dots,0).
	$$
	Hence $d\Phi_p(\doo_i|_p) = \doo_i|_{\Phi(p)}$ for $1 \leq i \leq r$ and $d\Phi_p(\doo_i|_p) = 0$ for $r+1 \leq i \leq n$. This means that $\text{dim}(\text{Im }d\Phi_p) = r$. Therefore $\text{dim} (\text{Ker }d\Phi'_p) = \text{dim} (\text{Ker }d\Phi_p)$.
\end{proof}

\begin{prop}[Exercise 5.42  : Prove Proposition 5.41.]
Suppose $M$ is a smooth $n$-dimensional manifold with boundary, $p \in \doo M$, and $(x^i)$ are any boundary coordinates defined on a neighbourhood of $p$. The \textbf{inward-pointing} vectors in $T_pM$ are precisely those with positive $x^n$-component, the \textbf{outward-pointing} ones are those with negative $x^n$-component, and the ones \textbf{tangent} to $\doo M$ are those with zero $x^n$-component. Thus, $T_pM$ is the disjoint union of $T_p\doo M$, the set of inward-pointing vectors, and the set of outward-pointing vectors, and $v \in T_pM$ is inward-pointing if and only if $-v$ is outward-pointing.
\end{prop}
\begin{proof}
Let $v \in T_pM$ be any inward-pointing vector. By definition, there exists a smooth curve $\gamma :[0,\varepsilon) \to M$ such that $\gamma(0)=p$ and $\gamma'(0) = v$. In boundary coordinate $(x^i)$, the representation of $\gamma$ is $\widehat{\gamma}(t) = (x^1(t),\dots,x^n(t))$, where $\widehat{\gamma}(0) = \widehat{p} = (x^1_p,\dots,x^{n-1}_p,0)$ and 
$$
\gamma'(0) = \frac{dx^i}{dt}(0) \frac{\doo}{\doo x^i}\bigg|_p=v = v^i\frac{\doo}{\doo x^i}\bigg|_p.
$$ 
Moreover, for any $t>0$, $x^n(t) > 0$. This means that
$$
v^n = \frac{dx^n}{dt}(0) = \lim_{h \to 0^+} \frac{x^n(0+h) - x^n(0)}{h} = \lim_{h\to 0^+} \frac{x^n(h)}{h} > 0 .
$$

If $v \in T_pM$ is an outward-pointing, then there exists smooth curve $\gamma : (-\varepsilon,0] \to M$ such that $\gamma(0) = p$ and $v = \gamma'(0)$. As before, in any boundary coordinate, $\widehat{\gamma}(t) = (x^1(t),\dots,x^n(t))$, where $\widehat{\gamma}(0) = \widehat{p} = (x^1_p,\dots,x^{n-1}_p,0)$, and $\gamma'(0) = \dot{x}^i \doo_i|_p = v^i \doo x^i|_p$. Only this time, for any $t<0$ $x^n(t)>0$. This implies
$$
v^n = \frac{dx^n}{dt}(0) = \lim_{h \to 0^-} \frac{x^n(0+h) - x^n(0)}{h} = \lim_{h \to 0^-} \frac{x^n(h)}{h} < 0.
$$

If $v \in T_p\doo M$, by Proposition 5.35, there exists a smooth curve $\gamma : (-\varepsilon,\varepsilon) \to M$ where $\gamma(-\varepsilon,\varepsilon) \subhim \doo M$, which is also a smooth map into $\doo M$, such that $\gamma(0) = p$ and $v = \gamma'(0)$. In any boundary chart $(x^i)$, the representation of $\gamma$ is $\widehat{\gamma}(t) = (x^1(t),\dots,x^{n-1}(t),0)$ for all $t$. Then it follows that $v^n =dx^n(0)/dt = 0$. The converse for all cases are already done in Proposition 3.23.
\end{proof}

\begin{prop}[Exercise 5.44]
Suppose $M$ is a smooth manifold with boundary, $f $ is a boundary defining function, and $p \in \doo M$. Show that a vector $v \in T_pM$ is inward-pointing if and only if $vf >0$, outward-pointing if and only if $vf<0$, and tangent to $\doo M$ if and only if $vf=0$.
\end{prop}
\begin{proof}
Let $p \in\doo M$ and $f : M \to [0,\infty)$ is the boundary defining function. Suppose $v \in T_pM$ be inward-pointing vector and $\gamma : [0,\varepsilon) \to M $ be the smooth curve such that $\gamma(0) = p$ and $\gamma'(0) = v$. Denote $\Id : [0,\infty) \to [0,\infty)$ as identity map, we have
\begin{align*}
vf &= v(\Id \circ f)=df_p(v)(\Id) = df_p \circ \gamma'(0) (\Id) \\ &= \frac{d}{dt}\bigg|_{t=0} (\Id \circ f \circ \gamma)(t) \\&= \frac{d}{dt}\bigg|_{t=0} ( f \circ \gamma)(t) \\&= \lim_{h \to 0^+} \frac{f(\gamma(0+h)) - f(\gamma(0))}{h} \\&= \lim_{h \to 0^+} \frac{f(\gamma(h)) - 0}{h} > 0 
\end{align*}
since $f(p)=0$ and $f(x)>0$ for all $x \in \text{Int }M$. The case for $v \in T_pM$ is outward-pointing and $v \in T_p\doo M$ are similar.  
\end{proof}


\subsection{Chapter 6 (Sard Theorem)}

\begin{prop}[Lemma 6.2]
Suppose $A \subhim \rn $ is a compact subset whose intersection with $\{c\} \times \mathbb{R}^{n-1}$ has $(n-1)$-dimensional measure zero for every $c \in \er$. Then $A$ has $n$-dimensional measure zero. 
\end{prop}
\begin{proof}

\end{proof}


\subsection{Chapter 7 (Lie Groups)}

\begin{prop}[Part of the proof in Theorem 7.21 \cite{LeeSM}]\label{[Part of the proof in Theorem 7.21}
Let $G$ a Lie group and $H\subseteq G$ is a Lie subgroup. Let $g \in \bar{H}$ and $(h_i)$ is a sequence in $H$ converging to $g$. Claim : $h_ig^{-1} \to e$.
\end{prop}
\begin{proof}
The map $f: G \times G \to G$ defined by $f(g,h)=gh^{-1}$ is a smooth map since $f$ just composition of smooth maps $\text{Id} \times i : G \times G \to G \times G$ defined as $(\text{Id} \times i) (g,h) = (\text{Id}(g),i(h)) = (g,h^{-1})$ with multiplication map $m : (g,h) \mapsto gh$. Since $f$ is also continous, then for any sequence $(p_i)$ in $G \times G$ such that $p_i \to p$, then $f(p_i) \to f(p)$. Since $h_i \to g$ hence the sequence $p_i = (h_i,g)$ in $G \times G$ converge to $p=(g,g) \in G \times G$. So by continuity $f(p_i) = h_i g^{-1} \to f(p)=gg^{-1} = e$.
\end{proof}
\begin{remark}
This proof require the proof for the claim the product of smooth maps (like $\text{Id} \times i : G \times G \to G \times G$ as above) is again a smooth map by making use of proposition 2.12 \cite{LeeSM}. I'll add this later.
\end{remark}

\begin{prop}[Problem 7-6 \cite{LeeSM}]
Suppose $G$ is a Lie group and $U$ is a neighbourhood of the identity. Show that there exist a neighbourhood $V$ of the identity such that $V\subseteq U$ and $gh^{-1} \in U$ whenever $g,h \in V$.
\end{prop}
\begin{proof}
By proposition \ref{[Part of the proof in Theorem 7.21}, the map $f : G \times G \to G$ defined as $f(g,h) = gh^{-1}$ is a smooth map. The neighbourhood $U \subset G$ is open so $W = f^{-1}(U)$ is open in $G \times G$. By Because $(e,e) \in W$, by basis criterion (\cite{LeeTM} \, exercise 2.40), there are open subsets $W_1,W_2 \subseteq G$ containing $e\in G$ such that $W_1\times W_2 \subseteq W$. Choose $V = (W_1 \cap W_2) \cap U$. Because $V \times V \subseteq W=f^{-1}(U)$ then $f(V \times V) \subseteq U$.
\end{proof}
\begin{remark}
This proof (although it is elegant), it's not mine. I'm trying to ptove it but turn out that my solution is false. Maybe i doesn't come up with the first because i'm ignoring "the hint", that is the multiplication $gh^{-1} \in U$ which is probably a sign for us to consider the map $(g,h) \mapsto gh^{-1}$.
\end{remark}

\begin{prop}[Theorem 7.21 \cite{LeeSM}] Suppose $G$ is a Lie group and $H \subseteq G$ is a Lie subgroup. Then $H$ is closed if and only if it is embedded.
\end{prop}
\begin{proof}
Assume first that $H \subseteq G$ is embdedded. We must show that $\bar{H} \subseteq H$ to show that $H$ is closed. So let $g \in \bar{H}$ and by definition, there is a sequence $(h_i)$ in $H$ converging to $g$
$$
h_i \to g \in \bar{H}.
$$
By continuity of the map $f: G \times G \to G$ defined by $f(g_1,g_2) = g_1g_2^{-1}$, this $h_i \to g$ implies that the sequence $(h_i g^{-1})$ converging to the identity $e$ 
(by this, we has made the problems easier because now we can work on the neighbourhood of identity since is many nice result already known). So now we have a sequence $(h_ig^{-1})$ in $G$ converging to $e$
$$
h_ig^{-1} \to e.
$$
\textbf{The myterious move  (motivated by some "observation"):} Take a neighbourhood $U$ of $e$ as the domain of some slice chart. Let $W \subset U$ be a smaller neighbourhood of $e$ \textbf{such that} $\bar{W}\subset U$. By Problem 7-6, there is a neighbourhood $V$ of $e$ with the property that $g_1g_2^{-1} \in W$ for any $g_1,g_2 \in V$. The fact that the sequence $(h_ig^{-1})$ converging to $e \in U$ tells us that we can assume $(h_ig^{-1})$ is sequence in $V$ because we can discarding finitely many terms of of the sequence outside $V$.

This implies that (\textbf{this is the "observation"}) 
$$
h_ih_j^{-1} = (h_ig^{-1})(h_jg^{-1})^{-1} \in W, \quad \forall i,j.
$$
And by fixing $j$ and let $i \to \infty$, we have
$$
h_ih_j^{-1} \to gh_j^{-1}
$$
by continuity of multiplication map $(g_1,g_2) \mapsto g_1g_2$. The limit point $gh_j^{-1} \in W\subseteq \bar{W} \subseteq U$, since $e,h_jg^{-1} \in V$ and $e \cdot (h_jg^{-1})^{-1} = gh_j^{-1}$. Since $H$ is subgroup, then $h_ih_j^{-1} \in H$, hence $h_ih_j^{-1} \in H \cap U$. Since $H \cap U$ is a slice, then it is closed in $U$. This implies the limit point $gh_j^{-1}$ of the sequence $(h_ih_j^{-1})$ in closed subset $H \cap U$ contain in itself. That is $gh_j^{-1} \in H \cap U $. In particular, $gh_j^{-1} \in H$. Since $H$ subgroup, then
$$
(gh_j^{-1}) \cdot h_j = g \in H.
$$ 
Therefore $H$ is closed.
\end{proof}

\textbf{Representations (Notes for page 169)}
Is all Lie group comes in form of subgroup of $\text{GL}(n,\er)$ and $\text{GL}(n,\mathbb{C})$ ? This question lead us to the concept of \textit{representations}.

Suppose that $V$ is a finite-dimensional real or complex vector space. We know the following facts about GL$(V)$ (the set of all invertible linear transformation of $V$) :
\begin{enumerate}[nolistsep]
\item[(1)] GL$(V)$ is a vector space, \text{dim}\,$\text{GL}(V) = (\text{dim}(V))^2$, hence
\item[(2)] GL$(V)$ is a smooth manifold, with global chart obtained by choose a basis for $V$ and represent the elements as matrix (Example 1.24 and Example 1.29 \cite{LeeSM}), hence
\item[(3)] GL$(V)$ is a Lie group, which is Lie group isomorphic to GL$(n,\er)$ or GL$(n,\mathbb{C})$.
\end{enumerate}
By this we come up with the following notion. If $G$ is a Lie group, \textit{\textbf{a (finite-dimensional) representation of}} $\boldmath{G}$ is a Lie group homomorphism $$\rho : G \to \text{GL}(V)$$ for some vector space $V$. It is said to be \textit{\textbf{faithful}} if it is also injective. By Proposition 7.17, (if we have an injective Lie group homomorphism then image is a Lie subgroup and the map is a Lie group isomorphism onto the image), then for a faithful representation $\rho : G \to \text{GL}(V)$, the image $\rho(G) \subhim \text{GL}(V) \isomorphic \GL$ or $\text{GL}(n,\mathbb{C})$ is a Lie subgroup and $\rho : G \to \rho(G)$ is a Lie group isomorphism. Thus, a Lie group admits a faithful representation if and only if it is isomorphic to Lie subgroup of $\GL$ or $\text{GL}(n,\mathbb{C})$. But, not every Lie group admits such representation.

There is a relation between representations and group actions. Suppose $G$ is a Lie group and $V$
be a finite-dimensional vector space. An action $\theta : G \times V \to V$ is said to be \textit{\textbf{linear action}} if for each $g \in G$, the map $\theta_g : V \to V$ ( or $x \mapsto g \cdot x$) is a linear map.

We know the following relation :
\begin{enumerate}[nolistsep]
\item[$\triangleright$] For a representation $\rho : G \to \GLV$, we have a linear action defined by
$$
\theta (g,x) := \rho(g)x, \quad \text{or} \quad g \cdot x := \rho(g)x
$$ 
which is clearly smooth and linear.
\item[$\triangleright$] The converse is also true, as proved in the following proposition.
\end{enumerate}
\begin{prop}[Proposition 7.37]
Let $G$ be a Lie group and $V$ be a finite-dimensional vector space. A smooth left-action of $G$ on $V$ is linear if and only if it is the form $g\cdot x = \rho(g)x$ for some representation $\rho$ of $G$.
\end{prop}
\begin{proof}
To prove the converse, assume we have a linear action of $G$ on $V$. Denote this by $\theta : G \times V \to V$. By hypothesis, for each $g \in G$, $\theta_g : V \to V$ is a linear map. This map is invertible since $\theta_{g^{-1}}$ is its inverse. Therefore at each $g \in G$ we have $\theta_g \in \GLV$. So define $\rho : G \to \GLV$ as
$$
\rho : g \mapsto \rho(g) := \theta_g.
$$
This map is a Lie group homomorphism since
$$
\rho(g_1g_2) = \theta_{g_1g_2} = \theta_{g_1} \circ \theta_{g_2} = \rho(g_1) \circ \rho(g_2).
$$
To show that $\rho$ is a smooth map, choose any basis $\{E_1,\dots,E_n\}$ for $V$. Then we have a global coordinates for $\GLV$ defined by its matrix representation. For any $E_i$, 
$$
\rho(g) E_i = \rho^j_i(p) E_j.
$$
By projection map $\pi^i : V \to \er$ defined as $\pi^i(x^jE_j) = x^j$, we have
\begin{align*}
\rho^i_j(g) = \pi^i (\rho(g)E_j) = \pi^i (\theta_g (E_j)) = (\pi^i \circ \theta) (g,E_j),
\end{align*} 
or more precisely
$$
\rho^i_j = \pi^i \circ \theta|_{G \times \{E_j\}} : g \mapsto \rho^i_j(g) \in \er,
$$
is a composition of smooth maps so it is smooth. Since its components is smooth on a global chart, therefore $\rho$ is smooth on $G$. Hence $\rho : G \to \GLV$ defined above is a representation of $G$.
\end{proof}

\subsection{Chapter 8 (Vector Fields)}

We need the following fact before proving Problem 8-1. Let $\mathfrak{X}(M)$ be the set of all smooth vector fields on a smooth manifold with or withhout boundary $M$
\begin{enumerate}
\item[(a)] $(\mathfrak{X}(M), \er)$ is a vector space under pointwise addition and scalar multiplication
$$
(aX + b Y)_p = aX_p + b Y_p.
$$
The zero element of $(\mathfrak{X}(M), \er)$ is \textbf{zero vector field} $X_0 : M \to TM$ defined as $X_0(p) = 0 \in T_pM$.
\item[(b)] $(\mathfrak{X}(M), C^{\infty}(M))$ is a module with scalar multiplication defined by 
$$
(fX)_p = f(p)X_p.
$$
\end{enumerate}

\begin{prop}[Exercise 8.9 \cite{LeeSM} \, : Prove Proposition 8.8]
Proposition 8.8 Let $M$ be a smooth manifold with or without boundary.
\begin{enumerate}
\item[(a)] If $X$ and $Y$ are smooth vector fields on $M$ and $f,g \in \CM$, then $fX+gY$ is a smooth vector field. 
\item[(b)] $\mathfrak{X}(M)$ is a module over a ring $\CM$
\end{enumerate}
\end{prop}
\begin{proof}
For (a), the map $Z :=fX+gY : M \to TM$ is smooth if and only if its restriction to a neighbourhood of each point is smooth. By proposition 8.1, its enough to show that its components function with respect to any chart $(U,\varphi)$ of $p$ are smooth. The representation in any chart $(U,\varphi)$ is
$$
(x^1,\dots,x^n,Z^1(x),\dots,Z^n(x)),
$$
with $Z^i(x) = \hat{f}(x)X^i(x)+\hat{g}(x)Y^i(x)$ certainly are smooth functions, since $X^i, Y^i,\hat{f},\hat{g}$ are all smooth. 

For (b), $\CM$ is a ring :

All the properties follows from properties of field $\er$
\begin{enumerate}
\item[$\bullet $] Abelian group under addition, $\forall f,g \in \CM, \, f+g \in \CM$
\begin{enumerate}[nolistsep]
\item Associative 
$$((f+g)+h)(p) = (f+g)(p)+ h(p) = f(p) + (g+h)(p) = (f+(g+h))(p).$$
\item Commutative 
$$(f+g)(p)=f(p) + g(p) = g(p) +f(p) = (g+f)(p).$$
\item Additive Identity $f_0 : M \to \er$, defined by  $f_0(p) = 0 \in \er$
$$(f+f_0)(p) = f(p).$$
\item Inverse $-f : M \to \er$ of $f \in \CM$ defined by $(-f)(p) = -f(p)$
$$
((-f) + f)(p) = -f(p) + f(p) = 0 = f_0(p).
$$
\end{enumerate}
\item[$\bullet $] Abelian semi-group under multiplication, $\forall f,g \in \CM, \, fg \in \CM$
\begin{enumerate}[nolistsep]
\item Associative
$$((fg)h)(p) = (fg)(p) h(p) = f(p)(gh)(p) = (f(gh))(p).$$
\item Commutative
$$(fg)(p) = f(p)g(p) = g(p)f(p) = (gf)(p). $$
\item Multiplicative Identity $f_1 : M \to \er$ defined by $f_1(p)=1 \in \er$
$$(ff_1)(p) = f(p)f_1(p) = f(p). $$
\end{enumerate}
\item[$\bullet $] Distributive law
$$
(f(g+h)) (p) = f(p) (g+h)(p) = (fg)(p) + (fh)(p) = (fg + fh)(p).
$$
\end{enumerate}
The set $\mathfrak{X}(M)$ is a module over $\CM$ : 

All the properties follows from properties of $(T_pM,\er)$.
\begin{enumerate}[nolistsep]
\item Abelian group under $+$
\begin{enumerate}[nolistsep]
\item[$\bullet$ ] Associative
\begin{align*}
((X+Y)+Z)(p) &= (X+Y)_p + Z_p \\ &= X_p+Y_p+Z_p \\ &= X_p + (Y+Z)_p \\ &= (X+(Y+Z))(p).
\end{align*}
\item[$\bullet$ ] Commutative
$$
(X+Y)(p) = X_p + Y_p  = Y_p + X_p = (Y+X)(p).
$$ 
\item[$\bullet$] Identity $X_0 : M \to TM$ defined as $X_0(p) = 0 \in T_pM$,
$$
(X+X_0)(p) = X(p) + (X_0)(p) = X(p) + 0 = X(p).
$$
\item[$\bullet$ ] Inverse of $X \in \mathfrak{X}(M)$ is $-X : M \to TM$ defined by $(-X)(p) = -X_p \in T_pM$ inverse of $X_p$ in $T_pM$,
$$
(-X + X)(p) = (-X)(p) + X(p) = -X_p + X_p = 0 = X_0(p).
$$
\end{enumerate}
\item $f(X+Y) = fX + fY$,
\begin{align*}
(f(X+Y))(p) &= f(p) (X_p + Y_p)\\ &= f(p)X_p + f(p)Y_p \\ &= (fX)(p) + (fY)(p) \\ &= (fX + fY)(p).
\end{align*}
\item $(f+g)X = fX + gX$,
\begin{align*}
((f+g)X)(p) &= (f+g)(p) X_p \\ &= (f(p)+g(p))X_p \\ &= f(p)X_p + g(p)X_p \\  &= (fX)(p) + (gX)(p) \\ &= (fX + gX)(p). 
\end{align*}
\item $f(gX) = (fg)X$,
\begin{align*}
(f(gX))(p) = f(p) (gX)(p) = f(p)g(p) X_p = (fg)(p)X_p = ((fg)X)(p).
\end{align*}
\item $f_1 X = X$,
$$
(f_1X)(p) = f_1(p) X_p = 1\cdot X_p = X(p). 
$$
\end{enumerate}
\end{proof}

\begin{prop}[Problem 8.1 : \textbf{Extension Lemma for Vector Fields} \cite{LeeSM}]
Let $M$ be a smooth manifold with or without boundary, and let $A \subhim M$ be a closed subset. Suppose $X$ is a smooth vector field along $A$. Given any open subset $U$ contain $A$, there exists a smooth global vector field $\widetilde{X}$ on $M$ such that $\widetilde{X}|_A = X$ and supp $\widetilde{X} \subhim U$.
\end{prop}
\begin{proof}
The proof is similar to (\ref{Extend f from A}), only need a few notes when apply the same steps. Here is the sketch : For each $p \in A$ we have neighbourhood $W_p \subhim U$ and a smooth vector field $\widetilde{X}^p : W_p \to TM$ such that $\widetilde{X}^p|_{W_p \cap A} = X|_{W_p \cap A}$. The collection $\mathcal{U} = \{ W_p : p \in A \} \cup \{ M \smallsetminus A \} $ is an open cover of $M$, and collection of functions $\{ \psi_p : p \in A \} \cup \{ \psi_0 \}$ is a smooth partition of unity subordinate to $\mathcal{U}$, with supp $\psi_p \subhim W_p$ and supp $\psi_0 \subhim M\smallsetminus A$.

By restrict $\psi_p$ to $W_p$, multiplying it with $\widetilde{X}^p$, we have smooth vector field on $W_p$ by Proposition 8.8. Extend the result using Gluing Lemma to $M$, we have global vector field 
$$
\psi_p \widetilde{X}^p : M \to TM, \quad \text{supp }\psi_p \widetilde{X}^p \subhim \text{supp }\psi_p \cap \text{supp } \widetilde{X}^p \subhim   \text{supp }\psi_p \subhim W_p .
$$
Define $\widetilde{X} : M \to TM$ by
$$
\widetilde{X}(x) = \sum_{p \in A} (\psi_p \widetilde{X}^p)(x), \quad x \in M.
$$
It follows that, \newline
$\bullet $ \textbf{$\widetilde{X}$ is smooth function :} Because the collection of support $\{\text{supp } \psi_p\}$ is locally finite (by definition of partition of unity), the vector field $\widetilde{X}$ is smooth on a neighbourhood of each point, hence smooth on $M$.\newline
$\bullet $ \textbf{$\widetilde{X}$ agrees with $X$ on $A$ :} If $x \in A$,
$$
\widetilde{X}(x) = \sum_{p\in A} \psi_p(x) \widetilde{X}^p_x = \Big(\psi_0(x) + \sum_{p \in A} \psi_p(x) \Big) X_x = X (x).
$$ 
\newline 
$\bullet $ \textbf{supp $\widetilde{X} \subhim U$ : }It follows from Lemma 1.13(b) that 
$$
\text{supp } \widetilde{X} \subhim \overline{\bigcup_{p \in A} \text{supp } \psi_p}  = \bigcup_{p \in A} \text{supp }\psi_p \subhim U. 
$$ 
\end{proof}

\begin{prop}[Problem 8-5 : Prove Proposition 8.11 \textbf{Completion of Local Frames}  \cite{LeeSM} ] 
Let $M$ be smooth $n$-manifold with or without boundary.
\begin{enumerate}
\item[(a)] If $(X_1,\dots,X_k)$ is a linear independent $k$-tuple of smooth vector fields on an open subset $U \subseteq M$, with $1 \leq k < n$, then for each $p \in U$, there exist smooth vector fields $X_{k+1},\dots,X_n$ in a neighbourhood $V$ of $p$ such that $(X_1,\dots,X_n)$ is a smooth local frame for $M$ on $U \cap V$.
\item[(b)] If $(v_1,\dots,v_k)$ is a linearly independent $k$-tuple of vectors in $T_pM$ for some $p \in M$, with $1\leq k \leq n$, then there exists a smooth local frame $(X_i)$ on a neighbourhood of $p$ such that $X_i|_p = v_i$ for $i =1,\dots,k$. 
\item[(c)] If $(X_1,\dots,X_n)$ is a linearly independent $n$-tuple of smooth vector fields along a closed subset $A\subhim M$, then there exists a smooth local frame $(\widetilde{X}_1,\dots,\widetilde{X}_n)$ on some neighbourhood of $A$ such that $\widetilde{X}_i|_A = X_i$ for $i=1,\dots,n$.
\end{enumerate}
\end{prop}

\begin{proof}
For (a), Let $p \in U$ and $(W,\varphi)$ is a smooth charts contain $p$. The representation of $X_i$ on $A := W\cap U$ is 
$$
\widehat{X}_i : \varphi(A) \rightarrow \varphi(A) \times \mathbb{R}^n
$$
with
$$
x \mapsto \widehat{X}_i(x) =(x,X_i(x))= (x,X_i^1(x), \dots, X_i^n(x)).
$$
and the last $n$-arguments are the components functions of $X_i$ on the chart. At $\varphi(p) = x_p \in \varphi(A) \subseteq \mathbb{R}^n$, we can choose $(n-k)$-tuple of vectors $v_j \in T_{x_p}\mathbb{R}^n$  for $k+1 \leq j \leq n$ such that  $$(X_1(x_p),\dots,X_k(x_p),v_{k+1},\dots,v_n)$$ is a basis of $T_{x_p}\mathbb{R}^n$. Now extend each vectors $v_j$ to the constant vector fields on $\varphi(A)$
$$
X_j : x \mapsto v_j(x) = v_j \qquad k+1 \leq j \leq n
$$
So we have vector fields $(X_1,\dots,X_k,X_{k+1},\dots, X_{n})$ on $\varphi(A)$,  such that at $x_p$, they are a basis for the tangent space $T_{x_p}\mathbb{R}^n$. Note that at any other point in $\varphi(A)$ other that $x_p$ one can't guaranteed that they form a basis or not. 

Next note that at $x_p$, the matrix $(X_i^j(x_p))$
$$
\begin{bmatrix}
    \Big( X^i_j(x_p) \Big)^{1\leq i \leq n}_{1 \leq j \leq k} &  \Bigg| & \Big( v^i_j \Big)^{1\leq i \leq n}_{k+1 \leq j \leq n} 
\end{bmatrix}
$$
are invertible (by l.i of $X_1(x_p),\dots,X_n(x_p)$). If we define a map by sending each point to its matrix defined as above and follows by determinant map of this matrices, then we have composition of smooth function
$$
\varphi(V) \xrightarrow{\text{M}} M(n \times n, \mathbb{R}) \xrightarrow{\text{det}}  \mathbb{R}
$$
At $x_p$, the value $\text{det}(M(x_p)) \neq 0$. Then by continuity of $M$ and determinant map, we have an open subset $V_0 \subseteq \varphi(A)$ contain $x_p$ where its matrix are invertible. This means that we have neighbourhood $V_0$ of $x_p$ such that the vectors fields $(X_1,\dots,X_n)$ constitute a local frame on $V_0$ with the last $(n-k)$ vector field are constant vector fields as constructed above. This proves (a).

For (b), let $p \in M$ and $(v_1,\dots,v_k)$ are vectors in $T_pM$. Choose $(v_{k+1},\dots,v_n)$ in $T_pM$ such that $(v_1,\dots,v_n)$ are basis for $T_pM$. Choose a smooth chart $(U,\varphi)$ contain $p$. In this chart, each vector represent as
$$
v_i = \sum_{j=1}^{n} v^j_i \frac{\doo }{\doo x^j}\bigg|_p.
$$
By extend each $v_i$ to a vector field with constant coefficient, that is by defining vector fields $X_i : U \to TM$for $i=1,\dots,n$ as
$$
X_i(x) = \sum_{j=1}^{n} v_i^j \frac{\doo}{\doo x^j}\bigg|_p, \quad \text{for any }x \in U
$$
we obtain linearly independent smooth vector fields $(X_i,\dots,X_n)$ on neighbourhood $U$ of $p$ such that $X_i|_x = v_i$ for any $x \in U$. 

For (c), Let $X_i : A \to TM,\, (i=1,\dots,n)$ are linearly independent smooth vector fields along closed subset $A \subhim M$. By definition, for a fix $i$, there are neighbourhood $U_{ip}$ of each point $p \in A$ and a smooth vector field $\widetilde{X}_{ip} : U_{ip} \to TM$ such that $\widetilde{X}_{ip}|_{U_{ip}\cap A} = X_i|_{U_{ip}\cap A}$. Therefore, $U_i = \{U_{ip} : p \in A \}$ is an open subset contain $A$. By Lemma 8.6, there are smooth vector field $\widetilde{X}_i : M \to TM$ such that $\widetilde{X}_i|_A = X_i$ and supp $\widetilde{X}_i \subhim U_i$. By doing this for all $i=1,\dots,n$, we obtain $n$-tuple global smooth vector fields $(\widetilde{X}_1,\dots,\widetilde{X}_n)$ such that $\widetilde{X}_i|_A = X_i$ and supp $\widetilde{X}_i \subhim U_i$ for every $i =1,\dots,n$. The restrictions of $\widetilde{X}_i$ to $U = \bigcap_{i=1}^n U_i$ is smooth. Denote these restrictions by same symbol for simplicity. Let $p$ be any point in $A\subhim U$. Since the vectors $(\widetilde{X}_1|_p,\dots,\widetilde{X}_n|_p)$ are linearly independent, then there is a smaller neighbourhood $V_p \subhim U$ of $p$ such that $(\widetilde{X}_1,\dots,\widetilde{X}_n)$ are linearly independent there\footnote[2]{Proved in the \textit{Remark} below.}. Therefore we have a neighbourhood $V := \{V_p : p\in A \} $ of $A$ such that $(\widetilde{X}_1,\dots,\widetilde{X}_n)$ are linearly independent on $V$ and $\widetilde{X}_i|_A = X_i$.
\end{proof}
\begin{remark}
\textbf{This remark contains proof for the claim in the proof of part (c) above.} 
 
 Suppose $M$ is an $n$-dimensional smooth manifold and $(X_1,\dots,X_n)$ are smooth vector fields defined on some open subset $U\subseteq M$ such that for a point $p \in U$, $X_1(p),\dots,X_n(p)$ are linearly independent. We claim that there are no neighbourhood $V$ of $p$ such that for any point $x \in V$, then vectors $X_1(x),\dots,X_n(x)$ are linearly independent. Stated it differently, every neighbourhood of $p$ contain a point where the vectors $X_i$ are linearly dependent. 

This will lead to contradiction as follows : Choose a  smooth charts $(U',x^i)$ contain $p$. By shrinking $U'$, assume that $U'\subseteq U$. In this chart, the value of vector fields $X_i$ at any point $x \in U'$ is
$$
X_i(x) = X_i^j(x) \frac{\partial}{\partial x^j}\bigg|_x.
$$

Define a map from $m : U' \to M(n\times n,\mathbb{R} )$  defined by 
$$
m : x \mapsto 
\begin{pmatrix}
    X_1^1(x) & \cdots & X_n^1(x) \\
    \vdots & \ddots & \vdots \\
    X_1^n(x) & \cdots & X_n^n(x)
\end{pmatrix} \in M(n\times n,\mathbb{R} ).
$$
This is a smooth map since the entries are smooth functions on $U'$. By composing this with determinant map $\text{det} : M(n\times n, \mathbb{R}) \to \mathbb{R}$, we have a smooth function $f = \text{det} \circ m : U' \to \mathbb{R}$. This function is also smooth and at $p \in U'$, $f(p) \neq 0$ since $X_1(p),\dots,X_n(p)$ are linearly independent vectors.

Now, by assumption, any neighbourhood $V_1 \subseteq U'$ of $p$ contain a point $p_1$ such that $X_1(p_1),\cdots,X_n(p_1)$ are linearly dependent. Therefore $f(p_1) = 0$. By doing this repeatly, we have a sequence $(p_k)_{k=1}^{\infty}$ in $U'$ converging to $p$ such that $f(p_k) =0$ for all $k$. Since $f$ is smooth (hence continous), then as $p_k \to p$, then $f(p_k) \to f(p)$. But this is not happen since $f(p_k)= 0$ is a constant sequence and $f(p) \neq 0$. Therefore $f$ is not continous. Contradiction. We can also proved this directly by (again using continuity of $f = \det \circ m$) obtain open neighbourhood of $p$ as the preimage of open subset of $f(p) \neq 0 \in \er$. $\qedsymbol$
\end{remark}

\begin{prop}[Problem 8.9 \cite{LWTu01} \textbf{Transforming Vectors to Coordinate Vectors}]
Let $X_1,\dots,X_n$ be  $n$ vector fields on an open subset $U$ of a manifold of dimension $n$ Suppose that at $p\in U$, the vectors $(X_1)_p,\dots,(X_n)_p$ are linearly independent. Show that there is a chart $(V,x^1,\dots,x^n)$ about $p$ such that $(X)_p=(\partial/\partial x^i)_p$ for all $i=1,\dots,n$.
\end{prop}
\begin{proof}
Suppose $(X_i)_{i=1}^n$ are local vector field on $U\subset M^n$ such that at $p\in U \subset M^n$, the $n$ vectors $X_i(p) \in T_pU$ are linear independent. If we expand each vector in terms of coordinate basis $(\frac{\partial}{\partial x^i}\big|_p)$ in $(U, \varphi)$ as 
$$
X_i(p) = \sum_jA_i^j \frac{\partial}{\partial x^j}\bigg|_p, \quad (i=1,,\dots,n)
$$
this means that the relation 
$$0=\sum _{i=1}^n c^i X_i(p) = \sum_{i,j}c^i A_i^j \frac{\partial}{\partial x^j}\bigg|_p \implies c^i=0 \quad \text{and} \quad \sum_iA^j_i c^i=0 $$ 
Which is means that $A^j_i$ is invertible. 

Now we want a new coordinate system $(V,\psi)$ with $\psi(p) = (\tilde{x}^1(p),\dots,\tilde{x}^n(p))$ such that
$$
X_i(p) = \frac{\partial}{\partial \tilde{x}^i} \bigg|_p, \quad \text{for all}\, i=1,\dots,n
$$
By using basis transformation rule 
$$\frac{\partial}{\partial \tilde{x}^i} \bigg|_p = \sum_j \frac{\partial x^j}{\partial \tilde{x}^i}(p) \frac{\partial}{\partial x^j} \bigg|_p$$
we have 
$$
\sum_j A_i^j \frac{\partial}{\partial x^j}\bigg|_p = \sum_j \frac{\partial x^j}{\partial \tilde{x}^i}(p) \frac{\partial}{\partial x^j} \bigg|_p.
$$
So we have $n\times n$ equation $A^i_j = \frac{\partial x^i}{\partial \tilde{x}^j}(p)$. Now apply inverse function theorem to our transformation map
$$
F : (x^1,\dots,x^n) \rightarrow (\tilde{x}^1,\dots,\tilde{x}^n)
$$
Because the differential $DF(p) =\frac{\partial x^i}{\partial \tilde{x}^j}(p) =A^i_j  $ is invertible at $p$, then there are smaller neighbourhood in $U$ and $V$ s.t the restriction of $F$ to this neighbourhood is diffeomorphism. Therefore the existence of such chart $(V,\psi)$ is guaranteed by this theorem. 
\end{proof}

\textbf{Vector Fields as Derivation of $\CM$. (Notes for p.180 \cite{LeeSM}) } \, An essential property of vector fields is that they define operators on the space of smooth real-valued functions. If $X \in \VF(M)$ and $f$ is a real-valued function defined on an open subset $U \subhim M$, we define $Xf : U \to \er$ by
$$
(Xf)(p) := X_pf
$$
where the vector $X_p$ understood as the corresponding vector on $T_pU$, that is the image of $X_p \in T_pM$ under isomorphism $d\iota_p : T_pU \to T_pM$. I.e., $X_p$ in the RHS above is actually $(d\iota_p)^{-1}(X_p) \in T_pU$. Because the action of a tangent vector, say $v\in T_pM$, to a smooth function $f \in \CM$ is determined by the values of the function in a arbitrary small neighbourhood, (that is for any $f,g \in \CM$ with $f|_U=g|_U$ then $vf = vg$ ) it follows that $Xf$ is locally determined. What it mean by that is this :

Consider $X \in \VF(M)$ and $f \in \Coo(M)$. The new function $Xf :U \to \er$. If $V \subhim U$ is an open subset in $U$, then the value of $Xf$ at $x \in V$ is
$$
(Xf)(x) = X_x f, \quad \text{with regard } X_x \in T_xV.
$$
But since it only depends on the the values of $f$ in $V$, then the values of $Xf$ on $V \subhim U$ is determined by the values of $f$ on $V$, then
$$
(Xf)|_V = X(f|_V).
$$
Another way to see it is that if $X_pf$ depends on the function $f$ globally, that is if we change $f$ in $X_pf$ with $g$, $f|_V = g|_V$, the values change, then we can't assert the above equality.
\begin{prop}[Proposition 8.14 \cite{LeeSM} : \textbf{Smoothness Criterion for Vector Fields}]
Let $M$ be a smooth manifold with or without boundary, and let $X : M \to TM$ be a rough vector field. The following are equivalent :
\begin{enumerate}[nolistsep]
\item[(a)] $X$ is smooth.
\item[(b)] For every $f \in \CM$, the function $Xf $ is smooth on $M$.
\item[(c)] For every open subset $U \subhim M$ and every $f \in C^{\infty}(U) $, the function $Xf$ is smooth in $U$. 
\end{enumerate}
\end{prop}
\begin{proof}
To prove (a) $\implies$ (b), assume $X$ is smooth. For any $f \in \CM$, the map $Xf : M \to \er$ defined as $Xf(p) = X_pf$. For any point $p \in M$, choose a smooth charts $(U,(x^i))$ contain $p$. In this coordinates, the value of $X$ at any point $x \in U$ is
$$
X(x) = X^i(x) \frac{\doo}{\doo x^i}\bigg|_x.
$$
So
$$
Xf(x) = X(x)f = \bigg(X^i(x) \frac{\doo}{\doo x^i}\bigg|_x \bigg)f = X^i(x) \frac{\doo f}{\doo x^i}(x).
$$
Since $X$ is smooth, then $X^i$ is smooth on $U$. It follows that $Xf$ is smooth on $U$. Since the same is true in any neighbourhood of each point, $Xf$ is smooth on $M$.

To show (b) $\implies$ (c), let $U \subhim M$ is open and $f \in C^{\infty}(U)$. We assume that (b) holds, that is, $X$ defines an operator on $\CM$. To show $Xf$ is smooth in $U$, we can show that for any point $p \in U$, there is a neighbourhood $V \subhim U$ of $p$, such that $(Xf)|_V$ is smooth. Let $p $ be any point in $U$ and choose a neighbourhood\footnote[2]{do this in a chart contain $p$} $V$ of $p$ such that $\overline{V} \subhim U$. By Proposition 2.25, choose a smooth bump function $\psi : M \to \er$, with $\psi \equiv 1$ on $\overline{V}$ and supp $\psi \subhim U$. Define function $\tilde{f} = \psi f : U \to \er$, with the function $\psi$ there understood as the restriction of $\psi$ to $U$.  Since supp $\psi f \subhim \text{supp }\psi \cap \text{supp }f \subhim \text{supp }\psi \subhim U$, we can extend $\tilde{f}$ to $M$ by Gluing Lemma with $\tilde{f}$ has zero value on $M \smallsetminus \text{supp }\psi$. Since by (b) $X\tilde{f} \in \CM$, the restriction $(X\tilde{f})$ to any open subset is smooth. In particular, $(X\tilde{f})|_V$ si smooth. Since
$$
(X\tilde{f})|_V = X(\tilde{f}|_V) = X(\psi f)|_V = X(f|_V) = (Xf)|_V,
$$
hence $Xf$ is smooth on the neighbourhood $V$ of $p$. Because $p \in U$ is arbitrary, we showed that $Xf$ is smooth on some neighbourhood of each point in $U$. Therefore $Xf$ is smooth on $U$.

To show $(c) \implies (a)$, assume that $Xf$ is smooth whenever $f$ is smooth on some open subset. Let $p$ be any point in $M$ and $(U,x^i)$ be any smooth chart contain $p$. Thinking $x^i : U \to \er$ as a smooth function on $U$, we have
$$
Xx^i = X^j \frac{\doo }{\doo x^j} (x^i) = X^i, 
$$ 
which is smooth by (b). Therefore by Proposition 8.1, $X$ is smooth.
\end{proof}
\begin{remark}
The consequence of above proposition is that any smooth vector field $X \in \VF(M)$ defines an operator
$$
X : \CM \to \CM
$$ 
defined as before $X : f \mapsto Xf$, where $Xf (p) := X_pf$. In addition, this operator is \textbf{linear over $\er$} and satisfy \textbf{product rule} for vector fields
\begin{enumerate}
\item[(D1)] For any $a,b \in \er$ and $ f,g \in \CM$
$$(Xf)(af+bg) = a (Xf)f + b(Xf)g$$
\item[(D2)] For any $f,g \in \CM$
$$
X(fg) = f\, Xg + g Xf
$$
\end{enumerate}
and therefore every $X \in \VF(M)$ defines a \textbf{derivation} $X : \CM \to \CM$. In fact, every derivation, that is operator $D : \CM \to \CM$ satisfy D1 and D2 determine a smooth vector field $X \in \VF(M)$ such that $Df=Xf$ (see prop. 8.15 \cite{LeeSM}). By this, we can identify a derivation with a smooth vector field. \newline

\textbf{Vector Fields and Submanifolds (Notes for page 184)} Suppose we have a smooth vector field $X \in \VF(M)$ on a smooth manifold $M$, and let $S\subhim M$ be an immersed or embeddeed submanifold of $M$. Given $p \in S$, we say \textbf{\boldmath$X$ is tangent to \boldmath$S$ at \boldmath$p$} if $X_p \in T_pS \subhim T_pM$. And \textbf{\boldmath$X$ tangent to \boldmath$S$} if it is tangent to every point $p$ in $S$. 


\textit{Comment : } Suppose we have a vector field $Y \in \VF(M)$ such that $Y$ is tangent to a immersed submanifold $ \iota : S \hookrightarrow M$. That is, at every point $p \in S$, we have a vector $X_p \in T_pS$ such that $d\iota_p(X_p) = Y_p$. I.e., we have a rough vector field on $S$, $X : S \to TS$ such that $X$ is $\iota$-related to $Y$. Intuitively, we can have the same vector field by restrict $Y \in \VF(M)$ to $S$. However, its still not clear that the restriction $X$ is a smooth vector field. If we view the restriction as the composition map $Y|_S := Y \circ \iota : S \to TM$, then its smooth by Theorem 5.27. But what we want here is that the vector field $X : S \to TS$. If its true that $TS$ is a submanifold, then by Theorem 5.29, we can restrict $X : S \to TM$ to the codomain $TS \subhim TM$ and obtain smooth vector field. But we dont know yet if this is true. So we will see a direct approach as showed by the following proposition, which shows that the vector field $X : S \to TS$ defined such that for each $p \in S$, $d\iota_p(X_p) = Y_p$ is a smooth vector field on $S$.

\begin{prop}[Proposition 8.23 : \textbf{Restricting Vector Fields to Submanifolds}]
Let $M$ be a smooth manifold, let $S \subhim M$ be an immersed submanifold with or without boundary, and let $\iota : S \hookrightarrow M$ denote the inclusion map. If $Y \in \VF(M)$ is tangent to $S$, then there is a unique smooth vector field on $S$, denoted by $Y|_S$, that is $\iota$-related to $Y$. 
\end{prop}
\begin{remark}
Note that $Y|_S$ is not $Y \circ \iota : S \to TM$. I dont know if eventually these two notions are same. Probably this will be revealed on Chapter 10 about Vector Bundles. Probably.
\end{remark}
\begin{proof}
The vector field $Y \in \VF(M)$ tangent to $S$ means that for any $p \in S$ there are a unique vector $X_p \in T_pS$ such that $Y_p = d\iota_p(X_p)$.  The vector $X_p$ is unique since $d\iota_p$ is injective. We then have a unique rough vector field $X : S \to TS$ defined as above that is $\iota$-related to $Y$. The only thing we need to check is the smoothness of $X$. It is suffice to show that $X$ is smooth on the neighbourhood of each point $p \in S$. 

Let $p $ be any point in $S$. Since any immersed submanifold (with or without boundary) is locally embedded, there is a neighbourhood $V$ of $p $ in $S$ that is embedded in $M$. Choose a slice chart (or boundary slice chart) $(U,(x^i))$ for $V$ centered at $p$, so that $V\cap U$ is the subset with $x^{k+1} = \cdots=x^n = 0$ (and $x^k\geq 0$ if $p \in \doo S$). We know that $(x^1,\dots,x^k)$ form a local coordinates in $V\cap U$ and therefore $\{ \doo/\doo x^1,\dots,\doo/\doo x^k \}$ is a local coordinate frame there. Let $Y = Y^i \doo/\doo x^i$ be the representation of $Y$ in $(U,(x^i))$. Since $Y$ is tangent at $S$, then for any $x \in U\cap V$,
$$
Y_x =\sum_{i=1}^{n} Y^i(x) \frac{\doo}{\doo x^i}\Big|_x, \quad \text{with } \quad Y^{k+1}(x) = \dots = Y^n(x) \equiv 0.
$$
By construction, $Y_x = d\iota_x(X_x)$ for any $x \in U \cap V$, so 
$$
\sum_{i=1}^{n} Y^i(x) \frac{\doo}{\doo x^i}\bigg|_x = d\iota_x \bigg( X^i(x) \frac{\doo}{\doo x^i}\bigg|_x \bigg) = \sum_{i=1}^{k}X^i(x) \frac{\doo}{\doo x^i}\bigg|_x,
$$
which is implies that the component functions $X^i = Y^i$ for $i=1,\dots,k$ are smooth. Therefore $X$ is smooth on $V \cap U$.
\end{proof}

\textbf{Lie Brackets (Notes for page 185)}
Another observation will lead to the following important result.  Given $X,Y \in \VF(M)$ and and any $f \in \CM$, we can apply $Y$ to $f$ to obtain smooth function
$$
f \mapsto Yf \in \CM,
$$ 
and then apply $X$ to $Yf$ to obtain another one 
$$
Yf \mapsto X(Yf) \in \CM.
$$ 
So in total we construct another operator 
$$
f \mapsto XYf \in \CM.
$$ 
The natural question is that : \textbf{is this operator a derivation (hence smooth vector field) ?}  In general, \textbf{the answer is no}. Consider the following example : $M = \er^2$ with $X = \frac{\doo}{\doo x}$ and $Y=x\frac{\doo}{\doo y}$ and $f(x,y) = x$, $g(x,y) = y$. A quick computation shows
$$
XY(fg) = 2x\quad \text{but} \quad  f\,XYg + g\, XYg = x.
$$
However, by applying $XYf$ and $YXf$ and then substract them, we obtain a derivation $[X,Y] : \CM \to \CM$ with $[X,Y] : = XYf - YXf$, called the \textbf{Lie Bracket} of $X$ and $Y$. The key fact is that $[X,Y]$ is a smooth vector field, in contrast with $XY$ or $YX$. The proof is straightforward, look Lemma 8.25 \cite{LeeSM}.
\end{remark}

\begin{prop}[Exercise 8.34 \cite{LeeSM}]
Verify that kernel and the image of a Lie Algebra homomorphism are Lie subalgebras.
\end{prop}
\begin{proof}
Suppose that we have a Lie algebra homomorphism $A : \mathfrak{g} \to \mathfrak{h}$, that is $A[X,Y] = [AX,AY]$ for any $X,Y \in \mathfrak{g}$. Let $0$ be the zero vector on $\mathfrak{h}$. The space $\text{ker}\,A = \{X \in \mathfrak{g} : A(X) = 0 \}$ is a linear subspace of $\mathfrak{g}$, and it closed under brackets $\mathfrak{g} \times \mathfrak{g} \to \mathfrak{g}$. For $X,Y \in \text{ker}\,A \subhim \mathfrak{g}$, the image of their bracket is 
$$
A[X,Y] = [AX,AY] = [0,0] = 0
$$ 
(since by antisymmetry $[X,X] = 0$), therefore $[X,Y] \in \text{ker}\,A$.

The image $\text{Im}\,A =\{AX \in \mathfrak{h} : \text{for all }X \in \mathfrak{g}  \} $ also a linear subspace of $\mathfrak{h}$ and closed under Lie bracket $\mathfrak{h} \times \mathfrak{h} \to \mathfrak{h}$. Since for any $Y_1,Y_2 \in \text{Im}\,A$, there are $X_1,X_2 \in \mathfrak{g}$ such that $AX_1=Y_1$ and $AX_2=Y_2$. Therefore
$$
[Y_1,Y_2] = [AX_1,AX_2] = A[X_1,X_2] \in \text{Im}\, A.
$$
Since both $\text{ker}\, A\subhim \mathfrak{g}$ and $\text{Im}\, A \subhim \mathfrak{h}$ are both linear subspace and closed under Lie bracket $\mathfrak{g} \times \mathfrak{g} \to \mathfrak{g}$ and $\mathfrak{h} \times \mathfrak{h} \to \mathfrak{h}$ respectively, then they both are Lie subalgebra. 
\end{proof}

\begin{prop}[Exercise 8.35 \cite{LeeSM}]
Suppose $\glie$ and $\hlie$ are finite-dimensional Lie algebra and $A : \glie \to \hlie$ are linear map. Show that $A$ is a Lie algebra homomorphism if and only if $A[E_i,E_j] = [AE_i,AE_j]$ for some basis $(E_1,\dots,E_n)$ of $\glie$.
\end{prop}
\begin{proof}
One direction is immidiate. For the converse, let $(E_1,\dots,E_n)$ be any basis for $\glie$ and $A : \glie \to \hlie$ is a linear map such that $A[E_i,E_j] = [AE_i,AE_j]$. For any $X,Y \in \glie$
\begin{align*}
A[X,Y] &= A[X^iE_i,Y^jE_j] \\ &= A( X^iY^j[E_i,E_j]) \\ &= X^iY^j A[E_i,E_j]\\ &= X^iY^j [AE_i,AE_j] \\
&= [X^iAE_i,Y^jAE_j] \\
&= [AX,AY]  
\end{align*}
Therefore $A$ is a Lie algebra homomorphism.
\end{proof}

\textbf{The Lie Algebra of a Lie Group.} \textbf{(Notes for page 189) \cite{LeeSM}}  We know that for any  $g \in G$ of a Lie group, the left-translation $L_g : G \to G $ is a diffeomorphism. So by Proposition 8.19, for any $X \in \VF(M)$ there is a unique smooth vector field $Y \in \VF(M)$ that is $L_g$-related to $X$, denote this as $(L_g)_* X$. We say a vector field $X \in \VF(M)$ is a \textbf{left-invariant vector field} if $X$ is $L_g$ related to itself for every $g \in G$. That is 
$$
d(L_g)_{g'} (X_{g'}) = X_{gg'}, \quad \text{for all }g,g' \in G.
$$ 
We abbreviate this by $(L_g)_{*}X=X$ for every $g \in G$.

The Lie algebra of a Lie Group, denote as Lie$(G)$, defined as the Lie algebra of all smooth left-invariant vector fields on a Lie group $G$.  Lie$(G)$ is a linear subspace of $\VF(M)$ that closed under Lie bracket $(X,Y) \mapsto [X,Y]$.
\begin{itemize}
\item Its a linear subspace of $\VF(M)$ since
$$
(L_g)_*(aX+bY) = a(L_g)_*X + b (L_g)_*Y = aX + bY,
$$
\item It is closed under Lie bracket since
$$
(L_g)_*[X,Y] = [(L_g)_*X,(L_g)_*Y] = [X,Y],
$$
for any $X,Y \in \text{Lie}(G)$ and $g \in G$.
\end{itemize}
In fact, Lie$(G)$ is finite dimensional and has dimension equal to dimension of $G$.
\begin{prop}[Theorem 8.37 \cite{LeeSM} : dim$(\text{Lie}(G)) = \text{dim} G$] Let $G$ be a Lie group. The evaluation map $\varepsilon : \text{Lie}(G) \to T_eG$ given by $\varepsilon(X) = X_e$, is a vector space isomorphism. Thus Lie$(G)$ is finite-dimensional, with dimension equal to dim$(G)$.
\end{prop}
\begin{proof}
The evaluation map is linear over $\er$ since
$$
\varepsilon(aX+bY) = (aX+bY)_e =aX_e + b Y_e = a\varepsilon(X) + b \varepsilon(Y). 
$$
To prove $\varepsilon$ is injective, suppose that $\varepsilon(X) =X_e= 0 \in T_eG$. Since $X \in \text{Lie}(G)$, then $(L_g)_* X = X$. This implies $X_g = d(L_g)_e X_e = 0$ for every $g \in G$. So $X=0$.

To prove $\varepsilon$ is surjective, for arbitrary $v \in T_eG$, we need to construct a left-invariant vector field $v^L$ such that $\varepsilon(v^L) = v^L|_e= v$. Let $v\in T_eG$ be an arbitrary vector. We want a left-invariant vector field $v^L$ whose value at $e\in G$ is $v$, i.e.,
$$
d(L_g)_{e}(v^L|_e) = d(L_g)_{e}(v) =  v^L|_g.
$$ 
So we define rough vector field $v^L : G \to TG$ by
$$
v^L|_g := d(L_g)_{e}(v).  
$$ 
To show that $v^L$ is smooth, it suffice to show that for any $f \in \Coo(G)$, $v^Lf : G \to \er$ defined by $(v^Lf)(g) = v^L|_gf$, is a smooth function. To do this, choose a smooth curve $\gamma : (-\delta,\delta) \to G$ such that $\gamma(0) =e$ and $\gamma'(0) = v$. Then for all $g \in G$,
\begin{align*}
(v^Lf)(g) &= v^L|_gf = d(L_g)_{e}(v)f = v(f \circ L_g) = \gamma'(0)(f \circ L_g) \\ &= \frac{d}{dt}\bigg|_{t=0} (f \circ L_g \circ \gamma)(t).
\end{align*}
If we define $\varphi : (-\delta,\delta) \times G \to \er$ by $\varphi (t,g) = (f\circ L_g \circ \gamma)(t) = f (g\gamma(t))$, the composition above shows that $(v^Lf)(g) = \frac{\doo \varphi}{\doo t}(0,g)$. Denote the "multiplication map" by $m : (g,h) \mapsto hg$. Because
$$
\varphi(t,g) =f(g\gamma(t)) = (f \circ m)(\gamma(t),g) = (f \circ m\circ (\gamma \times \text{Id}_G ))(t,g),
$$
that is $\varphi$ is a composition of smooth map, therefore $\varphi$ is smooth. It follows that $\frac{\doo \varphi}{\doo t}(0,g)$ depends smoothly on $g$, so $v^Lf$ is smooth.

Vector field $v^L$ is left-invariant, since for all $h,g \in G$,
\begin{align*}
d(L_h)_g (v^L|_g) &= (d(L_h)_g \circ d(L_g)_e)(v^L|_e) =(d(L_h)_g \circ d(L_g)_e)(v) \\ &=d(L_h \circ L_g)_e(v) \\ &= d(L_{hg})_e(v) \\ &= v^L|_{hg}.
\end{align*}
Therefore we have a left-invariant smooth vector field, $v^L \in \text{Lie}(G)$. The value of $v^L$ under $\varepsilon$ is
$$
\varepsilon(v^L) = v^L|_e = d(L_e)_e(v) = d(\text{Id}_G)_e(v) = \text{Id}_{T_eG}(v) = v.
$$ 
Hence $\varepsilon : \text{Lie}(G) \to T_eG$ is surjective.
\end{proof}
\begin{remark}
Given any  vector $v \in T_eG$, we denote the correspond smooth left-invariant vector field by $v^L$. That is,
$$
\varepsilon^{-1} : v\mapsto v^L .
$$
The above theorem produce these following interesting results : 
\begin{enumerate}
\item[(1)] The assumption of smoothness of vector fields in Lie$(G)$ is redundant,
\item[(2)] There is a global frame consist of left-invariant vector fields for any Lie group $G$.
\end{enumerate}
\end{remark}

\begin{prop}[Corollary 8.38 \cite{LeeSM} : \textbf{Smoothness is redundant in} \boldmath$\LieG$]
Every left-invariant rough vector field on a Lie group is smooth.
\end{prop}
\begin{proof}
Suppose $X : G \to TG$ is a left-invariant rough vector field. The corresponding element in Lie$(G)$ of the vector $v = X_e \in T_eG$ is $v^L$.  By  above theorem, $v^L : G \to TG$ defined as $v^L|_g := d(L_g)_e(v)$. Since $X$ is left-invariant,
$$
X_g = d(L_g)_e(X_e) = d(L_g)_e(v) = v^L|_g.
$$
Hence $X = v^L$ is smooth.
\end{proof}

\begin{prop}[Corollary 8.39 \cite{LeeSM} : \textbf{Lie Group is Parallelizable}]
Every Lie group admits a left-invariant smooth global frame, and therefore every Lie group is parallelizable. 
\end{prop}
\begin{proof}
Choose a basis $(v_1,\dots,v_n)$ for $T_eG$. By isomorphism $\varepsilon$, we have the corresponding left-invariant smooth vector fields $(v_1^L, \dots,v_n^L)$ on $G$ which is also a global frame for $G$. The linearly independent properties of $(v_1^L, \dots,v_n^L)$ can be checked directly. Since $v_i^L|_g = d(L_g)_e(v_i)$, then for any $g \in G$,
$$
0 = a^1v^L_1|_g + \cdots + a^nv^L_n|_g  = d(L_g)_e (a^1 v_1 + \cdots + a^nv_n) \implies a^i = 0,
$$
for all $1\leq i \leq n$.
\end{proof}

\textbf{An important example of Non-abelian Lie Algebra : }$\text{Lie}(\text{GL}(n,\er))$. (Notes page 193 \cite{LeeSM}) We know that through evaluation map we have natural isomorphism
$$
\text{Lie}(\text{GL}(n,\er)) \to \TInGL.
$$
Also, since $\GL$ is an open subset of $\Mtrix$, the tangent space is isomorphic to tangent space of $\Mtrix$, particularly $\TInGL \isomorphic T_{I_n}\Mtrix $. Since $\Mtrix \isomorphic T_{I_n}\Mtrix \isomorphic \lieMatrix$, we have isomorphism  
$$
\TInGL \to \lieMatrix.
$$
Therefore we have natural isomorphism
$$
\LieGL \to \TInGL \to \lieMatrix
$$
where both $\LieGL$ and $\lieMatrix$ are Lie algebra but with different structure$-$ the first coming from Lie bracket of vector fields, and the second from commutator bracket of matrices. The next proposition shows that the natural isomorphism above is in fact a Lie algebra isomorphism.

\begin{prop}[Proposition 8.41 \cite{LeeSM} : \textbf{Lie algebra of the General Linear group}]
The composition of the natural maps
$$
\LieGL \to \TInGL \to \lieMatrix
$$
gives a Lie algebra isomorphism between $\LieGL$ and the matrix algebra $\lieMatrix$.
\end{prop}
\begin{proof}
Using the matrix entries $(X^i_j)$ as global coordinates for $\GL \subhim \lieMatrix$, the natural isomorphism $\TInGL \to \LieGL$ takes the form
$$
A=A^i_j \frac{\doo}{\doo X^i_j}\bigg|_{I_n} \longleftrightarrow A = (A^i_j).
$$
Lets denote the isomorphism above as $ \ell : \lieMatrix \to \LieGL$. This map defined as 
$$
\ell : (A^i_j) \longmapsto A^i_j \frac{\doo}{\doo X^i_j}\bigg|_{I_n} \longmapsto \ell A = A^{\Ltegak}, 
$$
where 
$$
A^{\Ltegak}|_{X_0} = d(L_{X_0})_{I_n}(A) = d(L_{X_0})_{I_n}\bigg( A^i_j \frac{\doo}{\doo X^i_j}\bigg|_{I_n} \bigg).
$$
To show $\ell : A \mapsto \ell A = A^{\Ltegak}$ is a Lie algebra isomorphism, we have to show that $\ell[A,B]= [\ell A,\ell B]$ or
$$
 [A,B]^{\Ltegak} = [A^{\Ltegak}, B^{\Ltegak}] .
$$
To do this we need to find the coordinate expression for $A^{\Ltegak}$ first. Suppose that $X_0=(X_0)^i_j \in \GL$, then the representation of $L_X : \GL \to \GL$ in coordinates is
$$
(X^i_j) \mapsto (X_0)^i_k X^k_j.
$$
By linearity of differential and equation $(3.9) (\cite{LeeSM})$, we have
\begin{align*}
A^{\Ltegak}|_{X_0} &= d(L_{X_0})_{I_n}\bigg( A^i_j \frac{\doo}{\doo X^i_j}\bigg|_{I_n} \bigg) = A^i_j\,   d(L_{X_0})_{I_n}\bigg( \frac{\doo}{\doo X^i_j}\bigg|_{I_n} \bigg) \\
&= A^i_j \, \frac{\doo (L_{X_0})^k_m}{\doo X^i_j}(I_n) \frac{\doo}{\doo X^k_m}\bigg|_{X_0} = A^i_j \, \frac{\doo (X_0X)^k_m}{\doo X^i_j}(I_n) \frac{\doo}{\doo X^k_m}\bigg|_{X_0} \\
&= A^i_j \, \frac{\doo ((X_0)^k_l X^l_m)}{\doo X^i_j}(I_n) \frac{\doo}{\doo X^k_m}\bigg|_{X_0}  = A^i_j (X_0)^k_l \delta^{il}_{jm} \frac{\doo}{\doo X^k_m}\bigg|_{X_0}\\
&=   (X_0)^i_j  A^j_k \frac{\doo}{\doo X^i_k}\bigg|_{X_0}.
\end{align*}
By letting $X_0$ be arbitrary elements in $\GL$, we may change $X_0$ to $X$. So
$$
A^{\Ltegak}: X \longmapsto A^{\Ltegak}|_X = X^i_j A^j_k  \frac{\doo}{\doo X^i_k}\bigg|_{X}.
$$
By this formula, we have
\begin{align*}
[A^{\Ltegak},B^{\Ltegak}] &= \bigg[ X^i_j A^j_k  \frac{\doo}{\doo X^i_k} , X^p_q A^q_r  \frac{\doo}{\doo X^p_r}\bigg] \\
&=X^i_j A^j_k \frac{\doo}{\doo X^i_k} ( X^p_q A^q_r ) \frac{\doo}{\doo X^p_r}  -  X^p_q A^q_r \frac{\doo}{\doo X^p_r} (X^i_j A^j_k ) \frac{\doo}{\doo X^i_k} \\
&= (X^i_jA^j_k B^k_r  - X^i_jB^j_kA^k_r) \frac{\doo}{\doo X^i_r}.
\end{align*}
Whereas (by the same calculation),
\begin{align*}
[A,B]^{\Ltegak} &= X^i_j \, [A,B]^j_r \,  \frac{\doo}{\doo X^i_r} = X^i_j (A^j_kB^k_r - B^j_kA^k_r) \frac{\doo}{\doo X^i_r}\\ &= (X^i_jA^j_k B^k_r  - X^i_jB^j_kA^k_r) \frac{\doo}{\doo X^i_r}.
\end{align*}
Therefore $[A^{\Ltegak},B^{\Ltegak}] = [A,B]^{\Ltegak}$, and hence this natural isomorphism is a Lie algebra isomorphism.
\end{proof}

\begin{prop}[Theorem 8.44 : \textbf{Induced Lie Algebra Homomorphism.}]
Let $G$ and $H$ be Lie groups, and let $\glie$ and $\hlie$ be their Lie algebras. Suppose $F : G \to H$ is a Lie group homomorphism. For every $X \in \glie$, there is a unique vector field in $\hlie$ that is $F$-related to $X$. With this vector field denoted by $F_*X$, the map $F_* : \glie \to \hlie$ so defined is a Lie algebra homomorphism.
\end{prop}
\begin{proof}
Let $X \in \glie = \text{Lie}(G)$ of a Lie group $G$ with a Lie group homomorphism $F : G \to H$. If there is a vector field $Y \in \hlie = \text{Lie}(H)$ that is $F$-related to $X$, by definition it must satisfy
$$
dF_g(X_g) = Y_{F(g)} \quad \text{for any }g \in G.
$$
Particularly at $e \in G$, $dF_e(X_e) = Y_e$, therefore $Y$ is uniquely determined by
$$
Y = Y_e^{\Ltegak} = (dF_e(X_e))^{\Ltegak}
$$
(unique since the evaluation map is isomorphism).
To show that this $Y$ is $F$-related to $X$, that is $Y_{F(g)} = dF_g(X_g)$, compute $Y=Y^{\Ltegak}_e$ at $F(g)$ by definition
\begin{align*}
Y_{F(g)} &= Y_e^{\Ltegak}|_{F(g)} = d(L_{F(g)})_e(Y_e) = d(L_{F(g)})_e(dF_e(X_e)) \\ &= d(L_{F(g)} \circ F)_e (X_e).
\end{align*}
Now observe that for any $g_0 \in G$,
$$
(L_{F(g)} \circ F)(g_0) = L_{F(g)} (F(g_0)) = F(g)F(g_0) = F(gg_0) = F (L_g g_0) = (F \circ L_g)(g_0).
$$
Therefore
\begin{align*}
Y_{F(g)} &= d(L_{F(g)} \circ F)_e (X_e) = d(F \circ L_g)_e (X_e) \\
&= (dF_g \circ d(L_g)_e)(X_e) \\
&= dF_g(X_g). 
\end{align*}
Therefore $Y$ is $F$-related to $X$. For every $X \in \glie$, we denote this $F$-related vector field by $F_*X$. The map $F_* : \glie \to \hlie$ is a Lie algebra homomorphism since by naturality of Lie bracket (Corollary 8.31), for any $X_1,X_2 \in \glie$
$$
F_*[X_1,X_2] = [F_*X_1,F_*X_2].
$$ 
\end{proof}
\begin{remark}
Note that this theorem implies that for any $X \in \glie$, $F_*X$ is a well-defined smooth vector field on $H$, even though $F$ may not be a diffeomorphism (compare with Proposition 8.19).
\end{remark}

\begin{prop}[Proposition 8.45 : \textbf{Properties of Induced Homomorphism}]
The following are some properties of the induced Lie algebra homomorphism :
\begin{enumerate}
\item[(a)] The homeomorphism $(\Id_G)_* : \LieG \to \LieG $ induced by identity map $\Id_G : G \to G$ is the identity map on Lie$(G)$. 
\item[(b)] If $F_1 : G \to H$ and $F_2 : H \to K$ are Lie group homomorphisms, then
$$
(F_2 \circ F_1)_* = (F_2)_* \circ (F_1)_* : \LieG \to \Lie(K).
$$
\item[(c)] Isomorphic Lie groups have isomorphic Lie algebra.
\end{enumerate}
\end{prop}
\begin{proof}
To show $(a)$, let $X$ be any element of $\LieG$. The image of $X$ under homomorphism $(\Id_G)_* : \LieG \to \LieG$ is $(\Id_G)_* (X)$ a $\Id_G$-related vector field. The value at any $g \in G$ is
$$
\big((\Id_G)_*X\big)|_g = d(\Id_G)_g(X_g) = \Id_{T_gG} (X_g) = X_g
$$
by Proposition 3.6. Therefore $(\Id_G)_*$ is the identity of $\LieG$. For (b), the composition map $F_2 \circ F_1 : G \to K$ is a Lie group homomorphism, since for any $g,g_0 \in G$
$$
(F_2 \circ F_1)(gg_0) = F_2(F_1(gg_0)) = F_2(F_1(g)F_1(g_0)) = F_2(F_1(g)) F_2(F_1(g_0)).
$$
Hence the map $(F_2 \circ F_1)_* : \LieG \to \Lie(K)$ is a Lie group homomorphism that take any $X \in \LieG$ to $(F_2 \circ F_1)_*X$, the $(F_2 \circ F_1)$-related vector field in $\Lie(K)$. So at any $g\in G$ denote the image in $K$ as $k=F_2(F_1(g))$, we have
\begin{align*}
\big((F_2 \circ F_1)_*X\big)|_k &= d(F_2\circ F_1)_g (X_g) = (d(F_2)_{F_1(g)} \circ d(F_1)_g)(X_g) \\
&= d(F_2)_{F_1(g)} (F_{1*}X)|_{F_1(g)}\\
&= \big(F_{2*} (F_{1*}X)\big)|_{k}\\
&= \big( (F_{2*} \circ F_{1*})X \big)|_k
\end{align*}
Therefore $(F_2 \circ F_1)_* = F_{2*} \circ F_{1*}$. 
For (c), if $F : G \to H$ is an isomorphism, (a) and (b) imply that
$$
F_* \circ (F^{-1})_* = (F\circ F^{-1})_* = \Id = (F^{-1})_* \circ F_*.
$$
So $F_* : \LieG \to \Lie(K)$ is an isomorphism.
\end{proof}

\begin{prop}[Theorem 8.46 : \textbf{The Lie Algebra of a Lie subgroup}]
Suppose that $H \subhim G$ is a Lie subgroup, and $\iota : H \hookrightarrow G$ is the inclusion map.
There is a Lie subalgebra $\mathfrak{h} \subhim \LieG$ that is canonically isomorphic to $\Lie(H)$, characterized by either of the following descriptions:
\begin{align*}
\mathfrak{h} &= \iota_* (\Lie(H)) \\
&=\{ X \in \LieG : X_e \in T_eH  \}.
\end{align*}
\end{prop}
\begin{proof}
The inclusion map $\iota : H \hookrightarrow G$ is a Lie group homomorphism, since for any $h_1,h_2 \in H \subhim G$,
$$
\iota(h_1h_2) = h_1 h_2 = \iota(h_1)\iota(h_2).  
$$
This induce a Lie algebra homomorphism (by Theorem 8.44) 
$$
\iota_* : \Lie(H) \to \LieG.
$$
The image $\hlie = \iota_*(\Lie(H))$, is a Lie subalgebra of $\LieG$. Since by definition of $\iota_*$, for any $Y_1,Y_2 \in \iota_*(\Lie(H))$ there are $X_1,X_2 \in \Lie(H)$ such that $Y_i$ is $\iota$-related to $X_i$. So by naturality of Lie bracket,
$$
[Y_1,Y_2] = [\iota_*X_1,\iota_*X_2] = \iota_*[X_1,X_2] \in \iota_*(\Lie(H)).
$$ 
Now to show the equality of both characterization, let $X \in \iota_*(\Lie(H))$. By definition, $\iota_*(\Lie(H)) \subhim \LieG$, so $X \in \LieG$. Since by definition of $\iota_*$, $X_e = d\iota_e(v)$ for some $v \in T_eH$. Therefore $X_e \in T_eH$. Conversely if we have $X \in \LieG$ such that $X_e \in T_e H$, then there is a vector $v \in T_eH$ such that $X_e = d\iota_e(v)$. By evaluation map, we have left-invariant vector field $v^{\Ltegak}$ in $H$ such that $v^{\Ltegak}|_e = v$. The image of this vector field by $\iota_*$ is $\iota_*(v^{\Ltegak})$, which is by definition (Theorem 8.44) 
$$
\iota_*(v^{\Ltegak}) = \big(d\iota_e(v)\big)^{\Ltegak} = X_e^{\Ltegak}.
$$
But since the evaluation map is isomorphism, then $\iota_*(v^{\Ltegak}) = X \in \iota_*(\Lie(H))$.

The map $\iota_* : \Lie(H) \to \LieG$ is injective. For let $X,Y \in \Lie(H)$ such that $\iota_*(X) = \iota_*(Y)$. By definition, 
$$
(d\iota_e(X_e))^{\Ltegak} = \iota_*X = \iota_*Y = (d\iota_e(Y_e))^{\Ltegak}
$$
Since evaluation map is isomorphism and $d\iota_e$ is injective, then $X_e = Y_e$. Hence $X=Y$. Therefore $\iota_*$ is a isomorphism between $\Lie(H)$ and its image $\hlie$.
\end{proof}

\begin{prop}[Problem 8-4]
Let $M$ be a smooth manifold with boundary. Show that there exists a global smooth vector field on $M$ whose restriction to $\doo M$ is everywhere inward-pointing pointing, and one whose restriction to $\doo M$ is everywhere outward-pointing.
\end{prop}
\begin{proof}
Let $p$ be any point in $\doo M$ and $(U_p,x^i)$ be any boundary coordinate contain $p$. Define a local vector field $X_p : U_p \to TM$ as $X_p(x) = v^n_p \doo/\doo x^n|_x$ for a constant $v^n_p >0$ for any $x \in U_p$. This vector field is smooth and inward-pointing at $p$ by Proposition 5.41. Since $\doo M$ is a properly embedded submanifold of $M$, the collection $\{U_p : p \in \doo M\} \cup \{M \smallsetminus \doo M \}$ is an open cover of $M$. Let $\{\psi_p: p\in \doo M\} \cup \{\psi_0\}$ be the smooth partition of unity subordinate to this cover. For each $p \in \doo M$, define a global smooth vector field $\psi_p X_p : M \to TM$ by first restrict $\psi_p$ to $U_p$, multiplied by $X_p$, and then extend to all $M$ by gluing lemma. Define a vector field $\wtilde{X} : M \to TM$ as
$$
\wtilde{X}(x) = \sum_{p \in \doo M} (\psi_p X_p)(x). 
$$ 
Since supp $\psi_p$ is locally finite, then there is a neighbourhood of $x$ that intersect only finite number of support $\psi_p$. This means that the sum above has only finite number of nonzero terms, therefore this define a smooth function. Let $x \in \doo M$ arbitrary and $U$ be the neighbourhood such that the sum of the vector
$$
\wtilde{X}(x) = \sum_{p \in \doo M} \psi_p(x) X_p(x)
$$
has only finite number of nonzero terms. This neighbourhood is contain in one of the boundary coordinates $(U_p,x^i)$, and hence by restricting one of them to $U$, we obtain a boundary coordinate for $p$. In this coordinate, the $n$-th component of $\wtilde{X}_p$ is
$$
\wtilde{X}^n(x) = \sum_{p \in \doo M} \psi_p(x) X_p^n(x).
$$  
Since each $X_p^n(x)$ is inward-pointing and $\psi_p(x) >  0$, then $\wtilde{X}(x)$ is inward-pointing (by Proposition 5.41).
\end{proof}

\subsection{Chapter 9 (Integral Curves and Flows)}


\subsection{Chapter 10 (Vector Bundles)}
Here's an exercise from \cite{LeeJeff} that correspond to Exercise 10.1 \cite{LeeSM}. In this exercise, i'm using notation from \cite{WPoor}.
\begin{prop}[Proposition 1.5 \cite{WPoor}, Exercise 6.4 \cite{LeeJeff}]
The projection map $\pi : E \to M$ of a $C^{\infty}$ fibre bundle is a submersion. Each fibre $E_p = \pi^{-1}(p)$ is an embedded submanifold diffeomorphic to the standard fibre $F$ of the bundle. Furthermore, if $M$ and $F$ connected, then $E$ connected.
\end{prop}
\begin{proof}
Let $\xi \in E$ and $U$ be the open subset in $M$ contain $p = \pi(\xi)$. Take any bundle chart $(\pi,\varphi) : \pi^{-1}(U) \to U\times F$ over $U$ and consider the composition map $pr_1 \circ (\pi,\varphi) = \pi|_{\pi^{-1}(U)} : \pi^{-1}(U) \to U $.
\[
\begin{tikzcd}[column sep=small]
\pi^{-1}(U) \arrow{rr}{(\pi,\varphi)} \arrow[swap]{dr}{\pi}& &U \times F \arrow{dl}{pr_1}\\
& U & 
\end{tikzcd}
\]
Because the projection map $pr_1 : U \times F \to U$ is submersion on $ U \times F$ and $(\pi,\varphi)$ is a diffeomorphism (hence its differential is an isomorphism everywhere), then by chain rule
$$
d\pi_{\xi} = d(pr_1 \circ (\pi,\varphi))_{\xi} = d(pr_1)_{(p,v)} \circ d(\pi,\varphi)_{\xi}
$$
is surjective at $\xi$. Because $\xi$ arbitrary therefore $\pi$ is submersion. By Constant-Rank Level Set Theorem (Theorem 5.12 \cite{LeeSM}), each fibre $E_p=\pi^{-1}(p)$ is an embedded submanifold of $E$.  $E_p$ is diffeomorphic to $\{p\} \times F$ (hence to $F$), since $(\pi,\varphi)$ map $E_p$ diffeomorphically onto the embedded submanifold $\{p\} \times F$.

To show $E$ is connected if $M$ and $F$ connected, there are two alternative. First using contradiction. Suppose that $E$ is not connected, there exists a surjective continuous function $f:E\rightarrow \{0,1\}$ write$U=f^{-1}(0), V=f^{-1}(1)$, $E=U\cup V$ where $U$ and $V$ are not empty, let $\pi:E\rightarrow M$ the projection map. Remark that $\pi$ is an open map. This implies that $\pi(E)=M=\pi(U)\cup (V)$ is the  union of the open subsets $\pi(U)$ and $\pi(V)$ wich are not disjoint since $M$ is connected. Let $x\in \pi(U)\cap \pi(V)$, there exists $y_1\in U, y_2\in V$ such that $\pi(y_1)=\pi(y_2)=x$, since $\pi^{-1}(x)$ is connected, the restriction of $f$ to $\pi^{-1}(x)$ is constant, but $f(y_1)=0, f(y_2)=1$. Contradiction. 

The second is a direct approach using Theorem 26.15 \cite{Willard}  : If a topological space $X$ is connected and $\mathcal{U}$ is an open cover for $X$, then any two points can be connected by a simple chain consisting of elements of $\mathcal{U}$.

By local trivialization for each $p \in M$ we have an open subset $U \subset M$ containing $p$ and a diffeomorphism $\phi : \pi^{-1}(U) \rightarrow U \times F$. Let $\{\pi^{-1}(U)\}$ be the open cover for $E$. Because $F$ and $M$ connected, then  $U\subset M$ connected, $U \times F$ connected, $\phi^{-1}(U\times F) = \pi^{-1}(U)$ connected. With this open cover, we have a simple chain where each of its elements is connected (implies path-connectedness). By this we can easily make a path connecting $v,w \in E$ by joining the paths from each chain.
\end{proof}

After defining smooth vector bundle and look at some examples such as tangent bundle, Mobius bundle, and trivial bundle. We know that the non-trivial bundles requires more than one local trivializations. So now we want to know what is the transition function between those overlapping local trivializations look like. The following lemmas shows, that these functions has "simple form".

\begin{prop}[Lemma 10.5 : \textbf{Transition Function between Local Trivializations}]
Let $\pi : E \to M$ be a smooth vector bundle of rank $k$ over $M$. Suppose $\Phi : \pi^{-1}(U) \to U \times \rk$ and $\Psi : \pi^{-1}(V) \to V \times \rk$ are two smooth local trivializations of $E$ with $U \cap V \neq \emptyset$. There exists a smooth map $\tau : U \cap V \to \GLsaja(k,\er)$ such that the composition $\Phi \circ \Psi^{-1} : (U\cap V)\times \rk \to (U\cap V)\times \rk$ has the form 
$$
\Phi \circ \Psi^{-1} (p,v) = (p,\tau(p)v),
$$
where $\tau(p)v$ denote the usual action of the $k\times k$ matrix $\tau(p)$ on the vector $v \in \rk$.
\end{prop}
\begin{proof}
With the overlapping local trivializations $\Phi$ and $\Psi$ as above, we have the following commute diagram :
\[
\begin{tikzcd}[column sep=small]
(U\cap V)\times \rk \arrow[leftarrow]{r}{\Psi} \arrow[swap]{dr}{\pi_1} & \pi^{-1}(U \cap V) \arrow{r}{\Phi} \arrow[swap]{d}{\pi} & (U\cap V)\times \rk \arrow{dl}{\pi_1} \\
&U\cap V &
\end{tikzcd}
\]
where the maps on the top are the restriction the local trivializations $\Psi$ and $\Phi$ to $\pi^{-1}(U\cap V)$. Since $\pi^{-1}(U\cap V)$ is an open subset of $\pi^{-1}(U)$ and $\pi^{-1}(V)$ and $(U\cap V)\times \rk$ is open in $U\times \rk$ and $V\times \rk$, the restriction is still diffeomorphism. Therefore the composition map $\Phi \circ \Psi^{-1}$ is a diffeomorphism on $(U\cap V)\times \rk$, where
$$
\Phi \circ \Psi^{-1}(p,v) = (\delta(p,v), \sigma(p,v)),
$$ 
for some smooth functions $\delta = \pi_1 \circ (\Phi \circ \Psi^{-1}) : (U\cap V)\times \rk \to U\cap V $ and $\sigma = \pi_2 \circ (\Phi \circ \Psi^{-1}) : (U\cap V)\times \rk \to \rk$.

 By above diagram, the composition map $\Phi \circ \Psi^{-1} : (U\cap V)\times \rk \to (U\cap V)\times \rk$ satisfy
$$
\delta(p,v) = \pi_1 \circ (\Phi \circ \Psi^{-1})(p,v) = \pi_1(p,v)  = p.
$$
So we have 
$$
\Phi \circ \Psi^{-1}(p,v) = (p, \sigma(p,v)).
$$
Since the restriction of $\Psi$ and $\Phi$ to a fibre is an isomorphism, then by fixing $p$, the map $v \mapsto \sigma(p,v)$ is a invertible linear map on $\rk$. So there is a $k\times k$ matrix $\tau(p)$ such that $\sigma(p,v) = \tau(p)v$. So we have a map $\tau : U \cap V \to \GLsaja(k,\er)$. To show that this map is smooth, choose a basis $\{E_1,\dots,E_k\}$ for $\rk$. We can write the components functions of $\tau(p)$ as
$$
\tau^i_j(p) = (\pi^i \circ \sigma) (p,E_j),
$$
where $\pi^i : \rk \to \er$ defined as $\pi^i(v) = \pi^i (v^jE_j) = v^i$. This map is smooth since $\pi^i$ and $\sigma$ are smooth. Since these component functions is a global coordinates for $\GLsaja(k,\er)$ therefore $\tau$ is smooth.
\end{proof}

Suppose we have a smooth manifold $M$ and for each  point  $p\in M$ we are given a $k$-dimensional vector spaces $E_p$. By taking their disjoint union $E := \bigsqcup_{p\in M} E_p$, we have the basic ingredients to construct a smooth vector bundle
$$
E =  \bigsqcup_{p\in M} E_p \xrightarrow{\text{ }\pi \text{ } } M
$$
with $\pi$ is the natural projection from each fiber $E_p$. But to do this, we have to construct the \textbf{manifold topology} and \textbf{smooth structure} on $E$, and then construct the \textbf{local trivialization} for it. 

The next lemma provides the shortcut, by showing that it is sufficient to construct the local trivializations, as long as they overlap with smooth transition functions. So while the previous proposition tells us that
$$
\begin{matrix}
\text{Smooth rank-$k$ VB} 
\end{matrix} \xrightarrow{\quad\text{has}\quad }
\begin{matrix}
\text{Smooth Transition Function} \\
\tau(p) \in \GLsaja(k,\er)
\end{matrix}
$$ 
The next proposition tells us that
$$
\begin{matrix}
(E,\pi,M) \text{ with} \\
\text{(candidate) Smooth TF } \tau(p)
\end{matrix} \xrightarrow{\quad \text{formed} \quad }
\begin{matrix}
\text{Smooth VB}
\end{matrix}
$$ 
\begin{prop}[Lemma 10.6 : \textbf{Vector Bundles Chart Lemma}]
Let $M$ be a smooth manifold with or without boundary, and suppose that for each $p \in M$ we are given a real vector space $E_p$ of some fixed dimension $k$. Let $E = \bigsqcup_{p\in M} E_p$, and let $\pi : E \to M$ be the map that takes each element of $E_p$ to the point $p$. Suppose furthermore that we are given the following data:
\begin{enumerate}[nolistsep]
\item[(i)] an open cover $\{U_{\alfa}\}_{\alfa \in A}$ of $M$
\item[(ii)] for each $\alpha \in A$, a bijective map $\Phi_{\alpha} : \pi^{-1}_{\alfa} \to U_{\alfa} \times \rk $ whose restriction to each $E_p$ is a vector space isomorphism from $E_p$ to $\{p\} \times \rk \isomorphic \rk$
\item[(iii)] for each $\alfa, \beta \in A$ with $U_{\alfa} \cap U_{\beta} \neq \emptyset$, a smooth map $\tau : \Ualpha \cap \Ubeta \to \GLsaja(k,\er)$ such that the map $\Phi_{\alfa} \circ \Phi_{\beta}^{-1} $ from $(\Ualpha \cap\Ubeta)\times \rk$ to itself has the form 
$$
\Phi_{\alfa} \circ \Phi_{\beta}^{-1} (p,v) = (p,\tau_{\alfa\beta}(p) v)
$$
\end{enumerate}
Then $E$ has unique topology and smooth structure making it into a smooth manifold with or without boundary and a smooth rank-$k$ vector bundle over $M$, with $\pi$ as a projection $\{\Ualpha, \Phi_{\alfa}\}$ as smooth local trivialization.
\end{prop}
\begin{proof}
To be added.
\end{proof}
An example of the application of above lemma is the following.
\begin{prop}[Example 10.8 : \textbf{Restriction of a Vector Bundle}]
Suppose we have a rank-$k$ vector bundle $\pi : E \to M$. Let $S \subhim M$ be any subset. We define the \textbf{restriction of }$E$ \textbf{ to }$S$ to be the set $E|_S = \bigcup_{p \in S} E_p$ with the projection $\pi|_S : E|_S \to S$ obtain by restricting $\pi$. We will verify two things :
\begin{enumerate}[nolistsep]
\item[(1)] $E|_S \to S$ is a rank-$k$ vector bundle.
\item[(2)] If $S\subhim M$ is an immersed submanifold and $E \to M$ is a smooth rank-$k$ vector bundle, then $E|_S \to S$ is a smooth rank-$k$ vector bundle.
\end{enumerate}
To see (1), note that :
\begin{enumerate}[nolistsep]
\item[(a)] $E|_S$ and $S\subhim M$ are topological spaces with subspace topology. 
\item[(b)] The restriction of $\pi : E \to M$ to the domain $E|_S$ and codomain $S$, $\pi|_S : E|_S \to S$ is surjective and continous.
\item[(c)] Each $E_p$, for $p\in S$ is a $k$-dimensional vector space. 
\item[(d)] For each $p \in S\subhim M$, (by VB structure on $M$), we have open subset $U \subhim M$ and homeomorphism $\Phi : \pi^{-1}(U) \to U \times \rk$ such that $\pi_U \circ \Phi = \pi$  and the restriction to each fiber $E_p$ is a vector space isomorphism. Since $S$ is subspace of $M$, we have open subset $U \cap S$ containing $p$. The restriction map 
$$
\Phi|_U : (\pi|_S)^{-1}(U\cap S) \to (U\cap S)\times \rk
$$
is a bijective map, and with the domain and codomain endowed with subspace topology of $\pi^{-1}(U)$ and $U\times \rk$ respectively, it is a homeomorphism, and of course it is a vector space isomorphism when restricted to each fiber.
\end{enumerate}  
To show (2), that is $\pi_S : E|_S \to S$ is a smooth vector bundle, we only need to verify that hypothesis (i),(ii) and (iii) in the \textit{Vector Bundle Chart Lemma} are all satisfied. Note that :
\begin{enumerate}[nolistsep]
\item[(i)] If $\{ U_{\alpha} \}_{\alpha \in A}$ are open cover for $M$, then  $\{V_{\alpha} \, | \, V_{\alpha} = U_{\alpha} \cap S\}_{\alpha \in A}$ are open cover for $S$.
\item[(ii)] The maps $\Phi_{\alpha}|_{U_{\alpha}} : (\pi|_S)^{-1}(U_{\alpha} \cap S) \to (U_{\alpha} \cap S) \times \rk$ are bijective. To see this, let $U$ be any elements in $\{U_{\alpha}\}$ such that $U \cap S \neq \emptyset$ and $\Phi : \pi^{-1}(U) \to U \times \rk$ be the local trivialization for $E\to M$ over $U$. Since $(\pi|_S)^{-1}(U \cap S) = \bigcup_{p \in U \cap S} E_p = \pi^{-1}(U \cap S)$ and $\pi = \pi_U \circ \Phi$, then
$$
\pi_U \circ \Phi ((\pi|_S)^{-1}(U \cap S)) = \pi_U \circ \Phi (\pi^{-1}(U \cap S)) = \pi \circ \pi^{-1}(U \cap S) = U \cap S.
$$
Also since $\Phi|_{E_p}$ is an isomorphism, therefore $\Phi ((\pi|_S)^{-1}(U \cap S)) = (U\cap S) \times \rk$. Hence the restriction map $\Phi|_{U} : (\pi|_S)^{-1}(U \cap S) \to (U\cap S) \times \rk$ is bijective.
\item[(iii)] For any $V_{\alpha} \cap V_{\beta} \neq \emptyset$, the smooth map $\widetilde{\tau}_{\alpha \beta} : V_{\alpha} \cap V_{\beta} \to \GLsaja(k,\er)$ is smooth and the map $\Phi_{\alpha}|_{U_{\alpha}} \circ \Phi_{\beta}|_{U_{\beta}}^{-1} : (V_{\alpha} \cap V_{\beta})\times \rk \to (V_{\alpha} \cap V_{\beta})\times \rk$ has the form
\begin{equation}\label{transition map VB}\tag{$\star$}
\Phi_{\alpha}|_{U_{\alpha}} \circ \Phi_{\beta}|_{U_{\beta}}^{-1} (p,v) = (p, \widetilde{\tau}_{\alpha \beta} (p) v).
\end{equation}
To see this, we know that (iii) is satisfied by local trivializations of $E \to M$. First we have to show that $\widetilde{\tau}_{\alpha \beta}$ is smooth. This map is just the restriction of $\tau_{\alpha\beta} : U_{\alpha} \cap U_\beta \to \GLsaja(k ,\er)$ to $V_{\alpha} \cap V_{\beta} = (U_{\alpha} \cap S) \cap (U_{\beta} \cap S) = ( U_{\alpha} \cap U_\beta) \cap S$. Since $( U_{\alpha} \cap U_\beta) \cap S$ is an immersed submanifold of $U_{\alpha} \cap U_\beta$ (shown below) therefore $\widetilde{\tau}_{\alpha \beta} = \tau_{\alpha \beta}|_{V_{\alpha} \cap V_{\beta}}$ is smooth by Theorem 5.27. The form (\ref{transition map VB}) follows since all the maps are just restriction maps. \textit{Showing $( U_{\alpha} \cap U_\beta) \cap S \subhim  U_{\alpha} \cap U_\beta$ is an immersed submanifold :} Since $S \subhim M$ is an immersed submanifold, then by definition the inclusion map $\iota : S \hookrightarrow M$ is an injective smooth immersion. For simplicity, replace $U_{\alpha} \cap U_{\beta}$ with an open subset $U \subhim M$. The subsets $U \subhim M$ and $\iota^{-1}(U) \subhim S$ are both open subset and hence an embedded submanifold. The restriction of the smooth map $\iota : S \hookrightarrow M$ to the domain $E = \iota^{-1}(U) \subhim S$ and the codomain $U \subhim M$ (which are both embedded submanifold) is smooth. Denote such map as $\iota|_E : E \to U$. Since this map also still injective immersion and the image $\iota|_E (E) = \iota (\iota^{-1}(U)) = U \cap \iota(S) = U \cap S$, then $U \cap S $ is an immersed submanifold of $U$ by Proposition 5.18.
\end{enumerate}  
\end{prop}

\begin{prop}[Exercie 10.9 : \textbf{Zero Section is a continous (smooth) section}]
Show that the zero section of every vector bundle is continous and the zero section of every smooth vector bundle is smooth.
\end{prop}
\begin{proof}
The zero section $\xi$ is a global section  $\xi : M \to E$ defined by $\xi(p) = 0 \in E_p$. Suppose $p \in M$, by local trivialization we have the following composition 
$$
\Phi \circ \xi : U \to U \times \rk
$$
where $\Phi \circ \xi (p) = (p,0)$, which is a continous (smooth) map on the open set $U$. Since $\phi$ is a homeomorphism (diffeomorphism) then we can write
$$
\xi = \Phi^{-1} \circ (\Phi \circ \xi) 
$$  
which is a composition of continous (smooth) map. Therefore $\\xi$ is continous (smooth) on $U$. Since this is true locally for any point in $M$, then $\xi$ is continous (smooth) on $M$.
\end{proof}

\begin{prop}[Exercie 10.14]
Let $\pi : E \to M$ be a smooth vector bundle. Show that each element of $E$ is in the image of a smooth global section.
\end{prop}
\begin{proof}
Let $v$ be any element in $E_p$. Define a rough section $\sigma : \{p\} \to E$, defined as $\sigma(p) = v_p$. This section (on a closed set $\{p\}$) is smooth since we can extend it to a smooth local section on a neighbourhood of $p$. We do this as follows : Choose a local trivialization on some open subset $U$ containing $p$
$$
\Phi : \pi^{-1}(U) \to U \times \rk.
$$
Since $\sigma(p) = v_p \in E_p \subhim \pi^{-1}(U)$, then $\Phi (v_p) = (p,\bar{v})$ for an element $\bar{v} \in \rk$. Since $\Phi$ is a diffeomorphism, the map
$$
\Phi^{-1}|_{U\times \{\bar{v}\}} : U\times \{\bar{v}\} \to \pi^{-1}(U)
$$
is a restriction of smooth map $\Phi^{-1}$ to embedded submanifold $U \times \{\bar{v}\}$, hence it is a smooth map. By composing this with inclusion map $\iota : U \hookrightarrow U \times \{\bar{v}\}$ and $\iota' : \pi^{-1}U \hookrightarrow E$, we have a smooth map
$$
\sigma_U : =  \iota' \circ \Phi^{-1}|_{U\times \{\bar{v}\}} \circ \iota : U \to  E.
$$
This map is a smooth local section, since by property of local trivialization,
$$
\pi \circ \sigma_U(q) = \pi \circ \Phi^{-1} (q,\bar{v}) = \pi_U (q,\bar{v}) = q, \quad \forall q \in U
$$
Also $\sigma_U$ agree with $\sigma_p$ at $\{p\}$ by construction,
$$
\sigma_U(p) = \Phi^{-1} (p,\bar{v}) = v_p.
$$
Hence this map is a smooth local extension of $\sigma_p :\{p\} \to E$ on the neighbourhood $U$. It follows by Lemma 10.12 that there exists smooth global section $\widetilde{\sigma} : M \to E$ such that $\widetilde{\sigma}|_{\{p\}} = \sigma_p$ and supp $\widetilde{\sigma} \subhim U$. Hence $v_p \in E_p \subhim E$ is in the image of smooth global section $\widetilde{\sigma}$.
\end{proof}

\begin{prop}[Exercise 10.16 : Prove Proposition 10.15, \textbf{Completion of Local Frames for Vector Bundles}]
Suppose $\pi :E \to M$ is a smooth vector bundle of rank-$k$.
\begin{enumerate}
\item[(a)] If $(\sigma_1,\dots,\sigma_m)$ is a linearly independent $m$-tuple of smooth local sections of $E$ over an open subset $U \subhim M$, with $1 \leq m < k$, then for each $p \in U$ there exists smooth sections $\sigma_{m+1}, \dots , \sigma_k$ defined on some neighbourhood $V$ of $p$ such that $(\sigma_1,\dots, \sigma_k)$ is a smooth local frame for $E$ over $U \cap V$.
\item[(b)] If $(v_1,\dots,v_m)$ is a linearly independent $m$-tuple of elements of $E_p$ for some $p \in M$, with $1 \leq m \leq k$, then there exists smooth local frame $(\sigma_i)$ for $E$ over some neighbourhood of $p$ such that $\sigma_i(p) = v_i$ for $i=1,\dots,m$. 
\item[(c)] If $A \subhim M$ is a closed subset and $(\tau_1,\dots,\tau_k)$ is a linearly independent $k$-tuple of sections of $E|_A$ that are smooth in the sense described in Lemma 10.12, then there exists a smooth local frame $(\sigma_1,\dots,\sigma_k)$ for $E$ over some neighbourhood of $A$ such that $\sigma_i|_A = \tau_i$ for $i=1,\dots,k$.
\end{enumerate}
\end{prop}
\begin{proof}
For (a), it is more or less similar to its counterpart in Proposition 8.11. I'll add this later.

For (b), choose $(v_{m+1},\dots ,v_k)$ elements of $E_p$ such that $(v_1,\dots,v_k)$ is a basis for $E_p$.  Let $U\subhim M$ be any open subset containing $p$ and $\Phi : \pi^{-1}(U) \to U \times \rk$ be a smooth local trivialization. For any vector $v_i \in E_p \subhim \pi^{-1}(U)$, 
$$
\Phi : v_i \mapsto (p,(v^1_i,\dots,v^k_i)) \in U \times \rk.
$$
Define a smooth map $\widetilde{\sigma}_i : U \to U\times \rk$ as $\widetilde{\sigma}_i (p) = (p,(v_i^1 ,\dots , v_i^k))$. By this, we can define a smooth local frame $\sigma_1,\dots,\sigma_k : U \to E$ by
$$
\sigma_i (p) = (\Phi^{-1} \circ \widetilde{\sigma}_i) (p)= \Phi^{-1} (p,(v^1_i,\dots,v^k_i)) ,\quad \forall p\in U. 
$$
And since $\pi_1 \circ \Phi = \pi$, we have 
$$
\pi \circ \sigma_i (p) = \pi \circ \Phi^{-1}  (p,(v^1_i,\dots,v^k_i)) = \pi_1 (p,(v^1_i,\dots,v^k_i)) = p.
$$
Hence $\sigma_i$ is a local section.
\[
\begin{tikzcd}[row sep=large,column sep=tiny]
\pi^{-1}(U) \ar[rr,"\Phi"] \ar[dr,"\pi"] & & U \times \rk \ar[dl,swap,"\pi_1"] \\
& U \ar[ul, bend left=30, "\sigma_i"] \ar[ur, bend right=30, swap, "\widetilde{\sigma}_i"] & 
\end{tikzcd}
\] 
The sections $\sigma_1,\dots,\sigma_k$ are linearly independent at any point $p \in U$, since their image under isomorphism $\Phi|_{E_p} : E_p \to \{p\} \times \rk \isomorphic \rk$ are linearly independent vectors in $\rk$ by construction. Hence $(\sigma_i)$ is a local frame on the neighbourhood of $p$ such that $\sigma_i(p) = v_i$.

For (c), let $\tau_1,\dots,\tau_k : A \to E$ be a linearly independent $k$-tuple of smooth sections of $E|_A$. By applying Lemma 10.12, we have $k$-tuple of global smooth sections $\widetilde{\tau}_1,\dots,\widetilde{\tau}_k : M \to E$ such that $\widetilde{\tau}_i|_A = \tau_i$ and supp $\widetilde{\tau}_i \subhim U_i$ for some open subset $U_i$ contain $A$. By restrict each $\widetilde{\tau}_i$ to $U= \bigcap_{i=1}^k U_i$ we have a $k$-tuple of local sections on $U$ contain $A$. Since $(\widetilde{\tau}_i)$ agree with $(\tau_i)$ on $A$, then $(\widetilde{\tau}_i)$ are linearly independent at any $p \in A$. Let $\Phi : \pi^{-1}(U) \to U \times \rk$ be a local trivialization over $U$. So at any $p\in A$, we have
$$
\Phi \circ \widetilde{\tau}_i (p) = (p, (\widetilde{\tau}_i^1(p),\dots,\widetilde{\tau}_i^k(p)),
$$ 
for some smooth functions $\widetilde{\tau}_i^j : U \to \rk$. Then we have linearly independent $k$-tuple of vectors in $\rk$. By continuity, there are neighbourhood $V_p$ of $p$ such that $(\widetilde{\tau}_i)$ are linearly independent there. Since this holds for every point $p \in A$, the restrictions of the local sections $(\widetilde{\tau}_i)$ to $V = \bigcup_{p \in A} V_p$ is a local frame that agrees with $\tau_i$ on $A$. 
\end{proof}

\textbf{Local Frames Associated to Local Trivializations.} When we have a local trivialization of $E$ over some open subset $U$, 
$$
\Phi : \pi^{-1}(U) \to U \times \rk,
$$
we can define a local frame $\sigma_1,\dots,\sigma_k : U \to E$ as $\sigma_i(p) = \Phi^{-1} \circ \widetilde{e}_i (p) = \Phi^{-1}(p,e_i)$, where $\{e_i\}$ are the standard basis for $\rk$ :
\[
\begin{tikzcd}[row sep=large,column sep=tiny]
\pi^{-1}(U) \ar[rr,"\Phi"] \ar[dr,"\pi"] & & U \times \rk \ar[dl,swap,"\pi_1"] \\
& U \ar[ul, bend left=30, "\sigma_i"] \ar[ur, bend right=30, swap, "\widetilde{e}_i"] & 
\end{tikzcd}
\] 
We can easily check that $\sigma_i$ are linearly independent smooth section by above digram. However, the following proposition shows that the converse is true.
\begin{prop}
\textbf{Every smooth local frame for a smooth vector bundle is associated with a smooth local trivialization.} 
\end{prop}
\begin{proof}
Suppose that $\pi :E \to M$ is a smooth vector bundle and $(\sigma_i)$ is a smooth local frame for $E$ over some open subset $U \subhim M$. If there is a local trivialization $\Phi : \pi^{-1}(U) \to U \times \rk$ such that $(\sigma_i)$ is a local frame associated to it, that is they satisfy
$$
\sigma_i(p) = \Phi^{-1} \circ \widetilde{e}_i(p) = \Phi^{-1}(p,e_i),
$$
then for any element $v^ie_i \in \rk$,
$$
\Phi^{-1} (p,(v^1,\dots,v^k)) = v^i\sigma_i(p).
$$
So we define a map $\Psi : U \times \rk \to \pi^{-1}(U)$ by
$$
\Psi (p,(v^1,\dots,v^k)) = v^i \sigma_i(p),
$$
and then show that this map is indeed a local trivialization over $U$. First note that this map is bijective, since it sends basis $\{e_i\}$ of $\rk$ to basis $\sigma_i(p)$ of $E_p$. Since this map is \textbf{bijective}, to show that it is a \textbf{diffeomorphism} it suffice to show that it is a \textbf{local diffeomorphism}.   

Let $q \in U$ and $V$ be its neighbourhood in $M$ where there exists a smooth local trivialization $\Phi : \pi^{-1}(V) \to V \times \rk$. By shrinking $V$ if necessary, we assume that $V \subhim U$. Since $\Phi$ is a diffeomorphism, if we can show that $\Phi \circ \Psi|_{V \times \rk} : V \times \rk \to V \times \rk$ is a diffeomorphism, then it follows that the restriction $\Psi|_{V \times \rk} : V \times \rk  \to \pi^{-1}(V)$ is a diffeomorphism, hence $\Psi$ is a local diffeomorphism
\[
\begin{tikzcd}[row sep=large]
V \times \rk \arrow[r,"\Psi|_{V \times \rk}"] \arrow[dr,"\pi_1",swap] & \pi^{-1}(V) \arrow[d,"\pi",swap] \arrow[r,"\Phi"] & V \times \rk \arrow[dl,"\pi_1"] \\
&V&.
\end{tikzcd}
\]
For any element $(p,(v^1,\dots,v^k)) \in V \times \rk$,
$$
\Phi \circ \Psi(p,(v^1,\dots,v^k)) = \Phi (v^i\sigma_i(p)).
$$
Since $(\sigma_i)$ are smooth sections, then $\Phi \circ \sigma_i|_V : V \to V \times \rk$ is a smooth map. Thus there are smooth functions $\sigma_i^1,\dots,\sigma_i^k : V \to \er$ such that 
$$
\Phi \circ \sigma_i(p) = \Phi(\sigma_i(p)) = (p,(\sigma_i^1(p),\dots,\sigma_i^k(p)).
$$
Therefore
$$
\Phi \circ \Psi (p,(v^1,\dots,v^k)) = \Phi (v^i\sigma_i(p)) = (p,(v^i\sigma_i^1(p),\dots,v^i\sigma_i^k(p)),
$$
which is clearly smooth. To show that its inverse $(\Phi \circ \Psi)^{-1}$ is also smooth, we need to finc the expression for it. Suppose that for any $(p,(w^1,\dots,w^k)) \in V \times \rk$,
$$
(\Phi \circ \Psi)^{-1}(p,(w^1,\dots,w^k)) = (p,(\overline{w}^1,\dots,\overline{w}^k)),
$$
then
\begin{align*}
(p,(w^1,\dots,w^k)) &= (\Phi \circ \Psi) \circ (\Phi \circ \Psi)^{-1}(p,(w^1,\dots,w^k)) \\
&=(\Phi \circ \Psi) (p,(\overline{w}^1,\dots,\overline{w}^k))\\
&= (p,(\overline{w}^i \sigma_i^1(p),\dots,\overline{w}^i\sigma_i^k(p)),
\end{align*}
which is implies that $w^i = \sigma^i_j(p) \overline{w}^j$ for all $i=1,\dots,k$. Note that matrix $(\sigma^i_j(p))$ is invertible for each $p$, since $(\sigma_i)$ is a basis for $E_p$. So we have $\overline{w}^i = \tau^i_j(p) w^j$, where $(\tau^i_j(p))$ is the inverse matrix of $(\sigma^i_j(p))$. The entries ofthis matrix is a polynomial function of entries of $(\sigma_j^i(p))$, hence the entries is a smooth function on $V$. Therefore we can write the inverse map $(\Phi \circ \Psi)^{-1}$ as
$$
(\Phi \circ \Psi)^{-1}(p,(w^1,\dots,w^k)) = (p,(w^i\tau_i^1(p),\dots,w^i\tau_i^k(p)))
$$
which is a smooth map.
\end{proof}

\begin{prop}[Problem 10.11 : Prove Proposition 10.26]
\textbf{Proposition 10.26. } Suppose $E$ and $E'$ are smooth vector bundles over smooth manifold $M$ with or without boundary, and $F : E \to E'$ is a bijective smooth bundle homomorphism over $M$. Then $F$ is a smooth bundle isomorphism.
\end{prop}
\begin{proof}
To show that $F : E \to E'$ is a smooth bundle isomorphism, we need to show that its inverse $F^{-1} : E'\to E$ is also a smooth vector bundle homomorphism. First note that since $\pi' \circ F = \pi$, then $\pi \circ F^{-1} = \pi'$. The restriction $F^{-1}|_{E'_p} : E'_p \to E_p$ is linear since $F|_{E_p} : E_p \to E'_p$ is a linear bijective map. Since, for any $v_1',v_2' \in E_p'$ there are $v_1,v_2 \in E_p$ such that $F|_{E_p} (v_i) = v_i'$. Therefore
\begin{align*}
F^{-1}|_{E_p'} (av_1'+bv_2')&= F^{-1} (aF(v_1) + bF(v_2))\\ &= F^{-1}F(av_1+bv_2) \\ &= av_1 +b v_2 \\ &= aF^{-1}|_{E_p'}(v_1') + b F^{-1}|_{E_p'}(v_2')
\end{align*}
for any $a,b \in \er$. To show $F^{-1}$ is a smooth map, it is sufficient if we shows that $F^{-1}$ is smooth locally. Let $v'$ be any element of $E'_p$. Since $E$ and $E'$ are smooth vector bundles over $M$, there are some neighbourhoods $U,U'\subhim M$ of $p$, and local trivializations $\Phi : \pi^{-1}(U) \to U \times \rk$ and $\Phi' : \pi'^{-1}(U') \to U' \times \rk$. By replace $U$ and $U'$ with their intersection we may assume that  $U=U' \subhim M$. The following diagram is commute:
\[
\begin{tikzcd}[column sep=small]
\pi'^{-1}(U) \arrow[rr,"F^{-1}"] \arrow[dr,"\Phi'"] \arrow[ddr,"\pi'",swap] & & \pi^{-1}(U) \arrow[dl,"\Phi",swap] \arrow[ddl,"\pi"]  \\
& U \times \rk \arrow[d,"\pi_1",swap] & \\
&U&.
\end{tikzcd}
\]
For any $(p,v) \in U \times \rk$,
$$
(\Phi \circ F^{-1} \circ \Phi'^{-1}) (p,v) = (\sigma(p,v), \xi(p,v)) \in U \times \rk,
$$
for some functions $\sigma : U \times \rk \to U$ and $\xi : U \times \rk \to \rk$. Observe that
\begin{align*}
\sigma(p,v) &= \pi_1 \circ (\Phi \circ F^{-1} \circ \Phi'^{-1}) (p,v)\\ &= \pi \circ F^{-1} \circ \Phi'^{-1} (p,v) \\ &= \pi' \circ \Phi'^{-1} (p,v) \\ &= \pi_1 (p,v) \\ &= p.
\end{align*}
The restriction $(\Phi \circ F^{-1} \circ \Phi'^{-1})|_{\{p\} \times \rk} : \{p\} \times \rk \to \{p\} \times \rk$ is an invertible linear map. So there are $k\times k$ matrix $\tau(p) \in \GLsaja(k,\er)$ such that   
$$
(\Phi \circ F^{-1} \circ \Phi'^{-1}) (p,v) = (p, \tau(p)v).
$$
Also, since $\Phi' \circ F \circ \Phi^{-1} = (\Phi \circ F^{-1} \circ \Phi'^{-1})^{-1}$ is smooth and has the same form as above, then
\begin{align*}
(p,v) &= (\Phi' \circ F \circ \Phi^{-1}) \circ (\Phi \circ F^{-1} \circ \Phi'^{-1}) (p,v)\\ &= (\Phi' \circ F \circ \Phi^{-1}) (p,\tau(p)v)\\ &= (p, \widetilde{\tau}(p) \tau(p) v),
\end{align*}
for a smooth function $\widetilde{\tau} :U \to GL(k,\er)$. This implies that $\tau(p)$ is the inverse of $\widetilde{\tau}(p)$, and hence $\tau : U \to \GLsaja(k,\er)$ is a smooth map. Because $\Phi$ and $\Phi'$ are diffeomorphism, then $F^{-1}$ is smooth on $U$. This proves that $F^{-1}: E' \to E$ is a smooth map.
\end{proof}

\begin{prop}[Exercise 10.27]
Show that a smooth rank-$k$ vector bundle over $M$ is smoothly trivial if and only if it is smoothly isomorphic over $M$ to the product bundle $M \times \rk$.
\end{prop}
\begin{proof}
If a smooth rank-$k$ vector bundle $\pi : E \to M$ is smoothly trivial, then we have a global trivialization $\Phi : E \to M\times \rk$, which is by definition a bijective smooth bundle homomorphism. Hence $\Phi$ is a smooth bundle isomorphism by Proposition 10.26. Therefore $E$ is smoothly isomorphic over $M$ to the product bundle $M \times \rk$. Conversely, if $E$ is smoothly isomorphic over $M$ to $M \times \rk$, then we can choose the isomorphism between them as the global trivialization for $E$.
\end{proof}

\begin{prop}[Example 10.28 : \textbf{Bundles Homomorphisms.}] The following are the examples of smooth bundle homomorphism:

\begin{enumerate}[nolistsep]
\item[(a)] For a smooth map $F : M \to N$, the global differential $dF : TM \to TN$ is a smooth bundle homomorphism covering $F$.
\[
\begin{tikzcd}
TM \arrow[r,"dF"] \arrow[d,"\pi_M",swap] & TN \arrow[d,"\pi_N"] \\
M \arrow[r,"F"] & N
\end{tikzcd}
\]
The map $dF$ is smooth, since the representation on the natural coordinates for $TM$ and $TN$ is
$$
dF (x^1,\dots,x^m,v^1,\dots,v^m) = \Big( F^1(x),\dots,F^n(x), \frac{\doo F^1}{\doo x^i}(x) v^i, \dots, \frac{\doo F^n}{\doo x^i}(x) v^i \Big)
$$
which is clearly a smooth map. The restriction of $dF$ to the fiber is just the pointwise differential $dF_p : T_pM \to T_{F(p)} N$, which is a linear. It easy to check that $F \circ \pi_M = \pi_N \circ dF$.

\item[(b)] If $E \to M$ is a smooth vector bundle and $S \subhim M$ is an immersed submanifold with or without boundary, then the inclusion map $E|_S \hookrightarrow E$ is a smooth bundle homomorphism covering $\iota : S \hookrightarrow M$.
\[
\begin{tikzcd}
E|_S \arrow[r,"\iota'",hook] \arrow[d,"\pi|_S",swap] & E \arrow[d,"\pi"] \\
S \arrow[r,"\iota",hook] & M
\end{tikzcd}
\]
The map $\iota'$ smooth. To see this, let $v \in E_p \subhim E|_S \subhim E$ and $U \subhim M$ be a neighbourhood of $p$ such that $\Phi : \pi^{-1} (U) \to U \times \rk$ is a local trivialization of $E$ over $U$. By definition of $E|_S$,  the restriction $\Phi|_U : (U \cap S) \to (U\cap S)\times rk$ is a local trivialization of $E|_S$ over $U \cap S$. The composition map $\Phi \circ \iota' \circ \Phi|_U^{-1}$ is smooth since it has the following form.
\[
\begin{tikzcd}[column sep=large]
(\pi|_S)^{-1}(U\cap S) \arrow[r,"\iota'"] & \pi^{-1}(U) \arrow[d,"\Phi"] \\
(U \cap S) \times \rk \arrow[u,"\Phi|_U^{-1}"] \arrow[r,"\Phi \circ \iota' \circ \Phi|_U^{-1}"] & U \times \rk
\end{tikzcd}
\]  
By noting that $\iota'$ is just an identity map on the fiber, then for any $(q,(v^1,\dots,v^k) ) \in (U \cap S) \times \rk$,
\begin{align*}
(\Phi \circ \iota' \circ \Phi|_U^{-1}) (q,(v^1,\dots,v^k)) = \Phi \circ \iota' (v) = \Phi (v) = (q,(v^1,\dots,v^k))
\end{align*}
which is an identitiy map. Hence $\iota'$ is smooth. The restriction to each fiber $E_p$, $p \in S$, is just an identity map, hence linear. Also, it is easy to see that $\iota \circ  \pi|_S = \pi \circ \iota'$. 
\end{enumerate}
\end{prop}

\textbf{Notes for page 262} A smooth bundle homomorphism $F : E \to E'$ over $M$, between two smooth vector bundle $E \to M$ and $E' \to M$ induce a map $\widetilde{F} : \Gamma(E) \to \Gamma(E')$ between their spaces of smooth sections, defined as
$$
\widetilde{F}(\sigma)(p) = (F \circ \sigma)(p)= F(\sigma(p)).
$$ 
Its easy to verify that $\widetilde{F}(\sigma)$ is in $\Gamma(E')$. Hence this map is well defined. 
\[
\begin{tikzcd}[row sep=large,column sep=tiny]
E \ar[rr,"F"] \ar[dr,"\pi"] & & E' \ar[dl,swap,"\pi'"] \\
& M \ar[ul, bend left=30, "\sigma"] \ar[ur, bend right=30, swap, "\widetilde{F}(\sigma)"] & 
\end{tikzcd}
\]
Because bundle homomorphism is linear on fibers, then the induced map $\widetilde{F} : \Gamma(E) \to \Gamma(E')$ is linear over $\CM$. That is, for any $u_1,u_2 \in \CM$ and smooth sections $\sigma_1,\sigma_2 \in \Gamma(E)$,
$$
\widetilde{F}(u_1\sigma_1+ u_2 \sigma_2) = u_1 \widetilde{F}(\sigma_1) + u_2 \widetilde{F}(\sigma_2).
$$
The next lemma shows that the converse is also true.

\begin{prop}[Lemma 10.29 : \textbf{Bundle Homomorphism Characterization Lemma}]
Let $\pi :E \to M$ and $\pi' : E' \to M$ be smooth vector bundles over smooth manifold $M$ with or without boundary, and let $\Gamma(E)$, $\Gamma(E')$ denote their spaces of smooth sections. A map $\mathscr{F} : \Esection \to \Eprimesection$ is linear over $\CM$ if and only if there is a smooth bundle homomorphism $F : E \to E'$ over $M$ such that $\mathscr{F} (\sigma) = F \circ \sigma$ for all $\sigma \in \Esection$.
\end{prop}
\begin{proof}
One direction was clear. Conversely, suppose $\Fkeriting : \Esection \to \Eprimesection$ is linear over $\CM$. We want to define a smooth bundle homomorphism $F : E \to E'$ over $M$ such that $F \circ \sigma = \Fkeriting (\sigma)$ for all $\sigma \in \Esection$. I.e., we want map $F : E \to E'$ such that
\begin{enumerate}[nolistsep]
\item[(a)] $F \circ \sigma = \Fkeriting (\sigma)$, for all $\sigma \in \Esection$,
\item[(b)] $F$ is a smooth map,
\item[(c)] $F|_{E_p} :E_p \to E'_p$ is linear,
\item[(d)] $\pi = \pi' \circ F $.
\end{enumerate}
Seems like it is natural to define $F$ by $\Fkeriting$ so that the relation (a) will holds for free. It is seems natural that if we defined $F$ as
$$
F(v):= F(\widetilde{v}(p)) = \Fkeriting(\widetilde{v}) (p), \quad \text{for }v \in E_p \subhim E,
$$
where $\widetilde{v}$ is a smooth section of $E$ such that $\widetilde{v} (p) = v$ (which is according to Exercise 10.14, we can can always find such smooth section). Now the problem with this definition is that we defined the value $F(v)$ using a particular smooth section $\widetilde{v}$ which is not unique. So this definition will work if we can show that it will not depends on some particular smooth section that we choose. That is if $\sigma_1, \sigma_2 \in \Esection$ such that $\sigma_1(p) = \sigma_2(p) = v \in E_p$, then 
$$
\Fkeriting (\sigma_1) (p) \overset{?}{=} \Fkeriting (\sigma_1)(p).
$$
I.e., this same as asking if $\Fkeriting$ is act pointwise. If this is true, then our definition of $F$ is well defined. If $\Fkeriting$ act pointwise, then it must act locally also. That is if $\sigma_1,\sigma_2 \in \Esection$ such that $\sigma_1 \equiv \sigma_2$ on some neighbourhood $U$ of $p$, then
$$
\Fkeriting (\sigma_1) \overset{?}{\equiv} \Fkeriting (\sigma_2) \, \, \text{on } U.
$$
So we may try to see if this weaker property of $\Fkeriting$ holds or not. If this is true, then we will move on to the pointwise property. If this weaker property does not holds, then our definition will automatically fail.  

\textbf{The map $\boldmath{\Fkeriting}$ act locally: }  For any $\sigma_1,\sigma_2 \in \Esection$ such that $\sigma_1 \equiv \sigma_2$ on a neighbourhood $U$ of $p$, then $\Fkeriting (\sigma_1) \equiv \Fkeriting (\sigma_2) $ on $U$. By write $\tau = \sigma_1 - \sigma_2$, it is suffice to show that if $\tau \in \Esection$ vanish on $U$, then $\Fkeriting (\tau)$ does too. Given $p \in U$, let $\psi \in \CM$ be a smooth bump function supported in $U$ and equal to $1$ at $p$. Since $\psi \tau \equiv 0$ on $M$, then the fact that $\Fkeriting$ is $\CM $ linear implies 
$$
0 = \Fkeriting (\psi \tau) = \psi \Fkeriting(\tau).
$$ 
Evaluate at $p$, shows that $0 = \Fkeriting (\psi \tau)(p) = \psi (p) \Fkeriting(\tau)(p) = \Fkeriting (\tau)(p)$. Since this is true for every $p \in U$, then the claim follows.

\textbf{The map $\boldmath{\Fkeriting}$ is act pointwise: }For any $\sigma_1, \sigma_2 \in \Esection$ such that $\sigma_1(p) = \sigma_2(p)$ at a point $p \in M$, then $\Fkeriting (\sigma_1) (p) = \Fkeriting (\sigma_1)(p)$. Again it is suffice to show that if $\tau \in \Esection$ such that $\tau (p) = 0$, then $\Fkeriting(\tau)(p) = 0$. Choose a smooth local frame $(\sigma_1,\dots, \sigma_k)$ in some neighbourhood $U$ of $p$. In terms of this local frame, we can write $\tau$ as $\tau=u^i \sigma_i$ for some smooth functions $u^i : U \to \er $. Since $\tau(p)=0$, then $u^1(p) = \cdots=u^k(p) = 0$. By extension lemma for smooth function and smooth section, we can have smooth functions $\widetilde{u}^1,\dots,\widetilde{u}^k \in \CM $ such that $\widetilde{u}^i$ agrees with $u^i$ on some neighbourhood of $p$, and smooth sections $\widetilde{\sigma}_1,\dots,\widetilde{\sigma}_k \in \Esection$, such that $\widetilde{\sigma}_i$ agrees with $\sigma$ in a neighbourhood of $p$. So now we have $\tau = \widetilde{u}^i \widetilde{\sigma}_i$ in a (smaller) neighbourhood of $p$. Since $\Fkeriting$ act locally, then $\Fkeriting (\tau) = \Fkeriting (\widetilde{u}^i \widetilde{\sigma}_i)$. Evaluated at $p$
$$
\Fkeriting (\tau) (p) = \Fkeriting (\widetilde{u}^i \widetilde{\sigma}_i) (p) = \widetilde{u}^i(p) \Fkeriting ( \widetilde{\sigma}_i) (p) = u^i(p) \Fkeriting (\widetilde{\sigma}_i) (p) = 0.
$$

Therefore, our definition for $F : E \to E'$ above is well defined, since the value $F(v) := \Fkeriting (\widetilde{v}) (p)$ does not depend on the section $\widetilde{v}$. By this definition, for any $\sigma \in \Esection$, $F \circ \sigma = \Fkeriting (\sigma)$, since for any $p \in M$ and any $\widetilde{\sigma} \in \Esection$ such that $\widetilde{\sigma}(p) = \sigma(p)$, the pointwise property of $\Fkeriting$ imply 
$$
(F \circ \sigma)(p) = F(\sigma(p)) = \Fkeriting(\widetilde{\sigma})(p) = \Fkeriting(\sigma)(p).
$$

To show that $F|_{E_p} : E_p \to E'_p$ linear, let $a,b \in \er$ and $v,w \in E_p$, denote their extension as $\widetilde{v}$ and $\widetilde{w}$. Then
\begin{align*}
F|_{E_p} (av+bw) &= \Fkeriting(\widetilde{av+bw}) (p)\\ &= \Fkeriting (a \widetilde{v}+b \widetilde{w})\\ &= a \Fkeriting(\widetilde{v}) + b \Fkeriting(\widetilde{w}) \\ &= aF|_{E_p} (v) + b F|_{E_p} (w).
\end{align*}
To check that $\pi = \pi' \circ F$, let $v \in E_p$. The fact that $\Fkeriting (\sigma) \in \Eprimesection$ for any $\sigma \in \Esection$ imply
$$
(\pi' \circ F) (v) = \pi' (F(v)) = (\pi' \circ \Fkeriting(\widetilde{v}) )(p) = p = \pi(v).
$$
It remains to show that $F$ is smooth. It is suffice to show that $F$ is smooth in a neighbourhood of each point. Let $p \in M$ and let $(\sigma_i)$ be a local frame in a neighbourhood of $p$. By extension lemma, we have a global sections $\widetilde{\sigma}_i$ that agrees with $\sigma_i$ on a (smaller) neighbourhood $U$ of $p$. By shrinking $U$ if necessary, we also assume that there are local frame $(\sigma'_j)$ for $E'$ over $U$. Since $\Fkeriting : \widetilde{\sigma}_i \mapsto \Fkeriting(\widetilde{\sigma}_i) \in \Eprimesection$, then 
$$
\Fkeriting(\widetilde{\sigma}_i)|_U = A^j_i \sigma'_j
$$
for some smooth function $A^j_i \in C^{\infty}(U)$. Let $q \in U$ and $v \in E_q$. Write $v = v^i \sigma_i(q)$ for some real numbers $v^1,\dots,v^k$. Then
$$
F(v) = F(v^i \sigma_i(q)) = \Fkeriting(v^i \widetilde{\sigma}_i)(q) = v^i A^j_i(q) \sigma'_j(q).
$$
If $\Phi$ and $\Phi'$ are local trivializations of $E$ and $E'$ associated with frames $(\sigma_i)$ and $(\sigma'_j)$, that is $\Phi (\sigma_i(q)) =(q,e_i)$ and $\Phi' (\sigma'_j (q)) = (q,e_j)$. The map $\Phi' \circ F \circ \Phi^{-1} : U \times \rk \to U \times \rk$ has the form
$$
\Phi' \circ F \circ \Phi^{-1} (q, (v^1,\dots,v^k)) = (q, (v^i A_i^1(q),\dots, v^iA_i^k(q) ) )
$$  
which is smooth. Since $\Phi$ and $\Phi'$ are diffeomorphisms, hence $F$ is smooth on $\pi^{-1}(U)$.
\end{proof}

\textbf{Subbundle (Notes for  page 264).} Let $\pi_E : E \to M$ be a vector bundle. A \textbf{\textit{subbundle of \boldmath{E}}} is a vector bundle $\pi_D : D \to M$, such that 
\begin{enumerate}[nolistsep]
\item[(SB1)] $D \subhim E$ is a topological subspace,
\item[(SB2)] $\pi_D$ is the restriction of $\pi_E$ to $D$,
\item[(SB3)] For any $p \in M$, $D_p = D \cap E_p$ is a linear subspace of $E_p$ with inherit vector space structure of $E_p$.
\end{enumerate} 
If $E \to M$ is a smooth vector bundle, a subbundle of $E$ is called \textbf{\textit{smooth subbundle of \boldmath{E}}} if (SSB1) it is a smooth vector bundle and (SSB2) an embedded submanifold with or without boundary in $E$.

\begin{prop}[Exercise 10.31]
Given a smooth vector bundle $E \to M$ and a smooth subbundle $D \subhim E$, show that the inclusion map $\iota : D \hookrightarrow E$ is a smooth bundle homomorphism over $M$.
\end{prop}
\begin{proof}
By hypothesis $D \subhim E$ is a subbundle of $E$. So $D$ is an embedded submanifold of $E$. By definition, the inclusion map $\iota : D \hookrightarrow E$ is a smooth map, since it is a smooth embedding.
\[
\begin{tikzcd}[column sep=tiny, row sep=large]
D \arrow[rr,"\iota",hook] \arrow[dr,"\pi_D",swap] && E \arrow[dl,"\pi_E"] \\
&M&
\end{tikzcd}
\]
The linearity of $\iota|_{D_p} : D_p \to E_p$ is trivial, since $\iota|_{D_p}$ is just inclusion map of linear subspace $D_p \subhim E_p$. The relation $\pi_D = \pi_E \circ \iota$ follows easily. For, let $v \in D_p = D \cap E_p$, then $\pi_E \circ \iota(v) = \pi_E (v) = p = \pi_D (v)$.
\end{proof}

\begin{prop}[Lemma 10.32 : \textbf{Local Frame Criterion for Subbundles}]
Let $\pi : E \to M$ be a smooth vector bundle, and suppose that for each $p \in M$ we are given an $m$-dimensional linear subspace $D_p \subhim E_p$. Then $$D = \bigcup_{p \in M} D_p \subhim E \quad \text{is a smooth subbundle of }E$$ if and only if each point in $M$ has a neighbourhood $U$ on which there exists smooth local sections $\sigma_1,\dots,\sigma_m : U \to E$ with the property that $\sigma_1(p),\dots,\sigma_m(p)$ form a basis for $D_q$ at each $q \in U$.
\end{prop}
\begin{proof}
If $D \to M$ is a subbundle, then by definition for each $p \in M$ there exists a smooth local trivialization, hence there exists a smooth local frame associated to the local trivialization. So we have a smooth local sections $\tau_1 \dots,\tau_m : U \to D$ such that for any $q \in U$, $\tau_1(p),\dots, \tau_m(p)$ form a basis for $D_q$. By composing with inclusion map $\iota : D \to E$ (since $D$ is an embedded submanifold) we obtain the desired local sections $\sigma_i = \iota \circ \tau_i : U \to E$\, ($i = 1,\dots , m$).

Conversely, auppose $E \to M$ is a smooth vector bundle, and the union $D = \bigcup_{p \in M} D_p \subhim E$ of $m$-dimensional linear subspace $D_p \subhim E_p$ satisfy the above local sections hypothesis. We have to show that :
\begin{enumerate}[nolistsep]
\item[(1)] $D \subhim E$ is an embedded submanifold (with or without boundary),
\item[(2)] $\pi_D := \pi|_D : D \to M$ is a smooth vector bundle.
\end{enumerate}
To prove that $D \subhim E$ is an embedded submanifold with or withour boundary, it suffice to show any point $p \in M$ has a neighbourhood $U$ such that $D  \cap\pi^{-1}(U)$ is an embedded submanifold in $\pi^{-1}(U) \subhim E$. Given any $p \in M$, let $\sigma_1, \dots, \sigma_m : V \to E$ be smooth local sections such that $(\sigma_i(q))$ form a basis for $D_q$ for each $q \in U_p$. By Proposition 10.15, there are neighbourhood $W$ of $p$ and smooth local sections $\sigma_{m+1},\dots,\sigma_k : W \to E$ such that $(\sigma_1,\dots,\sigma_k)$ is a local frame for $E$ over $U= V \cap W$. This local frame associated to a local trivialization $\Phi :\pi^{-1}(U) \to U \times \rk$ defined by 
$$
\Phi(s^1 \sigma_1(q) + \cdots + s^k \sigma_k(q) ) = (q, (s^1 ,\dots, s^k) ).
$$
Since $D \cap  \pi^{-1}(U) = \big(\bigcup_{p \in M} D_p \big) \cap \big(\bigcup_{p \in U} E_p\big) = \bigcup_{p \in U} D_p$. This map takes $D \cap \pi^{-1}(U) = \{s = s^1 \sigma_1(q)+ \cdots + s^m \sigma_m(q) : q \in U \}$ to the subset
$$
\{ (q, (s^1, \dots, s^m, 0,\dots,0) ) \} \subhim  U \times \rk.
$$
So by using smooth chart from this local trivialization (Corollary 10.21), we conclude that $D \cap \pi^{-1}(U)$ is an embedded submanifold of $\pi^{-1}(U)$. It remain to show $\pi_D = \pi|_D : D \to M$ is a smooth vector bundle. The map $\pi_D = \pi|_D : D \to M$ is a smooth map, since $D$ is an embedded submanifold of $E$. By hypothesis, each fiber $D_p$ has vector space structure with fix dimension. By above construction, for each $p \in M$ we have an open subset 
$$
D \cap \pi^{-1}(U) = \bigcup_{p \in U} D_p = \pi_D^{-1}(U)
$$
(or by definition $\pi_D = \pi \circ \iota$, where $\iota : D \hookrightarrow E$, we have $\pi_D^{-1}(U) = (\pi \circ \iota)^{-1}(U) = \iota^{-1} \circ \pi^{-1} (U) = D \cap \pi^{-1} (U)$) together with local trivialization $\Phi : \pi^{-1}(U) \to U \times \rk$. Restrict this map to the submanifolds $ \pi_D^{-1}(U) = D\cap \pi^{-1}(U)$ and $U \times \mathbb{R}^m$, we have diffeomorphism
$$
\Psi : \pi_D^{-1}(U) \to U \times \mathbb{R}^m
$$  
defined as 
$$
\Psi ( s^1 \sigma_1(q)+ \cdots + s^m \sigma_m(q)) = (q, (s^1, \dots, s^m)),
$$
which is isomorphism on the fiber and satisfy $\pi_D = \pi_1 \circ \Psi$. Therefore $\pi_D : D \to M$ is a smooth vector bundle.
\end{proof}

\begin{prop}[Example 10.33]
The following are some example of subbundle:
\begin{enumerate}
\item[(a)] Suppose we have a nonvanishing vector field $V : M \to TM$ on a smooth manifold $M$. Define subset $D_p \subhim T_pM$ as the linear span of $V_p$. Since $V$ doesn't vanish on $M$, for each $p \in M$ we have $1$-dimensional linear subspace $D_p \subhim T_pM$. Then by Lemma 10.32, $D = \bigcup_{p \in M}D_p \subhim TM$ is a smooth subbundle of $TM$, since we have a global section $V : M \to TM$ which is, by construction, $V_p$ is basis for $D_p$ for each $p \in M$.
\item[(b)] Suppose $E \to M$ is any trivial bundle. For any global trivialization for $E$, let $(E_1,\dots,E_k)$ be the global frame associated to it. If $1\leq m \leq k$, the set $D=\bigcup_{p \in M}D_p \subhim E$, where for each $p$, 
$$
D_p:= \text{span }(E_1|_p ,\dots, E_m|_p),
$$
is a smooth subbundle of $E$.

\item[(c)] $\textbf{(Problem 10.14)}$ Suppose $M$ is a smooth manifold with or without boundary and $S\subhim M$ is an immersed $k$-submanifold with or without boundary. The subset $TS \subhim TM|_S$ is a smooth rank-$k$ subbundle of $TM|_S$. We will prove this using Lemma 10.32. 

\textbf{Setup : } Let $ \pi|_S : TM|_S \to S$ be the ambient tangent bundle, $\iota : S \hookrightarrow M$ is an immersed submanifold of dimension $k$ of a $n$-dimensional smooth manifold $M$.  For any $p \in S$, we have an $k$-dimensional subspace $d\iota_p (T_pS) \subhim T_pM$. We want to show that $\bigcup_{p \in S} d\iota_p(T_pS) \subhim TM|_S$ is a smooth subbundle of rank-$k$. To do this, it is enough to show that for any $p \in S$ there is a neighbourhood $V$ of $p$ and smooth local sections $\sigma_1,\dots,\sigma_k :V \to TM|_S$ such that at any $q \in V$, $\sigma_1(q),\dots,\sigma_k(q)$ form a basis for subspace $d\iota_q(T_qS)$.

\textbf{Proof : } Since $S \subhim M$ is an immersed submanifold, then $S$ locally embedded. So for any $p \in S$, there exists a neighbourhood $V \subhim S$ of $p$ such that $V$ is embedded submanifold of $M$. By shrinking $V$ if necessary, we may assume that $V \subhim S$ such that $\iota(V)=V \subhim U$, so $V = V \cap U$ is a single slice in $U$, where $(U,\varphi)$ is a slice chart for $V$. That is 
$$
\varphi(V) = \{(x^1,\dots,x^n) \in \varphi(U) : x^{m+1} = \cdots= x^n = 0   \} \subhim \varphi(U).
$$
By Theorem 5.8, we have smooth (global) chart $(V,\psi)$ where the coordinate functions $\psi (V) = \{ (x^1,\dots,x^m)\} \subhim \mathbb{R}^m$ are exactly the nonvanishing coordinates of image $\varphi(V)$ in $\varphi(U)$. Hence the representation of $\iota : S \hookrightarrow M$ in smooth charts $(V,\psi)$ and $(U,\varphi)$ is
\begin{equation}\label{represent.inclusionmap}\tag{$\star$}
(x^1,\dots,x^m) \mapsto (x^1,\dots,x^m,0,\dots,0).
\end{equation}
Now for $i=1,\dots,m$, the coordinate vector fields $\tau_i : V \to TS$ defined as $\tau_i(q) = \doo / \doo x^i\big|_q$ are smooth local frame (since the component functions are constant, hence smooth) for $TS$ over $V$. Define local sections $\sigma_i : V \to TM|_S$ defined as $\sigma_i(q) = d\iota_q (\tau_i(q))$, for $i =1,\dots,m$, where $d\iota_q : T_qS \to T_qM$ is the differential of $\iota : S \hookrightarrow M$ at $q\in S$. From (\ref{represent.inclusionmap}), we can compute $\sigma_i$ as
$$
\sigma_i(q) = d\iota_q \circ \tau_i(q) = d\iota_q \Big(\ddxi\bigg|_q \Big) = \ddxi\bigg|_q,
$$
which is form a basis for $d\iota_q(T_qS) \subhim T_qM$. 

We have to show that these sections are smooth sections on $TM|_S$. Since $\sigma_i (V) \subhim \bigcup_{q \in V} T_qM = (\pi|_S)^{-1}(V) = TM|_V$ and $TM|_V$ is open in $TM|_S$, then we only need to check the smoothness in $TM|_V$. For the local trivialization of $TM$ over slice chart $U$ as above is $\Phi : \pi^{-1}(U) \to U \times \rn$, defined as
$$
\Phi \Big(v^i \ddxip \Big) = (p,(v^1,\dots,v^n)).
$$
So the local trivialization for $TM|_V$ is just the restriction of $\Phi$ to $\pi^{-1}(U \cap V)$. Denote this as  $\Psi = \Phi|_U : \pi^{-1}(U \cap V) \to (U \cap V) \times \rn$. Therefore 
$$
\Psi \circ \sigma_i (q) =\Phi(\sigma_i(q)) =(q,e_i),
$$
which is obviously smooth. Hence $\sigma_i$ are the smooth local sections tha we seek. Since we can do this for every point $p \in S$, therefore $TS$ identified as its image $d\iota(TS) \subhim TM|_S$ is a smooth subbundle.
\end{enumerate}
\end{prop}

\begin{prop}[Theorem 10.34]
Let $E$ and $E'$ be smooth vector bundles over a smoorh manifold $M$, and let $F : E \to E'$ be a smooth bundle homomorphism over $M$. Define subsets $\text{Ker }F \subhim E$ and $\text{Im }F \subhim E'$ by
$$
\text{Ker }F =\bigcup_{p \in M} \text{Ker }\big(F|_{E_p}\big), \quad \text{Im }F = \bigcup_{p \in M} \text{Im } \big(F|_{E_p}\big) 
$$
Then $\text{Ker }F$ and $\text{Im }F$ are smooth vector subbudles of $E$ and $E'$, resprectively, if and oonly if $F$ has constant rank.
\end{prop}
\begin{proof}
\textbf{One direction is obvious:} If  $\text{Ker }F$ and $\text{Im }F$ are smooth subbundles, then by definition, the dimension of the fibers $ \text{Ker }\big(F|_{E_p}\big)$ and $\text{Im } \big(F|_{E_p}\big) $ are constant everywhere, therefore by rank-nullity theorem $$\text{dim }E_p = \text{dim }\big(\text{Im }(F|_{E_p}) \big)+ \text{dim }\big(\text{Ker }(F|_{E_p})\big),$$ the rank of linear map $F|_{E_p}$ is same everywhere. \textbf{Conversely,} suppose $F$ has constant rank $r$. Let $p \in M$ be arbitrary and $\sigma_1,\dots,\sigma_k : U \to E$ be a smooth local frame for $E$ over a neighbourhood $U$ of $p$.  For any $q \in U$ and $v \in E_q$,
$$
F(v) = F(v^i \sigma_i(q)) = v^i (F \circ \sigma_i)(q) \in \text{Im }(F|_{E_q}).
$$
Therefore $F \circ \sigma_1,\dots,F \circ \sigma_k : U \to E'$ are smooth local sections for $E'$ that $\text{span }(\text{Im }F)|_U$. By rearranging the indices as necessary, we can assume that $\{F \circ \sigma_1 (p), \dots, F \circ \sigma_r(p)\}$ form a basis for $\text{Im }(F|_{E_p})$. By continuity, they are still linearly independent in some neighbourhood $U_0$ of $p$. Since by assumption, $F$ has constant rank, then $\{F \circ \sigma_1 (q), \dots, F \circ \sigma_r(q)\}$ form a basis for $\text{Im }(F|_{E_q})$ for any $q \in U_0$. Therefore we obtain smooth local frame $F \circ \sigma_1,\dots,F\circ \sigma_r : U_0 \to E'$ for $\text{Im }F$ over neighbourhood $U_0$ of $p$. By Lemma 10.32, \textbf{\boldmath{$\text{Im }F$} is a smooth subbundle of \boldmath{$E'$}}.

\textbf{To prove $\text{Ker }F$ is a smooth subbundle,} let $U_0$ and $\{\sigma_1,\dots,\sigma_r,\sigma_{r+1},\dots,\sigma_k\}$ be as before. Since $(\sigma_i)_{i=1}^k$ is a smooth local frame for $E$ over $U_0$, then $E|_{U_0}$ is a smooth subbundle of $E$. Let $V = \bigcup_{p \in U_0} \text{span }(\sigma_1,\dots,\sigma_r) \subhim E|_{U_0}$. By definition, $V$ is a smooth subbundle of $E|_{U_0}$. The restriction map
$$
F|_V : V \to (\text{Im }F)|_{U_0}
$$
is smooth, since $F : E|_{U_0} \to (\text{Im }F)|_{U_0}$ is smooth, since $E|_{U_0}$ and $ (\text{Im }F)|_{U_0}$ are embedded submanifold of $E$ and $E'$ respectively and  the restriction to  $V$ and $(\text{Im }F)|_{U_0}$ also smooth, since they are both embedded submanifold of $E|_{U_0}$ and $ (\text{Im }F)|_{U_0}$ respectively. Since this map just a restriction of $F$, the pointwise behaviour is same as $F$. So $F|_V$ is a smooth bundle homomorphism. Also $F|_V$ is bijective, so by Proposition 10.26, \textbf{\boldmath{$F|_V$} is a smooth bundle isomorphism}. \textbf{Note that \boldmath{$V$} and \boldmath{$(\text{Ker }F)|_{U_0}$} span \boldmath{$E|_{U_0}$}}. For, let $v=v^1\sigma_1(q)+\cdots + v^k \sigma_k(q) \in E_q \subhim E|_{U_0}$. Collecting the terms
$$
v = u+w, \quad  u=\sum_{i=1}^{r} v^i \sigma_i(q), \text{ and } w=\sum_{j=r+1}^{k}v^j \sigma_j(q),
$$
we have $u$ is in $V$ by definition of $V$. To show that $w$ is in $(\text{Ker }F)|_{U_0}$, suppose that $F(w) \neq 0 \in (\text{Im }F)|_{U_0}$. But $F|_V$ is bijective, so there are real numbers $a^1,\dots,a^r$ such that $w = a^1\sigma(q) + a^r\sigma_r(q)$. So we have 
$$
a^1\sigma_1(q)+ \cdots + a^r \sigma_r(q) + v^{r+1}\sigma_{r+1}(q) + \cdots +v^k\sigma_k(q) = 0
$$
with not all the coefficient are zero, which is impossible since $(\sigma_i(q))_{1}^k$ are linearly independent.

\textbf{Define a smooth bundle homomorphism}
$$
\Psi : E|_{U_0} \to E|_{U_0}, \quad \text{by} \quad \Psi (v) = v- (F|_V)^{-1} \circ F(v). 
$$
\begin{enumerate}
\item[(1)] If $v \in V$, then $F(v) = F|_{V}(v)$. So $F(\Psi(v)) = F(v) - F \circ (F|_V)^{-1} \circ (F|_V)(v) = 0$.
\item[(2)] If $v \in (\text{Ker }F)_{U_0}$, then $\Psi(v) = v$, so again $F(\Psi(v)) = F(v) = 0$.
\end{enumerate}
So $\Psi ( E|_{U_0}) \subhim (\text{Ker }F)_{U_0}$. By (2), $(\text{Ker }F)_{U_0} \subhim \Psi ( E|_{U_0})$. Therefore we have $\Psi ( E|_{U_0}) = (\text{Ker }F)|_{U_0}$. Since $(\text{Im }F)|_{U_0}$ is a smooth subbundle, then the dimension of $\text{Im }F|_{E_q}$ is same for all $q \in U_0$. This implies that $\text{dim }\big(\text{Im }(\Psi|_{E_q}) \big) = \text{dim }\big(\text{Ker }(F|_{E_q}) \big)$ is same for all $q \in U_0$. So $\Psi$ has constant rank. By the preceding result, $\text{Im }\Psi = (\text{Ker }F)_{U_0}$ is a smooth subbundle of $E|_{U_0}$. Since we can do the same thing in a neighbourhood of each point, then $\text{Ker }F$ is a smooth subbundle of $E$.
\end{proof}

By this result, we can construct interesting subbundles of the tangent bundle over submanifolds of $\rn$ as shown by the following lemma.

\begin{prop}[Lemma 10.35 : \textbf{Orthogonal Complement Bundles}]
Let $M$ be an immersed submanifold with or without boundary in $\rn$, and $D$ be a smooth rank-$k$ subbundle of $T\rn|_M$. For each $p \in M$, let $D_p^{\perp}$ denote the orthogonal complement of $D_p$ in $T_p\rn$ with respect to the Euclidean dot product, and let $D^{\perp} \subhim T\rn|_M$ be the subset 
$$
D^{\perp} = \{ (p,v)\in T\rn : p\in M, v \in D_p^{\perp} \}.
$$ 
Then $D^{\perp}$ is a smooth rank-$(n-k)$ subbundle of $T\rn|_M$. For each $p \in M$, there is a smooth orthogonal frame for $D^{\perp}$ on a neighbourhood of $p$.
\end{prop}
\begin{proof}
Let $p \in M$ be arbitrary, and let $X_1,\dots,X_k : V \to D$ is a smooth local frame for $D \to M$ over a neighbourhood $V$ of $p$ in $M$. Because immersed submanifold is locally embedded, by shrinking $V$ if necessary, we may assume that $V$ is a single slice in some coordinate ball of half ball in $\rn$. Since $D \hookrightarrow T\rn|_M$ is smooth, then $X_1,\dots,X_k : V \to T\rn|_M$ are linearly independent smooth local sections of $T\rn|_M$ over open subset $V \subhim M$. From Proposition 10.15 (a), we can complete $X_1,\dots,X_k$ to a local frame of $T\rn|_M$ over a smaller neighbourhood of $p$. By shrinking $V$ and $U$ again, we may assume that $X_1,\dots,X_n : V \to T\rn|_M$ is a local frame over a neighbourhood $V \subhim M$ which is a single slice of coordinate ball $U \subhim \rn$. Note that $V$ closed in $U$ and each $X_i$ admit a smooth extension in a neighbourhood of each $p\in V$ in $\rn$ (by extend it as a constant coefficient vector field in the slice chart $U$). Therefore by Proposition 10.15 (c), we can extend $(X_1,\dots,X_n)$ to a smooth local frame $\widetilde{X}_1,\dots,\widetilde{X_n} : U \to T\rn$ for $T\rn$ over $U$, and then Lemma 8.13 yields a smooth \textit{orthonormal} frame $(E_j)$ over $U$ such that $\text{span } (E_1|_p,\dots,E_k|_p) = \text{span }(X_1|_p,\dots,X_k|_p) = D_p$ for each $p \in V$. It follows that $(E_{k+1},\dots,E_n)$ restrict to a smooth orthonormal frame for $D^{\perp}$ over $V$. Thus $D^{\perp}$ satisfies the local frame criterion, and is therefore a smooth subbundle of $T\rn|_M$.
\end{proof}

\begin{prop}[Corollary 10.36 : \textbf{The Normal Bundle to a Submanifold of \boldmath{$\rn$}}]
If $M \subhim \rn$ is an immersed $m$-dimensional submanifold with or withour boundary , its normal bundle $NM$ is a smooth rank-$(n-m)$ subbundle of $T\rn|_M$. For each $p \in M$, there exists a smooth orthonormal frame for $NM$ on a neighbourhood of $p$.
\end{prop}
\begin{proof}
Apply Lemma 10.35 to the smooth subbundle $TM \subhim T\rn|_M$.
\end{proof}

\subsection{Chapter 11 (Cotangent Bundle)}

\textbf{Cotangent bundle of \boldmath{$M $} }is the disjoint union $T^*M =\bigsqcup_{p \in M} T^*_pM$  where $T^*_pM = (T_pM)^*$. It has \textit{natural projection} $\pi : T^*M \to M$ sending $\omega \in T^*_pM$ to $p \in M$.

Since we have coordinate frame $(\doo / \doo x^i|_p )$ for any smooth local coordinates $(U,x^i)$, for any $p \in U$ we have a basis $(\lambda^i|_p)$ for $T^*_pM$ dual to $(\doo / \doo x^i|_p )$. This define $n$ maps $\lambda_1,\dots,\lambda_n : U \to T^*M$, called \textbf{coordinate covector fields}.

\begin{prop}[Proposition 11.9 : \textbf{The Cotangent Bundle as a Vector Bundle}]
Let $M$ be a smooth $n$-manifold with or without boundary. With its standard projection map and the natural vector space structure on each fiber, the cotangent bundle $T^*M$ has a unique topology and smooth structure making it into a smooth rank-$n$ vector bundle over $M$ for which all coordinate covector fields are smooth local sections.
\end{prop}
\begin{proof}
We want to prove this using vector bundle chart lemma. Let $(U,\varphi)$ be a smooth chart on $M$. For any $p \in U$ we have basis $(\lambda^i|_p)$ for $T^*_pM$ dual to the coordinate basis vector  $(\doo / \doo x^i|_p )$. So for any $\xi \in T^*_pM$ we can write it as $\xi = \xi_i \lambda^i|_p$. Define the map $ \Phi : \pi^{-1}(U) = \bigcup_{p \in U} T^*_pM \to U \times \rn$ as
$$
\Phi (\xi) = \Phi (\xi_i \lambda^i|_p) = (p, (\xi_1,\dots,\xi_n) ).
$$ 
The restriction $\Phi|_{\tpstarm} : \tpstarm \to \{p\} \times \rn$ is certainly a vector space isomorphism. Therefore $\Phi$ is a bijective map. 

Suppose $(\widetilde{U},\wtilde{\varphi})$ is another smooth chart on $M$ such that $U \cap \wtilde{U} \neq \emptyset$, and define $\wtilde{\Phi} : \pi^{-1}(\wtilde{U}) \to \wtilde{U} \times \rn$ as before
$$
\wtilde{\Phi} (\xi) = \wtilde{\Phi} (\tilde{\xi}_i \wtilde{\lambda}^i|_p) = (p, (\tilde{\xi}_1,\dots,\tilde{\xi}_n)),
$$
where $(\wtilde{\lambda}^i|_p)$ is a basis of $\tpstarm$ dual to $(\doo / \doo \wtilde{x}^i|_p )$. Consider the map $\Phi \circ \wtilde{\Phi}^{-1} : (U \cap \wtilde{U}) \times \rn \to (U \cap \wtilde{U}) \times \rn$. For any $(p,(\tilde{\xi}_1,\dots,\tilde{\xi}_n) ) \in (U \cap \wtilde{U}) \times \rn \subhim \wtilde{U} \times \rn$,
$$
(\Phi \circ \wtilde{\Phi}^{-1}) (p,(\tilde{\xi}_1,\dots,\tilde{\xi}_n)) = \Phi(\xi) = (p,(\xi_1,\dots,\xi_n)).
$$
By transformation law for vector field, we derive that the components of covector field $\xi=\xi_i \lambda^i|_p = \tilde{\xi}_i \wtilde{\lambda}^i|_p$ transform as 
$$
\xi_i = \frac{\doo \wtilde{x}^j}{\doo x^i}(p) \tilde{\xi}_j.
$$
Therefore we have 
\begin{align*}
(\Phi \circ \wtilde{\Phi}^{-1}) (p,(\tilde{\xi}_1,\dots,\tilde{\xi}_n)) &= \bigg(p, \bigg( \frac{\doo \wtilde{x}^j}{\doo x^1}(p) \tilde{\xi}_j,\dots, \frac{\doo \wtilde{x}^j}{\doo x^n}(p) \tilde{\xi}_j  \bigg)\bigg).
\end{align*}
The function $\tau : U \cap \wtilde{U} \to \GL$ defined as 
$$
\tau(p) = \bigg[ \frac{\doo \wtilde{x}^i}{\doo x^j}(p) \bigg]^T
$$
is smooth. It follows from the vector bundle chart lemma that $\tpstarm$ has a smooth structure making it into a smooth vector bundle for which maps $\Phi$ are smooth local trivializations. The coordinate covector fields $\lambda^1 ,\dotsm\lambda^n : U \to T^*M$ are smooth local sections, since by definition of local trivialization $\Phi$ is $\lambda^i (p)  = \Phi^{-1} \circ e_i (p)$ where $e_i : U \to U \times \rn$ defined as $e_i (p) = (p, (0,\dots,1,\dots,0) )$. 
\end{proof}

Inspired by above construction of $T^*M$, we can also construct dual bundle $E^* \to M$ of a vector bundle $E \to M$ as the following proposition shows.

\begin{prop}[Exercise 11.10 : \textbf{Construction of Dual Bundle}]
Suppose $M$ is a smooth manifold and $E \to M$ is a smooth vector bundle over $M$. Define the \textit{\textbf{dual bundle to E}} to be the bundle $E^* \to M$ whose total space is the disjoint union $E^* = \bigsqcup_{p \in M} E^*_p$, where $E^*_p$ is the dual space to $E_p$, with the obvious projection. Show that $E^* \to M$ is a smooth vector bundle, whose transition functions are given by $\tau^*(p) = \big(\tau(p)^{-1} \big)^T$ for any transition function $\tau : U \to \GLsaja(k,\er)$ of $E$.   
\end{prop}
\begin{proof}
Similar to the construction of $\tpstarm$, we try to construct $E^* \to M$ using vector bundle chart lemma. Let $p \in M$ arbitrary and suppose $\Phi : \pi^{-1}(U) \to U \times \rk$ and $\wtilde{\Phi} : \pi^{-1}(\wtilde{U}) \to \wtilde{U} \times \rk$ are local trivializations of $E$ over open subsets $U, \wtilde{U} \subhim M$ contain $p$. For each local trivialization we have a local frame associated to it, denoted by $\sigma_i : U \to E$ defined by $\sigma_i(p) = \Phi^{-1} (p,e_i)$ and $\wtilde{\sigma}_i : \wtilde{U} \to E$ defined by $\wtilde{\sigma}_i(p) = \wtilde{\Phi}^{-1}(p,e_i)$, where $(e_i)$ is the standard basis for $\rk$. For any $v \in E_p \subhim \pi^{-1}(U \cap \wtilde{U})$, we can write $v = v^i \sigma_i(p) = \wtilde{v}^i \wtilde{\sigma}_i(p)$, and by definition 
$$
\Phi (v) = \Phi (v^i \sigma_i(p)) = (p, (v^1,\dots,v^k)),
$$
and
$$
\wtilde{\Phi} (v) = \wtilde{\Phi} (\wtilde{v}^i \wtilde{\sigma}_i(p)) = (p, (\wtilde{v}^1,\dots,\wtilde{v}^k )).
$$
We also know that there is a smooth map $\tau : U \cap \wtilde{U} \to \GLkr$ such that the map $\Phi \circ \wtilde{\Phi}^{-1} : (U \cap \wtilde{U}) \times \rk \to (U \cap \wtilde{U}) \times \rk$ has the following form
$$
\Phi \circ \wtilde{\Phi}^{-1} (p, (\wtilde{v}^i) ) = \Phi(v) = (p, (v^i)) = (p, \tau(p) (\wtilde{v}^i) ), 
$$
(here we denote $(v^i)$ and $(\wtilde{v}^i)$ as elements of $\rk$ for simplicity). So we have $(v^i) = \tau(p) (\wtilde{v}^i)$, or in terms of components $v^i = \tau^i_j(p) \wtilde{v}^j$. Since $v = v^i \sigma_i(p) = \wtilde{v}^i \wtilde{\sigma}_i(p)$, 
\begin{align*}
\wtilde{v}^j \wtilde{\sigma}_j|_p =  v^i \sigma_i|_p = \big(\tau^i_j(p) \wtilde{v}^j \big) \sigma_i|_p = \wtilde{v}^j \Big(\tau^i_j(p) \sigma_i|_p \Big) \implies \wtilde{\sigma}_j(p) = \tau^i_j (p) \sigma_i(p).
\end{align*}

To show that $E^* \to M$ is a smooth vector bundle using vector bundle chart lemma, we need to find a bijective map
$$
\Psi : \pi^{-1}(U) = \bigcup_{p \in U} E^*_p \to U \times \rk
$$
for every open subset $U \subhim M$, which is linear on the fiber and then show that the overlap map between any two such map has desired form. Since for each open subset $U \subhim M$ we have a local frame $(\sigma_i)$ associated to the local trivialization $\Phi : \pi^{-1}(U) \to U \times \rk$ for $E$, then for every $p \in U$ we has a basis $\{\varepsilon^i(p) \, | \, i =1,\dots,k\}$ for $E^*_p$ dual to $\{\sigma_i(p) \,| \, i=1\dots,k\}$, which is basis for $E_p$. Therefore we can define the map $\Psi : \pi^{-1}(U) = \bigcup_{p \in U} E^*_p \to U \times \rk$ as
$$
\Psi (\omega) = \Psi (\omega_i \varepsilon^i(p)) = (p , (\omega_1,\dots,\omega_k)), \quad \text{for any } \omega \in E^*_p.
$$
It is easy to see that this map is bijective and linear on each fiber.

After defining the maps, consider any two such map $\Psi : \pi^{-1}(U) \to U \times \rk$ and $\wtilde{\Psi} : \pi^{-1} (\wtilde{U}) \to \wtilde{U} \times \rk$ with $U \cap \wtilde{U} \neq \emptyset$. Let $p \in U \cap \wtilde{U}$ and $\{\varepsilon^i(p)\}$ and $\{\wtilde{\varepsilon}^i(p)\}$ be the bases for $E^*_p$ dual to $\{\sigma_i(p)\}$ and $\{\wtilde{\sigma}_i(p)\}$ respectively. For any $\omega \in E^*_p \subhim \pi^{-1} ( U \cap \wtilde{U})$ we can write $\omega = \omega_i \varepsilon^i(p) = \wtilde{\omega}_i \wtilde{\varepsilon}^i(p)$. Consider the map $\Psi \circ \wtilde{\Psi}^{-1} : (U\cap \wtilde{U}) \times \rk \to (U\cap \wtilde{U}) \times \rk $, 
$$
\Psi \circ \wtilde{\Psi}^{-1} (p, (\wtilde{\omega}_1,\dots,\wtilde{\omega}_k) ) = \Psi (\omega) = (p, (\omega_1,\dots, \omega_k)).
$$
To find the relation between $\omega_i$ and $\wtilde{\omega}_i$, we have to find the relation between $\wtilde{\varepsilon}^i(p)$ and $\varepsilon^i$. Let $\wtilde{\varepsilon}^i(p) = A^i_j \varepsilon^j(p)$ for some numbers $A^i_j$. Since we have $\wtilde{\sigma}_j(p) = \tau^i_j (p) \sigma_i(p)$, then (by omiting $p$ for a moment)
\begin{align*}
\delta^i_j=\wtilde{\varepsilon}^i (\wtilde{\sigma}_j) = (A^i_r \varepsilon^r) (\tau^l_j \sigma_l)= \tau^l_j A^i_r \varepsilon^r( \sigma_l)  = (\tau^l_j  A^i_r) \delta^r_l =   A^i_l \tau^l_j.
\end{align*}
This implies that $[A^i_j] = [\tau^i_j(p)]^{-1} =  \tau(p)^{-1}$. Denote the inverse of $\tau(p)$ as $\overline{\tau}(p)$, we have
$$
\wtilde{\varepsilon}^i(p) = \overline{\tau}^i_j(p)  \varepsilon^j(p).
$$
Hence the components of $\omega = \omega_i \varepsilon^i(p) = \wtilde{\omega}_i \wtilde{\varepsilon}^i(p)$ transform as $\omega_i =  \overline{\tau}^j_i(p) \wtilde{\omega}_j $. Therefore the transition map has the following form
$$
\Psi \circ \wtilde{\Psi}^{-1} (p, (\wtilde{\omega}_1,\dots,\wtilde{\omega}_k) )  =(p, (\overline{\tau}^j_1(p) \wtilde{\omega}_j,\dots,\overline{\tau}^j_k(p) \wtilde{\omega}_j) ).
$$
Since in matrix form $(\wtilde{\omega}_i) \mapsto [\overline{\tau}_i^j(p)]^T (\wtilde{\omega}_j) = \big(\tau(p)^{-1} \big)^T (\wtilde{\omega}_j) $, then we have the smooth transition map  $\tau^* = \big(\tau(p)^{-1} \big)^T :  U \cap \wtilde{U} \to \GLkr$. This conclude that $E^* \to M$ is a smooth vector bundle over $M$.
\end{proof}

\begin{prop}[Exercise 11.12 : \textbf{Smoothness Criteria for Covector Fields}]
Let $M$ be a smooth manifold with or without boundary, and let $\omega : M \to T^*M$ be a rough covector field. Show that the following are equivalent:
\begin{enumerate}[nolistsep]
\item[(a)] $\omega$ is smooth.
\item[(b)] In every smooth coordinate chart, the components functions of $\omega$ are smooth.
\item[(c)] Each point of $M$ is contained in some coordinate chart in which $\omega$ has smooth component functions.
\item[(d)] For every smooth vector field $X \in \VF(M)$, the function  $\omega(X)$ is smooth on $M$.
\item[(e)] For every open subset $U \subhim M$ and every smooth vector field $X$ on $U$, the function $\omega(X) : U \to \er$ is smooth on $U$.
\end{enumerate} 
\end{prop}
\begin{proof}
We will show that $(a) \Rightarrow (b) \Rightarrow (c) \Rightarrow (a)$ and $(c) \Rightarrow (d) \Rightarrow (e) \Rightarrow (b)$. To show $(a) \Rightarrow (b)$, suppose that $(a)$ is true. This implies that $\omega$ is smooth locally. Since for any smooth chart $(U,\varphi)$ the restriction $\omega|_U = \omega_i \lambda^i$ is smooth, therefore $\omega_i$ are smooth. $(b) \Rightarrow (c)$ and $(c) \Rightarrow (a)$ are obvious.

To show $(c) \Rightarrow (d)$, suppose $(c)$ holds. It is enough to show that $\omega(X)$ is smooth on some neighbourhood $U \subset M$ of each point in $M$. Let $p \in M$ and $X \in \VF(M)$ are arbitrary. Since $(c)$ holds, we can find a coordinate chart $(U,\varphi)$ such that $\omega_i : U \to \er$ are smooth. Since the representation of $\omega(X)$ on this chart is $\omega(X) = \omega_iX^i$, hence $\omega(X)$ is smooth on $U$. 

To show $(d) \Rightarrow (e)$, suppose $(d)$ is true. Let $U$ be any open subset of $M$ and $X \in \VF(U)$. To show that $\omega(X)$ is smooth on $U$, it is enough to show that it is smooth on a neighbourhood of each point. Suppose $p$ be any point in $U$ and $V \subhim U$ is a neighbourhood of $p$. Choose a smooth bump function $\varphi : M \to \er$ such that $\varphi \equiv 1$ on $V$ and supp $\varphi \subhim U$. By restrict $\varphi$ to $U$ and multiply it with $X : U \to TM$ we obtain a smooth vector field $\varphi X : U \to TM$ which is equal to $X$ in $V$ and supp $\varphi X \subhim U$. By gluing lemma, we obtain smooth vector field $\wtilde{X} : M \to TM $ such that supp $\wtilde{X}=$ supp $\varphi X \subhim U$, $\wtilde{X} \equiv 0$ on $M \smallsetminus \text{ supp }\wtilde{X}$, and $\wtilde{X}|_V = X|_V$.  Since (d) holds, then $\omega(\wtilde{X})$ is smooth on $M$, in particular it is smooth on $V$. By construction,
$$
\big(\omega(\wtilde{X})\big)|_V = \omega (\wtilde{X}|_V) = \omega(X).
$$
This shows that $\omega(X)$ is smooth on $V$. Since $p \in V \subhim U$ arbitrary, then $\omega(X)$ is smooth on $U$.

For $(e) \Rightarrow (b)$, suppose $(e)$ holds. Let $(U,x^i)$ be any smooth chart in $M$. Since the coordinate vector fields $(\partial / \doo x^i)$ are smooth vector fields in $U$, (e) implies that $\omega_i = \omega(\partial / \doo x^i)$ is smooth on $U$. 
\end{proof}

\begin{prop}[Exercise 11.16 : \textbf{Coframe Criterion for Smoothness of Covector Fields}]
Let $M$ be a smooth manifold with or without boundary, and let $\omega$ be a rough covector field on $M$. If $(\varepsilon^i)$ is a smooth coframe on an open subset $U \subhim M$, then $\omega$ is smooth on $U$ if and only if its component functions with respect to $(\varepsilon^i)$ are smooth. 
\end{prop}
\begin{proof}
Suppose that $\omega $ is smooth on $U$. By Lemma 11.14, the frame $(E_i)$ dual to $(\varepsilon^i)$ is a smooth frame. Since $\omega$ is smooth on $U$ and $E_i \in \VF(U)$, then by Proposition 11.11 (d), the function $\omega(E_i) = \omega_i$ are smooth on $U$. Conversely, suppose that the component functions $\omega_i = \omega(E_i)$are smooth on $U$ and $X \in \VF(U)$. Since $(E_i)$ is a coframe on $U$, we can write $X$ as $X = X^i E_i$. The function $\omega(X) = \omega_i \varepsilon^i (X^j E_j) = \omega_i X^i$ is smooth on $U$ since $X^i$ and $\omega_i$ are smooth. Therefore by Proposition 11.11, $\omega$ is smooth on $U$. 
\end{proof}

\begin{prop}[Exercise 11.21 : \textbf{Properties of Differential}]
Let $M$ be a smooth manifold with or withour boundary, and let $f,g \in \CM$.
\begin{enumerate}[nolistsep]
\item[(a)] If $a$ and $b$ are constants, then $d(af+bg) = adf +b dg$.
\item[(b)] $d(fg) = fdg + g df$.
\item[(c)] $d(f/g) = (gdf - fdg)/g^2$ on the set where $g\neq 0$.
\item[(d)] If $J \subhim \er$ is an interval containing the image of $f$, and $h : J \to \er$ is a smooth function, then $d(h \circ f) = (h' \circ f) df$.
\item[(e)] If $df$ is constant, then $df=0$. 
\end{enumerate}
\end{prop}
\begin{proof}
Part (a),(b) and (e) follows directly from the linearity and Leibnitz rule for tangent vector. For (c), let $v \in T_pM$ with $p$ is any point such that $g \neq 0$. Then 
$$
d(f/g)_p(v) = v(f/g) = v(fg^{-1}) = f(p) v(g^{-1}) + v(f)/g(p).
$$
By continuity, $g \neq 0$ on some neighbourhood of $p$. On this neighbourhood $g$ has an inverse $g^{-1}$. So we can compute 
$$
0 = v(1) = v(g g^{-1}) = g(p) v(g^{-1}) + v(g)/g(p).
$$
Thus we have $d(f/g)_p(v) = (g(p) df(v) - f(p) dg(v))/g(p)^2$.

To show (d), let $p \in M$ and $v \in T_pM$. Denote $T_p f : T_pM \to T_{f(p)} \er$ as the differential map, we have 
\begin{align*}
d(h \circ f)_p (v) &= v(h \circ f) = T_pf(v) h = dh_{f(p)} \big(T_pf(v) \big) \\ &= h'(f(p))\, dx \, \big(T_pf(v)\big)\\ &= (h' \circ f)(p)\, T_pf(v) (x) \\ &= (h' \circ f)(p)\, v(x \circ f) \\ &= (h' \circ f)(p)\, v(f) \\ &= (h' \circ f)(p)\, df_p(v).
\end{align*}
This proves (d).
\end{proof}

\begin{prop}[Exercise 11.30]
Suppose $M$ is a smooth manifold with or without boundary and $S \subhim M$ is an immersed submanifold with or without boundary. For any $f \in \CM$, show that $d(f|_S)= \iota^*(df)$. Conclude that the pullback of $df$ to $S$ is zero if and only if $f$ is constant on each compoenent of $S$. 
\end{prop}
\begin{proof}
Let $\iota :S \hookrightarrow M$ be the inclusion. The map $f|_S = f \circ \iota : S \to \er$ is a smooth real-valued function on $S$ where its differential $d(f|_S)$ defined as
$$
d(f|_S)_p(v) = v (f \circ \iota), \quad \forall v \in T_pS.
$$
On the other hand, since $f \in \CM$, $df \in \VF^*(M)$. For any $v \in T_pS$,  the value of pullback $\iota^*(df)$ at $v$ is
$$
\iota^*(df)_p (v) = (df)_{\iota(p)} (d\iota_p(v)) = d\iota_p(v) (f) = v(f \circ \iota).
$$
So $d(f|_S) = \iota^*(df)$. By Proposition 11.22, the pullback $\iota^*(df)=d(f|_S) \in \VF^*(S)$ is zero if and only if $f|_S$ is constant on each component of $S$.
\end{proof}

\textbf{Line Integrals } Let $I = [a,b] \subhim \er$ and $\omega \in\VF^*(I)$. If we denote $t$ as a standard coordinate in $[a,b]$, then we can write $\omega_t = f(t) dt$ for some smooth function $f : [a,b] \to \er$. Define \textbf{\textit{integral of $\omega$ over $[a,b]$}} as
$$
\int_{[a,b]} \omega = \int_{a}^{b} \omega_t =\int_{a}^{b} f(t) dt.
$$

If $\gamma : [a,b] \to M$ is a smooth curve segment and $\omega \in \VF^*(M)$ then define \textbf{\textit{the line integral of $\omega$ over $\gamma$ }} to be the real number
$$
\int_{\gamma} \omega = \int_{[a,b]} \gamma^*\omega.
$$
More generally, if $\gamma$ is a piecewise smooth curve segment, then define
$$
\int_{\gamma} \omega = \sum_{i=1}^{k} \int_{[a_{i-1},a_i]} \gamma^*\omega,
$$ 
where $[a_{i-1},a_i]$, $i=1,\dots,k$, are subintervals where $\gamma$ is smooth. 


Before we continue to establish the properties of line integrals, it is much easier if we first prove Proposition 11.38.

\begin{prop}[Proposition 11.38]
If $\gamma : [a,b] \to M$ is a piecewise smooth curve segment, the line integral of $\omega$ over $\gamma$ can also be expressed as the ordinary integral
$$
\int_{\gamma} \omega = \int_{a}^{b} \omega_{\gamma(t)} \big( \gamma'(t) \big) dt.
$$
\end{prop}
\begin{proof}
First, assume that $\gamma$ is a smooth segment and the image $\gamma([a,b])$ is contained in a single smooth chart. Let $(U,(x^i))$ be a smooth chart contain $\gamma([a,b])$. By definition, we have to evaluate
$$
\int_{\gamma} \omega = \int_{[a,b]} \gamma^*\omega = \int_{a}^{b} (\gamma^*\omega)_t.
$$
Since in this coordinates $\gamma(t) = (\gamma^1(t),\dots,\gamma^n(t))$ and $\omega = \omega_i dx^i$, and from Proposition 11.25 we have
\begin{align*}
(\gamma^*\omega)_t &= (\gamma^*(\omega_i dx^i))_t\\ &= (\omega_i \circ \gamma)(t) \, (\gamma^*(dx^i))_t\\ &= \omega_i(\gamma(t)) \, d(x^i \circ \gamma)_t \\ &= \omega_i(\gamma(t))\, d(\gamma^i)_t  \\
&= \omega_i(\gamma(t)) \, \dot{\gamma}^i dt \\
&= \omega_i(\gamma(t)) \, dx^i (\gamma'(t)) dt\\
&= \omega_{\gamma(t)} (\gamma'(t)) dt.
\end{align*}
Therefore 
$$
\int_{\gamma} \omega = \int_{[a,b]} \gamma^*\omega = \int_{a}^{b} (\gamma^*\omega)_t dt = \int_a^b \omega_{\gamma(t)} (\gamma'(t)) dt.
$$
If the image $\gamma([a,b])$ does not contain in a single chart we do the following. For each point $p=\gamma(t) \in \gamma[a,b]$ choose a smooth chart $(U_p,\varphi_p)$ contain $p=\gamma(t)$. The collection $\bigcup_{p=\gamma(t)} U_p$ is an open cover for $\gamma([a,b])$. Since $\gamma([a,b])$ is compact, there is a finite subcover $U_{0},\dots,U_{k}$ for it. After reordering the indices, we may assume that $\gamma(a) \in U_0$ and $\gamma(b)\in U_k$. Since the intersections $U_{i-1} \cap U_i \neq \emptyset$ for $i=1,\dots,k$, choose a point $p_i \in U_{i-1}\cap U_i$ for each $i$ such that $p_i=\gamma(a_i) \in \gamma([a,b])$. Therefore we have a finite partition $a_0=a < a_1 <\cdots <a_k=b$ of $[a,b]$ where $\gamma(a_i) = p_i$ for each $i$, and $\gamma([a_{i-1},a_i])$ contained in a single chart. So we can apply above argument on each subinterval.

Finally if $\gamma$ is only piecewise smooth curve segment, we can simply apply the same argument on each subinterval on which $\gamma$ is smooth.    
\end{proof}




\begin{prop}[Exercise 11.35 : Prove Proposition 11.34 \textbf{Properties of Line Integrals}]
Let $M$ be a smooth manifold with or without boundary. Suppose $\gamma : [a,b] \to M$ is a piecewise smooth curve segment, and $\omega,\omega_1,\omega_2 \in \VF^*(M)$.
\begin{enumerate}[nolistsep]
\item[(a)] For any $c_1,c_2 \in \er$,
$$
\int_{\gamma} (c_1\omega_1 + c_2 \omega_2) = c_1 \int_{\gamma} \omega_1 + c_2 \int_{\gamma} \omega_2
$$
\item[(b)] If $\gamma$ is a constant map, then $\int_{\gamma} \omega = 0$.
\item[(c)] If $\gamma_1 = \gamma|_{[a,c]}$ and $\gamma_{2} = \gamma|_{[c,b]}$ with $a <c<b$, then
$$
\int_{\gamma} \omega = \int_{\gamma_1} \omega + \int_{\gamma_2} \omega.
$$
\item[(d)] If $F : M \to N$ is any smooth map and $\eta \in \VF^*(N)$, then 
$$
\int_{\gamma} F^*\eta = \int_{F \circ \gamma} \eta.
$$  
\end{enumerate}
\end{prop}
\begin{proof}
The proof rely on the proposition above. For (a),
\begin{align*}
\int_{\gamma} (c_1 \omega_1 + c_2 \omega_2) &= \int_a^b (c_1 \omega_1 + c_2 \omega_2)_{\gamma(t)} (\gamma'(t)) dt \\ &= \int_a^b \Big(c_1 \omega_1|_{\gamma(t)} + c_2 \omega_2|_{\gamma(t)} \Big)  (\gamma'(t)) dt \\ &= \int_a^b \Big(c_1 \omega_1|_{\gamma(t)} (\gamma'(t)) + c_2  \omega_2|_{\gamma(t)} (\gamma'(t)) \Big) dt \\
&= c_1 \int_a^b \omega_1|_{\gamma(t)} (\gamma'(t)) dt + c_2 \int_a^b \omega_2|_{\gamma(t)} (\gamma'(t)) dt \\ &=
  c_1 \int_{\gamma} \omega_1 + c_2 \int_{\gamma} \omega_2.
\end{align*}
For (b), suppose $\gamma$ is a constant map. Then $\gamma'(t) = 0$ for all $t \in [a,b]$. So
$$
\int_{\gamma} \omega = \int_a^b \omega_{\gamma(t)} (\gamma'(t)) dt = \int_a^b \omega_{\gamma(t)} (0) dt = \int_a^b \, 0\, dt = 0.
$$ 
For (c), 
\begin{align*}
\int_{\gamma} \omega  &= \int_a^b \omega_{\gamma(t)} (\gamma'(t)) dt\\  &= \int_a^c \omega_{\gamma(t)} (\gamma'(t)) dt + \int_c^b \omega_{\gamma(t)} (\gamma'(t)) dt\\ &= \int_{\gamma|_{[a,c]}} \omega_{\gamma(t)} (\gamma'(t)) dt + \int_{\gamma|_{[c,b]}} \omega_{\gamma(t)} (\gamma'(t)) dt \\ &= \int_{\gamma_1} \omega + \int_{\gamma_2} \omega.
\end{align*}
For (d), note that $F^*\eta \in \VF^*(M)$ and $F\circ \gamma :[a,b] \to N$ is a smooth curve in $N$, and $(F \circ \gamma)'(t) =d(F \circ \gamma)_t (d/dt|_t) =dF_{\gamma(t)} (\gamma'(t))$. So by definition on line integral of $F^*\eta$ over $\gamma$,
\begin{align*}
\int_{\gamma} F^*\eta &= \int_{[a,b]} \gamma^*(F^*\eta) = \int_a^b (F^*\eta)_{\gamma(t)} (\gamma'(t)) dt \\ &= \int_{a}^{b} dF^*_{\gamma(t)} (\eta_{F \circ \gamma (t)}) (\gamma'(t)) dt \\ &=  \int_a^b  \eta_{F \circ \gamma (t)} \big(dF_{\gamma(t)} (\gamma'(t))\big) dt \\
&= \int_a^b \eta_{F \circ \gamma (t)} \big((F \circ \gamma)'(t) \big) dt \\
&= \int_{[a,b]} (F \circ \gamma)^* \eta \\
&= \int_{F \circ \gamma} \eta.
\end{align*}
\end{proof}

\begin{prop}[Proposition 11.37 \textbf{Parameter Independence of Line Integrals}]
Suppose $M$ is a smooth manifold with or without boundary, $\omega \in \VF^*(M)$, and $\gamma$ is a piecewise smooth curve segment in $M$. For any reparametrization $\wtilde{\gamma}$ of $\gamma$, we have 
$$
\int_{\wtilde{\gamma}} \omega = \pm \int_{\gamma} \omega,
$$
with $+$ sign if $\wtilde{\gamma}$ is a forward reparametrization, and $-$ sign if $\wtilde{\gamma}$ is a backward reparametrization.
\end{prop}
\begin{proof}
First suppose that $\gamma : [a,b] \to M$ is smooth, and $\wtilde{\gamma} = \gamma \varphi$, where $\varphi : [c,d] \to [a,b]$ is an increasing diffeomorphism. Then proposition 11.31 implies
$$
\int_{\wtilde{\gamma}} \omega = \int_{[c,d]} (\gamma \circ \varphi)^* \omega = \int_{[c,d]} \varphi^*\gamma^*\omega = \int_{[a,b]} \gamma^*\omega = \int_{\gamma} \omega.
$$
The only difficulty from above arguments is the second equality. We can check that as follows
\begin{align*}
\int_{[c,d]} (\gamma \circ \varphi)^* \omega &= \int_c^d \big( (\gamma \circ \varphi)^* \omega \big)_t dt, \\ &= \int_c^d d(\gamma \circ \varphi)^*_t (\omega_{\gamma \circ \varphi (t)}) dt, \\ &= \int_c^d (d\varphi_t^* \circ d\gamma_{\varphi(t)}^* )(\omega_{\gamma \circ \varphi(t)}) dt, \quad \text{from property of dual map} \\&= \int_c^d d\varphi_t^* \big(( \gamma^* \omega )_{\varphi(t)}\big) dt, \\ &=\int_c^d \big(\varphi^* (\gamma^* \omega) \big)_t dt, \\ &= \int_{[c,d]}\varphi^*\gamma^*\omega.
\end{align*}
If $\gamma$ only piecewise smooth, the results follows simply by applying the preceding argument on each subinterval where $\gamma$ is smooth. 
\end{proof}

We say that a smooth covector field $\omega$ is \textit{\textbf{conservative}} if the line integral of $\omega$ over every piecewise smooth closed curve segment is zero.

\begin{prop}[Exercise 11.41 : Prove Proposition 11.40]
A smooth covector field is conservative if and only if its line integrals are path independent, in the sense that $\int_{\gamma} \omega = \int_{\wtilde{\gamma}} \omega $ whenever $\gamma$ and $\wtilde{\gamma}$ are piecewise smooth curve segments with the same starting and end points.
\end{prop}
\begin{proof}
Suppose that $\omega$ is a smooth conservative covector field and $p$ and $q$ be any points in $M$. Let $\gamma : [0,1] \to M$ and $\wtilde{\gamma} : [0,1] \to M$ be any piecewise smooth curve segments such that $p=\gamma(0) = \wtilde{\gamma}(0)$ and $q=\gamma(1) = \wtilde{\gamma} (1)$ (since line integrals does not depend on the parameter, we may set $[0,1]$ as the domain). We have to show that
$$
\int_{\gamma} \omega = \int_{\wtilde{\gamma}} \omega.
$$
Define a curve $\hat{\gamma} : [0,1] \to M$ as
$$
\hat{\gamma}(t) = 
\begin{cases}
\gamma(2t) & \text{for } 0 \leq t \leq \seperdua,\\
\wtilde{\gamma}(2(1-t)) & \text{for } \seperdua \leq t \leq 1.
\end{cases}
$$
This is a piecewise smooth closed curve segment starting at $\hat{\gamma}(0) = p$. Since $\omega$ is conservative, then $\int_{\hat{\gamma}} \omega = 0$. Observe that
$$
\gamma_1 := \hat{\gamma}|_{[0,\seperdua]} = \gamma \circ \varphi_1
$$
where $\varphi_1 :[0,\seperdua] \to [0,1]$ defined as $\varphi(t) = 2t$ is a forward reparametrization, and
$$
\gamma_2:= \hat{\gamma}|_{[\seperdua,1]} = \wtilde{\gamma} \circ \varphi_2, 
$$
where $\varphi_2 :[\seperdua,1] \to [0,1]$ is a backward reparametrization defined as $\varphi_2(t) = 2(1-t)$. By properties of line integrals we have
\begin{align*}
0 &= \int_{\hat{\gamma}} \omega = \int_{\gamma_1} \omega + \int_{\gamma_2} \omega \\ &= \int_{[0,\seperdua]} (\gamma \circ \varphi_1)^* \omega + \int_{[\seperdua,1]} (\wtilde{\gamma} \circ \varphi_2)^* \omega \\ &= \int_{[0,\seperdua]} \varphi_1^* \gamma^* \omega + \int_{[\seperdua,1]} \varphi_2^* \wtilde{\gamma}^* \omega \\  &= \int_{[0,1]} \gamma^* \omega - \int_{[0,1]} \wtilde{\gamma}^* \omega \\ &= \int_{\gamma} \omega - \int_{\wtilde{\gamma}} \omega \implies \int_{\gamma} \omega = \int_{\wtilde{\gamma}} \omega.
\end{align*}

Conversely, suppose that $\omega$ is a smooth covector field on $M$ such that its line integrals are path independent. Let $p \in M$ be arbitrary and $\gamma : [a,b] \to M$ is a piecewise smooth closed curve segment starting at $p$. Choose $c \in [a,b]$ such that $a<c<b$, and define 
$$
\gamma_1 = \gamma|_{[a,c]} : [a,c] \to M,
$$
and
$$
\gamma_2 = \gamma|_{[c,b]} \circ \varphi : [0,1] \to M 
$$
where $\varphi : [0,1] \to [c,b]$ is a backward reparametrization of $\gamma|_{[c,b]}$ defined as $\varphi(t) = t(c-b) + b$. The curves $\gamma_1$ and $\gamma_2$ are piecewise smooth curve segment with the same starting and end points. Since by hypothesis the integrals of $\omega$ are path independent, then $\int_{\gamma_1} \omega = \int_{\gamma_2} \omega$. So
\begin{align*}
0 &= \int_{\gamma_1} \omega - \int_{\gamma_2} \omega\\ &= \int_{\gamma|_{[a,c]}} \omega - \int_{[0,1]} (\gamma|_{[c,b]} \circ \varphi)^*\omega \\ &= \int_{\gamma|_{[a,c]}} \omega - \int_{[0,1]} \varphi^*(\gamma|_{[c,b]})^*\omega \\ &= \int_{\gamma|_{[a,c]}} \omega + \int_{[c,b]} (\gamma|_{[c,b]})^*\omega \\&= \int_{\gamma|_{[a,c]}} \omega + \int_{\gamma|_{[c,b]}} \omega \\&= \int_{\gamma} \omega.
\end{align*}
This shows that $\omega$ is conservative. 
\end{proof}

\subsection{Chapter 13 (Riemannian Metrics)}
\begin{prop} [Proposition 13.3 : \textbf{Existence of Riemannian Metrics}] Every smooth manifold with or without boundary admits a Riemannian Metric.
\end{prop}
\begin{proof}
Misalkan $M$ adalah (smooth) manifold yang dikover oleh koleksi koordinat chart $\{U_{\alpha}, \varphi_{\alpha}\}$. Untuk tiap koordinat domain $U_{\alpha}$, kita memiliki diffeomorfisma
$$
\varphi_{\alpha} : U_{\alpha} \rightarrow \hat{U}_{\alpha} = \varphi_{\alpha}(U_{\alpha}) \subset \mathbb{R}^n 
$$
Kita definisikan metrik pada $\hat{U}_{\alpha} \subset \mathbb{R}^n$ sebagai restriksi dari metrik $\bar{g}$ pada $\hat{U}_{\alpha}$. Agar sederhana, restriksi ini tetap dilambangkan sebagai $\bar{g}$. Metrik pada $U_{\alpha}$ diperoleh dari \textit{pull-back} $\bar{g}$ oleh $\varphi_{\alpha}$, 
$$
g_{\alpha}:= \varphi_{\alpha}^* \bar{g} : U_{\alpha} \rightarrow T^2(T^*M)
$$
Pilih \textit{partition of unity} $\{\psi_{\alpha}\}$ subordinat terhadap kover buka $\{U_{\alpha}\}$. Yaitu untuk tiap $U_{\alpha}$ kita punya fungsi terdiferensialkan $\psi_{\alpha} : M \rightarrow \mathbb{R}$ dimana $\text{supp } \psi_{\alpha} \subseteq U_{\alpha}$. Dengan mengalikan restriksi $\psi_{\alpha}|_{U_{\alpha}}$ dengan $g_{\alpha}$, kita peroleh metrik $\psi_{\alpha} g_{\alpha} : U_{\alpha} \rightarrow T^2(T^*M)$ dengan $\text{supp }\psi_{\alpha} g_{\alpha} \subset U_{\alpha}$. Dengan properti ini, kita dapat mengekstensi $\psi_{\alpha}g_{\alpha}$ ke seluruh $M$ dimana nilainya pada $M\smallsetminus \text{supp }\psi_{\alpha}g_{\alpha}$ adalah nol. Dengan melakukan ini untuk seluruh $g_{\alpha}$, kita definisikan metrik pada $M$ sebagai 
$$
g = \sum_{\alpha} \psi_{\alpha}g_{\alpha},
$$
dimana tiap suku $\psi_{\alpha}g_{\alpha}$ kita anggap sebagai ekstensi $\psi_{\alpha}g_{\alpha}$ diatas. Dari sifat terhingga lokal (\textit{local finiteness}), hanya ada berhingga banyaknya suku untuk suatu lingkungan dari suatu sebarang $p \in M$. Sehingga metrik $g$ diatas terdefinisi dengan baik (tidak ada isu kekonvergenan). Metrik diatas simetrik, karena tiap $g_{\alpha}$ simetrik. Sehingga tinggal sifat positif definit yang perlu diverifikasi. Untuk sebarang $p \in M$ dan vektor tak-nol $v \in T_pM$,
$$
g_p(v,v) = \sum_{\alpha} \psi_{\alpha}(p) g_{\alpha}|_p (v,v)
$$
tak-negatif karena $ 0 \leq\psi_{\alpha}(p) \leq 1$ dan 
$$
g_{\alpha}|_p(v,v) = (\varphi^*_{\alpha} \bar{g})_p (v,v) = \bar{g} \, (d\varphi_{\alpha}|_p (v) , d\varphi_{\alpha}|_p (v)) > 0
$$
Dengan demikian $g$ merupakan metrik pada $M$.
\end{proof}

\begin{prop}[Exercise 13.10 :  \textbf{Pullback Metric Criterion}]
Suppose $F : M \rightarrow N$ is a smooth map and $g$ is a Riemannian metric on $N$. Then $F^*g$ is a Riemannian metric on $M$ if and only if $F$ is a smooth immersion.
\end{prop}
\begin{proof}
Tensor $F^*g$ simetrik karena $g$ simetrik. Untuk menunjukan bahwa $F$ adalah smooth immersion kita tunjukan bahwa $\forall p \in M$, $\text{Ker}\,dF_p = \{0\}$. Misalkan $v \in T_pM$ dimana $dF_p(v) = 0$. Karena $F^*g$ adalah metrik Riemann maka
$$
(F^*g)_p(v,v)  = g_{F(p)} (dF_p(v),dF_p(v)) = g_{F(p)} (0,0) = 0 \implies v=0.
$$ 
Sehingga $F$ adalah smooth immersion. 

Untuk konversnya, misalkan $F$ smooth immersion. Untuk tiap $p\in M$ dan $v \in T_pM$, kita harus menunjukan bahwa
$$
(F^*g)_p(v,v)  = g_{F(p)} (dF_p(v),dF_p(v)) > 0 \quad \text{jika } v \neq 0
$$
dan $(F^*g)_p(v,v) = 0$ j.h.j $v = 0$. Dari hipotesis, $\text{Ker}\,dF_p = \{0\}$. Sehingga untuk $v \neq 0$, $dF_p(v) \neq 0$. Dari sifat positif definit $g$, kita simpulkan $(F^*g)_p(v,v) > 0$. Bila $v=0$ maka $dF_p(v) = 0$. Sehingga $(F^*g)_p(v,v) = 0$. Jika $(F^*g)_p(v,v)= 0$ maka $g_{F(p)} (dF_p(v),dF_p(v)) = 0$. Sifat positif definit $g$ memberikan $dF_p(v) = 0$. Karena $\text{Ker}\,dF_p = \{0\}$ maka $v = 0$.
\end{proof}

\begin{prop}[Problem 13.2 \cite{LeeSM}]
Suppose $E$ is a smooth vector bundle over a smooth manifold $M$ with or without boundary, and $V \subhim E$ is an open subset with the property that for each $p \in M$, the intersection of $V$ with the fiber $E_p$ is convex and nonempty. By a "section of $V$", we mean a (local or global) section of $E$ whose image lies in $V$.
\begin{enumerate}[nolistsep]
\item[(a)] Show that there exists a smooth global section of $V$.
\item[(b)] Suppose $\sigma : A \to V$ is a smooth section of $V$ defined on a closed subset $A \subhim M$. (This means that $\sigma$ extend to a smooth section of $V$ in a neighbourhood of each point of $A$.) Show that there exists a smooth global section $\wtilde{\sigma}$ of $V$ whose restriction to $A$ is equal to $\sigma$. Show that if $V$ contains the image of the zero section of $E $, then $\wtilde{\sigma} $ can be chosen to be supported in any predetermined neighbourhood of $A$.
\end{enumerate} 
\end{prop}
\begin{proof}
For (a), first we have to define local section on a neighbourhood of each points in $M$. Let $p \in M$ and $v \in V\cap E_p$ arbitrary. Choose a local trivialization $\Phi : \pi^{-1}(U) \to U\times \rk$ over a neighbourhood $U$ of $p$. Suppose that $\Phi(v) = (p, (v^1,\dots,v^n))$, define a smooth local section over $U$ with constant coefficient $\sigma_p : U \to E$ by $\sigma_p(q) = \Phi^{-1} (q,(v^1,\dots,v^n))$ for any $q \in U$. Since $v \in V\cap E_p \subhim V$ and $V$ is open, we can find an open neigbourhood $W$ of $v$ such that $W \subhim V$. By continuity, $\sigma_p^{-1}(W)$ is an open subset of $p$ contained in $U$. So therefore by shrinking the domain of $\sigma_p$ to $U_p = \sigma^{-1}(W) \subhim U$, we obtain a smooth local section of $V$ $\sigma_p : U_p \to E$ over a neighbourhood of each point in  $M$. Let $\{\psi_p : M \to \er  : \forall p \in M \}$ be the partition of unity subordinate to the open cover $\{U_p : \forall p \in M \}$ of $M$. For each $p \in M$, we multiply $\sigma_p$ with $\psi_p|_{U_p}$ and extend it to $M$, we obtain global section $\psi_p \sigma_p : M \to E$ such that supp $\psi_p \sigma_p \subhim U_p$. Define a map $\sigma : M \to E$ by 
$$
\sigma(x) = \sum_{p \in M} (\psi_p \sigma_p)(x).
$$
Because the collection of supports $\{\text{supp }\psi_p \}$ is locally finite, the sum has only finite number of terms in a neighbourhood of any point of $M$, and therefore define a smooth global section. To see that this is a section of $V$, just note that since $\sum_{p \in M} \psi_p(x) = 1$ for any $x \in M$ and each $\sigma_p$ is a section of $V$, therefore the convexity of $V \cap E_p$ implies that $\sigma(x) = \sum_{i=1}^{k} \psi_{p_i}(x) \sigma_{p_i}(x) \in V$ (look remark below).

For (b), let $p$ be any point in $A$ and $\sigma_p : U_p \to V$ be the local section of $V$ over a neighbourhood $U_p$ of $p$ such that $\sigma_p|_A = \sigma|_A$. The collection $\mathcal{U} = \{U_p : \forall p\in A \} \cup \{M \smallsetminus A\}$ is an open cover of $M$. Let $\{\psi_p  : \forall p \in A \} \cup \{\psi_0\}$ be the smooth partition of unity subordinate to this open cover. By (a), we have a global section of $V$ $\sigma' : M \to V$. Define the map $\wtilde{\sigma} : M \to E$ as
$$
\wtilde{\sigma} = \psi_0 \sigma' + \sum_{p \in A} \psi_p \sigma_p,
$$ 
where the multiplication $\psi_p \sigma_p$ interpreted as before. By the same argument as (a), $\wtilde{\sigma}$ is a smooth global section of $E$. Since supp $\psi_0 \subhim M \smallsetminus A$, then for any point $x \in A$
$$
\wtilde{\sigma}(x) = \psi_0(x) \sigma'(x) + \sum_{p \in A} \psi_p(x) \sigma_p(x) =  0 + \big( \sum_{p \in A} \psi_p(x) \big) \sigma(x) = \sigma(x).
$$  
So $\wtilde{\sigma}$ agree with $\sigma$ on $A$. If $x \in M \smallsetminus A$, the value $\wtilde{\sigma}(x) = \psi_0(x) \sigma'(x) + \sum_{p \in A} \psi_p(x) \sigma_p(x)$ is just the linear combination of vectors in $V \cap E_x$. Since $V \cap E_x$ convex and $\psi_0(x) + \sum_{p \in A} \psi_p(x) = 1$ then $\wtilde{\sigma}(x) \in V \cap E_x$. Therefore $\wtilde{\sigma} : M \to V$ is a smooth global section of $V$ that agree with $\sigma$ on $A$. 

Suppose that $V$ contains the image of the zero section of $E$. By choosing $U_p \subhim U$ from the beginning (where $U $ is a predetermined neighbourhood of $A$), we obtained a smooth global section of $V$, $\wtilde{\sigma} : M \to E$  defined by $\wtilde{\sigma} = \sum_{p \in A} \psi_p \sigma_p$, where 
$$
\text{supp } \wtilde{\sigma} \subhim \overline{\bigcup_{p \in A} \text{supp } \psi_p}  = \bigcup_{p \in A} \text{supp }\psi_p \subhim U
$$ 
by virtue of Lemma 1.13(b).
\end{proof}

\begin{remark}[\textbf{About Covex Set}]
If $C \subhim V$ is a convex subset of a vector space $V$, then for any $v_0,\dots,v_n \in C$ and real numbers $t_0,\dots,t_n \in \er$ such that $0 \leq t_i \leq 1$ and $\sum_{i=0}^{n} t_i = 1$
$$
\sum_{i=0}^{n}t_iv_i \in C.
$$
The converse is also true, but we only need $n=1$.

To prove this, note that for $n=1$, the sum $t_0 v_0 + t_1 v_1$ is in $C$ since 
$$
t_0v_0 + t_1 v_1 = t_0 v_0 + (1-t_0)v_1 = v_1 + t_0(v_0 -v_1) 
$$ 
which is an element on the line segment from $v_1$ to $v_0$. Suppose that $\sum_{i=0}^{k-1} t_iv_i = \sum_{j=1}^{k} t_jv_j \in C$ holds for an integer $k$. Then we can write the sum $\sum_{i=0}^{k}t_iv_i$ as
$$
\sum_{i=0}^{k}t_iv_i = t_0v_0 + \sum_{i=1}^{k}t_iv_i =  t_0v_0 + \sum_{i=1}^{k} t_i \cdot \frac{\sum_{i=1}^{k} t_i v_i}{\sum_{i=1}^{k} t_i} = t_0 v_0 + \big(\sum_{i=1}^{k} t_i \big) v
$$ 
with $v = \sum_{i=1}^{k} t_i v_i/\sum_{i=1}^{k} t_i$. By assumption, the vector $v$ is in $C$ since 
$$
v =  \frac{\sum_{i=1}^{k} t_i v_i}{\sum_{i=1}^{k} t_i} = \sum_{i=1}^{k} a_i v_i, \quad \text{with } a_i =  \frac{t_i}{\sum_{i=1}^{k} t_i} \text{ and }  \sum_{i=1}^{k} a_i = 1.
$$
So we have $v_0,v \in C$ and $t_0 + \sum_{i=1}^{k}t_i = 1$. This implies 
$$
\sum_{i=0}^{k}t_iv_i  = t_0 v_0 + \big(\sum_{i=1}^{k} t_i \big) v \in C. 
$$
\end{remark}

\begin{prop}[Problem 13.21]
Let $(M,g)$ be a Riemannian manifold, let $f \in \CM$, and let $p\in M$ be a regular point of $f$.
\begin{enumerate}[nolistsep]
\item[(a)] Show that among all unit vectors $v \in T_pM$, the directional derivative $vf$ is greatest when $v$ points in the same direction as $\text{grad }f|_p$, and the length of $\text{grad }f|_p$ is equal to the value of the directional derivative in that direction.
\item[(b)] Show that $\text{grad }f|_p$ is normal to the level set of $f$ through $p$. 
\end{enumerate}
\end{prop}
\begin{proof}
By definition, $\text{grad }f = \widehat{g}^{-1} (df) \in \VF(M)$ such that for any $v \in T_pM$
$$
\langle \text{grad }f|_p, v\rangle_g = \widehat{g}(\text{grad }f|_p) (v) = \widehat{g}(\widehat{g}^{-1}(df)(p)) (v) = df_p(v) = vf.
$$
So by Cauchy-Schwarz inequality we have
$$
 \big|\langle\text{grad }f|_p, v\rangle_g\big|^2 \leq \big|\text{grad }f|_p\big|^2 \big|v\big|^2 = \big|\text{grad }f|_p\big|^2.
$$
So $vf$ maximize when $vf = \big|\text{grad }f|_p\big|$. This unit vector is actually the vector that point in the direction of $\text{grad }f|_p$, since
$$
\text{grad }f|_p (f) = \langle \text{grad }f|_p,\text{grad }f|_p \rangle_g = \big|\text{grad }f|_p\big|^2 = \big|\text{grad }f|_p\big| \cdot vf, 
$$
which is gives us 
$$
vf = \frac{\text{grad }f|_p}{\big|\text{grad }f|_p\big|} f.
$$ 

For (b), let $S\subhim M$ is the level set through $p$, that is $S = f^{-1}(f(p))$. Since $T_pS = \text{Ker }df_p$, then for any $v \in T_pS$
$$
 \langle \text{grad }f|_p,v \rangle_g = vf = df_p(v) = 0.
$$
Therefore $ \text{grad }f|_p$ normal to $T_pS$.
\end{proof}

\subsection{Chapter 15 (Orientations)}

\begin{prop}[Exercise 15.4]
Suppose $M$ is an oriented smooth $n$-manifold with or without boundary, and $n\geq 1$. Show that every local frame with connected domain is either positively oriented or negatively oriented. Show that connected assumption is necessary.
\end{prop}
\begin{proof}
Let $\{\sigma_i : U \to TM \, | \, i=1,\dots,n \}$ be any local frame over an open connected subset $U \subhim M$ of the oriented manifold $(M , \mathcal{O})$. By definition, $\mathcal{O}$ is an continous pointwise orientation, which is means that any point is in the domain of an oriented local frame. Let $p$ be an arbitrary point in $U$ and $\wtilde{\sigma}_i : U_p \to TM$ be an oriented local frame with domain $U_p\subhim U$ contain $p$. Since $\{\sigma_i(p)\}$ and $\{\wtilde{\sigma}_i(p)\}$ are both basis for $T_pM$ then we have a relation $\wtilde{\sigma}_j(p) = A^i_j(p) \sigma_i(p)$. Denote $(\varepsilon^i)$ as the local coframe dual to $(\sigma_i)$, then we have a continous functions $A^i_j : U_p \to \er $ for $i,j=1,\dots,n$, defined as
$$
A^i_j(q) = \varepsilon^i (\wtilde{\sigma}_j)(q) = \varepsilon^i(q) \wtilde{\sigma}_j(q), \quad \forall q \in U_p.
$$  
By composing the continous function $A : U_p \to \GL$ defined as $A(q) = [A^i_j(q)]$ with the determinant function, we obtain a continous function $O = \det \circ A : U_p \to \er$. Since $U_p$ connected and $O(q) \neq 0$ for all $q\in U_p$, it follows that this function is either positive or negative on $U_p$.  Therefore, the given frame $\{\sigma_i\}$ is either positively oriented or negatively oriented on a neighbourhood $U_p$ of each point $p \in U$. 
 
Now, let $p$ and $q$ be any two distinct points on $U$. 
Since $U$ is connected, we can find a continous curve $\gamma : [0,1] \to U$ such that $\gamma(0)=p$ and $\gamma(q) = 1$. By above argument, for any $x \in \gamma([0,1])$ we can always find a neighbourhood $U_x$ of $x$ such that $\{\sigma_i\}$ is positively oriented or negatively oriented on it. The collection $\{U_x\}$ is an open cover for $\gamma([0,1])$. By continuity, there exists a finite number of such neighbourhoods $\{U_{x_i}\}_{i=1}^k$. Assume that $x_1= p$ and $x_k=q$. Since the neighbourhoods intersect nontrivially, then once $\{\sigma_i\}$ is positively (or negatively) oriented on $U_p$, then it follows that $\{\sigma_i\}$ will also positively (negatively) oriented on $U_q$. Because $p$ and $q$ are arbitrary, therefore $\{\sigma_i\}$ is either positively oriented or negatively oriented on $U$.       
\end{proof}

\begin{prop}[Exercise 15.10 : Prove Proposition 15.9]
Let $M$ be a connected, orientable smooth manifold with or without boundary. Then $M$ has exactly two orientations. If two orientations of $M$ agree at one point, they are equal.
\end{prop}
\begin{proof}
Let $(M,\mathcal{O})$ be the connected smooth oriented manifold. By Proposition 15.5, there exists a nonvanishing $n$-form $\omega$ on $M$ such that $\omega$ is positively oriented at each point. Any other choice of orientation $\wtilde{\mathcal{O}}$ for $M$ will induce a nonvanishing $n$-form $\wtilde{\omega}$ on $M$. Since $\omega = f \wtilde{\omega}$ for some smooth function $f : M \to \er$ such that $f(p) \neq 0$ for all $p \in M$, then by the connectedness of $M$, the image $f(M) \subhim \er$ is a connected subset which does not contain $0$. In the other words, the function $f$ either always positive or negative on $M$. If $f$ is positive, then by Proposition 15.3, $\omega_p$ and $\wtilde{\omega}_p$ determine the same orientation at $T_pM$ for each $p \in M$. Hence $\wtilde{\mathcal{O}} = \mathcal{O}$. So lets assume that $f$ is negative. By the similar argument $\omega_p$ and $\wtilde{\omega}_p$ determine different orientation at $T_pM$  for each $p \in M$. So $\mathcal{O}$ is different than $\wtilde{\mathcal{O}}$. By similar argument as above, any other orientation $\mathcal{O}'$ for $M$ will induce a non-vanishing $n$-form $\omega'$ such that $\omega' = g \omega$ for either $g : M \to \er$ is positive or negative function. If $g$ is positive, then $\mathcal{O}' = \mathcal{O}$. If $g$ is negative then $\mathcal{O}'$ different than $\mathbb{O}$ and then the product $gf$ is a positive function. Hence above relation $\omega' =g \omega= gf \wtilde{\omega}$ implies that $\mathcal{O} = \wtilde{\mathcal{O}}$. This proves that there are exactly two orientation on $M$. 

Suppose $\mathcal{O}_1$ and $\mathcal{O}_2$ are orientation for $M$ and $\omega_1$ and $\omega_2$ are the orientation forms for $\mathcal{O}_1$ and $\mathcal{O}_2$ respectively. Let $\omega_1 = f \omega_2$. If they agree at a point $p \in M$, then the orientation forms is positive multiples of each other at $p$. This implies that $f$ is a positive function. Hence $\omega_1$ and $\omega_2$ determines the same orientation for each point in $M$, i.e., $\mathcal{O}_1 = \mathcal{O}_2$.  
\end{proof}

\begin{prop}[Exercise 15.12 : Prove Proposition 15.11 \textbf{Orientations of Codimension-\boldmath$0$ Submanifolds}]
Suppose $M$ is an oriented smooth manifold with or without boundary, and $D \subhim M$ is a smooth codimension-$0$ submanifold with or without boundary. Then the orientation of $M$ restrict to an orientation of $D$. If $\omega$ is an orientation form for $M$, then $\iota^*_D\omega$ is an orientation form for $D$.  
\end{prop}
\begin{proof}
By Proposition 5.49, the codimension-0 submanifold with or without boundary $D \subhim M$ is an open subset of $M$. Give $(M,\mathcal{O})$, the pointwise orientation for $D$ is given by restrict $\mathcal{O}$ to $D$. By restrict every oriented local frames for $M$ that has nontrivial intersection with $D$, we conclude that $\mathcal{O}|_D$ is a continous orientation for $D$.

Suppose that $\omega$ is an orientation form for $M$, that is a nonvanishing $n$-form in $M$. The $n$-form $\iota_D^*\omega$ is nonvanishing. Let $p \in D\subhim M$ arbitrary and let  $v_1,\dots,v_n \in T_pD$ be linear independent vectors. Because $d(\iota_D)_p : T_pD \to T_pM$ is an isomorphism, we have
$$
(\iota_D^*\omega)_p(v_1,\dots,v_n) = \omega_p(d(\iota_D)_p(v_1),\dots,d(\iota_D)_p(v_n)) \neq 0.
$$
Hence $\iota^*_D\omega$ is an orientation form of $D$.

\end{proof}

\begin{prop}[Exercise 15.13]
Suppose $M$ and $N$ are oriented positive-dimensional smooth manifold with or without boundary, and $F : M \to N$ is a local diffeomorphism. Show that the following are equivalent.
\begin{enumerate}[nolistsep]
\item[(a)] $F$ is orientation preserving.
\item[(b)] With respect to any oriented smooth charts for $M$ and $N$, the Jacobian matrix of $F$ has positive determinant.
\item[(c)] For any positively oriented orientation form $\omega$ for $N$, the form $F^*\omega$ is positively oriented for $M$.
\end{enumerate}
\end{prop}
\begin{proof}
We will show that $(a)\Rightarrow (b)$, $(b) \Rightarrow (c)$ and $(c) \Rightarrow (a)$. For $(a) \Rightarrow (b)$, let $F : M \to N$ be an orientation preserving local diffeomorphism. Let $p \in M$ be arbitrary and $U$ be a neighbourhood of $p$ such that $F|_{U} : U \to F(U)$ is a diffeomorphism. By shrinking $U$ if necessary, we may assume that $(U,x^i)$ is an oriented smooth chart contain $p$ and $(F(U),y^i)$ is an oriented smooth chart for $N$ contain $F(p)$. By assumption, for any $q \in U$, the isomorphism $dF_q$ takes oriented basis $\doo/\doo x^i|_q$ of $T_qM$ to oriented basis $dF_p(\doo/\doo x^i|_q)$ of $T_{F(q)}N$. Since both $\doo/\doo y^i|_{F(q)}$ and $dF_q(\doo/\doo x^i|_q)$ are positively oriented, then by definition the transition matrix $(B^i_j)$ defined by $dF_q(\doo/\doo x^i|_q) = B^j_i\doo/\doo y^j|_{F(q)} $, has positive determinant. Straighforward computation shows that
$$
dF_q\bigg( \frac{\doo}{\doo x^i}\bigg|_q \bigg) = \frac{\doo \widehat{F}^j}{\doo x^i}(q) \frac{\doo}{\doo y^j}\bigg|_{F(q)} \implies B_i^j = \frac{\doo \widehat{F}^j}{\doo x^i}(q). 
$$
Since this is true for any $q \in U$, then the Jacobian matrix has positive determinant on $U$. This proves $(a)\Rightarrow (b)$. 

To show $(b) \Rightarrow (c)$, suppose that $(b)$ holds and $\omega$ is a positively oriented orientation form for $N$. For any $p\in M$ and oriented smooth charts $(U,x^i)$ and $(V,y^i)$ as before we have 
\begin{align*}
(F^*\omega)_p\bigg( \frac{\doo}{\doo x^1}\bigg|_p,\dots,\frac{\doo}{\doo x^n}\bigg|_p \bigg) &= \omega_{F(p)} \bigg(dF_p\bigg(\frac{\doo}{\doo x^1}\bigg|_p\bigg), \dots, dF_p\bigg(\frac{\doo}{\doo x^n}\bigg|_p\bigg)\bigg) \\
&= \det \bigg(\frac{\doo \widehat{F}^j}{\doo x^i}(p) \bigg)\, \omega_{F(p)} \bigg( \frac{\doo}{\doo y^1}\bigg|_p,\dots,\frac{\doo}{\doo y^n}\bigg|_p \bigg).
\end{align*} 
Since $(b)$ holds and $\omega$ is positively oriented, then $F^*\omega$ is positively oriented. This proves $(b) \Rightarrow (c)$.

For $(c) \Rightarrow (a)$, suppose $(c)$ holds, $p\in M$ arbitrary and let $(E_i)$ be any oriented basis of $T_pM$. We have to show that $\wtilde{E}_i = dF_p(E_i)$ is an oriented basis for $T_{F(p)}N$. Since $\omega$ and $F^*\omega$ are both positively oriented by assumption, then
$$
(F^*\omega)_p(E_1,\dots,E_n) = \omega_{F(p)} (dF_p(E_1),\dots, dF_p(E_n)) = \omega_{F(p)} (\wtilde{E}_1,\dots,\wtilde{E}_n) > 0. 
$$
This shows that $(\wtilde{E}_i)$ is positively oriented.
\end{proof}

\begin{prop}[Exercise 15.14]
Show that a composition of orientation-preserving maps is orientation-preserving.
\end{prop}
\begin{proof}
Suppose that $F : M \to N$ and $G : N \to P$ are both local diffeomorphism and orientation preserving maps. Since every composition of local diffeomorphism is a local diffeomorphism, then the map $G \circ F : M \to P$ is a local diffeomorphism. Let $p \in M$ arbitrary and $(E_i)$ be any oriented basis of $T_pM$. By assumption,
$$
d(G \circ F)_p(E_i) = dG_{F(p)} \circ dF_p(E_i) = dG_{F(p)} (dF_p(E_i))
$$
is an oriented basis of $T_{G \circ F(p)}P$. So $G \circ F$ is an orientation-preserving map.
\end{proof}

\begin{prop}[Exercise 15.16]
Suppose $F : M \to N$ and $G : N \to P$ are local diffeomorphisms and $\mathcal{O}$ is an orientation on $P$. Show that $(G \circ F)^*\mathcal{O} = F^*(G^*\mathcal{O})$. 
\end{prop}
\begin{proof}
By Proposition 15.15 the orientation $(G \circ F)^*\mathcal{O}$ is a unique orientation on $M$ such that $G \circ F$ is orientation-preserving. Let $\omega$ be the orientation form for $P$. Then Exercise 15.13. implies that $(G \circ F)^*\omega=F^*(G^* \omega)$ is a positively oriented orientation form for $M$. Since $(G \circ F)^*\mathcal{O}$ unique, then $(G \circ F)^*\omega$ determine $(G \circ F)^*\mathcal{O}$. Therefore $(G \circ F)^*\mathcal{O} = F^*(G^*\mathcal{O})$ since $(G \circ F)^*\omega=F^*(G^* \omega)$ determine both $(G \circ F)^*\mathcal{O}$ and  $F^*(G^*\mathcal{O})$.
\end{proof}

\begin{prop}[Exercise 15.30]
Suppose $(M,g)$ and $(\wtilde{M},\wtilde{g})$ are positive-dimensional Riemannian manifolds with or without boundary, and $F : M \to \wtilde{M}$ is a orientation-preserving local isometry. Show that $F^*\omega_{\wtilde{g}} = \omega_g$. 
\end{prop}
\begin{proof}
The existence of Riemannian volume form $\omega_{\wtilde{g}} \in \Omega^n(\wtilde{M})$ and $\omega_g \in \Omega^n(M)$ is guaranteed by Proposition 15.29. Since these Riemannian volume form is unique, its enough to show that the $n$-form $F^*\omega_{\wtilde{g}}$ satisfy
$$
(F^*\omega_{\wtilde{g}}) (E_1,\dots,E_n) =1
$$
for every local orthonormal frame $(E_i)$ on $M$. Suppose that $(E_i)$ is a local oriented orthonormal frame on an open subset $U \subhim M$. Let $p \in U$ be arbitrary. Since $F : M \to \wtilde{M}$ is a local isometry, we can find a neighbourhood $V$ of $p$ such that $F(V)$ is open, $F : V \to F(V)$ is a diffeomorphism and $F^*\wtilde{g}=g$ on $V$. By shrinking $V$ if necessary, we may assume that $V \subhim U$, and therefore $(E_i)$ is an oriented local orthonormal frame on $V$. Hence for any $q \in V$,
$$
\delta_{ij} = \metric{E_i|_q,E_j|_q}_g = (F^*\wtilde{g})_q(E_i|_q,E_j|_q) = \metric{dF_q(E_i|_q),dF_q(E_j|_q)}_{\wtilde{g}}.
$$
This shows that $(dF_q(E_i|_q))$ is an oriented (by assumption on $F$) orthonormal basis for $T_{F(q)}\wtilde{M}$. In fact, by smoothness of $dF :TM \to T\wtilde{M}$, then $(dF(E_i))$ is a local oriented orthonormal frame on $F(V)$. By all of this facts we have
$$
(F^*\omega_{\wtilde{g}})(E_1,\dots,E_n) = \omega_{\wtilde{g}} (dF(E_1),\dots,dF(E_n)) = 1. 
$$
This proves that $F^*\omega_{\wtilde{g}} = \omega_{g}$. 
\end{proof}

\subsection{Chapter 16 (Integration on Manifold)}

\begin{prop}[Exercise 16.18]
	Prove that $\bar{\mathbb{R}}^n_+$ is homeomorphic to upper half-space $\hn$.
\end{prop}
\begin{proof}
	Construct a homeomorphism $f : \bar{\mathbb{R}}_+^2 \to \mathbb{H}^2$ defined by restriction of the map $f(z) = z^2$ to $\bar{\mathbb{R}}_+^2=[0,\infty) \times [0,\infty) \subset \mathbb{R}^2 \approx \mathbb{C}$. So we have $\mathbb{H}^2\approx \bar{\mathbb{R}}^2_+$.
	Suppose that $\mathbb{H}^n\approx \bar{\mathbb{R}}^n_+$ for some $n$. Clearly this hold for $n=1$. And 	we know that if $A \approx B$ then $A \times C \approx B \times C$, since the map $f : A \times C \to B \times C$ defined by $f(a,c) = (\varphi(a),c)$ (where $\varphi$ is the homeomorphism between $A$ and $B$) is a homeomorphism. By this
	\begin{align*}
	\bar{\mathbb{R}}^{n+1}_+ = \bar{\mathbb{R}}^{n}_+ \times \bar{\mathbb{R}}_+ &\approx \mathbb{H}^n \times \bar{\mathbb{R}}_+ = \mathbb{R}^{n-1} \times \bar{\mathbb{R}}_+ \times \bar{\mathbb{R}}_+  \\ &\approx \mathbb{R}^{n-1} \times \mathbb{H}^2 = \mathbb{R}^{n-1} \times \mathbb{R} \times  \bar{\mathbb{R}}_+ = \mathbb{R}^n \times \bar{\mathbb{R}}_+ = \mathbb{H}^{n+1}.
	\end{align*}
\end{proof}
%%%%%%%%%%%%%%%%%%%%%%%%%%%%%%%%%%%%%%%%%%% M O R S E - T H E O R Y
\section{Morse Theory and Floer Homology}
The following are my notes (mostly consist of filling in the details, some comments and exercise) from Mich\`ele Audin's $\textit{Morse Theory and Floer Homology}$ \cite{Audin}, \cite{BH}, \cite{MilnorM}, and \cite{YukioM}.


\begin{prop}[Proposition 1.2.1 \cite{Audin} The Existence of Morse Functions]
Let $V \subhim \rn$ be a submanifold. For almost every point $p$ of $\rn$, the function 
$$
f_p : V \to \er, \, x  \mapsto \norm{x-p}^2
$$
is a Morse function.
\end{prop}



%%%%%%%%%%%%%%%%%%%%%%%%%%%%%%%%%%%%%%%%%%%%% Differential Topology
\section{Differential Topology (Milnor)}

\begin{prop}[page 8 \cite{MilnorDT}]
Let $F : M \to N$ is a smooth map between manifolds of same dimension. If $M$ is compact and $y \in N$ is a regular value, then $M$ is finite (possibly empty).
\end{prop}
\begin{proof}
Since $F^{-1}(y)$ is a closed subset of the compact space $M$, then $F^{-1}(y)$ is compact. For any $p \in F^{-1}(y)$, we have a nbd $U_p$ of $p$, and nbd $V$ of $y$ such that $F|_{U_p} : U_p \to V$ is a diffeomorphism. Since $F(p)=y$ and $F|_{U_p}$ is one to one, then there is no other points of $F^{-1}(y)$ in $U_p$. So we have $U_p \cap F^{-1}(y) = \{p\}$. This means that $\{p\}$ is open in $F^{-1}(y)$. Since $p \in F^{-1}(y)$ arbitrary, then all points in $F^{-1}(y)$ is open in $F^{-1}(y)$, and  we have an open cover $\{\{p\} : \forall p \in F^{-1}(y) \}$  for $F^{-1}(y)$. By compactness, there is finite subcover $\{\{p_i\} : i=1,\dots,k \}$. This clearly contradict if we assume $F^{-1}(y)$ is infinite. Hence,
$$
F^{-1}(y) = \{p_1,\dots,p_k\}.
$$
\end{proof}
\section{Riemannian Geometry}


\section{Algebraic Topology}
Here i am following \cite{bredon} entirely. The first two chapters dealing with basic material; general topology and differentiable manifold.

\subsection*{I.13 Quotient Spaces.} 
Quotient space sometimes can have very non-Hausdorff topologies, as showed by the following problem.

\begin{prop}[Problem 6, Ch I, Sec 13 \cite{bredon}]
Consider the real line $\er$, with the equivalence relation $x \sim y \Leftrightarrow x - y$ is rational. Show that $\er/{\sim}$ has an uncountable number of points, but its topology is the trivial one.
\end{prop} 
\begin{proof}
 For any two points $x$ and $y$ in $\er = \rational \cup \rational^c$, there are three possibility: (1) both rational (2) one rational and another irrational, and (3) both irrational. For (1), if $x,y \in \rational$ then $x-y$ is rational. For (2), $x \in \rational$ and $y \in \rational^c$, $x-y \in \rational^c$ (if not then $y$ is rational, which is a contradiction). For the last, $x,y \in \rational^c$, observe the following:  For any irrational number $x$, there are infinitely many irrational numbers $y$ such that $x+y$ is rational. To see this, take $y = q-x$ where $q \in \rational$. Also, for any irrational $x$ the numbers $y =qx$ (where $q \in \rational$ arbitrary) is irrational. Hence $x+y$ is irrational. The observation above tell us that all rationals are in the same equivalence class, say $[q]$, and there are no irrationals in $[q]$. The rest of the equivalence classes are irrationals. Suppose that $\er/{\sim}$ is countable. By this we can label the classes such as 
 $$
 [q] \quad [x_1] \quad [x_2] \quad  [x_3] \quad \cdots
 $$
 where $x_i$ are irrationals. Suppose that for a fix $i$, any irrational $y \in [x_i]$ must have the form $y = x-q$ for some $q \in \rational$. Since $\rational$ is countable, then the irrationals in each class $[x_i]$ is countable. Since their union is equal to $\er$, then this implies that $\er$ is countable, which is obviously false. Therefore $\er/{\sim}$ is uncountable.  
 
 Let $\pi  : \er \to \er/{\sim}$ be the canonical projection and $U \subset \er/{\sim}$ be an open subset in the quotient topology of $\er/{\sim}$. For any $[y] \in \er/{\sim}$, the preimage is
 $$
 \pi^{-1}([y]) = \{z\in \er : z-y=q \text{ for some }q \in \rational \}.
 $$
 This means that $z = y+ \rational$. Since $\rational$ is dense in $\er$, then $y+\rational$ intersects open subset $\pi^{-1}(U)$. This holds for all $[y]\in \er/\sim$. Therefore $\er/{\sim} \subset U$, so $U = \er/{\sim}$ and $\er/{\sim}$ has trivial topology. 
\end{proof}

\begin{prop}[13.6. Example \cite{bredon}]

\end{prop}


\section{Miscellaneous}

\subsection*{Set Theory}

\begin{prop}[\textbf{Inclusion Map}]
	Let $X$ be a set. For any subset $U \subset X$ define a map $\iota : U \hookrightarrow X$ as $\iota(x) = x$, called the \textit{inclusion map} of $U$. For any subset $A \subhim X$, then
	$$
	\iota^{-1}(A) = U \cap A.
	$$ 
\end{prop}

\begin{prop}[\textbf{Image of Complement equal to Complement of the Image}]
	Let $f : X \to Y$ be a map between sets and $A \subhim X$ is a subset. Then 
	\begin{enumerate}[nolistsep]
		\item[(a)]	$$
		f(X \smallsetminus A) \supseteq Y \smallsetminus f(A) \Leftrightarrow f \text{ is surjective}.$$
		\item[(b)] $$
		f(X \smallsetminus A) \subhim Y \smallsetminus f(A) \Leftrightarrow f \text{ is injective}.$$
		\item[(c)] $$
		f(X\smallsetminus A) = Y \smallsetminus f(A) \Leftrightarrow f \text{ is bijective}.
		$$ 
	\end{enumerate}
\end{prop}
\begin{proof}
	For (a), let the inclusion holds and take $A = \emptyset$. This implies $f(X) \supseteq Y$. Since $f(X) \subhim Y$ always hold, then $f(X) = Y$. Conversely, suppose $f$ is surjective and let $y \in Y \smallsetminus f(A)$. By hypothesis, there exists $x \in X$ such that $f(x) = y$. Since $f \ni f(A)$ then $x \in X \smallsetminus A$. This means that $y \in f(X \smallsetminus A)$. So $f(X \smallsetminus A) \supseteq Y \smallsetminus A$. 
	
	For (b), suppose that $f$ is not injective, that is there are distinct points $x_1,x_2 \in X$ such that $f(x_1)=f(x_2)=y$. Let $A = \{x_1\}$ so $x_2 \in X \smallsetminus A$. We have $f(x_2)=y \in f(X\smallsetminus A)$ but $y \ni Y \smallsetminus f(A) = Y \smallsetminus \{y\}$. Conversely, let $f$ be injective. Choose any $y \in f(X\smallsetminus A)$. This means that there exists $x \in X \smallsetminus A$ such that $f(x) = y$. If $y \in f(A)$, then $\exists a \in A$ with $f(a) = y=f(x)$. By hypothesis $x = a$ which is contradict with $x \ni A$. So $y \in Y \smallsetminus f(A)$. This proves (b), and (c) follows from (a) and (b). 
\end{proof}

\begin{prop}[\textbf{Image of Intersection Equal to Intersection of Image}]
Let $f : X \to Y$ be a map between two set $X$ and $Y$. For any subsets $A,B \subhim X$
$$
f(A \cap B) \subhim f(A) \cap f(B). 
$$
The equality hold if and only if $f : X \to Y$ is injective.
\end{prop}
\begin{proof}
For any $y \in f(A\cap B)$, there is some $x \in A \cap B$ such that $f(x) = y$. Since $x \in A$, then $y = f(x) \in f(A)$. Similarly $y=f(x) \in f(B)$. Hence $y \in f(A) \cap f(B)$. Here's the counterexample for which the reverse direction is fail. Take $X = \{0,1\}$ and $Y= \{0\}$ and define $f : X \to Y$ as $f(0) = f(1) = 0$. For the subsets $A=\{0\}, B = \{1\}$, we have $f(A\cap B) = \emptyset$ but $f(A)\cap f(B) = 0$. 

For the second, suppose that $f : X \to Y$ is injective and $A,B$ be any subsets of $X$. We have to show that 
$$
f(A) \cap f(B) \subhim f(A\cap B).
$$
Let $y \in f(A)\cap f(B)$. So we have $a \in A$ and $b \in B$ such that $f(a) = f(b) = y$. Since $f$ is injective then $a=b$. Therefore $y \in f(A\cap B) $. Conversely, suppose $f$ is not injective. So there are distinct points $x_1,x_2 \in X$ such that $f(x_1) = f(x_2) = y$. Take $A=\{x_1\}$ and $B\{x_2\}$ as subsets of $X$ we have $A\cap B  = \emptyset$. Therefore  
$$
f(A)\cap f(B) = f(x_1) \cap f(x_2) = y \nsubseteq f(A\cap B) = \emptyset.
$$
\end{proof}

\begin{prop}[\textbf{Characterization of Injective and Surjective Map}]
Let $f : X \to Y$ be any map between two sets $X$ and $Y$. Show that  :
\begin{enumerate}
\item[(a)] $f$ is injective if and only if for any subset $A \subhim X$, $f^{-1}(f(A)) = A$. 
\item[(b)] $f$ is surjective if and only if for any $B \subhim Y$, $f(f^{-1}(B)) = B$.
\end{enumerate}
\end{prop}
\begin{proof}
For (a), note that for any $A \subhim X$, $A \subhim f^{-1}(f(A))$ always holds even $f$ is not injective (consider $f : \er \to \er$, defined as $f : x\mapsto x^2$). Since for any $a \in A$, $f(a) \in f(A)$, hence $a \in f^{-1}(f(A))$. So what we actually need to show is that for any $A \subhim X$,
$$
f \text{ is injective } \iff f^{-1}(f(A)) \subhim A.
$$
First, suppose $f$ is injective and $A \subhim X$. For any $x \in f^{-1}(f(A))$, means that $f(x)=y \in f(A)$. Since $y \in f(A)$, then there exist $a \in A$ such that $f(a) = y = f(x)$. By hypothesis $f$ is injective. Therefore $x=a \in A$.  Conversely assume that $f^{-1}(f(A)) \subhim A$ for any $A \subhim X$. Let $x_1,x_2 \in X$ such that $f(x_1)  = f(x_2)$. Choose $A = \{x_1\}$, we have $f(A) = \{f(x_1)\}$ and $f^{-1}\big(\{f(x_1)\}\big) \subhim \{x_1\}$. Moreover $f^{-1}\big(\{f(x_1)\}\big) = \{x_1\}$, since the other direction is always true. Similarly, let $B = \{x_2\}$, then $f^{-1}\big(\{f(x_2)\}\big) = \{x_2\}$. Since $f(x_1) = f(x_2)$, therefore
$$
\{x_1\} = f^{-1}\big(\{f(x_1)\}\big) = f^{-1}\big(\{f(x_2)\}\big) = \{x_2\}.
$$
So $f$ is injective.

For (b), also note that for any $B \subhim Y$, $f(f^{-1}(B)) \subhim B$ always holds even though $f$ is not surjective (consider $f : \er \to \er$ defined as $f : x \mapsto \arctan(x)$). Since for any $y \in f(f^{-1}(B))$, then there exists $x \in f^{-1}(B)$ such that $f(x) = y$, which is means that $f(x) = y \in B$. So we have to show that for any $B \subhim Y$,
$$
f \text{ is surjective } \iff B \subhim f(f^{-1}(B)).
$$
First, suppose that $f$ is surjective and $B \subhim Y$. Let $y \in B$ arbitrary. By hypothesis, there exists $x \in X$ such that $f(x) = y$. So $x \in f^{-1}(\{y\})$. Since $\{y\} \subhim B \implies f^{-1}(\{y\}) \subhim f^{-1}(B)$, then $x \in f^{-1}(B)$. Hence $y=f(x) \in f(f^{-1}(B))$. Conversely, let $B \subhim f(f^{-1}(B))$ holds. Suppose by contrary that $f$ is not surjective. Then there is $y \in Y$ such that $f^{-1}(\{y\}) = \emptyset$. Choose $B = \{y\}$, we must have $\{y\} \subhim f(f^{-1}(\{y\})) = f(\emptyset) = \emptyset$. Contradiction.
\end{proof}

Compare the following proposition with the fact that any subspace of a second countable space is also second countable.

\subsection*{Topology : From Appendix of \cite{LeeSM}}
\begin{prop}[Exercise A.22 \cite{LeeSM}]\label{Exercise A.22}
Let $X$ be a topological space, and suppose $X$ admits a countable open cover $\mathcal{O} =\{U_i\}$ such that each set $U_i$ is second-countable in the subspace topology. Show that $X$ is second-countable.
\end{prop}
\begin{proof}
For each $U_i \in \mathcal{O}$, let $\mathcal{B}_i$ be its countable basis. Since $U_i$ is open in $X$, then each elements in $\mathcal{B}_i$ are also open in $X$. Then the union $\mathcal{B} = \bigcup \mathcal{B}_i$ is a countable collection of open subsets in $X$. To show $\mathcal{B}$ is a basis for the topology of $X$, we have to show that every open subset of $X$ is the union of some collection of elements in $\mathcal{B}$. Let $U$ be any open subset of $X$ and $x$ be any point in $U$. Since $\mathcal{O}$ cover $X$, then there is $U_i$ such that $x \in U_i$. Since $x \in U_i \cap U \subhim U_i$, then there are $B \in \mathcal{B}_i$ such that $x \in B \subhim U_i \cap U$. Therefore for any $x \in U$, we have $B_x \in \mathcal{B}$ contained in $U$. Hence we have $\bigcup_{x\in U}B_x = U$. So $\mathcal{B}$ is a countable basis for the topology of $X$. So $X$ is second -countable.
\end{proof}


\subsection*{Linear Algebra}

\begin{prop}[\textbf{The Invariant Property of Determinant and Trace of a Linear Map}]
Let $A : V \to V$ be a linear transformation on a $n$-dimensional vector space $V$. The determinant $\det A$ and trace $\text{Tr} A$ of the matrix representation are independent of the choice of basis of $V$.
\end{prop}
\begin{proof}
Let $\{e_i\}$ and $\{\wtilde{e_i} \}$ are basis for $V$ and $\{\omega^i\}$ and $\{\wtilde{\omega}^i\}$ are the dual basis respectively. By $Ae_i = A_i^j e_j$ and $A\wtilde{e_i} = \wtilde{A}_i^je_j$, we have
$$
A_i^j = \omega^j(Ae_i) \quad \text{and} \quad \wtilde{A}_i^j = \wtilde{\omega}^j(A\wtilde{e}_i).
$$
Write the relation between basis as $\wtilde{e}_i = B_i^k e_k$ and $\wtilde{\omega}^i = \bar{B}^i_k\omega^k$, where $\bar{B} = B^{-1}$, we have
$$
\wtilde{A}^i_j = \bar{B}^i_k  A^k_l B_j^l,
$$
which shows that $[\wtilde{A}^i_j] = B^{-1} [A^k_l] B$. And therefore by property of determinant, $\det [\wtilde{A}^i_j] = \det [A^i_j]$. And the trace is
\begin{align*}
\text{Tr} \wtilde{A} = \sum \wtilde{A}^i_i = \sum \wtilde{\omega}^i(A\wtilde{e}_i) = (B_i^j \bar{B}^i_k) \omega^k (Ae_j) = \sum A^k_k = \text{Tr} A.
\end{align*}
\end{proof}

\begin{prop}[Exercise B.10 \cite{LeeSM}]
	Let $V$ be a vector space, and let $v+S$ be an affine subspace of $V$ parallel to $S$.
	\begin{enumerate}[nolistsep]
		\item [(a)] Show that $v+S$ is a linear subspace if and only if it contains $0$, which is tru if and only if $v\in S$.
		\item [(b)] Show that $v+S = \tilde{v}+\tilde{S}$ if and only if $S=\tilde{S}$ and $v-\tilde{v} \in S$.
	\end{enumerate}
\end{prop}
\begin{proof}
	For (a), we have to show that the statement 
	$$
	v+S \text{ linear subspace } \Leftrightarrow 0 \in v+S
	$$
	is true if and only if $v \in S$. If it is true, then $0 \in v+S$ imeplies that $v = -w \in S$ for some $w \in S$. Conversely, let $v \in S$. The $\Rightarrow$ part is immidiate. For $\Leftarrow$, $v+S$ is closed since 
	$$
	(v+s_1) + (v+s_2) = v + (v+s_1+s_2) = v+ w \in v+S.
	$$
	The zero vector is $0$, and for any $v+s \in v+ S$, the inverse is given by $v+\bar{s}$, where $\bar{s} = -2v - s$.
	
	For (b), suppose $v+ S = \tilde{v} + \tilde{S}$. Since $S,\tilde{S}$ are subspace, then they contain $0$. Then there exist $\tilde{w} \in \tilde{S}$ and $w \in S$ such that $v+0 = \tilde{v} + \tilde{w}$ and $v+w = \tilde{v}+ 0 $. These implies $v-\tilde{v} = \tilde{w} \in \tilde{S}$ and $\tilde{v} - v = w \in S$. Since $S,\tilde{S}$ are subspace then $w,\tilde{w}=-w \in S \cap \tilde{S}$. So for any $s \in S$ we have some $\tilde{s} \in \tilde{S}$ such that
	$v + s = \tilde{v} + \tilde{s}$ which is implies $s = (\tilde{v} - v) + \tilde{s} \in \tilde{S}$. Hence $S \subhim \tilde{S}$. By the same way, $S \supseteq \tilde{S}$. The other direction is immidiate.
\end{proof}

\begin{prop}[Exercise B.11 \cite{LeeSM}]
Suppose $V$ is a vector space and $S$ is a linear subspace of $V$. Define vector addition and scalar multiplication of cosets $V/S:= \{v+S : v \in V\}$ by
$$
(v+S)+(w+S) = (v+w)+S, \qquad c(v+S) = (cv)+S.
$$	
\begin{enumerate}[nolistsep]
	\item [(a)] Show that the quotient $V/S$ is a vector space under this operation.
	\item [(b)] Show that if $V$ is finite-dimensional, then $\text{dim }V/S = \text{ dim }V - \text{ dim }S$.
\end{enumerate}
\end{prop}
\begin{proof}
	The operations are obviously closed. The zero vector is $0+S$. For any $v+S \in V/S$, the inverse is $(-v)+S$.
\end{proof}

\begin{prop}[Exercise B.48 \cite{LeeSM}]
For any $A \in \text{M}(m\times n, \er)$ and $B \in \text{M}(n\times r, \er)$, show that 
$$
|AB| \leq |A||B|.
$$
\end{prop}
\begin{proof}
Since $|A|^2 = \sum_{i,p} (A^i_p)^2$ and $|B|^2 = \sum_{q,j} (B^q_j)^2$, we have
\begin{align*}
|A|^2|B|^2 &=\sum_{i,p} (A^i_p)^2 \, \sum_{q,j} (B^q_j)^2 \\
&= \sum_i \Big(\sum_{p=1}^n (A^i_p)^2 \Big) \sum_{j} \Big( \sum_{q=1}^n (B^q_j)^2 \Big) \\
&= \sum_{i,j} \Big( \sum_{p,q=1}^n (A^i_p)^2(B^q_j)^2 \Big) \\
&= \sum_{i,j} \Big( \sum_{p\neq q} (A^i_p)^2(B^q_j)^2 + \sum_{p=q} (A^i_p)^2(B^q_j)^2 \Big) \\
&= \sum_{i,j} \Big( \sum_{p\neq q} (A^i_p)^2(B^q_j)^2 + \sum_{k=1}^n (A^i_k)^2(B^k_j)^2 \Big) \\
&= \sum_{i,j} \sum_{p\neq q} (A^i_p)^2(B^q_j)^2 + \sum_{i,j} \big[(AB)^i_j\big]^2 \\
&= \sum_{i,j} \sum_{p\neq q} (A^i_p)^2(B^q_j)^2 + |AB|^2 \geq |AB|^2.
\end{align*}
Hence $|AB| \leq |A||B|$.

\end{proof}

\subsection*{Multivariable Calculus}

\begin{prop}[Exercise 1.8.9 \cite{Moskowitz}] 
Let $S$ be a bounded set in $\er$. Prove that 
$$
\text{sup}(S) - \text{inf}(S) = \text{sup} \{ |x-y|: \forall x,y \in S \}.
$$
\end{prop}
\begin{proof}
It is immidiate from definition that the number  $s = \text{sup}(S) - \text{inf}(S)$ is an upper bound of $ \text{sup} \{ |x-y|: \forall x,y \in S \}$. Now, given any $\epsilon >0$ we have to show that there exists $x_0,y_0 \in S$ such that $s - \epsilon < |x_0-y_0|$. By definition $\text{sup}(S)$ and $\text{inf}(S)$, there exists $x_0,y_0 \in S$ so that $x_0 > \text{sup}(S) - \epsilon/2$ and $y_0 < \text{inf}(S) + \epsilon/2$. With this we have
$$
x_0-y_0 > \text{sup}(S) - \text{inf}(S) -\epsilon = s-\epsilon.
$$ 
Therefore $|x_0-y_0|> s-\epsilon$. This completes the proof.
\end{proof}


\begin{thebibliography}{9}
\bibitem[\textbf{Audin}]{Audin}
Audin, Mich\'ele and Mihai Damian: Morse Theory and Floer Homology. Springer, London (2014)

\bibitem[\textbf{Banyaga}]{BH}
Banyaga, Augustin and David Hurtubise: Lectures on Morse homology. Vol. 29. Springer Science \& Business Media, 2013.

\bibitem[\textbf{Bredon}]{bredon}
Bredon, Glen E.: Topology and Geometry. Springer, New York (1993)

\bibitem[\textbf{doCarmo}]{doCarmo}
do Carmo, Manfredo P.: Riemannian Geometry. Birkhauser, Boston (1992). Translated by the second Portuguese edition by Francis Flaherty.

\bibitem[\textbf{Hatcher}]{Hatcher}
Hatcher, Allen: Algebraic Topology. Cambridge University Press, Cambridge, UK, 2002

\bibitem[\textbf{Kosinski}]{Kosinski}
Kosinski, A.: Differential Manifolds. Dover Publication, 2007.

\bibitem[\textbf{LeeTM}]{LeeTM} Lee, John M.: Introduction to Topological Manifolds, 2nd ed. Springer GTM, New York (2011).

\bibitem[\textbf{LeeSM}]{LeeSM}
Lee, John M.: Introduction to Smooth Manifolds, 2nd ed. Springer GTM, New York (2003).

\bibitem[\textbf{LeeRM}]{LeeRM}
Lee, John M.: Riemannian Manifolds: An Introduction to Curvature. Springer GTM, New York (1997).

\bibitem[\textbf{LeeJeff}]{LeeJeff} Lee, Jeffrey M.: Manifolds and Differential Geometry. Am. Math. Soc., Providence (2009).

\bibitem[\textbf{LWTu01}]{LWTu01} Tu, Loring W.: An Introduction to Manifolds. Springer, New York (2011).

\bibitem[\textbf{LWTu02}]{LWTu02} Tu, Loring W.: Differential Geometry. Springer GTM, New York (2017).

\bibitem[\textbf{MilnorDT}]{MilnorDT}
Milnor, John: Topology from the Differentiable Viewpoint. rev. ed. Princeton University
Press, 1997.

\bibitem[\textbf{MilnorM}]{MilnorM}
Milnor, John: Morse Theory.(AM-51) Vol. 51. Princeton university press, 2016.

\bibitem[\textbf{MoP}]{Moskowitz} Moskowitz, Martin, Paliogiannis, Fotios : Functions of Several Real Variables. World Scientific (2011).

\bibitem[\textbf{Rotman}]{Rotman} Rotman, Joseph J.: An Introduction to Algebraic Topology. (Vol 119) Springer Science \& Business Media.

\bibitem[\textbf{VINK}]{Viro} O. Ya. Viro, O. A. Ivanov, N. Yu. Netsvetaev, V. M. Kharlamov : Elementary Topology : Problem Textbook. American Mathematical Society (2008).

\bibitem[\textbf{Willard}]{Willard}
Willard, Stephen: General Topology. Dover Publication, 2004.

\bibitem[\textbf{MunTop}]{Munkres}
Munkres, James R.: Topology, 2nd edn, Prentice-Hall, Upper Saddle River (2000)

\bibitem[\textbf{WPoor}]{WPoor}
Poor, Walter: Differential Geometric Structures. Dover Publication, 2007.

\bibitem[\textbf{YukioM}]{YukioM}
Matsumoto, Yukio: An introduction to Morse theory. Vol. 208. American Mathematical Soc., 2002.



\end{thebibliography}


\end{document}